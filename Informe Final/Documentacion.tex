\documentclass[a4paper,oneside]{book}
\usepackage[top=2.54cm, bottom=2.54cm, left=3.17cm, right=3.17cm]{geometry}
\usepackage[spanish,es-tabla]{babel} % Traduce cuadros por tablas
% La siguiente linea es para que se reconozcan los acentos.
% Si falla probar con la siguiente
\usepackage[utf8]{inputenc}
\usepackage[T1]{fontenc}
\usepackage{graphicx}
\usepackage{fontawesome}
\graphicspath{ {Imagenes/} }
\DeclareGraphicsExtensions{.png,.jpg}
\setlength{\parindent}{12pt}
\setcounter{secnumdepth}{3}
\usepackage{fancyhdr}
\pagestyle{fancy}
\fancyhf{}
\chead{Sistema de gestión de mueblería}
\lfoot{\leftmark}
\rfoot{\thepage}
\usepackage{titlesec}
% \setcounter{tocdepth}{3} para que aparezcan las subsecciones en el indice https://www.lawebdelprogramador.com/foros/TeX-Latex/619886-subsubsection.html
% \usepackage[latin1]{inputenc}
\usepackage[none]{hyphenat}
\usepackage{listings}
\usepackage{color}

\definecolor{dkgreen}{rgb}{0,0.6,0}
\definecolor{gray}{rgb}{0.5,0.5,0.5}
\definecolor{mauve}{rgb}{0.58,0,0.82}

\lstset{frame=tb,
  language=SQL,
  aboveskip=3mm,
  belowskip=3mm,
  showstringspaces=false,
  columns=flexible,
  basicstyle={\small\ttfamily},
  numbers=none,
  numberstyle=\tiny\color{gray},
  keywordstyle=\color{blue},
  commentstyle=\color{dkgreen},
  stringstyle=\color{mauve},
  breaklines=true,
  breakatwhitespace=true,
  tabsize=3
}

\usepackage[
pdfauthor={Nicolás Bachs, Loïk Choua},
pdftitle={Sistema de Gestión de mueblería},
pageanchor,
hidelinks,
plainpages=false,
pdfpagelabels,
hypertexnames=false,
unicode	]{hyperref}
% Poner links en el proyecto
% \url{http://www.latex-project.org/}
% \href{http://www.latex-project.org/}{latex project}
\usepackage{wrapfig}
\usepackage{lscape}
\usepackage{rotating}
\usepackage{epstopdf}
\usepackage{multirow, array}
\usepackage{array,longtable}
\usepackage{float}
\usepackage[table,xcdraw]{xcolor}
\usepackage[euler]{textgreek}
\usepackage{makecell}
\usepackage{caption}
% Comandos para los diagramas de actividad
\newcommand{\activityDiagram}[2]{
	\begin{figure}[H]
		\centering
		\includegraphics[width=\textwidth,height=0.95\textheight,keepaspectratio]{DiagramasActividad/DiagramaDeActividad/#1}
		\caption{#2}
	\label{fig:#1}
	\end{figure}
}

% Comandos para los diagramas de estado
\newcommand{\stateDiagram}[2]{
	\begin{figure}[H]
		\centering
		\includegraphics[width=\textwidth,height=0.95\textheight,keepaspectratio]{DiagramasEstado/DiagramaDeEstado/#1}
		\caption{#2}
	\label{fig:#1}
	\end{figure}
}


% Cargar imagen de escenario
% Parámetro 1: nombre de imagen - Parámetro 2: epígrafe
\newcommand{\stageImage}[2]{
	\begin{figure}[H]
		\centering
		\includegraphics[width=\textwidth,height=0.35\textheight,keepaspectratio]{escenarios/#1}
		\caption{#2}
	\label{fig:#1}
	\end{figure}
}

\newcommand{\stageTitle}[2]{
	\noindent\begin{minipage}{\textwidth}
	{#2}
	\stageImage{#1}{#2}
	\end{minipage}
}

\newcommand{\stageTitleD}[3]{
	\noindent\begin{minipage}{\textwidth}
	{#3}
	\begin{figure}[H]
		\centering
		\includegraphics[width=\textwidth,height=0.35\textheight,keepaspectratio]{escenarios/#1}
	\end{figure}
	\begin{figure}[H]
		\centering
		\includegraphics[width=\textwidth,height=0.35\textheight,keepaspectratio]{escenarios/#2}
		\caption{#3}
	\label{fig:#1}
	\end{figure}
	\end{minipage}
	}

% Comandos para numerar los escenarios
% Ejemplo de uso: \stageLogin{00} --> AI_01_00
\newcommand{\stageLogin}[1]{\hyperref[fig:login]{AI\_01\_#1}}
\newcommand{\stageMain}[1]{\hyperref[fig:main]{AI\_02\_#1}}
\newcommand{\stageHeader}[1]{\hyperref[fig:header]{AI\_03\_#1}}
\newcommand{\stageAlarms}[1]{\hyperref[fig:alarms]{AI\_04\_#1}}
\newcommand{\stageCommunicationErrors}[1]{\hyperref[fig:communicationErrors]{AI\_05\_#1}}
\newcommand{\stageSitesMap}[1]{\hyperref[fig:sitesMap]{AI\_06\_#1}}
\newcommand{\stageFilters}[1]{\hyperref[fig:filters]{AI\_07\_#1}}

\newcommand{\stageDashboard}[1]{\hyperref[fig:dashboard]{AI\_08\_#1}}
\newcommand{\stageManagement}[1]{\hyperref[fig:management]{AI\_09\_#1}}

\newcommand{\stageSiteData}[1]{\hyperref[fig:siteData]{AI\_10\_#1}}
\newcommand{\stageSiteLocation}[1]{\hyperref[fig:siteLocation]{AI\_11\_#1}}
\newcommand{\stageSiteComponent}[1]{\hyperref[fig:siteComponent]{AI\_12\_#1}}
\newcommand{\stageChannel}[1]{{AI\_13\_#1}}
\newcommand{\stageAdministration}[1]{{AI\_14\_#1}}

\newcommand{\stageComponentConfig}[1]{\hyperref[fig:componentConfig]{AI\_15\_#1}}
\newcommand{\stageRequestConfig}[1]{\hyperref[fig:requestConfig]{AI\_16\_#1}}
\newcommand{\stageInputConfig}[1]{\hyperref[fig:inputConfig]{AI\_17\_#1}}
\newcommand{\stageSiteConfig}[1]{\hyperref[fig:siteConfig]{AI\_18\_#1}}
\newcommand{\stageUnifilarConfig}[1]{\hyperref[fig:unifilarConfig]{AI\_19\_#1}}
\newcommand{\stageUnifilarConfigOp}[1]{\hyperref[fig:unifilarConfigOp]{AI\_20\_#1}}
\newcommand{\stageUnifilarConfigOpInputs}[1]{\hyperref[fig:unifilarConfigOpInputs]{AI\_21\_#1}}
\newcommand{\stageUnifilarConfigOpRequests}[1]{\hyperref[fig:unifilarConfigOpRequests]{AI\_22\_#1}}
\newcommand{\stageUnifilarConfigOpStates}[1]{\hyperref[fig:unifilarConfigOpStates]{AI\_23\_#1}}

\newcommand{\stageUserData}[1]{\hyperref[fig:userData]{AI\_24\_#1}}
\newcommand{\stageUserSite}[1]{\hyperref[fig:userSite]{AI\_25\_#1}}
\newcommand{\stageProfile}[1]{\hyperref[fig:profile]{AI\_26\_#1}}

\newcommand{\stageChannelCommunication}[1]{\hyperref[fig:channelCommunication]{AI\_27\_#1}}

\newcommand{\stageUnifilar}[1]{\hyperref[fig:unifilar]{AI\_28\_#1}}
\newcommand{\stageDevices}[1]{\hyperref[fig:devices]{AI\_29\_#1}}
\newcommand{\stageSiteRequests}[1]{\hyperref[fig:requestsSite]{AI\_30\_#1}}
\newcommand{\stageAnalogInputs}[1]{\hyperref[fig:analog]{AI\_31\_#1}}
\newcommand{\stageDigitalInputs}[1]{\hyperref[fig:digital]{AI\_32\_#1}}
\newcommand{\stageCommands}[1]{\hyperref[fig:commands]{AI\_33\_#1}}
\newcommand{\stageRequests}[1]{\hyperref[fig:requests]{AI\_34\_#1}}
\newcommand{\stageTasks}[1]{\hyperref[fig:tasks]{AI\_35\_#1}}

\newcommand{\stageHistory}[1]{{UI\_36\_#1}}

\newcommand{\stageName}[2]{Escenario #1 : #2}
\newcommand{\stageLoginName}{\stageName{\stageLogin{00}}{Inicio de sesión}}
\newcommand{\stageHeaderName}{\stageName{\stageHeader{00}}{Encabezado del sistema}}
\newcommand{\stageMainName}{\stageName{\stageMain{00}}{Vista principal}}
\newcommand{\stageAlarmsName}{\stageName{\stageAlarms{00}}{Alarmas}}
\newcommand{\stageCommunicationErrorsName}{\stageName{\stageCommunicationErrors{00}}{Errores de comunicación}}
\newcommand{\stageSitesMapName}{\stageName{\stageSitesMap{00}}{Mapa de sitios}}
\newcommand{\stageFiltersName}{\stageName{\stageFilters{00}}{Filtros de mapa}}

\newcommand{\stageDashboardName}{\stageName{\stageDashboard{00}}{Panel de control}}
\newcommand{\stageManagementName}{\stageName{\stageManagement{00}}{Gestión de entidades}}
\newcommand{\stageSiteDataName}{\stageName{\stageSiteData{00}}{Sitios - Datos generales}}
\newcommand{\stageSiteLocationName}{\stageName{\stageSiteLocation{00}}{Sitios - Ubicación}}
\newcommand{\stageSiteComponentName}{\stageName{\stageSiteComponent{00}}{Sitios - Configuración de componentes}}
\newcommand{\stageUserDataName}{\stageName{\stageUserData{00}}{Usuarios - Datos generales}}
\newcommand{\stageUserSiteName}{\stageName{\stageUserSite{00}}{Usuarios - Asignación de sitios}}
\newcommand{\stageProfileName}{\stageName{\stageProfile{00}}{Perfiles}}

\newcommand{\stageComponentConfigName}{\stageName{\stageComponentConfig{00}}{Configuración de componentes - Datos generales}}
\newcommand{\stageRequestConfigName}{\stageName{\stageRequestConfig{00}}{Configuración de componentes - Pedidos}}
\newcommand{\stageInputConfigName}{\stageName{\stageInputConfig{00}}{Configuración de componentes - Entradas}}
\newcommand{\stageSiteConfigName}{\stageName{\stageSiteConfig{00}}{Configuración de sitios - Datos generales}}
\newcommand{\stageUnifilarConfigName}{\stageName{\stageUnifilarConfig{00}}{Configuración de sitios - Creación de diagrama unifilar}}
\newcommand{\stageUnifilarConfigOpName}{\stageName{\stageUnifilarConfigOp{00}}{Configuración de diagrama unifilar}}
\newcommand{\stageUnifilarConfigOpInputsName}{\stageName{\stageUnifilarConfigOpInputs{00}}{Configuración de diagrama unifilar - Entradas}}
\newcommand{\stageUnifilarConfigOpRequestsName}{\stageName{\stageUnifilarConfigOpRequests{00}}{Configuración de diagrama unifilar - Pedidos}}
\newcommand{\stageUnifilarConfigOpStatesName}{\stageName{\stageUnifilarConfigOpStates{00}}{Configuración de diagrama unifilar - Estados de dispositivos}}
\newcommand{\stageChannelCommunicationName}{\stageName{\stageChannelCommunication{00}}{Sistema de comunicaciones}}

\newcommand{\stageUnifilarName}{\stageName{\stageUnifilar{00}}{Unifilar}}
\newcommand{\stageDevicesName}{\stageName{\stageDevices{00}}{Dispositivos}}
\newcommand{\stageSiteRequestsName}{\stageName{\stageSiteRequests{00}}{Pedidos de sitio}}
\newcommand{\stageAnalogInputsName}{\stageName{\stageAnalogInputs{00}}{Entradas analógicas}}
\newcommand{\stageDigitalInputsName}{\stageName{\stageDigitalInputs{00}}{Entradas digitales}}
\newcommand{\stageCommandsName}{\stageName{\stageCommands{00}}{Comandos}}
\newcommand{\stageRequestsName}{\stageName{\stageRequests{00}}{Pedidos}}
\newcommand{\stageTasksName}{\stageName{\stageTasks{00}}{Tareas}}

\newcommand{\stageHistoryName}{\stageName{\stageHistory{00}}{Históricos}}
\newcommand{\stageChannelName}{\stageName{\stageChannel{00}}{Canales}}
\newcommand{\stageAdministrationName}{\stageName{\stageAdministration{00}}{Administraciones}}


% Comandos para numerar los casos de uso
\newcounter{caseUseCounter}
\setcounter{caseUseCounter}{0}

\newcommand{\caseUseNext}[1]{%
	\refstepcounter{caseUseCounter}\label{CUL#1}
	\expandafter\gdef\csname CU#1\endcsname{CU\ref{CUL#1}}%
}

\newcommand{\caseUseNumber}[1]{%
	CU\ref{CUL#1}
}

% Comando para un diagrama de secuencia pequeño
% Parámetro 1: nombre de imagen - Parámetro 2: epígrafe
\newcommand{\SmallSecuenceD}[3]{
\begin{minipage}{0.45\textwidth}
{#3}
\end{minipage}%
\hfill
\begin{minipage}{0.45\textwidth}
\begin{tabular}{|p{\textwidth}}
\begin{figure}[H]
		\centering
		\includegraphics[width=\textwidth,height=0.35\textheight,keepaspectratio]{#1}
		\caption{#2}
	\label{fig:#1}
	\end{figure}
\end{tabular}
\end{minipage}%
}

% Comandos para las tablas de acciones
\newcommand{\actionFileName}{sinAcentos}
\newcommand{\actionName}{Nombre completo}
\newcommand{\actionsFunctions}{Nombre completo}


\definecolor{tableCaseUseBackground}{HTML}{2A7F84}
\definecolor{tableCaseUseFontColor}{HTML}{FFFFFF}

% Variables para el llenado de las tablas de casos de uso
\newcommand{\caseUseRow}[1]{& \multicolumn{2}{L{\secondColumnWidth}|}{{#1}} \\}
\newcommand{\caseUseCreated}{Editar fecha de creación}
\newcommand{\caseUseModified}{Editar fecha de modificación}
\newcommand{\caseUseName}{Usar renewcommand}
\newcommand{\caseUseShortName}{sinAcentos}
\newcommand{\caseUseSummary}{Editar resumen}
\newcommand{\caseUsePeople}{Editar personas}
\newcommand{\caseUsePreconditions}{Editar precondiciones}
\newcommand{\caseUsePostconditions}{Editar postcondiciones}
\newcommand{\caseUseScene}{Editar escenario}
\newcommand{\caseUseRequirementsGUI}{Editar requisitos}
\newcommand{\caseUseResponseTime}{Editar tiempo de respuesta}
\newcommand{\caseUseConcurrence}{Editar concurrencia}
\newcommand{\caseUseAvailability}{Editar disponibilidad}
\newcounter{rownumbers}
\newcommand{\rownumber}{\stepcounter{rownumbers}\arabic{rownumbers} }
\newcommand{\secondColumnWidth}{0.7\textwidth}

\newcolumntype{L}[1]{>{\raggedright\arraybackslash}p{#1}}
\newcommand{\addCaseUseRow}[1]{\caseUseRow{{#1}}}
\newcommand{\addCaseUseStep}[1]{\caseUseRow{\rownumber {#1}}}


\newcommand{\caseUseWidthScene}{0.7\textwidth}
\setcounter{rownumbers}{0}
	% @params 1: Nombre de la tabla de caso de uso - 2: Nombre del caso de uso
	\newcommand{\itemCaseUse}[2]{%
	\caseUseNext{#1}%
	\item \hyperref[tabla:#1]{\caseUseNumber{#1} {#2} (Página \pageref{tabla:#1})}}

	%Flujos alternativos
	\newcounter{beginAlternative} %Contador para las filas de cada flujo alternativo
	\newcommand{\alternativeCaseUse}{Editar flujo alternativo} %Contenido de la fila de casos de usos alternativos
	
	% @params 1: Nombre del caso de uso - 2: Paso en el que comienza
	\newcommand{\newAlternative}[2]{\setcounter{beginAlternative}{#2} \caseUseRow{\textbf{{#1}}}}
	\newcommand{\beginAlternative}{\stepcounter{beginAlternative}\arabic{beginAlternative} } %Devuelve el número de una nueva fila
	\newcommand{\alternativeRow}[1]{\caseUseRow{\beginAlternative {#1}}}


\usepackage{multicol}

\begin{document}
\sloppy

\begin{titlepage}
	\centering
	{\scshape universidad nacional de tucumán - facultad de ciencias exactas y tecnología\par}
	{\scshape departamento de electricidad, electrónica y computación\par}
	{\scshape\Large ingeniería en computación\par}
	\vspace{0.5cm}
	{\scshape noviembre 2020\par} %modificar e ingresar fecha cuando corresponda
	\vfill
	{\LARGE\mdseries Trabajo de Graduación\par}
	\vspace{0.5cm}
	{\scshape\huge Sistema de Gestión de mueblería\par}
	{\scshape revisión 3\par}
	\vfill
	{\mdseries autores\par}
	\par\noindent\rule{0.9\textwidth}{0.2pt}
	\begin{multicols}{2}
	{\large Nicolás Bachs\par}
	{\small CX1401187\par}
	\columnbreak
	{\large Loïk Choua\par}
	{\small CX1707456\par}
	\end{multicols}
	\vspace{0.5cm}
	{\mdseries tutores\par}
	\par\noindent\rule{0.9\textwidth}{0.2pt}
	\begin{multicols}{2}
	{\large Ing. Maximiliano Odstricil\par}
	{\small tutor\par}
	\columnbreak
	{\large Ing. Matías Mendiondo\par}
	{\small co-tutor\par}
	\end{multicols}
\end{titlepage}
\stepcounter{page}

\chapter*{Agradecimientos}
\addcontentsline{toc}{chapter}{Agradecimientos}
\markboth{Agradecimientos}{Agradecimientos}
\paragraph\indent
%Agradecemos enormemente a nuestros familiares, los cuales nos brindaron todo su apoyo y conocimiento durante todos estos años de vida.

\paragraph\indent
%A nuestros compañeros y amigos, con los cuales compartimos una gran cantidad de momentos vividos, gracias por darnos la posibilidad de contar con ustedes.

\paragraph\indent
%A todos los maestros y profesores que tuvimos en las diferentes etapas, todos ellos nos llevaron hasta aquí, en especial a todos aquellos que hicieron más de lo que les correspondía.

\paragraph\indent
%Particularmente a todas las personas que hicieron de este, un gran proyecto.

\begin{itemize}
	\item Ing. Matias Mendiondo
	\item Ing. Maximiliano Odstrcil
\end{itemize}


\chapter*{Dedicatorias}
\addcontentsline{toc}{chapter}{Dedicatorias}
\markboth{Dedicatorias}{Dedicatorias}

\paragraph\indent
% Escribir aca Bachs

\begin{flushright}
Nicolás Bachs
\end{flushright}

\paragraph\indent
% Escribir aca Loïk

\begin{flushright}
Loïk Choua
\end{flushright}

\tableofcontents
\chapter{Introducción}
\label{ch:capitulo1} 
\markboth{CAPÍTULO \ref*{ch:capitulo1}. Introducción}{CAPÍTULO \ref*{ch:capitulo1}. Introducción}
\section{Introducción}
\paragraph\indent
El presente documento corresponde al trabajo de graduación de la carrera de Ingeniería en Computación de la Facultad de Ciencias Exactas y Tecnología - Universidad Nacional de Tucumán, de los alumnos Nicolás Bachs y Loïk Choua.

Esta documentación surgió a partir de sucesivas versiones del mismo que fueron revisadas por los usuarios finales del sistema, los tutores y jurados del proyecto.

\section{Objetivos}
\subsection{Generales}
\paragraph\indent
Con respecto al trabajo de graduación, el objetivo principal poner en práctica todos los conocimientos y habilidades adquiridas durante el transcurso de la carrera, mediante la construcción de un producto de software que satisfaga las necesidades y genere valor agregado a un cliente determinado.

\subsection{Específicos}
\begin{itemize}
	\item Diseñar un sistema intuitivo, fácil de utilizar y que pueda ser implementado rápidamente.
	\item Diseñar el sistema de manera que se puedan añadir nuevas funcionalidades al mismo, teniendo en cuenta el constante avance de la tecnología.
\end{itemize}

\chapter{Selección del modelo de ciclo de vida y metodología de desarrollo de Software}
\label{ch:capitulo3}
\markboth{CAPÍTULO \ref*{ch:capitulo3}. Modelo de ciclo de vida y gestión del proyecto}{CAPÍTULO \ref*{ch:capitulo3}. Modelo de ciclo de vida y gestión del proyecto}
\section{Introducción}
\paragraph\indent
El ciclo de vida es la transformación que el producto software sufre a lo largo de su vida, desde que nace hasta que muere. El resultado de las transformaciones que sufre el producto software a lo largo de su ciclo de vida representa en esencia el producto mismo y se denomina estado.
\paragraph\indent
Un ciclo de vida determina el orden de las fases de un proceso software y establece los criterios de transición de una fase a la otra. El proceso software es una colección de actividades que comienzan con la identificación de una necesidad y concluye con el retiro del software que satisface dicha necesidad. 
\paragraph\indent
Al comienzo de un proyecto software se debe elegir el ciclo de vida que seguirá el producto a construir. El modelo de ciclo de vida elegido llevará a encadenar las tareas y actividades del proceso software de una determinada manera. Se debe tener en cuenta que algunas tareas serán realizadas una vez y otras deberán realizarse más de una vez. El ciclo de vida apropiado se elige en base a la cultura de la corporación, el dominio del problema, la comprensión de los requisitos y la volatilidad de los mismos.  
\paragraph\indent
Un proyecto sin estructura es un proyecto inmanejable, no puede ser planificado ni estimado ni mucho menos alcanzar un compromiso de costos y tiempo. La idea de buscar ciclos de vida que describan las actividades a realizar para transformar el producto surgen de tener un esquema que sirvan como base para: 
\begin{itemize}
	\item Planificar
	\item Organizar
	\item Asignar personal
	\item Coordinar
	\item Presupuestar
	\item Dirigir
\end{itemize}
\section{Selección de un modelo de ciclo de vida}
\paragraph\indent
El ciclo de vida para el desarrollo del sistema de gestión de la muebleria es el de prototipado evolutivo. Para la selección del mismo se tuvo en cuenta tanto la necesidad de brindar a los usuarios las características del nuevo sistema a medida que estas sean implementadas, como así también las partes volátiles del proyecto.
\paragraph\indent
Un prototipo es un sistema a escala que tiene las características del sistema disminuidas. Se usa cuando el cliente no tiene una idea muy detallada de lo que necesita o el ingeniero en software no está muy seguro de la viabilidad de la solución.
\paragraph\indent
Las evaluaciones del usuario retroalimentan el proceso para refinar los diseños y especificaciones del sistema emergente. Los prototipos pueden además utilizarse para verificar la viabilidad del diseño del sistema y como una herramienta iterativa del desarrollo de software donde el prototipo evoluciona hasta llegar al sistema final.
\paragraph\indent
Los prototipos evolutivos son fácilmente modificables y ampliables. Una vez definidos estos requisitos, el prototipo evolucionará hasta el sistema final. Un prototipo evolutivo tiene como característica que sigue el ciclo de vida estándar, pero con el tiempo de desarrollo bastante reducido y, además, la aplicación de estándares no es muy rigurosa.
\paragraph\indent
El modelo de desarrollo basado en prototipos es una versión modificada del modelo en cascada o clásico con el fin de contrarrestar las limitaciones que este posee. Las fases del ciclo de vida clásico quedan modificadas de la siguiente manera debido a la introducción del uso de prototipos:
\begin{itemize}
	\item Análisis preliminar y especificación de requisitos de usuarios: En esta fase se hace un primer análisis de las necesidades del usuario, especificaciones generales del sistema y estudios de viabilidad. Estas especificaciones preliminares forman la base sobre la que se apoya el diseño y la implementación del prototipo. 
	\item Diseño, desarrollo e implementación del prototipo: Lo importante al desarrollar el prototipo es que su implantación sea rápida y su coste de desarrollo sea bajo. Existe una serie de factores que deben ser tenidos en cuenta para conseguir dichos objetivos:
	\begin{itemize}
		\item Énfasis en la interfaz del usuario, la cual debe permitir completar los requerimientos después del desarrollo.
		\item Desarrollar el prototipo con un pequeño equipo de desarrollo que minimice los problemas de comunicación y sean los mismos que desarrollaron las especificaciones.
		\item Utilizar un lenguaje adecuado para el desarrollo del prototipo, que permita una rápida detección de errores y facilidades de manipulación de datos.
		\item Buscar las herramientas adecuadas para un desarrollo rápido.
  \end{itemize}
	\item Prueba del prototipo: Es importante desarrollar adecuadamente esta actividad y extraer el máximo de conclusiones de la experiencia de los usuarios en el uso del prototipo. Como elementos claves se debe señalar:
	\begin{itemize}
		\item La asignación de un tiempo suficiente a la actividad de planificación del desarrollo, como para que los usuarios puedan probar el prototipo y comunicar sus experiencias.
		\item Los departamentos de usuarios deben comprometerse a probar adecuadamente dicho prototipo.
		\item Planificar la formación y entrenamiento de los usuarios en el uso de prototipos.
		\item Desarrollar metodologías para recoger las impresiones de los usuarios.
  \end{itemize}
	\item Refinamiento iterativo del prototipo: Con la información proporcionada por el usuario, debe modificarse el prototipo en forma rápida para que pueda ser probado nuevamente por los usuarios. Esta experiencia y refinamiento se continúa hasta alcanzar el estado donde los beneficios de mejorar aún más el prototipo sean menores que el tiempo y costo requerido para tales modificaciones.
	\item Refinamiento de las especificaciones de requisitos: Toda la información aportada por los usuarios se analiza, y a partir de la misma se revisan las especificaciones de requisitos. Sobre ellas se procede al diseño e implementación del sistema de producción.
	\item Diseño e implantación del sistema de producción: Para ello se sigue el modelo clásico de V, al que se habrá aportado una gran intuición acerca de cómo se debería desarrollar el sistema real. 
\end{itemize}
\paragraph\indent
Existen distintos modelos de prototipos, se tienen los prototipos desechables, las maquetas y los prototipos evolutivos, este último es el que se utilizará. Es un modelo de trabajo del sistema propuesto, que aporta a los usuarios una representación física de las partes claves del sistema antes de la implantación. Es fácilmente modificable y ampliable, y una vez definidos todos los requisitos, el prototipo evolucionará hacia el sistema final. Es allí donde se implantan primero aquellos requisitos y necesidades que son claramente entendidos, y para aquellos que son críticos se realiza un diseño y análisis en detalle.
\section{Selección de metodología de desarrollo de software}
	\paragraph\indent
	Una metodología indica el camino a seguir para el desarrollo de un proyecto de software. Para el desarrollo del sistema de gestión de la mueblería se utilizará la metodología V-Script.
	\paragraph\indent
	La metodología Script o V-Script es una metodología de desarrollo de software que tiene un alto componente dinámico, orientado hacia la interfaz de usuario. Mediante el proceso Script, se capturan las necesidades del usuario mediante la construcción de maquetas o prototipos desechables, tratando de capturar la expectativa del usuario: qué es lo que el usuario espera que haga el producto.
	\paragraph\indent
	La metodología permite integrar perfectamente los procesos de estimación, gestión de calidad, gestión de configuración y verificación y validación del software. Esta metodología se adapta perfectamente al estándar IEEE 1074-1991: Standard for Developing Software Life Cycle Process.
	\paragraph\indent
	La selección de esta metodología implica que el ciclo de vida mostrado a continuación será el utilizado para completar cada iteración del prototipado evolutivo, recorriendo el modelo una vez por cada iteración. Siendo fundamental la primera iteración para implementar el núcleo.
	\paragraph\indent
		\begin{figure}[H]
            \centering
            \includegraphics[width=0.8\textwidth,height=0.45\textheight,keepaspectratio]{Extras/DiagramaScript}
            \caption{Diagrama Script}
        \label{fig:diagramaScript}
        \end{figure}
	\paragraph\indent
	En el diagrama puede observarse dos ejes fundamentales:
	\begin{itemize}
		\item \textbf{Eje horizontal}: divide al ciclo de vida en dos partes: una orientada hacia el cliente y otra orientada hacia el desarrollador computadora.
		\item \textbf{Eje vertical}: divide al ciclo de vida en etapas de desarrollo y etapas de prueba del software.
  	\end{itemize}
	Después de finalizada cada etapa, se determina una línea de base. Una línea de base es un punto del ciclo de vida que permite tomar decisiones condicionantes. Se evalúa todo lo realizado hasta ese momento mediante revisiones formales y se decide seguir adelante o bien continuar en cada etapa.
	\paragraph\indent
	Cada etapa Script de desarrollo tiene asociada una etapa de prueba al mismo nivel de abstracción. Estas etapas permiten verificar y validar el producto en los diferentes puntos del camino del producto desde la necesidad hacia la máquina.
%\input{capitulos/capitulo2/cap2sec4.tex}
%\input{capitulos/capitulo2/cap2sec4-1.tex}
%\input{capitulos/capitulo2/cap2sec4-2.tex}
%\input{capitulos/capitulo2/cap2sec4-3.tex}
%\input{capitulos/capitulo2/cap2sec4-4.tex}
%\input{capitulos/capitulo2/cap2sec4-5.tex}
%\input{capitulos/capitulo2/cap2sec4-6.tex}
%\input{capitulos/capitulo2/cap2sec5.tex}
\chapter{Especificación de Requisitos Complementarios del software (ANSI/IEEE 830)}
\label{ch:capitulo2} 
\markboth{CAPÍTULO \ref*{ch:capitulo2}. Especificación de requisitos de software}{CAPÍTULO \ref*{ch:capitulo2}. Especificación de requisitos de software}
\section{Análisis de requisitos del sistema}
	\paragraph\indent
	Esta especificación tiene como objetivo analizar y documentar las necesidades funcionales que deberán ser soportadas por el sistema a desarrollar. Para ello, se identificarán los requisitos que ha de satisfacer el nuevo sistema mediante entrevistas, el estudio de los problemas de las unidades afectadas y sus necesidades actuales. Además de identificar los requisitos se deberán establecer prioridades, lo cual proporciona un punto de referencia para validar el sistema final que compruebe que se ajusta a las necesidades del usuario. 
	\subsection{Identificación de los usuarios participantes}
		\paragraph\indent
			Los objetivos de esta tarea son identificar a los responsables de cada una de las unidades implicadas y a los principales usuarios implicados. En la organización se identificaron los siguientes usuarios:
		\begin{itemize}
			\item Gerente de Empresa Zimmerman Muebles SRL: es el solicitante de la aplicación.
			\item Vendedores: Formado por los usuarios capaces de realizar funciones del sistema relacionadas con presupuestos y ventas.
            \item Fabricantes: Formado por aquellos usuarios que llevan a cabo la producción de muebles.
            \item Administradores: Formado por aquellos usuarios que poseen los permisos para realizar todas las funciones del sistema.
		\end{itemize}
		\paragraph\indent
			Es de destacar la necesidad de una participación activa de los usuarios del futuro sistema en las actividades de desarrollo del mismo, con objeto de conseguir la máxima adecuación del sistema a sus necesidades y facilitar el conocimiento paulatino de dicho sistema, permitiendo una rápida implantación.
	\subsection{Catálogo de requisitos del sistema}
		\paragraph\indent
        El objetivo de la especificación es definir en forma clara, precisa, completa y verificable todas las funcionalidades y restricciones del sistema que se desea construir. Esta documentación está sujeta a revisiones por el grupo de usuarios que se recogerán por medio de sucesivas versiones del documento, hasta alcanzar su aprobación por parte de la dirección de Zimmerman Muebles SRL. y del grupo de usuarios. Una vez aprobado, servirá de base al equipo para la construcción del nuevo sistema. 
        \paragraph\indent
			Esta especificación se ha realizado de acuerdo al estándar ``IEEE Recommended Practice for Software Requirements Specifications (IEEE/ANSI 830-1998)''.
		\subsubsection{Objetivos y alcance del sistema}
			\paragraph\indent
            El principal objetivo es desarrollar una aplicación web para el uso exclusivo del personal interno de la mueblería Zimmerman Muebles SRL, a partir de ahora ZM, con el fin de automatizar el proceso de manejo de stock, desde la emision de presupuestos, pasando por las ventas, hasta la entrega de productos.
			\paragraph\indent
            Se podrá realizar la gestión de usuarios, clientes, productos, presupuestos, ventas, órdenes de producción, remitos y órdenes de resposición. Además, se podrá gestionar la entrega de los productos. El futuro sistema llevará el nombre de \textbf{ZMGestion}.
            \paragraph\indent
            En esta versión se realizará lo recién mencionado dejando para futuras versiones la  contabilidad y la gestión con los proveedores.
            \paragraph\indent
            El desarrollo lo llevará a cabo \textbf{NL}, con opción a ser resopnsable del posterior mantenimiento de éste.
        \subsubsection{Definiciones, acrónimos y abreviaturas}
            \underline{Definiciones:}
            \begin{itemize}
                \item Interfaz: En este contexto, llamamos interfaz a la pantalla que una persona puede acceder para recibir o transmitir información.
                \item Cliente: toda aquella persona que solicitó un presupuesto y/o efectuó una compra.
                \item ZM: refiere a la empresa Zimmerman Muebles SRL.
            \end{itemize}
            \underline{Acrónimos:}
            \begin{itemize}
                \item ZMGestion: refiere al sistema a desarrollar.
            \end{itemize}
            \underline{Abreviaturas:}
            \begin{itemize}
                \item IEEE: Institute of Electrical and Electronics Engineers.
                \item CUIT: Clave Única de Identificación Tributaria.
                \item CUIL: Clave Única de Identificación Laboral.
            \end{itemize}
            
		\subsubsection{Descripción general}
			\paragraph\indent
				Esta sección presenta una descripción general del sistema con el fin de conocer las funciones que debe soportar, los datos asociados, las restricciones impuestas y cualquier otro factor que pueda influir en la construcción del mismo.
				\paragraph\indent\textbf{Usuarios:}
				\paragraph\indent
				Para poder acceder a ZM Gestion se deberá contar con una cuenta (nombre de usuario y contraseña). Cada empleado de la mueblería tendrá una sola cuenta de usuario, de ellos se desea saber: nombres, apellidos, tipo y número de documento, fecha de nacimiento, fecha de inicio de actividad laboral en la empresa, estado civil, cantidad de hijos, número de telefono de contacto, correo electrónico y contraseña. Todos los datos recién mencionados serán de carácter obligatorio. Todas las funciones del sistema requerirán que los usuarios inicien sesión. Los usuarios tendrán dos estados posibles: alta o baja. Cuando su estado sea dado de alta, tendran la posibilidad de iniciar sesión y ejercer pleno uso de sus funciones. Mientras que cuando estén dados de baja no podrán iniciar sesión.
				\paragraph\indent\textbf{Roles:}
				\paragraph\indent	
				Cada empleado tendrá un rol, pudiendo ser: Administrador, Vendedor o Productor. Dependiendo del rol que ocupen en la mueblería, tendrán distintos permisos lo que les permitirá realizar distintas funciones del sistema. Los roles se podrán gestionar (crear, modificar, borrar, dar de alta y dar de baja) por aquellos usuarios que tengan el permiso de hacerlo.
				\paragraph\indent\textbf{Gestión empleados:}
				\paragraph\indent	
				Exisitira una gestión de empleados (crear, modificar, borrar, dar de alta y dar de baja) que será realizada por los usuarios con los debidos permisos. No se podrán borrar aquellos usuarios que hayan realizado algún presupuesto, venta u orden de producción. O bien, tengan asociada alguna orden de producción.
				\paragraph\indent\textbf{Clientes:}
				\paragraph\indent	
				Un cliente es aquella persona (física o jurídica) que solicita un presupuesto o realiza una compra. Para que un cliente sea dado de alta se necesita saber: el tipo de persona (física o jurídica), nombres y apellidos (o razón social en caso de las personas jurídicas), tipo y número de documento (CUIT en caso de las personas jurídicas), correo electrónico, nacionalidad (en caso de las personas físicas), número de teléfono y domicilio. Siendo obligatorio: el tipo de persona, nombres y apellidos (o razón social), tipo y número de documento y correo electrónico. El correo electrónico será único entre los usuarios y se podrán dar de alta clientes que tengan el mismo tipo y número de documento, emitiendo una advertencia que ya existe un cliente con los mismos datos. Los clientes se podrán gestionar (crear, modificar, borrar, dar de alta y dar de baja) por aquellos usaurios que cuenten con los permisos necesarios. No se podrá borrar un cliente que tenga asociado un presupuesto o venta.
				\paragraph\indent\textbf{Presupuestos:}
				\paragraph\indent	
				Al momento de presentarse un cliente solicitando un presupuesto se comprobará si está activo para ello se preguntará el tipo y número de documento y se verificará el resto de los datos. En caso de no existir se procederá a crearlo. Un presupuesto tiene un código de identificación, fecha en la que se realiza, periodo de validez (previamente asignado por los administradores), cliente asociado, una o más líneas de presupuesto y estado (pudiendo ser: en creacion, incompleto, creado,  vendido y expirado). Los presupuestos se podrán gestionar (crear, modificar y borrar) por aquellos usuarios que cuenten con los permisos necesarios. No se podrán borrar presupuestos que hayan sido utilizados para realizar una venta. Los presupuestos serán enviados por correo electrónico siempre y cuando el cliente haya proporcionado dicha información.
				\paragraph\indent\textbf{Líneas de presupuesto:}
				\paragraph\indent	
				Una línea de presupuesto está formada por un producto, la cantidad solicitada de dicho producto, el importe unitario, el importe total para dicha cantidad y un estado (pudiendo ser: pendiente, utilizado o no utilizado). Las líneas de presupuestos tendrán como estado utilizado o no utilizado, cuando el presupuesto se encuentre en estado vendido. Si un presupuesto no se encuentra en estado vendido, las líneas de presupuestos deberán estar en estado pendiente. Las líneas de presupuesto se podrán gestionar (crear, modificar y borrar) por aquellos usuarios que cuenten con los permisos necesarios. Además algunos usuarios contarán con un permiso que les permitirá modificar el importe unitario o total de una línea de presupuesto.
				\paragraph\indent\textbf{Productos:}
				\paragraph\indent	
				Un producto tiene nombre, código identificador (único para cada producto), tipo de producto ( con proceso de producción, sin proceso de producción o a medida), lustre y tela. Además cada producto pertenecerá a una categoría y grupo. Todos los datos de los productos son obligatorios excepto la tela y/o lustre para aquellos sin proceso de producción. Los usuarios que cuenten con los permisos necesarios podrán gestionar productos (crear, modificar, borrar, dar de alta y dar de baja). No se podrán borrar productos que hayan sido utilizados en algún presupuesto, venta, orden de producción u orden de resposición.
				\paragraph\indent\textbf{Categorías de producto:}
				\paragraph\indent	
				Los productos perteneceran a una unica categoria y una categoría podrá tener cero o mas productos. De la categoria se desea saber el nombre y estado (pudiendo ser alta o baja). Las categorías se podrán gestionar (crear, modificar, borrar, dar de alta y dar de baja) por aquellos usuarios que cuenten con  los permisos suficientes. No se podran borrar categorias que tengan al menos un producto asociado.
				\paragraph\indent\textbf{Grupos de productos:}
				\paragraph\indent	
				Los productos se encontrarán asociados en grupos de productos. Estos estarán compuestos por cero o mas productos. De los grupos se desea saber: codigo, estado (pudiendo ser alta o baja) y descripción. Los grupos de productos se podrán gestionar (crear, modificar, borrar, dar de alta y dar de baja) por aquellos usuarios que cuenten con  los permisos suficientes. No se podrán borrar grupos que estén compuestos por al menos un producto.
				\paragraph\indent\textbf{Ubicaciones y Stock:}
				\paragraph\indent	
				Los productos tendran asociado una o mas ubicaciones geograficas. De las ubicaciones se desea saber: nombre y dirección. Las ubicaciones podran tener almacenados cero o mas productos. Las ubicaciones se podrán gestionar (crear, modificar, borrar, dar de alta y dar de baja) por aquellos usuarios que cuenten con  los permisos suficientes.. Por cada producto se desea saber: cantidad por ubicación a una fecha dada y observaciones.
				\paragraph\indent\textbf{Lista de precios:}
				\paragraph\indent	
				Cada producto tendrá asociado uno o más precios con su respectiva fecha de establecimiento. El precio vigente será el ultimo que se haya establecido. La actualización de precios se podrá realizar a todos los productos pertenecientes a un determinado grupo en igual proporción (incremento o decremento) respecto del precio vigente de cada producto. La lista de precio se podrá generar en formato PDF.
				\paragraph\indent\textbf{Ventas:}
				\paragraph\indent	
				Una venta se podrá realizar a partir de uno o más presupuestos existentes o no. En caso de que un cliente quiera realizar una compra de uno o más productos que hayan sido presupuestados previamente, un usuario deberá buscar dicho/s presupuesto/s a través de su código de identificación o bien con datos del cliente para así acceder a sus presupuestos. De los presupuestos se podrán elegir cuales líneas de presupuesto el cliente desea comprar, pudiendo modificar la cantidad del mismo y asignar el precio a aquellos que no tuviesen, o bien el precio del producto haya cambiado. 
				\paragraph\indent
				Las líneas de presupuesto que hayan sido seleccionadas, pasarán al estado de utilizados, mientras que las descartadas pasarán al estado de no utilizadas. A su vez, el presupuesto utilizado pasará al estado vendido.
				\paragraph\indent
				El cliente deberá realizar un pago total o parcial y se le solicitará, en caso que no lo haya otorgado, su número de teléfono, una dirección e información de entrega (nombre y apellido de la persona que recibirá el o los productos en la dirección de entrega), siendo de carácter obligatorio el número de teléfono. Una venta tendrá un cliente, código de identificación, una o más factura, de las cuales sólo una podrá estar activa, uno o más recibos, fecha de venta, monto total, una o más líneas de producto, una orden de producción, uno o más remitos, ubicación donde se generó la venta, un usuario asociado y un estado. Dichos datos serán obligatorios.
				\paragraph\indent
				Los posibles estados de una venta son los siguientes:
				\begin{itemize}
					\item \textbf{En creación:} Cuando no tiene lineas de producto o todas las lineas de producto se encuentran en creación.
					\item \textbf{Congelada:} Cuando se quita la posibilidad de modificar o borrar la venta para esperar la adhesión de un pago.
					\item \textbf{En revisión:} Cuando se necesita que un administrador revise una venta ya que una linea de producto perteneciente a la venta no tiene el precio actual.
					\item \textbf{Pendiente:} Cuando al menos una linea de producto se encuentra en estado distinto de en creación, cancelada o entregada.
					\item \textbf{Cancelada:} Cuando todas las líneas de producto se encuentran en estado cancelada. El/Los presupuesto/s y sus respectivas líneas vuelven al estado en que se encontraban antes de realizar la venta. Además se dará de baja todas las facturas y recibos asociados, y se realizará una nota de crédito de ser necesario.
					\item \textbf{Entregada:} Cuando todas las líneas de producto no canceladas estén en estado entregadas.
				\end{itemize}
				\paragraph\indent
				Cuando todas las líneas de producto no canceladas de una venta se encuentren en estado verificada se enviará un correo electrónico al cliente (en caso de haberlo proporcionado) notificándole su deuda en caso de existir e informándole que los productos que compró ya están disponibles.
				\paragraph\indent\textbf{Lineas de producto:}
				\paragraph\indent	
				Una venta tiene una o mas líneas de producto. Una linea de producto está compuesta por: un producto, la cantidad solicitada de dicho producto, cero o más observaciones, el precio por la cantidad solicitada y estado. Siendo de caracter obligatorio producto, cantidad y estado.
				\paragraph\indent
				Los posibles estados de una linea de producto son:
				\begin{itemize}
					\item \textbf{En creación:} Cuando una linea de producto se está creando.
					\item \textbf{Pendiente:} Cuando la linea de producto fue creada y se debe determinar si es necesario asignarlo a una orden de producción.
					\item \textbf{Cancelada:} Cuando un cliente solicita una cancelación de la linea de producto.
					\item \textbf{Pendiente de producción:} Cuando un administrador seleccionó la linea para ser producida pero aún no se está realizando ninguna de sus tareas de producción.
					\item \textbf{En producción:} Cuando se están realizando las tareas de producción asignadas para su producción.
					\item \textbf{Entregada:} Cuando una linea de producto fue entregada.
					\item \textbf{Verificado:} Cuando un producto ya está listo para ser entregado (haya pasado, o no, por un proceso de producción). El/Los productos podrán ser entregados si el credito es superior al total de dicha linea de producto. Se debe tener en cuenta que:
				\end{itemize}
				\begin{center}
					CREDITO = MONTO TOTAL DE VENTA - (TOTAL EN PRODUCTOS ENTREGADO + DEUDA)

					DEUDA = MONTO TOTAL - SUMA DE RECIBOS
				\end{center}
				\paragraph\indent\textbf{Tareas de producción:}
				\paragraph\indent	
				Al momento de producir una linea de producto se deben realizar una serie de tareas. Las tareas pueden tener encadenada una tarea a realizarse posterior a que se ha verificado su finalización o no. Las tareas son asignadas a un fabricante y deben ser verificadas por un administrador. Los posibles estados de las tareas son:
				\begin{itemize}
					\item \textbf{Pendiente:} Cuando se creo y asigno a un fabricante una tarea.
					\item \textbf{En proceso:} Cuando el fabricante indica que ha comenzado a realizar la tarea.
					\item \textbf{Cancelada:} Cuando un administrador desea cancelar la ejecución de la tarea.
					\item \textbf{Pausada:} Cuando un administrador desea pausar la ejecución de una tarea.
					\item \textbf{Finalizada:} Cuando el fabricante indica que ha finalizado su tarea.
					\item \textbf{Verificada:} Cuando un administrador verifica que la tarea se ha finalizado de forma correcta.
				\end{itemize}

				\paragraph\indent\textbf{Facturas:}
				\paragraph\indent	
				Una venta tendrá una o más facturas asociadas, de las facturas se desea saber: número de factura, venta asociada, fecha en la que se realizó, uno o más recibos, cero o más notas de crédito y estado (pudiendo ser: alta o baja). Si una venta se encuentra en estado pendiente o entregada, deberá tener una factura en estado activo. Las facturas se podrán gestionar (crear, modificar, borrar, dar de alta y dar de baja) por aquellos usuarios que cuenten con los permisos necesarios.
				\begin{center}
					MONTO TOTAL DE VENTA = SUMA TOTAL DE LOS RECIBOS ASOCIADOS A LA FACTURA ACTIVA DE DICHA VENTA - MONTO TOTAL DE LA NOTA DE CRÉDITO ASOCIADA A LA FACTURA ACTIVA DE DICHA VENTA (EN CASO DE EXISTIR)
				\end{center}
				\paragraph\indent
				Al momento de cancelar una venta se podrá cancelar la factura (pasandola al estado baja) o bien asociarle una o más notas de crédito.
				\paragraph\indent\textbf{Notas de crédito:}
				\paragraph\indent	
				Una nota de crédito tendrá un número de identificación, una fecha en la que se realizó, el monto de la misma, una factura asociada y un estado (pudiendo ser: alta o baja).  Las notas de crédito se podrán gestionar (crear, borrar, dar de alta y dar de baja) por aquellos usuarios que cuenten con los permisos necesarios.
				\paragraph\indent\textbf{Recibos:}
				\paragraph\indent	
				Cada factura tendrá uno o más recibos asociados. De los recibos interesa saber: código de identificación, fecha en que se realizó, cantidad de dinero recibido, medio de pago utilizado, factura asociada y descripción. Los recibos podrán gestionarse (crear, modificar y borrar) por aquellos usuarios que cuenten con los permisos necesarios para hacerlo.
				\paragraph\indent\textbf{Órdenes de Producción:}
				\paragraph\indent	
				Una orden de producción está compuesta por una o mas líneas de producto, fecha de creación, fecha en la que se finaliza, usuario que la crea, usuario que marca como finalizada y usuario que verifica que este finalizada y estado.
				\paragraph\indent
				Los estados de una orden de producción pueden ser:
				\begin{itemize}
					\item \textbf{En creación:} Cuando se estan agregando líneas de producto a la orden de producción.
					\item \textbf{Pendiente:} Cuando las lineas de producto no canceladas, no verificadas y no entregadas de la orden de producción se encuentran en estado pendiente de producción.
					\item \textbf{En producción:} Cuando al menos una linea de producto de la orden de producción se encuentran en estado de en producción.
					\item \textbf{Cancelada:} Cuando todas las líneas de producto de la orden de producción se encuentran en estado cancelada.
					\item \textbf{Verificada:} Cuando todas las lineas de productos no canceladas se encuentran verificadas o entregadas.
				\end{itemize}
				\paragraph\indent
				Una orden de producción puede ser creada a partir de una venta existente o no. En el caso de que sea creada a partir de una venta, no se podrán agregar nuevas líneas de producto para producir.
				\paragraph\indent
				Cuando una orden de producción pase al estado de verificada se deberá asignar una ubicación, de manera que todos las líneas de producto producidas se agregarán al stock correspondiente.
				\paragraph\indent\textbf{Remitos:}
				\paragraph\indent	
				Un remito tendrá una fecha y dirección de entrega, una o más líneas de producto y una venta asociada, por cada linea de producto, se podrá seleccionar la ubicación de la cual se extraerá. Al generarse el remito se deberá cambiar el estado de la línea de producto de la venta a entregado.  Además se decrementará el stock correspondiente.
				\paragraph\indent\textbf{Órdenes de reposición:}
				\paragraph\indent	
				Los productos existentes se podrán mover de una ubibación a otra, para ello se deberá generar una orden de reposición en la cual se indicará la cantidad de productos a mover, ubicación origen y ubicación destino. Decrementando el stock en la ubicación origen e incrementando el stock en la ubicación destino.
		
\section{Suposiciones y dependencias}
\paragraph\indent 	
\textbf{Suposiciones:} Se asume que los requisitos en este documento son estables una vez que sean aprobados por la Dirección de Zimmerman Muebles SRL. Cualquier petición de cambios en la especificación debe ser aprobada por todas las partes intervinientes y será gestionada por el equipo de desarrollo. 
\paragraph\indent
\textbf{Dependencias:} No posee dependencias.

\section{Requisitos de usuario y tecnológicos}
\textbf{Requisitos de usuario:} Los usuarios serán todas aquellas personas mayores de 18 años que puedan acceder a la aplicación Web, sean empleados de la empresa Zimmerman Muebles SRL y tengan una dirección de correo electrónico. Las interfaces deben ser intuitivas, fáciles de usar y amigables, de manera que con unas breves instrucciones sean capaces de usarla.

\textbf{Requisitos tecnológicos:} En vista de que la aplicación debe correr en diferentes navegadores de diferentes dispositivos, y teniendo en cuenta un futuro crecimiento de ZM Gestión, se optará por un entorno económico y fácil de instalar. La aplicación se ejecutará sobre un esquema Cliente/Servidor Internet, con los procesos ejecutándose parte en el servidor web y de bases de datos, y la interfaz de usuario y procesos de interfaz ejecutándose en los clientes y éstos solicitando requerimientos al servidor vía el protocolo HTTP. El navegador del cliente debe ser HTML 2.0 compatible y los servidores serán dimensionados en base a los servicios web, de bases de datos y conexiones simultáneas; la aplicación debe estar disponible en un régimen de 24x7 y el número esperado de usuarios será de 100, con un factor de simultaneidad de 40 porciento. 

\section{Requisitos de interfaces externas}
\textbf{Interfaces de usuario:} La interfaz de usuario debe ser web responsive.

\textbf{Interfaces hardware:} Pantalla 320 x 80 mínimo, teclado alfanumérico, y dispositivo señalizador o bien pantalla táctil. 

\textbf{Interfaces software:} La aplicación deberá interactuar con un sistema de envíos automático de correos electrónicos tipo MailGun.

\section{Requisitos de rendimiento}
El tiempo de respuesta de la aplicación a cada función solicitada por el usuario no debe ser superior a los 10 segundos en una velocidad efectiva de conexión de 512 Kbps. El tiempo de respuesta a los listados dependerá de la cantidad de datos solicitados.

\section{Requisitos de desarrollo}
El ciclo de vida será el de Prototipado Evolutivo, debiendo orientarse hacia el desarrollo de un sistema flexible que permita incorporar de manera sencilla cambios y nuevas funcionalidades. 

\section{Restricciones de diseño}
\paragraph\textbf{Ajuste a estándares:} No se han definido.

\paragraph\textbf{Seguridad:} La seguridad de la comunicación será establecida por https con ssh de 128 bits y la de los datos por el Sistema Gestor de Base de Datos Relacional que se emplee. 

\paragraph\textbf{Política de Respaldo:} La seguridad de la comunicación será establecida por https con ssh de 128 bits y la de los datos por el Sistema Gestor de Base de Datos Relacional que se emplee. 
\begin{itemize}
	\item 1 Backup completo por semana.
	\item 1 Backup transaccional por dia. 
\end{itemize}

\paragraph\textbf{Base de Datos:} El Sistema Gestor de Base de Datos debe ser relacional.

\paragraph\textbf{Política de Borrado:} No se ha definido.


\chapter{Especificación C}
\label{ch:capitulo4} 
\markboth{CAPÍTULO \ref*{ch:capitulo4}. Especificación C}{CAPÍTULO \ref*{ch:capitulo4}. Especificación C}
\section{Modelización del sistema}
	\subsection{Actores}
		\paragraph\indent
		Un actor interactúa con el sistema, pudiendo ser estos un usuario u otro sistema. Los actores identificados son:
		\begin{itemize}
			\item Vendedores
			\item Fabricantes
			\item Administradores
		\end{itemize}
	\subsection{Diagrama de contexto}
	\begin{figure}[H]
		\centering
		\includegraphics[width=\textwidth,height=0.40\textheight,keepaspectratio]{DiagramaContexto}
		\caption{Diagrama de contexto}
	\label{fig:DiagramaContexto}
    \end{figure}
    \subsection{Diagrama de subsistema}
	\begin{figure}[H]
		\centering
		\includegraphics[width=\textwidth,height=0.40\textheight,keepaspectratio]{DiagramaSubsistema}
		\caption{Diagrama de subsistema}
	\label{fig:DiagramaSubsistema}
    \end{figure}
    
	\subsection{Diagrama de casos de uso}
	\begin{figure}[H]
		\centering
		\includegraphics[width=\textwidth,height=0.90\textheight,keepaspectratio]{Sesiones}
		\caption{Diagrama de casos de uso para sesiones}
	\label{fig:Sesiones}
	\end{figure}
	\begin{figure}[H]
		\centering
		\includegraphics[width=\textwidth,height=0.90\textheight,keepaspectratio]{GestionEmpleados}
		\caption{Diagrama de casos de uso para la gestión de empleados}
	\label{fig:GestionEmpleados}
	\end{figure}
	\begin{figure}[H]
		\centering
		\includegraphics[width=\textwidth,height=0.90\textheight,keepaspectratio]{GestionRoles}
		\caption{Diagrama de casos de uso para la gestión de roles}
	\label{fig:GestionRoles}
    \end{figure}
    \begin{figure}[H]
		\centering
		\includegraphics[width=\textwidth,height=0.90\textheight,keepaspectratio]{GestionProductos}
		\caption{Diagrama de casos de uso para la gestión de productos}
	\label{fig:GestionProductos}
    \end{figure}
    \begin{figure}[H]
		\centering
		\includegraphics[width=\textwidth,height=0.90\textheight,keepaspectratio]{GestionTelas}
		\caption{Diagrama de casos de uso para la gestión de telas}
	\label{fig:GestionTelas}
    \end{figure}
    \begin{figure}[H]
		\centering
		\includegraphics[width=\textwidth,height=0.90\textheight,keepaspectratio]{GestionProductosFinales}
		\caption{Diagrama de casos de uso para la gestión de productos finales}
	\label{fig:GestionProductosFinales}
    \end{figure}
    \begin{figure}[H]
		\centering
		\includegraphics[width=\textwidth,height=0.90\textheight,keepaspectratio]{GestionGruposProducto}
		\caption{Diagrama de casos de uso para la gestión de grupos de producto}
	\label{fig:GestionGruposProducto}
    \end{figure}
    \begin{figure}[H]
		\centering
		\includegraphics[width=\textwidth,height=0.90\textheight,keepaspectratio]{GestionUbicaciones}
		\caption{Diagrama de casos de uso para la gestión de ubicaciones}
	\label{fig:GestionUbicaciones}
    \end{figure}
    \begin{figure}[H]
		\centering
		\includegraphics[width=\textwidth,height=0.90\textheight,keepaspectratio]{GestionClientes}
		\caption{Diagrama de casos de uso para la gestión de clientes}
	\label{fig:GestionClientes}
    \end{figure}
    \begin{figure}[H]
		\centering
		\includegraphics[width=\textwidth,height=0.90\textheight,keepaspectratio]{GestionPresupuestos}
		\caption{Diagrama de casos de uso para la gestión de presupuestos}
	\label{fig:GestionPresupuestos}
    \end{figure}
    \begin{figure}[H]
		\centering
		\includegraphics[width=\textwidth,height=0.90\textheight,keepaspectratio]{GestionLineasPresupuesto}
		\caption{Diagrama de casos de uso para la gestión de líneas de presupuesto}
	\label{fig:GestionLineasPresupuesto}
    \end{figure}
    \begin{figure}[H]
		\centering
		\includegraphics[width=\textwidth,height=0.90\textheight,keepaspectratio]{GestionVentas}
		\caption{Diagrama de casos de uso para la gestión de ventas}
	\label{fig:GestionVentas}
    \end{figure}
    \begin{figure}[H]
		\centering
		\includegraphics[width=\textwidth,height=0.90\textheight,keepaspectratio]{GestionLineasVenta}
		\caption{Diagrama de casos de uso para la gestión de líneas de venta}
	\label{fig:GestionLineasVenta}
    \end{figure}
    \begin{figure}[H]
		\centering
		\includegraphics[width=\textwidth,height=0.90\textheight,keepaspectratio]{GestionFacturas}
		\caption{Diagrama de casos de uso para la gestión de facturas}
	\label{fig:GestionFacturas}
    \end{figure}
    \begin{figure}[H]
		\centering
		\includegraphics[width=\textwidth,height=0.90\textheight,keepaspectratio]{GestionRecibos}
		\caption{Diagrama de casos de uso para la gestión de recibos}
	\label{fig:GestionRecibos}
    \end{figure}
    \begin{figure}[H]
		\centering
		\includegraphics[width=\textwidth,height=0.90\textheight,keepaspectratio]{GestionNotasCredito}
		\caption{Diagrama de casos de uso para la gestión de notas de crédito}
	\label{fig:GestionNotasCredito}
    \end{figure}
    \begin{figure}[H]
		\centering
		\includegraphics[width=\textwidth,height=0.90\textheight,keepaspectratio]{GestionRemitos}
		\caption{Diagrama de casos de uso para la gestión de remitos}
	\label{fig:GestionRemitos}
    \end{figure}
    \begin{figure}[H]
		\centering
		\includegraphics[width=\textwidth,height=0.90\textheight,keepaspectratio]{GestionOrdenesReposicion}
		\caption{Diagrama de casos de uso para la gestión de órdenes de reposición}
	\label{fig:GestionOrdenesReposicion}
    \end{figure}
    \begin{figure}[H]
		\centering
		\includegraphics[width=\textwidth,height=0.90\textheight,keepaspectratio]{GestionOrdenesProduccion}
		\caption{Diagrama de casos de uso para la gestión de órdenes de producción}
	\label{fig:GestionOrdenesProduccion}
    \end{figure}
    \begin{figure}[H]
		\centering
		\includegraphics[width=\textwidth,height=0.90\textheight,keepaspectratio]{GestionLineasOrdenesProduccion}
		\caption{Diagrama de casos de uso para la gestión de líneas de órdenes de producción}
	\label{fig:GestionLineasOrdenesProduccion}
    \end{figure}
    \begin{figure}[H]
		\centering
		\includegraphics[width=\textwidth,height=0.90\textheight,keepaspectratio]{GestionObservaciones}
		\caption{Diagrama de casos de uso para la gestión de observaciones}
	\label{fig:GestionObservaciones}
    \end{figure}
    
\input{Capitulos/Capitulo4/cap4sec2.tex}
\section{Identificación de roles, sus funciones y restricciones:}
    \subsection{Roles}
    Los usuarios finales en el sistema pueden cumplir uno de los siguientes roles:
    \begin{itemize}
        \item Administrador.
        \item Vendedor.
        \item Fabricante.
    \end{itemize}
    \subsection{Funciones de los usuarios por rol}
    \begin{itemize}
        \item Rol administrador:
        \begin{itemize}
            \item Sesiones
            \begin{itemize}
                \item Iniciar sesión
                \item Cerrar sesión
            \end{itemize}
            \item Gestión empleados
            \begin{itemize}
                \item Crear empleado
                \item Buscar avanzado empleados
                \item Dar de baja empleado
                \item Dar de alta empleado
                \item Modificar empleado
                \item Borrar empleado
            \end{itemize}
            \item Gestión roles
            \begin{itemize}
                \item Crear rol
                \item Listar roles
                \item Modificar rol
                \item Borrar rol
            \end{itemize}
            \item Gestión productos
            \begin{itemize}
                \item Crear producto
                \item Buscar avanzado productos
                \item Dar de baja producto
                \item Dar de alta producto
                \item Modificar producto
                \item Borrar producto
            \end{itemize}
            \item Gestión telas
            \begin{itemize}
                \item Crear tela
                \item Listar telas
                \item Dar de baja tela
                \item Dar de alta tela
                \item Modificar tela
                \item Borrar tela
            \end{itemize}
            \item Gestión productos finales
            \begin{itemize}
                \item Crear producto final
                \item Buscar avanzado productos finales
                \item Dar de baja producto final
                \item Dar de alta producto final
                \item Modificar producto final
                \item Borrar producto final
            \end{itemize}
            \item Gestión grupos de producto
            \begin{itemize}
                \item Crear grupo de producto
                \item Listar grupos de producto
                \item Dar de baja grupo de producto
                \item Dar de alta grupo de producto
                \item Modificar grupo de producto
                \item Borrar grupo de producto
                \item Listar productos por grupo
                \item Modificar precios de producto por grupo
            \end{itemize}
            \item Gestión ubicaciones
            \begin{itemize}
                \item Crear ubicación
                \item Listar ubicaciones
                \item Dar de baja ubicación
                \item Dar de alta ubicación
                \item Modificar ubicación
                \item Borrar ubicación
            \end{itemize}
            \item Gestión remitos
            \begin{itemize}
                \item Crear remito
                \item Buscar avanzado remitos
                \item Cancelar remito
                \item Descancelar remito
                \item Entregar remito
                \item Borrar remito
                \item Listar líneas de remito
            \end{itemize}
            \item Gestión líneas de remito
            \begin{itemize}
                \item Crear línea remito
                \item Modificar línea de remito
                \item Borrar línea de remito
            \end{itemize}
            \item Gestión clientes
            \begin{itemize}
                \item Crear cliente
                \item Buscar avanzado clientes
                \item Dar de baja cliente
                \item Dar de alta cliente
                \item Modificar cliente
                \item Borrar cliente
            \end{itemize}
            \item Gestión presupuestos
            \begin{itemize}
                \item Crear presupuesto
                \item Buscar avanzado presupuestos
                \item Modificar presupuesto
                \item Borrar presupuesto
                \item Transformar presupuesto en venta
                \item Generar presupuesto en formato PDF
                \item Enviar presupuesto por correo electrónico
                \item Listar líneas de presupuesto
            \end{itemize}
            \item Gestión líneas de presupuesto
            \begin{itemize}
                \item Crear línea de presupuesto
                \item Modificar línea de presupuesto
                \item Borrar línea de presupuesto
            \end{itemize}
            \item Gestión ventas
            \begin{itemize}
                \item Crear venta
                \item Buscar avanzado ventas
                \item Listar líneas de venta
                \item Modificar venta
                \item Borrar venta
                \item Generar orden de producción a partir de venta
                \item Generar remito a partir de venta
                \item Crear comprobante 
                \item Revisar venta
            \end{itemize}
            \item Gestión líneas de venta
            \begin{itemize}
                \item Crear línea de venta
                \item Modificar línea de venta
                \item Borrar línea de venta
                \item Cancelar línea de venta
            \end{itemize}
            \item Gestión comprobantes
            \begin{itemize}
                \item Buscar avanzado comprobantes
                \item Modificar comprobante
                \item Borrar comprobante
            \end{itemize}
            \item Gestión órdenes de producción
            \begin{itemize}
                \item Crear orden de producción
                \item Buscar avanzado órdenes de producción
                \item Modificar orden de producción
                \item Listar líneas de orden de producción
                \item Borrar orden de producción
                \item Cancelar orden de producción
                \item Listar observaciones de línea de orden de producción
                \item Listar tareas de línea de orden de producción
            \end{itemize}
            \item Gestión líneas de órdenes de producción
            \begin{itemize}
                \item Crear línea de orden de producción
                \item Modificar línea de orden de producción
                \item Borrar línea de orden de producción
                \item Verificar línea de orden de producción
                \item Cancelar línea de orden de producción	
                \item Reanudar línea de orden de producción		
            \end{itemize}
            \item Gestión observaciones
            \begin{itemize}
                \item Crear observación
                \item Borrar observación
            \end{itemize}
            \item Gestión tareas
            \begin{itemize}
                \item Crear tarea
                \item Borrar tarea
                \item Finalizar tarea
                \item Verificar tarea
                \item Pausar tarea
                \item Reanudar tarea
                \item Cancelar tarea
                \item Ejecutar tarea
            \end{itemize}
        \end{itemize}
        
        \item Rol vendedor:
        \begin{itemize}
            \item Sesiones
            \begin{itemize}
                \item Iniciar sesión
                \item Cerrar sesión
            \end{itemize}
            \item Gestión productos
            \begin{itemize}
                \item Buscar avanzado productos
            \end{itemize}
            \item Gestión telas
            \begin{itemize}
                \item Listar telas
            \end{itemize}
            \item Gestión productos finales
            \begin{itemize}
                \item Crear producto final
                \item Buscar avanzado productos finales
            \end{itemize}
            \item Gestión ubicaciones
            \begin{itemize}
                \item Listar ubicaciones
            \end{itemize}
            \item Gestión remitos
            \begin{itemize}
                \item Crear remito
                \item Buscar avanzado remitos
                \item Entregar remito
                \item Listar líneas de remito
            \end{itemize}
            \item Gestión líneas de remito
            \begin{itemize}
                \item Crear línea remito
                \item Modificar línea de remito
            \end{itemize}
            \item Gestión clientes
            \begin{itemize}
                \item Crear cliente
                \item Buscar avanzado clientes
                \item Modificar cliente
                \item Borrar cliente
            \end{itemize}
            \item Gestión presupuestos
            \begin{itemize}
                \item Crear presupuesto
                \item Buscar avanzado presupuestos
                \item Modificar presupuesto
                \item Transformar presupuesto en venta
                \item Generar presupuesto en formato PDF
                \item Enviar presupuesto por correo electrónico
                \item Listar líneas de presupuesto
            \end{itemize}
            \item Gestión líneas de presupuesto
            \begin{itemize}
                \item Crear línea de presupuesto
                \item Modificar línea de presupuesto
                \item Borrar línea de presupuesto
            \end{itemize}
            \item Gestión ventas
            \begin{itemize}
                \item Crear venta
                \item Buscar avanzado ventas
                \item Listar líneas de venta
                \item Modificar venta
                \item Generar remito a partir de venta
                \item Crear comprobante
            \end{itemize}
            \item Gestión líneas de venta
            \begin{itemize}
                \item Crear línea de venta
                \item Modificar línea de venta
                \item Borrar línea de venta
            \end{itemize}
            \item Gestión comprobantes
            \begin{itemize}
                \item Buscar avanzado comprobantes
            \end{itemize}
        \end{itemize}
        
        \item Rol fabricante:
        \begin{itemize}
            \item Sesiones
            \begin{itemize}
                \item Iniciar sesión
                \item Cerrar sesión
            \end{itemize}
            \item Gestión órdenes de producción
            \begin{itemize}
                \item Buscar avanzado órdenes de producción
                \item Listar líneas de orden de producción
                \item Listar observaciones de línea de orden de producción
                \item Listar tareas de línea de orden de producción
            \end{itemize}
            \item Gestión tareas
            \begin{itemize}
                \item Finalizar tarea
                \item Ejecutar tarea
            \end{itemize}
        \end{itemize}
    \end{itemize}
    \subsection{Restricciones de datos}
    Todos los roles tendrán restricción para insertar, modificar, borrar y leer cualquier tabla. Las consultas se realizarán a través de los Stored Procedures.
\chapter{Especificación D}
\label{ch:capitulo4} 
\markboth{CAPÍTULO \ref*{ch:capitulo5}. Especificación D}{CAPÍTULO \ref*{ch:capitulo5}. Especificación D}
\input{Capitulos/Capitulo5/cap5sec1.tex}
%\input{capitulos/capitulo4/cap4sec2.tex}
%\section{Identificación de roles, sus funciones y restricciones:}
    \subsection{Roles}
    Los usuarios finales en el sistema pueden cumplir uno de los siguientes roles:
    \begin{itemize}
        \item Administrador.
        \item Vendedor.
        \item Fabricante.
    \end{itemize}
    \subsection{Funciones de los usuarios por rol}
    \begin{itemize}
        \item Rol administrador:
        \begin{itemize}
            \item Sesiones
            \begin{itemize}
                \item Iniciar sesión
                \item Cerrar sesión
            \end{itemize}
            \item Gestión empleados
            \begin{itemize}
                \item Crear empleado
                \item Buscar avanzado empleados
                \item Dar de baja empleado
                \item Dar de alta empleado
                \item Modificar empleado
                \item Borrar empleado
            \end{itemize}
            \item Gestión roles
            \begin{itemize}
                \item Crear rol
                \item Listar roles
                \item Modificar rol
                \item Borrar rol
            \end{itemize}
            \item Gestión productos
            \begin{itemize}
                \item Crear producto
                \item Buscar avanzado productos
                \item Dar de baja producto
                \item Dar de alta producto
                \item Modificar producto
                \item Borrar producto
            \end{itemize}
            \item Gestión telas
            \begin{itemize}
                \item Crear tela
                \item Listar telas
                \item Dar de baja tela
                \item Dar de alta tela
                \item Modificar tela
                \item Borrar tela
            \end{itemize}
            \item Gestión productos finales
            \begin{itemize}
                \item Crear producto final
                \item Buscar avanzado productos finales
                \item Dar de baja producto final
                \item Dar de alta producto final
                \item Modificar producto final
                \item Borrar producto final
            \end{itemize}
            \item Gestión grupos de producto
            \begin{itemize}
                \item Crear grupo de producto
                \item Listar grupos de producto
                \item Dar de baja grupo de producto
                \item Dar de alta grupo de producto
                \item Modificar grupo de producto
                \item Borrar grupo de producto
                \item Listar productos por grupo
                \item Modificar precios de producto por grupo
            \end{itemize}
            \item Gestión ubicaciones
            \begin{itemize}
                \item Crear ubicación
                \item Listar ubicaciones
                \item Dar de baja ubicación
                \item Dar de alta ubicación
                \item Modificar ubicación
                \item Borrar ubicación
            \end{itemize}
            \item Gestión remitos
            \begin{itemize}
                \item Crear remito
                \item Buscar avanzado remitos
                \item Cancelar remito
                \item Descancelar remito
                \item Entregar remito
                \item Borrar remito
                \item Listar líneas de remito
            \end{itemize}
            \item Gestión líneas de remito
            \begin{itemize}
                \item Crear línea remito
                \item Modificar línea de remito
                \item Borrar línea de remito
            \end{itemize}
            \item Gestión clientes
            \begin{itemize}
                \item Crear cliente
                \item Buscar avanzado clientes
                \item Dar de baja cliente
                \item Dar de alta cliente
                \item Modificar cliente
                \item Borrar cliente
            \end{itemize}
            \item Gestión presupuestos
            \begin{itemize}
                \item Crear presupuesto
                \item Buscar avanzado presupuestos
                \item Modificar presupuesto
                \item Borrar presupuesto
                \item Transformar presupuesto en venta
                \item Generar presupuesto en formato PDF
                \item Enviar presupuesto por correo electrónico
                \item Listar líneas de presupuesto
            \end{itemize}
            \item Gestión líneas de presupuesto
            \begin{itemize}
                \item Crear línea de presupuesto
                \item Modificar línea de presupuesto
                \item Borrar línea de presupuesto
            \end{itemize}
            \item Gestión ventas
            \begin{itemize}
                \item Crear venta
                \item Buscar avanzado ventas
                \item Listar líneas de venta
                \item Modificar venta
                \item Borrar venta
                \item Generar orden de producción a partir de venta
                \item Generar remito a partir de venta
                \item Crear comprobante 
                \item Revisar venta
            \end{itemize}
            \item Gestión líneas de venta
            \begin{itemize}
                \item Crear línea de venta
                \item Modificar línea de venta
                \item Borrar línea de venta
                \item Cancelar línea de venta
            \end{itemize}
            \item Gestión comprobantes
            \begin{itemize}
                \item Buscar avanzado comprobantes
                \item Modificar comprobante
                \item Borrar comprobante
            \end{itemize}
            \item Gestión órdenes de producción
            \begin{itemize}
                \item Crear orden de producción
                \item Buscar avanzado órdenes de producción
                \item Modificar orden de producción
                \item Listar líneas de orden de producción
                \item Borrar orden de producción
                \item Cancelar orden de producción
                \item Listar observaciones de línea de orden de producción
                \item Listar tareas de línea de orden de producción
            \end{itemize}
            \item Gestión líneas de órdenes de producción
            \begin{itemize}
                \item Crear línea de orden de producción
                \item Modificar línea de orden de producción
                \item Borrar línea de orden de producción
                \item Verificar línea de orden de producción
                \item Cancelar línea de orden de producción	
                \item Reanudar línea de orden de producción		
            \end{itemize}
            \item Gestión observaciones
            \begin{itemize}
                \item Crear observación
                \item Borrar observación
            \end{itemize}
            \item Gestión tareas
            \begin{itemize}
                \item Crear tarea
                \item Borrar tarea
                \item Finalizar tarea
                \item Verificar tarea
                \item Pausar tarea
                \item Reanudar tarea
                \item Cancelar tarea
                \item Ejecutar tarea
            \end{itemize}
        \end{itemize}
        
        \item Rol vendedor:
        \begin{itemize}
            \item Sesiones
            \begin{itemize}
                \item Iniciar sesión
                \item Cerrar sesión
            \end{itemize}
            \item Gestión productos
            \begin{itemize}
                \item Buscar avanzado productos
            \end{itemize}
            \item Gestión telas
            \begin{itemize}
                \item Listar telas
            \end{itemize}
            \item Gestión productos finales
            \begin{itemize}
                \item Crear producto final
                \item Buscar avanzado productos finales
            \end{itemize}
            \item Gestión ubicaciones
            \begin{itemize}
                \item Listar ubicaciones
            \end{itemize}
            \item Gestión remitos
            \begin{itemize}
                \item Crear remito
                \item Buscar avanzado remitos
                \item Entregar remito
                \item Listar líneas de remito
            \end{itemize}
            \item Gestión líneas de remito
            \begin{itemize}
                \item Crear línea remito
                \item Modificar línea de remito
            \end{itemize}
            \item Gestión clientes
            \begin{itemize}
                \item Crear cliente
                \item Buscar avanzado clientes
                \item Modificar cliente
                \item Borrar cliente
            \end{itemize}
            \item Gestión presupuestos
            \begin{itemize}
                \item Crear presupuesto
                \item Buscar avanzado presupuestos
                \item Modificar presupuesto
                \item Transformar presupuesto en venta
                \item Generar presupuesto en formato PDF
                \item Enviar presupuesto por correo electrónico
                \item Listar líneas de presupuesto
            \end{itemize}
            \item Gestión líneas de presupuesto
            \begin{itemize}
                \item Crear línea de presupuesto
                \item Modificar línea de presupuesto
                \item Borrar línea de presupuesto
            \end{itemize}
            \item Gestión ventas
            \begin{itemize}
                \item Crear venta
                \item Buscar avanzado ventas
                \item Listar líneas de venta
                \item Modificar venta
                \item Generar remito a partir de venta
                \item Crear comprobante
            \end{itemize}
            \item Gestión líneas de venta
            \begin{itemize}
                \item Crear línea de venta
                \item Modificar línea de venta
                \item Borrar línea de venta
            \end{itemize}
            \item Gestión comprobantes
            \begin{itemize}
                \item Buscar avanzado comprobantes
            \end{itemize}
        \end{itemize}
        
        \item Rol fabricante:
        \begin{itemize}
            \item Sesiones
            \begin{itemize}
                \item Iniciar sesión
                \item Cerrar sesión
            \end{itemize}
            \item Gestión órdenes de producción
            \begin{itemize}
                \item Buscar avanzado órdenes de producción
                \item Listar líneas de orden de producción
                \item Listar observaciones de línea de orden de producción
                \item Listar tareas de línea de orden de producción
            \end{itemize}
            \item Gestión tareas
            \begin{itemize}
                \item Finalizar tarea
                \item Ejecutar tarea
            \end{itemize}
        \end{itemize}
    \end{itemize}
    \subsection{Restricciones de datos}
    Todos los roles tendrán restricción para insertar, modificar, borrar y leer cualquier tabla. Las consultas se realizarán a través de los Stored Procedures.
\chapter{Codificación}
\label{ch:capitulo6} 
\markboth{CAPÍTULO \ref*{ch:capitulo6}. Codificación}{CAPÍTULO \ref*{ch:capitulo6}. Codificación}
\section{Elección del lenguaje de programación}
	\paragraph\indent
	Independientemente del paradigma de la ingeniería del software, el lenguaje de programación tendrá impacto en la planificación, el análisis, el diseño, la codificación, la prueba y el mantenimiento de un proyecto. Los lenguajes elegidos para la construcción del sistema fueron:

		\begin{itemize}
			\item Golang: es un lenguaje de programación concurrente y compilado inspirado en la sintaxis de C, que intenta ser dinámico como Python y con el rendimiento de C o C++. Ha sido desarrollado por Google y sus diseñadores iniciales fueron Robert Griesemer, Rob Pike y Ken Thompson. Actualmente está disponible en formato binario para los sistemas operativos Windows, GNU/Linux, FreeBSD y Mac OS X, pudiendo también ser instalado en estos y en otros sistemas mediante el código fuente.
			\item MySQL: es un sistema de gestión de bases de datos relacional desarrollado bajo licencia dual: Licencia pública general/Licencia comercial por Oracle Corporation y está considerada como la base de datos de código abierto más popular del mundo, y una de las más populares en general junto a Oracle y Microsoft SQL Server, todo para entornos de desarrollo web.
			\item Dart: es un lenguaje de programación de código abierto, desarrollado por Google. Fue relevado en la conferencia goto; en Aarhus, Dinamarca el 10 de octubre de 2011. El objetivo de Dart no es reemplazar JavaScript como el principal lenguaje de programación web en los navegadores web, sino ofrecer una alternativa más moderna. El espíritu del lenguaje puede verse reflejado en las declaraciones de Lars Bak, ingeniero de software de Google, que define a Dart como un "lenguaje estructurado pero flexible para programación Web".
			
	Se utiliza principalmente en su forma del lado del cliente, implementado como parte de un navegador web permitiendo mejoras en la interfaz
		\end{itemize}
\section{Frameworks y librerías}
		
		\begin{itemize}
		
			\item \textbf{Flutter:} es un SDK de código fuente abierto de desarrollo de aplicaciones móviles creado por Google. Suele usarse para desarrollar interfaces de usuario para aplicaciones en Android, iOS y Web así como método primario para crear aplicaciones para Google Fuchsia.
			
			\item \textbf{Material Design:} es un concepto, una filosofía, unas pautas enfocadas al diseño utilizado en Android, pero también en la web y en cualquier plataforma.

			\item \textbf{apiDoc:} libreria para generar documentación a partir de comentarios en una API.
			
		\end{itemize}

	
\section{Herramientas de desarrollo}
	
		\begin{itemize}
			\item  \textbf{Visual Studio Code:} es un editor de código fuente desarrollado por Microsoft para Windows, Linux y macOS. Incluye soporte para la depuración, control integrado de Git, resaltado de sintaxis, finalización inteligente de código, fragmentos y refactorización de código. También es personalizable, por lo que los usuarios pueden cambiar el tema del editor, los atajos de teclado y las preferencias. Es gratuito y de código abierto, aunque la descarga oficial está bajo software propietario. Se escogió esta herramienta por las numerosas ventajas que brinda en el momento de la programación. Además cuenta con una gran cantidad de extensiones muy útiles para el usuario.

			\item \textbf{MySQL Workbench:} es el entorno integrado oficial de MySQL. Fue desarrollado por MySQL AB, y permite a los usuarios administrar gráficamente las bases de datos MySQL y diseñar visualmente las estructuras de las bases de datos. MySQL Workbench reemplaza el anterior paquete de software, MySQL GUI Tools. Similar a otros paquetes de terceros, pero aún considerado como el front end autorizado de MySQL, MySQL Workbench permite a los usuarios administrar el diseño y modelado de bases de datos, el desarrollo de SQL (reemplazando al MySQL Query Browser) y la administración de bases de datos (reemplazando al MySQL Administrator).

			MySQL Workbench está disponible en dos ediciones, la habitual Edición Comunitaria gratuita y de código abierto que puede descargarse del sitio web de MySQL, y la Edición Estándar patentada que amplía y mejora el conjunto de características de la Edición Comunitaria. 
			
			Se lo escogió por ser el cliente recomendado por MySQL para la gestión de la base de datos.
			
			\item \textbf{Git:} es un software de control de versiones diseñado por Linus Torvalds, pensando en la eficiencia y la confiabilidad del mantenimiento de versiones de aplicaciones cuando éstas tienen un gran número de archivos de código fuente. Su propósito es llevar registro de los cambios en archivos de computadora y coordinar el trabajo que varias personas realizan sobre archivos compartidos.
				\begin{itemize}
				\item \textbf{GitHub:} GitHub es una plataforma de desarrollo colaborativo para alojar proyectos utilizando el sistema de control de versiones Git. Se utiliza principalmente para la creación de código fuente de programas de ordenador.
				\item \textbf{GitKraken:} es un cliente con una GUI de Git que cuenta con versiones para Linux, Mac y Windows. Ofrece dos versiones, una gratuita y una paga.
				\end{itemize}
			
			\item \textbf{Postman:} Postman nace como una herramienta que principalmente nos permite crear peticiones sobre APIs de una forma muy sencilla y poder, de esta manera, probar las APIs. Todo basado en una extensión de Google Chrome. El usuario de Postman puede ser un desarrollador que esté comprobando el funcionamiento de una API para desarrollar sobre ella o un operador el cual esté realizando tareas de monitoreo sobre un API.
			
			\item  \textbf{Compute Engine:} Es el IaaS (Infraestructura como servicio) de la plataforma de Google Cloud (GCP), en este servicio se pueden crear máquinas virtuales (instancias) de recursos personalizados (CPU, RAM, disco), su costo se factura por minuto de cada recurso. Se lo escogió por ser una tecnología de gran uso en la actualidad, además tener experiencia utilizando esta herramienta nos permite insertarnos de mejor manera en el mundo laboral.
			\item \textbf{Firebase Hosting:} es un servicio de hosting de contenido web con nivel de producción orientado a desarrolladores. Con un solo comando, puedes implementar aplicaciones web y entregar contenido dinámico y estático en una CDN (red de distribución de contenidos) global rápidamente. También se puede sincronizar Firebase Hosting con Cloud Functions o Cloud Run para compilar y alojar microservicios en Firebase.
					
		\end{itemize}
\section{Herramientas de documentación}
	
		\begin{itemize}
			\item \textbf{LaTeX:} es un sistema de composición de textos, orientado a la creación de documentos escritos que presenten una alta calidad tipográfica. Por sus características y posibilidades, es usado de forma especialmente intensa en la generación de artículos y libros científicos que incluyen, entre otros elementos, expresiones matemáticas.
			LaTeX está formado por un gran conjunto de macros de TeX, escrito por Leslie Lamport en 1984, con la intención de facilitar el uso del lenguaje de composición tipográfica, TeX, creado por Donald Knuth.
				\begin{itemize}
				\item \textbf{MiKTeX:} es una distribución TeX/LaTeX para Microsoft Windows que fue desarrollada por Christian Schenk.
				Las características más apreciables de MiKTeX son su habilidad de actualizarse por sí mismo descargando nuevas versiones de componentes y paquetes instalados previamente, y su fácil proceso de instalación.
				\item \textbf{TeXnicCenter:} es un editor software libre de LaTeX para Windows, el cual integra en sí mismo las herramientas necesarias para la composición de texto científico, desde una ventana de compilación integrada, una completa ayuda y manual de LaTeX para los usuarios primerizos, así como un entorno personalizable para los usuarios avanzados.
				\end{itemize}
			
			\item \textbf{PlantUML:} es una herramienta de código abierto que permite a los usuarios crear diagramas UML a partir de un lenguaje de texto plano. Utiliza el software Graphviz para diseñar sus diagramas.
		\end{itemize}
		
	\section{Código}
	El código fuente se encuentra en el CD adjunto.
%\input{capitulos/capitulo4/cap4sec2.tex}
%\section{Identificación de roles, sus funciones y restricciones:}
    \subsection{Roles}
    Los usuarios finales en el sistema pueden cumplir uno de los siguientes roles:
    \begin{itemize}
        \item Administrador.
        \item Vendedor.
        \item Fabricante.
    \end{itemize}
    \subsection{Funciones de los usuarios por rol}
    \begin{itemize}
        \item Rol administrador:
        \begin{itemize}
            \item Sesiones
            \begin{itemize}
                \item Iniciar sesión
                \item Cerrar sesión
            \end{itemize}
            \item Gestión empleados
            \begin{itemize}
                \item Crear empleado
                \item Buscar avanzado empleados
                \item Dar de baja empleado
                \item Dar de alta empleado
                \item Modificar empleado
                \item Borrar empleado
            \end{itemize}
            \item Gestión roles
            \begin{itemize}
                \item Crear rol
                \item Listar roles
                \item Modificar rol
                \item Borrar rol
            \end{itemize}
            \item Gestión productos
            \begin{itemize}
                \item Crear producto
                \item Buscar avanzado productos
                \item Dar de baja producto
                \item Dar de alta producto
                \item Modificar producto
                \item Borrar producto
            \end{itemize}
            \item Gestión telas
            \begin{itemize}
                \item Crear tela
                \item Listar telas
                \item Dar de baja tela
                \item Dar de alta tela
                \item Modificar tela
                \item Borrar tela
            \end{itemize}
            \item Gestión productos finales
            \begin{itemize}
                \item Crear producto final
                \item Buscar avanzado productos finales
                \item Dar de baja producto final
                \item Dar de alta producto final
                \item Modificar producto final
                \item Borrar producto final
            \end{itemize}
            \item Gestión grupos de producto
            \begin{itemize}
                \item Crear grupo de producto
                \item Listar grupos de producto
                \item Dar de baja grupo de producto
                \item Dar de alta grupo de producto
                \item Modificar grupo de producto
                \item Borrar grupo de producto
                \item Listar productos por grupo
                \item Modificar precios de producto por grupo
            \end{itemize}
            \item Gestión ubicaciones
            \begin{itemize}
                \item Crear ubicación
                \item Listar ubicaciones
                \item Dar de baja ubicación
                \item Dar de alta ubicación
                \item Modificar ubicación
                \item Borrar ubicación
            \end{itemize}
            \item Gestión remitos
            \begin{itemize}
                \item Crear remito
                \item Buscar avanzado remitos
                \item Cancelar remito
                \item Descancelar remito
                \item Entregar remito
                \item Borrar remito
                \item Listar líneas de remito
            \end{itemize}
            \item Gestión líneas de remito
            \begin{itemize}
                \item Crear línea remito
                \item Modificar línea de remito
                \item Borrar línea de remito
            \end{itemize}
            \item Gestión clientes
            \begin{itemize}
                \item Crear cliente
                \item Buscar avanzado clientes
                \item Dar de baja cliente
                \item Dar de alta cliente
                \item Modificar cliente
                \item Borrar cliente
            \end{itemize}
            \item Gestión presupuestos
            \begin{itemize}
                \item Crear presupuesto
                \item Buscar avanzado presupuestos
                \item Modificar presupuesto
                \item Borrar presupuesto
                \item Transformar presupuesto en venta
                \item Generar presupuesto en formato PDF
                \item Enviar presupuesto por correo electrónico
                \item Listar líneas de presupuesto
            \end{itemize}
            \item Gestión líneas de presupuesto
            \begin{itemize}
                \item Crear línea de presupuesto
                \item Modificar línea de presupuesto
                \item Borrar línea de presupuesto
            \end{itemize}
            \item Gestión ventas
            \begin{itemize}
                \item Crear venta
                \item Buscar avanzado ventas
                \item Listar líneas de venta
                \item Modificar venta
                \item Borrar venta
                \item Generar orden de producción a partir de venta
                \item Generar remito a partir de venta
                \item Crear comprobante 
                \item Revisar venta
            \end{itemize}
            \item Gestión líneas de venta
            \begin{itemize}
                \item Crear línea de venta
                \item Modificar línea de venta
                \item Borrar línea de venta
                \item Cancelar línea de venta
            \end{itemize}
            \item Gestión comprobantes
            \begin{itemize}
                \item Buscar avanzado comprobantes
                \item Modificar comprobante
                \item Borrar comprobante
            \end{itemize}
            \item Gestión órdenes de producción
            \begin{itemize}
                \item Crear orden de producción
                \item Buscar avanzado órdenes de producción
                \item Modificar orden de producción
                \item Listar líneas de orden de producción
                \item Borrar orden de producción
                \item Cancelar orden de producción
                \item Listar observaciones de línea de orden de producción
                \item Listar tareas de línea de orden de producción
            \end{itemize}
            \item Gestión líneas de órdenes de producción
            \begin{itemize}
                \item Crear línea de orden de producción
                \item Modificar línea de orden de producción
                \item Borrar línea de orden de producción
                \item Verificar línea de orden de producción
                \item Cancelar línea de orden de producción	
                \item Reanudar línea de orden de producción		
            \end{itemize}
            \item Gestión observaciones
            \begin{itemize}
                \item Crear observación
                \item Borrar observación
            \end{itemize}
            \item Gestión tareas
            \begin{itemize}
                \item Crear tarea
                \item Borrar tarea
                \item Finalizar tarea
                \item Verificar tarea
                \item Pausar tarea
                \item Reanudar tarea
                \item Cancelar tarea
                \item Ejecutar tarea
            \end{itemize}
        \end{itemize}
        
        \item Rol vendedor:
        \begin{itemize}
            \item Sesiones
            \begin{itemize}
                \item Iniciar sesión
                \item Cerrar sesión
            \end{itemize}
            \item Gestión productos
            \begin{itemize}
                \item Buscar avanzado productos
            \end{itemize}
            \item Gestión telas
            \begin{itemize}
                \item Listar telas
            \end{itemize}
            \item Gestión productos finales
            \begin{itemize}
                \item Crear producto final
                \item Buscar avanzado productos finales
            \end{itemize}
            \item Gestión ubicaciones
            \begin{itemize}
                \item Listar ubicaciones
            \end{itemize}
            \item Gestión remitos
            \begin{itemize}
                \item Crear remito
                \item Buscar avanzado remitos
                \item Entregar remito
                \item Listar líneas de remito
            \end{itemize}
            \item Gestión líneas de remito
            \begin{itemize}
                \item Crear línea remito
                \item Modificar línea de remito
            \end{itemize}
            \item Gestión clientes
            \begin{itemize}
                \item Crear cliente
                \item Buscar avanzado clientes
                \item Modificar cliente
                \item Borrar cliente
            \end{itemize}
            \item Gestión presupuestos
            \begin{itemize}
                \item Crear presupuesto
                \item Buscar avanzado presupuestos
                \item Modificar presupuesto
                \item Transformar presupuesto en venta
                \item Generar presupuesto en formato PDF
                \item Enviar presupuesto por correo electrónico
                \item Listar líneas de presupuesto
            \end{itemize}
            \item Gestión líneas de presupuesto
            \begin{itemize}
                \item Crear línea de presupuesto
                \item Modificar línea de presupuesto
                \item Borrar línea de presupuesto
            \end{itemize}
            \item Gestión ventas
            \begin{itemize}
                \item Crear venta
                \item Buscar avanzado ventas
                \item Listar líneas de venta
                \item Modificar venta
                \item Generar remito a partir de venta
                \item Crear comprobante
            \end{itemize}
            \item Gestión líneas de venta
            \begin{itemize}
                \item Crear línea de venta
                \item Modificar línea de venta
                \item Borrar línea de venta
            \end{itemize}
            \item Gestión comprobantes
            \begin{itemize}
                \item Buscar avanzado comprobantes
            \end{itemize}
        \end{itemize}
        
        \item Rol fabricante:
        \begin{itemize}
            \item Sesiones
            \begin{itemize}
                \item Iniciar sesión
                \item Cerrar sesión
            \end{itemize}
            \item Gestión órdenes de producción
            \begin{itemize}
                \item Buscar avanzado órdenes de producción
                \item Listar líneas de orden de producción
                \item Listar observaciones de línea de orden de producción
                \item Listar tareas de línea de orden de producción
            \end{itemize}
            \item Gestión tareas
            \begin{itemize}
                \item Finalizar tarea
                \item Ejecutar tarea
            \end{itemize}
        \end{itemize}
    \end{itemize}
    \subsection{Restricciones de datos}
    Todos los roles tendrán restricción para insertar, modificar, borrar y leer cualquier tabla. Las consultas se realizarán a través de los Stored Procedures.
\chapter{Pruebas}
\label{ch:capitulo7} 
\markboth{CAPÍTULO \ref*{ch:capitulo7}. Pruebas}{CAPÍTULO \ref*{ch:capitulo7}. Pruebas}
\section{Introducción}
	\paragraph\indent
	A cada etapa de desarrollo le corresponde una etapa de prueba del mismo nivel y según a quién está orientada la misma se puede clasificar en:
	\begin{itemize}
		\item Pruebas orientadas al desarrollo:
			\begin{itemize}
				\item Test de unidades: prueba de las unidades individuales de código.
				\item Test de módulos: prueba de módulos funcionales del sistema.
				\item Test de integración: prueba de la estructura modular del programa y su interacción.
			\end{itemize}
		\item Pruebas orientadas al Cliente:
			\begin{itemize}
				\item Test de Aceptación: prueba de la estructura modular del programa y su interacción.
			\end{itemize}
	\end{itemize}
	
\section{Test de unidades}
\subsection{Pruebas de caja blanca}

\paragraph\indent
Es un tipo de método de prueba que permite detectar errores internos del código de cada módulo.

\paragraph\indent
Con estas pruebas de pueden garantizar que se ejercitan por lo menos una vez todos los caminos independientes de cada módulo, que las decisiones lógicas se evalúan en sus variantes verdadera y falsa, que se ejecutan todos los bucles en sus límites operacionales y, por último, que se ejercitan las estructuras internas de datos para asegurar su validez.

\paragraph\indent
Para realizar este test se procedió primero a determinar el conjunto de datos representativos del dominio de la prueba, de manera tal que se
se atraviesen todas las bifuraciones del código, decisiones y loop. 

\paragraph\indent
Se utilizó una herramienta para realizar tests automatizados que viene incorporado con Golang: el paquete test.

\paragraph\indent
Los resultados de este test fueron exitosos.

\section{Test de módulos}
\subsection{Pruebas de caja negra}

\paragraph\indent
Se ve a cada módulo como una caja negra y se generan conjuntos de condiciones de entrada que ejerciten completamente todos los requisitos funcionales del programa, observando las salidas. La prueba de la caja negra centra su atención en la información y la clave está en generar el conjunto de datos o condiciones de entrada. Se detectan los siguientes errores:
\begin{itemize}
	\item Funciones incorrectas o ausentes.
	\item Errores de interfaz.
	\item Errores en estructuras de datos o en accesos a bases de datos externas.
	\item Errores de rendimiento.
	\item Errores de inicialización y terminación.
\end{itemize}

\paragraph\indent
Para realizar este test se eligieron datos representativos del dominio, verificando sus salidas. Los resultados de este test fueron exitosos.

\subsection{Prueba de estrés}

\paragraph\indent
Se centra en realizar el análisis de valores límite, y en condiciones límite, ya que se ha demostrado que los errores tienden a darse más en los límites del campo de entrada y sometidos a condiciones límite.

\paragraph\indent
Se sometió a la aplicación al 500\% de su carga máxima, mediante un autómata escrito en un script MySQL que se ejecuta mediante un evento, duplicando el número de conexiones simultáneas.

\paragraph\indent
Los resultados del test fueron exitosos.

\section{Test de integración}
Los errores que surgen de integrar los módulos son: 
\begin{itemize}
	\item Los datos se pueden perder en una interfaz: un módulo puede tener un efecto adverso e inadvertido sobre otro.
	\item Las subfunciones, cuando se combinan, pueden no producir la función principal.
	\item Las estructuras de datos globales pueden presentar problemas.
\end{itemize}

El objetivo es tomar los módulos probados y construir una estructura de programa que esté de acuerdo con lo que dicta la Especificación C.

Existen dos tipos de integración:
\begin{itemize}
	\item Integración descendente: se integran los módulos moviéndose hacia abajo por la jerarquía de control, comenzando con el módulo de control principal.
	\item Integración ascendente: se integran los módulos atómicos (niveles más bajos) primero y luego se continúa con el nivel inmediato superior.
\end{itemize}

\section{Test de aceptación}
\subsection{Pruebas alfa y beta}

La prueba \textalpha \, es conducida por el cliente en el lugar de desarrollo. Se usa el software de forma natural (previa capacitación), con el encargado de desarrollo mirando por encima del hombro del usuario y registrando errores y problemas de uso. Se llevan a cabo en un entorno controlado.

La prueba \textbeta \,se lleva a cabo en uno o más lugares de clientes, por los usuarios finales de software. El encargado de desarrollo a cabo no está presente. El cliente registra todos los problemas (reales e imaginarios) que encuentra durante la prueba e informa a intervalos regulares al equipo de desarrollo.

Tanto los planes como los procedimientos de prueba, estarán diseñados para asegurar que se satisfacen todos los requisitos funcionales y que se alcanzan todos los requisitos de rendimientos.

Los tests de aceptación fueron realizados con el cliente en la mueblería, verificando el funcionamiento del sistema y que el mismo cumpla con las expectativas del cliente. Los resultados fueron exitosos.
%\input{capitulos/capitulo4/cap4sec2.tex}
%\section{Identificación de roles, sus funciones y restricciones:}
    \subsection{Roles}
    Los usuarios finales en el sistema pueden cumplir uno de los siguientes roles:
    \begin{itemize}
        \item Administrador.
        \item Vendedor.
        \item Fabricante.
    \end{itemize}
    \subsection{Funciones de los usuarios por rol}
    \begin{itemize}
        \item Rol administrador:
        \begin{itemize}
            \item Sesiones
            \begin{itemize}
                \item Iniciar sesión
                \item Cerrar sesión
            \end{itemize}
            \item Gestión empleados
            \begin{itemize}
                \item Crear empleado
                \item Buscar avanzado empleados
                \item Dar de baja empleado
                \item Dar de alta empleado
                \item Modificar empleado
                \item Borrar empleado
            \end{itemize}
            \item Gestión roles
            \begin{itemize}
                \item Crear rol
                \item Listar roles
                \item Modificar rol
                \item Borrar rol
            \end{itemize}
            \item Gestión productos
            \begin{itemize}
                \item Crear producto
                \item Buscar avanzado productos
                \item Dar de baja producto
                \item Dar de alta producto
                \item Modificar producto
                \item Borrar producto
            \end{itemize}
            \item Gestión telas
            \begin{itemize}
                \item Crear tela
                \item Listar telas
                \item Dar de baja tela
                \item Dar de alta tela
                \item Modificar tela
                \item Borrar tela
            \end{itemize}
            \item Gestión productos finales
            \begin{itemize}
                \item Crear producto final
                \item Buscar avanzado productos finales
                \item Dar de baja producto final
                \item Dar de alta producto final
                \item Modificar producto final
                \item Borrar producto final
            \end{itemize}
            \item Gestión grupos de producto
            \begin{itemize}
                \item Crear grupo de producto
                \item Listar grupos de producto
                \item Dar de baja grupo de producto
                \item Dar de alta grupo de producto
                \item Modificar grupo de producto
                \item Borrar grupo de producto
                \item Listar productos por grupo
                \item Modificar precios de producto por grupo
            \end{itemize}
            \item Gestión ubicaciones
            \begin{itemize}
                \item Crear ubicación
                \item Listar ubicaciones
                \item Dar de baja ubicación
                \item Dar de alta ubicación
                \item Modificar ubicación
                \item Borrar ubicación
            \end{itemize}
            \item Gestión remitos
            \begin{itemize}
                \item Crear remito
                \item Buscar avanzado remitos
                \item Cancelar remito
                \item Descancelar remito
                \item Entregar remito
                \item Borrar remito
                \item Listar líneas de remito
            \end{itemize}
            \item Gestión líneas de remito
            \begin{itemize}
                \item Crear línea remito
                \item Modificar línea de remito
                \item Borrar línea de remito
            \end{itemize}
            \item Gestión clientes
            \begin{itemize}
                \item Crear cliente
                \item Buscar avanzado clientes
                \item Dar de baja cliente
                \item Dar de alta cliente
                \item Modificar cliente
                \item Borrar cliente
            \end{itemize}
            \item Gestión presupuestos
            \begin{itemize}
                \item Crear presupuesto
                \item Buscar avanzado presupuestos
                \item Modificar presupuesto
                \item Borrar presupuesto
                \item Transformar presupuesto en venta
                \item Generar presupuesto en formato PDF
                \item Enviar presupuesto por correo electrónico
                \item Listar líneas de presupuesto
            \end{itemize}
            \item Gestión líneas de presupuesto
            \begin{itemize}
                \item Crear línea de presupuesto
                \item Modificar línea de presupuesto
                \item Borrar línea de presupuesto
            \end{itemize}
            \item Gestión ventas
            \begin{itemize}
                \item Crear venta
                \item Buscar avanzado ventas
                \item Listar líneas de venta
                \item Modificar venta
                \item Borrar venta
                \item Generar orden de producción a partir de venta
                \item Generar remito a partir de venta
                \item Crear comprobante 
                \item Revisar venta
            \end{itemize}
            \item Gestión líneas de venta
            \begin{itemize}
                \item Crear línea de venta
                \item Modificar línea de venta
                \item Borrar línea de venta
                \item Cancelar línea de venta
            \end{itemize}
            \item Gestión comprobantes
            \begin{itemize}
                \item Buscar avanzado comprobantes
                \item Modificar comprobante
                \item Borrar comprobante
            \end{itemize}
            \item Gestión órdenes de producción
            \begin{itemize}
                \item Crear orden de producción
                \item Buscar avanzado órdenes de producción
                \item Modificar orden de producción
                \item Listar líneas de orden de producción
                \item Borrar orden de producción
                \item Cancelar orden de producción
                \item Listar observaciones de línea de orden de producción
                \item Listar tareas de línea de orden de producción
            \end{itemize}
            \item Gestión líneas de órdenes de producción
            \begin{itemize}
                \item Crear línea de orden de producción
                \item Modificar línea de orden de producción
                \item Borrar línea de orden de producción
                \item Verificar línea de orden de producción
                \item Cancelar línea de orden de producción	
                \item Reanudar línea de orden de producción		
            \end{itemize}
            \item Gestión observaciones
            \begin{itemize}
                \item Crear observación
                \item Borrar observación
            \end{itemize}
            \item Gestión tareas
            \begin{itemize}
                \item Crear tarea
                \item Borrar tarea
                \item Finalizar tarea
                \item Verificar tarea
                \item Pausar tarea
                \item Reanudar tarea
                \item Cancelar tarea
                \item Ejecutar tarea
            \end{itemize}
        \end{itemize}
        
        \item Rol vendedor:
        \begin{itemize}
            \item Sesiones
            \begin{itemize}
                \item Iniciar sesión
                \item Cerrar sesión
            \end{itemize}
            \item Gestión productos
            \begin{itemize}
                \item Buscar avanzado productos
            \end{itemize}
            \item Gestión telas
            \begin{itemize}
                \item Listar telas
            \end{itemize}
            \item Gestión productos finales
            \begin{itemize}
                \item Crear producto final
                \item Buscar avanzado productos finales
            \end{itemize}
            \item Gestión ubicaciones
            \begin{itemize}
                \item Listar ubicaciones
            \end{itemize}
            \item Gestión remitos
            \begin{itemize}
                \item Crear remito
                \item Buscar avanzado remitos
                \item Entregar remito
                \item Listar líneas de remito
            \end{itemize}
            \item Gestión líneas de remito
            \begin{itemize}
                \item Crear línea remito
                \item Modificar línea de remito
            \end{itemize}
            \item Gestión clientes
            \begin{itemize}
                \item Crear cliente
                \item Buscar avanzado clientes
                \item Modificar cliente
                \item Borrar cliente
            \end{itemize}
            \item Gestión presupuestos
            \begin{itemize}
                \item Crear presupuesto
                \item Buscar avanzado presupuestos
                \item Modificar presupuesto
                \item Transformar presupuesto en venta
                \item Generar presupuesto en formato PDF
                \item Enviar presupuesto por correo electrónico
                \item Listar líneas de presupuesto
            \end{itemize}
            \item Gestión líneas de presupuesto
            \begin{itemize}
                \item Crear línea de presupuesto
                \item Modificar línea de presupuesto
                \item Borrar línea de presupuesto
            \end{itemize}
            \item Gestión ventas
            \begin{itemize}
                \item Crear venta
                \item Buscar avanzado ventas
                \item Listar líneas de venta
                \item Modificar venta
                \item Generar remito a partir de venta
                \item Crear comprobante
            \end{itemize}
            \item Gestión líneas de venta
            \begin{itemize}
                \item Crear línea de venta
                \item Modificar línea de venta
                \item Borrar línea de venta
            \end{itemize}
            \item Gestión comprobantes
            \begin{itemize}
                \item Buscar avanzado comprobantes
            \end{itemize}
        \end{itemize}
        
        \item Rol fabricante:
        \begin{itemize}
            \item Sesiones
            \begin{itemize}
                \item Iniciar sesión
                \item Cerrar sesión
            \end{itemize}
            \item Gestión órdenes de producción
            \begin{itemize}
                \item Buscar avanzado órdenes de producción
                \item Listar líneas de orden de producción
                \item Listar observaciones de línea de orden de producción
                \item Listar tareas de línea de orden de producción
            \end{itemize}
            \item Gestión tareas
            \begin{itemize}
                \item Finalizar tarea
                \item Ejecutar tarea
            \end{itemize}
        \end{itemize}
    \end{itemize}
    \subsection{Restricciones de datos}
    Todos los roles tendrán restricción para insertar, modificar, borrar y leer cualquier tabla. Las consultas se realizarán a través de los Stored Procedures.
\chapter{Conclusiones}
\label{ch:capitulo8} 
\markboth{CAPÍTULO \ref*{ch:capitulo8}. Conclusiones}{CAPÍTULO \ref*{ch:capitulo8}. Conclusiones}

\section{Sobre el proyecto}

En primera medida se desarrolló el proyecto cumpliendo con los objetivos propuestos
incialmente. Se pusieron en práctica conocimientos incorporados en la carrera Ingeniería en
Computación, y sobre todo en la PPS. Tambien se logró aprender y profundizar acerca de nuevas herramientas
necesarias para el desarrollo del proyecto y para una mejor formación profesional.

\paragraph\indent
El sistema permite llevar a cabo todo el proceso que se inicia con la presupuestación y
finaliza con la entrega de los productos, involucrando la venta y fabricación de los mismos. De
esta manera se consiguieron optimizar los procesos que se llevan a cabo en la mueblería.

\paragraph\indent
El sistema permite generar y actualizar la lista de precios de forma automática, reduciendo
los tiempos que tomaba generarlas y distribuirlas en todas las sucursales de la mueblería. Además
otorga la posibilidad de conocer en tiempo real la disponibilidad de los productos en las distintas
sucursales de la empresa.

\paragraph\indent
Al usar tecnologías web, se consiguió que el sistema pueda ser ejecutado mediante cualquier
navegador web de los sistemas operativos de mayor uso en la actualidad, Windows, Linux y
MacOS.

\paragraph\indent
La implementación del sistema permitirá llevar a cabo un estudio estadístico, mejorando
la toma de decisiones de la mueblería. Además el sistema cuenta con la función de generar
documentos de forma digital, reemplazando los manuscritos, con el objetivo de reducir el impacto
ambiental y el tiempo para su elaboración.

\paragraph\indent
La comunicación con los servidores de datos se encuentra cifrada utilizando protocolos
modernos como ser HTTPS.

\paragraph\indent
En cuanto a las dificultades, una de las principales fue la instalación de los distintos
servidores en entornos Cloud. Para ello se investigó, se tomaron cursos y se realizaron consultas a
expertos en el área. 

\paragraph\indent
Por otro lado, otro reto que se presentó fue el despliegue del sistema utilizando servicios CI/CD automatizado. Para resolver esta problemática se investigó y se realizaron diversas pruebas hasta lograr el objetivo. 

\paragraph\indent
Por último el uso de lenguajes de programación modernos, sin conocimientos previos, obligó a realizar una capacitación en ellos.

\section{Personales}

\paragraph\indent
Desde el punto de vista personal, el desarrollo de este proyecto implicó diversos desafios que supimos atravesar con éxito. El trabajo en equipo, las largas discusiones sobre diseño, los errores cometidos, los aciertos y el trato con personas son algunos ejemplos.

\paragraph\indent
Con el desarrollo de este proyecto nos enfrentamos a problemas reales, en donde tuvimos que tomar desiciones de diseño ingenieríl a fin de encontrar soluciones eficientes que resolviesen los requerimientos del cliente, asi como también, otros problemas que se nos presentaron. 

\paragraph\indent
Durante este tiempo mejoramos el trabajo colaborativo, la forma en la que nos relacionamos e interactuamos con el cliente y también supimos diferenciar discusiones relacionadas al proyecto con nuestra relación personal como compañeros.

\paragraph\indent
Tuvimos la oportunidad de hacer nuestra PPS en una empresa que se dedica al desarrollo de software, participando en proyectos cuyos productos tuvieron certificación de calidad internacional. Pudimos aprender los procesos software, la forma de elaborar documentación para certificación, y aprender nuevas herramientas de desarrollo, las cuales fueron aplicadas al proyecto.

\paragraph\indent
Poder ver el sistema de gestión de la mueblería en funcionamiento nos llena de orgullo y nos inspira a seguir trabajando y formándonos en esta hermosa profesión que hemos elegido.
\chapter{Bibliografía}
\label{ch:capitulo9} 
\markboth{CAPÍTULO \ref*{ch:capitulo9}. Bibliografía}{CAPÍTULO \ref*{ch:capitulo9}. Bibliografía}

\begin{enumerate}
	\item Documentación oficial de Golang - \url{https://golang.org/doc/} - 01/06/2020
	\item Documentación oficial de MySQL - \url{https://dev.mysql.com/} - 05/05/2020
	\item Documentación oficial de Dart - \url{https://dart.dev/guides} - 01/06/2020
	\item Documentación oficial de Flutter - \url{https://flutter.dev/docs} - 01/06/2020
	\item Documentación oficial de Echo - \url{https://echo.labstack.com/guide} - 01/06/2020
	\item Documentación oficial de Compute Engine/GCP - \url{https://cloud.google.com/compute/docs} - 05/05/2020
	\item Documentación oficial de Git - \url{https://git-scm.com/doc} - 01/02/2020
	\item Documentación oficial de PlantUML - \url{https://plantuml.com/es/} - 01/03/2020
	\item Documentación oficial de ApiDoc - \url{https://apidocjs.com/} - 01/06/2020
	\item Foro de ayuda de Latex - \url{https://tex.stackexchange.com/} - 01/02/2020
	\item Guía básica de Latex - \url{http://minisconlatex.blogspot.com/} - 01/02/2020
	\item Apuntes de clase de Ingeniería de Software I - 01/02/2020
	\item Ingeniería del Software, un enfoque práctico - Tercera edición - 01/02/2020
	\item UML y Patrones - Segunda edición - 01/02/2020
\end{enumerate}
\chapter*{Anexo}
\addcontentsline{toc}{chapter}{Anexo}
\markboth{Anexo}{Anexo}
\section*{Descripción textual de los casos de uso y diagramas de actividad}

%Sesiones
\renewcommand{\caseUseShortName}{iniciarSesion}
\renewcommand{\caseUseCreated}{16/01/2020}
\renewcommand{\caseUseModified}{16/01/2020}
\renewcommand{\caseUseName}{\CUiniciarSesion - Iniciar sesión}
\renewcommand{\caseUseSummary}{Este caso de uso permite a un usuario iniciar sesión en el sistema.}
\renewcommand{\caseUsePeople}{Usuarios: quiere ingresar al sistema.}
\renewcommand{\caseUsePreconditions}{
	\caseUseRow{El usuario se encuentra creado y activo en ZMGestion.}
}
\renewcommand{\caseUsePostconditions}{
	\caseUseRow{Se valida al usuario y se le muestra al usuario las opciones personales disponibles.}
}
\renewcommand{\caseUseScene}{
	\addCaseUseStep{El usuario ingresa la dirección de la aplicación en un dispositivo conectado a Internet.}
	\addCaseUseStep{ZMGestion muestra un formulario para que el usuario ingrese su nombre de usuario y contraseña.}
	\addCaseUseStep{El usuario introduce su nombre de usuario y contraseña.}
	\addCaseUseStep{ZMGestion trae los permisos del usuario y le muestra sus opciones.}
	
}
\renewcommand{\alternativeCaseUse}{
	\newAlternative{A1: El nombre de usuario no existe en ZMGestion.}{3}
	\caseUseRow{La secuencia A1 comienza luego del punto 3 del escenario principal.}
	\alternativeRow{ZMGestion muestra un mensaje de error.}
	\caseUseRow{El escenario vuelve al punto 2.}
	\caseUseRow{}
	
	\newAlternative{A2: El usuario no se encuentra activo.}{3}
	\caseUseRow{La secuencia A2 comienza luego del punto 3 del escenario principal.}
	\alternativeRow{ZMGestion informa al usuario que el mismo no se encuentra activo y que debe comuicarse con un administrador.}
	\caseUseRow{El escenario vuelve al punto 2.}
	\caseUseRow{}

	\newAlternative{A3: La contraseña ingresada es incorrecta y el número de intentos no supero el límite permitido.}{3}
	\caseUseRow{La secuencia A3 comienza luego del punto 3 del escenario principal.}
	\alternativeRow{ZMGestion informa al usuario que la contraseña ingresada es incorrecta.}
	\caseUseRow{El escenario vuelve al punto 2.}
	\caseUseRow{}

	\newAlternative{A4: La contraseña ingresada es incorrecta y el número de intentos supero el límite permitido.}{3}
	\caseUseRow{La secuencia A4 comienza luego del punto 3 del escenario principal.}
	\alternativeRow{ZMGestion informa al usuario que la contraseña es incorrecta, que el usuario ha sido ha sido dado de baja y que debe comunicarse con un administrador.}
	\caseUseRow{El escenario vuelve al punto 2.}
	\caseUseRow{}	
}

\item Caso de uso \caseUseName

%DESCRIPCION TEXTUAL
\renewcommand*{\arraystretch}{1.3}
\begin{longtable}[c]{|>{\raggedright}p{0.3\textwidth} | >{\raggedright}p{0.2\textwidth} | p{0.5\textwidth} |}
\caption{\hyperref[sec:listadoCasoUso]{\caseUseName}}
\label{tabla:\caseUseShortName}\\
\hline
\rowcolor{tableCaseUseBackground}

\multicolumn{3}{|l|}{\textcolor{tableCaseUseFontColor}{Descripción textual del caso de uso: \caseUseName}} \\ \hline

Fecha de Creación: & \multicolumn{2}{L{\secondColumnWidth}|}{\caseUseCreated}\\ \hline

Fecha de Modificación: & \multicolumn{2}{L{\secondColumnWidth}|}{\caseUseModified} \\ \hline

Versión: & \multicolumn{2}{L{\secondColumnWidth}|}{1} \\ \hline

Resumen: & \multicolumn{2}{L{\secondColumnWidth}|}{\caseUseSummary} \\ \hline

Personas involucradas y metas: & \multicolumn{2}{L{\secondColumnWidth}|}{\caseUsePeople} \\ \hline

Precondiciones: \caseUsePreconditions \hline

Postcondiciones: \caseUsePostconditions \hline

Escenario principal: \caseUseScene \hline

Flujos alternativos: \alternativeCaseUse \hline

Requisitos de interfaz de usuario: \caseUseRequirementsGUI \hline
\multirow{3}{*}{Requisitos funcionales:}  & Tiempo de respuesta: & \caseUseResponseTime \\ \cline{2-3} 
& Concurrencia: & \caseUseConcurrence \\ \cline{2-3} 
& Disponibilidad: & \caseUseAvailability \\ \hline
\end{longtable}

\setcounter{rownumbers}{0}

\renewcommand{\alternativeCaseUse}{
	\caseUseRow{No existen flujos alternativos.}
}

%DIAGRAMA DE ACTIVIDAD
%\lineabreak[0]
\activityDiagram{\caseUseShortName}{Diagrama de actividad - \caseUseName}

\renewcommand{\caseUseShortName}{cerrarSesion}
\renewcommand{\caseUseCreated}{16/01/2020}
\renewcommand{\caseUseModified}{16/01/2020}
\renewcommand{\caseUseName}{\CUcerrarSesion - Cerrar sesión}
\renewcommand{\caseUseSummary}{Este caso de uso permite a un usuario cerrar sesión en el sistema.}
\renewcommand{\caseUsePeople}{Usuarios: quiere salir del sistema.}
\renewcommand{\caseUsePreconditions}{
	\caseUseRow{Tener una sesión iniciada en el sistema.}
}
\renewcommand{\caseUsePostconditions}{
	\caseUseRow{Se borra el token de sesión almacenado en el dispositivo del usuario.}
}
\renewcommand{\caseUseScene}{
	\addCaseUseStep{El usuario accede a la pantalla destinada para cerrar sesión.}
	\addCaseUseStep{ZMGestion elimina el token de sesión del usuario y muestra un mensaje informando el éxito de la operación.}
}
\renewcommand{\alternativeCaseUse}{
	\caseUseRow{Ninguno.}
	\caseUseRow{}
}

\item Caso de uso \caseUseName

%DESCRIPCION TEXTUAL
\renewcommand*{\arraystretch}{1.3}
\begin{longtable}[c]{|>{\raggedright}p{0.3\textwidth} | >{\raggedright}p{0.2\textwidth} | p{0.5\textwidth} |}
\caption{\hyperref[sec:listadoCasoUso]{\caseUseName}}
\label{tabla:\caseUseShortName}\\
\hline
\rowcolor{tableCaseUseBackground}

\multicolumn{3}{|l|}{\textcolor{tableCaseUseFontColor}{Descripción textual del caso de uso: \caseUseName}} \\ \hline

Fecha de Creación: & \multicolumn{2}{L{\secondColumnWidth}|}{\caseUseCreated}\\ \hline

Fecha de Modificación: & \multicolumn{2}{L{\secondColumnWidth}|}{\caseUseModified} \\ \hline

Versión: & \multicolumn{2}{L{\secondColumnWidth}|}{1} \\ \hline

Resumen: & \multicolumn{2}{L{\secondColumnWidth}|}{\caseUseSummary} \\ \hline

Personas involucradas y metas: & \multicolumn{2}{L{\secondColumnWidth}|}{\caseUsePeople} \\ \hline

Precondiciones: \caseUsePreconditions \hline

Postcondiciones: \caseUsePostconditions \hline

Escenario principal: \caseUseScene \hline

Flujos alternativos: \alternativeCaseUse \hline

Requisitos de interfaz de usuario: \caseUseRequirementsGUI \hline
\multirow{3}{*}{Requisitos funcionales:}  & Tiempo de respuesta: & \caseUseResponseTime \\ \cline{2-3} 
& Concurrencia: & \caseUseConcurrence \\ \cline{2-3} 
& Disponibilidad: & \caseUseAvailability \\ \hline
\end{longtable}

\setcounter{rownumbers}{0}

\renewcommand{\alternativeCaseUse}{
	\caseUseRow{No existen flujos alternativos.}
}

%DIAGRAMA DE ACTIVIDAD
%\lineabreak[0]
%\activityDiagram{AD_\caseUseShortName}{Diagrama de actividad - \caseUseName}


%GestionEmpleados

\renewcommand{\caseUseShortName}{crearEmpleado} %cammelCase name

\renewcommand{\caseUseCreated}{27/01/2020} %Fecha creación
\renewcommand{\caseUseModified}{27/01/2020} %Fecha modificación
\renewcommand{\caseUseName}{CU03 - Crear empleado} %{\CUcammelCase - Title}

\renewcommand{\caseUseSummary}{Este caso de uso permite a un administrador de ZMGestion crear empleados y asignarle un rol en el sistema.} %Resumen
\renewcommand{\caseUsePeople}{Administradores: quiere crear un empleado.} %Actor: Meta
\renewcommand{\caseUsePreconditions}{
	\caseUseRow{Haber iniciado sesión en el sistema y tener el permiso necesario para realizar esta función.} %Precondiciones
}
\renewcommand{\caseUsePostconditions}{
	\caseUseRow{Ninguna.} %Postcondiciones
}
\renewcommand{\caseUseScene}{ %Escenario principal
    \addCaseUseStep{El administrador accede a la pantalla para crear empleados}
    \addCaseUseStep{ZMGestion muestra un formulario para que el usuario ingrese: Nombres, apellidos, correo electrónico, número de teléfono, nombre de usuario, contraseña, fecha de inicio de actividad laboral, cantidad de hijos, estado civil, tipo de documento, documento, teléfono, fecha de nacimiento, rol del empleado que desea agregar y su ubicación en la cual desempeñará su tarea. Indicando que son requeridos todos los campos.}
    \addCaseUseStep{El administrador completa los campos del formulario.}
    \addCaseUseStep{ZMGestion crea el empleado con los campos ingresados por el usuario y muestra un mensaje indicando el éxito de la operación.}
}
\renewcommand{\alternativeCaseUse}{ %Flujos alternativos
	\newAlternative{A1: El nombre de usuario ingresado ya existe.}{3} %Flujo alternativo A1.
	\caseUseRow{La secuencia A1 comienza luego del punto 3 del escenario principal.} %¡Indicar número paso!
    \alternativeRow{ZMGestion muestra un mensaje de error informando que el nombre de usuario ingresado ya está en uso.}
    \caseUseRow{El escenario vuelve al punto 2.}
    \caseUseRow{}
    \newAlternative{A2: El correo electrónico ingresado ya existe.}{3} %Flujo alternativo A2.
	\caseUseRow{La secuencia A2 comienza luego del punto 3 del escenario principal.} %¡Indicar número paso!
    \alternativeRow{ZMGestion muestra un mensaje de error informando que el correo electrónico ingresado ya está en uso.}
    \caseUseRow{El escenario vuelve al punto 2.}
    \caseUseRow{}
    \newAlternative{A3: El documento y tipo de documento ingresado ya existe.}{3} %Flujo alternativo A3.
	\caseUseRow{La secuencia A3 comienza luego del punto 3 del escenario principal.} %¡Indicar número paso!
    \alternativeRow{ZMGestion muestra un mensaje de error informando que el documento y tipo de documento ya existe.}
    \caseUseRow{El escenario vuelve al punto 2.}
    \caseUseRow{}
    \newAlternative{A4: El usuario ha dejado un campo requerido vacío.}{3} %Flujo alternativo A3.
	\caseUseRow{La secuencia A4 comienza luego del punto 3 del escenario principal.} %¡Indicar número paso!
    \alternativeRow{ZMGestion informa al usuario que dicho campo es requerido.}
    \caseUseRow{El escenario vuelve al punto 2.}
    \caseUseRow{}
}

%\item Caso de uso \caseUseName
\renewcommand*{\arraystretch}{1.3}
\begin{longtable}[c]{|>{\raggedright}p{0.3\textwidth} | >{\raggedright}p{0.2\textwidth} | p{0.5\textwidth} |}
\caption{\hyperref[sec:listadoCasoUso]{\caseUseName}}
\label{tabla:\caseUseShortName}\\
\hline
\rowcolor{tableCaseUseBackground}

\multicolumn{3}{|l|}{\textcolor{tableCaseUseFontColor}{Descripción textual del caso de uso: \caseUseName}} \\ \hline

Fecha de Creación: & \multicolumn{2}{L{\secondColumnWidth}|}{\caseUseCreated}\\ \hline

Fecha de Modificación: & \multicolumn{2}{L{\secondColumnWidth}|}{\caseUseModified} \\ \hline

Versión: & \multicolumn{2}{L{\secondColumnWidth}|}{1} \\ \hline

Resumen: & \multicolumn{2}{L{\secondColumnWidth}|}{\caseUseSummary} \\ \hline

Personas involucradas y metas: & \multicolumn{2}{L{\secondColumnWidth}|}{\caseUsePeople} \\ \hline

Precondiciones: \caseUsePreconditions \hline

Postcondiciones: \caseUsePostconditions \hline

Escenario principal: \caseUseScene \hline

Flujos alternativos: \alternativeCaseUse \hline

Requisitos de interfaz de usuario: \caseUseRequirementsGUI \hline
\multirow{3}{*}{Requisitos funcionales:}  & Tiempo de respuesta: & \caseUseResponseTime \\ \cline{2-3} 
& Concurrencia: & \caseUseConcurrence \\ \cline{2-3} 
& Disponibilidad: & \caseUseAvailability \\ \hline
\end{longtable}

\setcounter{rownumbers}{0}

\renewcommand{\alternativeCaseUse}{
	\caseUseRow{No existen flujos alternativos.}
}

%DIAGRAMA DE ACTIVIDAD
%\lineabreak[0]
%\activityDiagram{\caseUseShortName}{Diagrama de actividad - \caseUseName}

\renewcommand{\caseUseShortName}{buscarAvanzadoEmpleados} %cammelCase name

\renewcommand{\caseUseCreated}{27/01/2020} %Fecha creación
\renewcommand{\caseUseModified}{27/01/2020} %Fecha modificación
\renewcommand{\caseUseName}{\CUbuscarAvanzadoEmpleados - Buscar avanzado empleados} %{\CUcammelCase - Title}

\renewcommand{\caseUseSummary}{Este caso de uso permite a los administradores buscar empleados a partir de una cadena de texto de búsqueda.} %Resumen
\renewcommand{\caseUsePeople}{Administradores: quiere encontrar un empleado existente en el sistema.} %Actor: Meta
\renewcommand{\caseUsePreconditions}{
	\caseUseRow{Haber iniciado sesión en el sistema y tener el permiso necesario para realizar esta función.} %Precondiciones
}
\renewcommand{\caseUsePostconditions}{
	\caseUseRow{Ninguna.} %Postcondiciones
}
\renewcommand{\caseUseScene}{ %Escenario principal
    \addCaseUseStep{El administrador accede a la pantalla para realizar la búsqueda de empleados.}
    \addCaseUseStep{ZMGestion muestra un formulario para que el usuario ingrese una cadena de búsqueda, si la busqueda incluye usuarios dados de baja y rol del usuario que está buscando.}
    \addCaseUseStep{El administrador ingresa los campos solicitados.}%3
    \addCaseUseStep{ZMGestion realiza la busqueda por nombres, apellidos, correo electrónico, estado, rol y nombre de usuario.}
    \addCaseUseStep{ZMGestion lista las coincidencias encontradas.}
}
\renewcommand{\alternativeCaseUse}{ %Flujos alternativos
	\newAlternative{A1: No se encontró ninguna coincidencia.}{4} %Flujo alternativo A1.
	\caseUseRow{La secuencia A1 comienza luego del punto 4 del escenario principal.} %¡Indicar número paso!
    \alternativeRow{ZMGestion informa al usuario que no se encontraron resultados para su búsqueda.}
    \caseUseRow{El escenario vuelve al punto 2.}
}

\item Caso de uso \caseUseName
\renewcommand*{\arraystretch}{1.3}
\begin{longtable}[c]{|>{\raggedright}p{0.3\textwidth} | >{\raggedright}p{0.2\textwidth} | p{0.5\textwidth} |}
\caption{\hyperref[sec:listadoCasoUso]{\caseUseName}}
\label{tabla:\caseUseShortName}\\
\hline
\rowcolor{tableCaseUseBackground}

\multicolumn{3}{|l|}{\textcolor{tableCaseUseFontColor}{Descripción textual del caso de uso: \caseUseName}} \\ \hline

Fecha de Creación: & \multicolumn{2}{L{\secondColumnWidth}|}{\caseUseCreated}\\ \hline

Fecha de Modificación: & \multicolumn{2}{L{\secondColumnWidth}|}{\caseUseModified} \\ \hline

Versión: & \multicolumn{2}{L{\secondColumnWidth}|}{1} \\ \hline

Resumen: & \multicolumn{2}{L{\secondColumnWidth}|}{\caseUseSummary} \\ \hline

Personas involucradas y metas: & \multicolumn{2}{L{\secondColumnWidth}|}{\caseUsePeople} \\ \hline

Precondiciones: \caseUsePreconditions \hline

Postcondiciones: \caseUsePostconditions \hline

Escenario principal: \caseUseScene \hline

Flujos alternativos: \alternativeCaseUse \hline

Requisitos de interfaz de usuario: \caseUseRequirementsGUI \hline
\multirow{3}{*}{Requisitos funcionales:}  & Tiempo de respuesta: & \caseUseResponseTime \\ \cline{2-3} 
& Concurrencia: & \caseUseConcurrence \\ \cline{2-3} 
& Disponibilidad: & \caseUseAvailability \\ \hline
\end{longtable}

\setcounter{rownumbers}{0}

\renewcommand{\alternativeCaseUse}{
	\caseUseRow{No existen flujos alternativos.}
}

%DIAGRAMA DE ACTIVIDAD
%\lineabreak[0]
%\activityDiagram{\caseUseShortName}{Diagrama de actividad - \caseUseName}

\renewcommand{\caseUseShortName}{darBajaEmpleado} %cammelCase name

\renewcommand{\caseUseCreated}{27/01/2020} %Fecha creación
\renewcommand{\caseUseModified}{27/01/2020} %Fecha modificación
\renewcommand{\caseUseName}{\CUdarBajaEmpleado - Dar de baja empleado} %{\CUcammelCase - Title}

\renewcommand{\caseUseSummary}{Este caso de uso permite a un administrador dar de baja un empleado que se encuentra en el estado de Activo.} %Resumen
\renewcommand{\caseUsePeople}{Administradores: quiere dar de baja un empleado que se encuentra en estado Activo.} %Actor: Meta
\renewcommand{\caseUsePreconditions}{
	\caseUseRow{Haber realizado con éxito el \CUbuscarAvanzadoEmpleados (Buscar avanzado empleados).} %Precondiciones
}
\renewcommand{\caseUsePostconditions}{
	\caseUseRow{Se cierra la sesión del empleado que se está dando de baja eliminando el token de sesión del mismo.} %Postcondiciones
}
\renewcommand{\caseUseScene}{ %Escenario principal
    \addCaseUseStep{El administrador indica el usuario que desea dar de baja.}
    \addCaseUseStep{ZMGestión da de baja al empleado indicado y muestra un mensaje informando que la operación se realizó con éxito.}
    \addCaseUseStep{ZMGestión cierra la sesión del usuario que se dió de baja eliminando su token de sesión.}
}
\renewcommand{\alternativeCaseUse}{ %Flujos alternativos
	\newAlternative{A1: El usuario ya se encuentra en estado Baja.}{1} %Flujo alternativo A1.
	\caseUseRow{La secuencia A1 comienza luego del punto 1 del escenario principal.} %¡Indicar número paso!
    \alternativeRow{ZMGestion informa que el usuario indicado ya que encuentra en estado Baja.}
    \caseUseRow{El escenario vuelve al punto 1.}
}

\item Caso de uso \caseUseName
\renewcommand*{\arraystretch}{1.3}
\begin{longtable}[c]{|>{\raggedright}p{0.3\textwidth} | >{\raggedright}p{0.2\textwidth} | p{0.5\textwidth} |}
\caption{\hyperref[sec:listadoCasoUso]{\caseUseName}}
\label{tabla:\caseUseShortName}\\
\hline
\rowcolor{tableCaseUseBackground}

\multicolumn{3}{|l|}{\textcolor{tableCaseUseFontColor}{Descripción textual del caso de uso: \caseUseName}} \\ \hline

Fecha de Creación: & \multicolumn{2}{L{\secondColumnWidth}|}{\caseUseCreated}\\ \hline

Fecha de Modificación: & \multicolumn{2}{L{\secondColumnWidth}|}{\caseUseModified} \\ \hline

Versión: & \multicolumn{2}{L{\secondColumnWidth}|}{1} \\ \hline

Resumen: & \multicolumn{2}{L{\secondColumnWidth}|}{\caseUseSummary} \\ \hline

Personas involucradas y metas: & \multicolumn{2}{L{\secondColumnWidth}|}{\caseUsePeople} \\ \hline

Precondiciones: \caseUsePreconditions \hline

Postcondiciones: \caseUsePostconditions \hline

Escenario principal: \caseUseScene \hline

Flujos alternativos: \alternativeCaseUse \hline

Requisitos de interfaz de usuario: \caseUseRequirementsGUI \hline
\multirow{3}{*}{Requisitos funcionales:}  & Tiempo de respuesta: & \caseUseResponseTime \\ \cline{2-3} 
& Concurrencia: & \caseUseConcurrence \\ \cline{2-3} 
& Disponibilidad: & \caseUseAvailability \\ \hline
\end{longtable}

\setcounter{rownumbers}{0}

\renewcommand{\alternativeCaseUse}{
	\caseUseRow{No existen flujos alternativos.}
}

%DIAGRAMA DE ACTIVIDAD
%\lineabreak[0]
%\activityDiagram{AD_\caseUseShortName}{Diagrama de actividad - \caseUseName}

\renewcommand{\caseUseShortName}{darAltaEmpleado} %cammelCase name

\renewcommand{\caseUseCreated}{27/01/2020} %Fecha creación
\renewcommand{\caseUseModified}{27/01/2020} %Fecha modificación
\renewcommand{\caseUseName}{\CUdarAltaEmpleado - Dar de alta empleado} %{\CUcammelCase - Title}

\renewcommand{\caseUseSummary}{Este caso de uso permite a un administrador dar de alta un empleado que se encuentra en el estado de Baja.} %Resumen
\renewcommand{\caseUsePeople}{Administradores: quiere dar de alta un empleado que se encuentra en estado Baja.} %Actor: Meta
\renewcommand{\caseUsePreconditions}{
	\caseUseRow{Haber realizado con éxito el \CUbuscarAvanzadoEmpleados (Buscar avanzado empleados).} %Precondiciones
}
\renewcommand{\caseUsePostconditions}{
	\caseUseRow{Ningúna.} %Postcondiciones
}
\renewcommand{\caseUseScene}{ %Escenario principal
    \addCaseUseStep{El administrador indica el usuario que desea dar de alta.}
    \addCaseUseStep{ZMGestión da de alta el usuario indicado y muestra un mensaje informando que la operación se realizó con éxito.}
}
\renewcommand{\alternativeCaseUse}{ %Flujos alternativos
	\newAlternative{A1: El usuario ya se encuentra en estado Alta.}{1} %Flujo alternativo A1.
	\caseUseRow{La secuencia A1 comienza luego del punto 1 del escenario principal.} %¡Indicar número paso!
    \alternativeRow{ZMGestion informa que el usuario indicado ya que encuentra en estado Alta.}
    \caseUseRow{El escenario vuelve al punto 1.}
}

\item Caso de uso \caseUseName
\renewcommand*{\arraystretch}{1.3}
\begin{longtable}[c]{|>{\raggedright}p{0.3\textwidth} | >{\raggedright}p{0.2\textwidth} | p{0.5\textwidth} |}
\caption{\hyperref[sec:listadoCasoUso]{\caseUseName}}
\label{tabla:\caseUseShortName}\\
\hline
\rowcolor{tableCaseUseBackground}

\multicolumn{3}{|l|}{\textcolor{tableCaseUseFontColor}{Descripción textual del caso de uso: \caseUseName}} \\ \hline

Fecha de Creación: & \multicolumn{2}{L{\secondColumnWidth}|}{\caseUseCreated}\\ \hline

Fecha de Modificación: & \multicolumn{2}{L{\secondColumnWidth}|}{\caseUseModified} \\ \hline

Versión: & \multicolumn{2}{L{\secondColumnWidth}|}{1} \\ \hline

Resumen: & \multicolumn{2}{L{\secondColumnWidth}|}{\caseUseSummary} \\ \hline

Personas involucradas y metas: & \multicolumn{2}{L{\secondColumnWidth}|}{\caseUsePeople} \\ \hline

Precondiciones: \caseUsePreconditions \hline

Postcondiciones: \caseUsePostconditions \hline

Escenario principal: \caseUseScene \hline

Flujos alternativos: \alternativeCaseUse \hline

Requisitos de interfaz de usuario: \caseUseRequirementsGUI \hline
\multirow{3}{*}{Requisitos funcionales:}  & Tiempo de respuesta: & \caseUseResponseTime \\ \cline{2-3} 
& Concurrencia: & \caseUseConcurrence \\ \cline{2-3} 
& Disponibilidad: & \caseUseAvailability \\ \hline
\end{longtable}

\setcounter{rownumbers}{0}

\renewcommand{\alternativeCaseUse}{
	\caseUseRow{No existen flujos alternativos.}
}

%DIAGRAMA DE ACTIVIDAD
%\lineabreak[0]
%\activityDiagram{AD_\caseUseShortName}{Diagrama de actividad - \caseUseName}

\renewcommand{\caseUseShortName}{modificarEmpleado} %cammelCase name

\renewcommand{\caseUseCreated}{27/01/2020} %Fecha creación
\renewcommand{\caseUseModified}{27/01/2020} %Fecha modificación
\renewcommand{\caseUseName}{\CUmodificarEmpleado - Modificar empleado} %{\CUcammelCase - Title}

\renewcommand{\caseUseSummary}{Este caso de uso permite a un administrador de ZMGestion modificar un empleado existente.} %Resumen
\renewcommand{\caseUsePeople}{Administradores: quiere modificar un empleado.} %Actor: Meta
\renewcommand{\caseUsePreconditions}{
	\caseUseRow{Haber realizado con éxito el \CUbuscarAvanzadoEmpleados (Buscar avanzado empleados).} %Precondiciones
}
\renewcommand{\caseUsePostconditions}{
	\caseUseRow{Ninguna.} %Postcondiciones
}
\renewcommand{\caseUseScene}{ %Escenario principal
    \addCaseUseStep{El administrador indica el usuario que desea modificar.}
    \addCaseUseStep{ZMGestion muestra un formulario autocompletado con los datos del usuario seleccionado para que el administrador modifique: Nombres, apellidos, correo electrónico, número de teléfono, nombre de usuario, contraseña, fecha de inicio de actividad laboral, cantidad de hijos, estado civil, tipo de documento, documento, teléfono, fecha de nacimiento y/o rol del empleado. Indicando que son requeridos todos los campos.}
    \addCaseUseStep{El usuario modifica los campos que desea cambiar.}
    \addCaseUseStep{ZMGestion modifica el usuario con los nuevos valores de los campos solicitados.}
}
\renewcommand{\alternativeCaseUse}{ %Flujos alternativos
	\newAlternative{A1: El nombre de usuario ingresado está siendo usado por otro empleado.}{3} %Flujo alternativo A1.
	\caseUseRow{La secuencia A1 comienza luego del punto 3 del escenario principal.} %¡Indicar número paso!
    \alternativeRow{ZMGestion muestra un mensaje de error informando que el nombre de usuario ingresado ya está en uso.}
    \caseUseRow{El escenario vuelve al punto 2.}
    \caseUseRow{}
    \newAlternative{A2: El correo electrónico ingresado está siendo usado por otro empleado.}{3} %Flujo alternativo A2.
	\caseUseRow{La secuencia A2 comienza luego del punto 3 del escenario principal.} %¡Indicar número paso!
    \alternativeRow{ZMGestion muestra un mensaje de error informando que el correo electrónico ingresado ya está en uso.}
    \caseUseRow{El escenario vuelve al punto 2.}
    \caseUseRow{}
    \newAlternative{A3: El documento y tipo de documento ingresado está siendo usado por otro empleado.}{3} %Flujo alternativo A3.
	\caseUseRow{La secuencia A3 comienza luego del punto 3 del escenario principal.} %¡Indicar número paso!
    \alternativeRow{ZMGestion muestra un mensaje de error informando que el documento y tipo de documento ya existe.}
    \caseUseRow{El escenario vuelve al punto 2.}
    \caseUseRow{}
    \newAlternative{A4: El usuario ha modificado el rol del usuario.}{3} %Flujo alternativo A3.
	\caseUseRow{La secuencia A4 comienza luego del punto 3 del escenario principal.} %¡Indicar número paso!
    \alternativeRow{ZMGestion cierra la sesión del usuario que se está modificando.}
    \caseUseRow{El escenario continúa desde el punto 4.}
    \caseUseRow{}
    \newAlternative{A5: El usuario ha dejado un campo requerido vacio.}{3} %Flujo alternativo A3.
	\caseUseRow{La secuencia A5 comienza luego del punto 3 del escenario principal.} %¡Indicar número paso!
    \alternativeRow{ZMGestion informa al usuario que dicho campo es requerido.}
    \caseUseRow{El escenario vuelve al punto 2.}
    \caseUseRow{}
}

\item Caso de uso \caseUseName
\renewcommand*{\arraystretch}{1.3}
\begin{longtable}[c]{|>{\raggedright}p{0.3\textwidth} | >{\raggedright}p{0.2\textwidth} | p{0.5\textwidth} |}
\caption{\hyperref[sec:listadoCasoUso]{\caseUseName}}
\label{tabla:\caseUseShortName}\\
\hline
\rowcolor{tableCaseUseBackground}

\multicolumn{3}{|l|}{\textcolor{tableCaseUseFontColor}{Descripción textual del caso de uso: \caseUseName}} \\ \hline

Fecha de Creación: & \multicolumn{2}{L{\secondColumnWidth}|}{\caseUseCreated}\\ \hline

Fecha de Modificación: & \multicolumn{2}{L{\secondColumnWidth}|}{\caseUseModified} \\ \hline

Versión: & \multicolumn{2}{L{\secondColumnWidth}|}{1} \\ \hline

Resumen: & \multicolumn{2}{L{\secondColumnWidth}|}{\caseUseSummary} \\ \hline

Personas involucradas y metas: & \multicolumn{2}{L{\secondColumnWidth}|}{\caseUsePeople} \\ \hline

Precondiciones: \caseUsePreconditions \hline

Postcondiciones: \caseUsePostconditions \hline

Escenario principal: \caseUseScene \hline

Flujos alternativos: \alternativeCaseUse \hline

Requisitos de interfaz de usuario: \caseUseRequirementsGUI \hline
\multirow{3}{*}{Requisitos funcionales:}  & Tiempo de respuesta: & \caseUseResponseTime \\ \cline{2-3} 
& Concurrencia: & \caseUseConcurrence \\ \cline{2-3} 
& Disponibilidad: & \caseUseAvailability \\ \hline
\end{longtable}

\setcounter{rownumbers}{0}

\renewcommand{\alternativeCaseUse}{
	\caseUseRow{No existen flujos alternativos.}
}

%DIAGRAMA DE ACTIVIDAD
%\lineabreak[0]
%\activityDiagram{\caseUseShortName}{Diagrama de actividad - \caseUseName}

\renewcommand{\caseUseShortName}{borrarEmpleado} %cammelCase name

\renewcommand{\caseUseCreated}{27/01/2020} %Fecha creación
\renewcommand{\caseUseModified}{27/01/2020} %Fecha modificación
\renewcommand{\caseUseName}{\CUdarBajaEmpleado - Borrar empleado} %{\CUcammelCase - Title}

\renewcommand{\caseUseSummary}{Este caso de uso permite a un administrador borrar un empleado.} %Resumen
\renewcommand{\caseUsePeople}{Administradores: quiere borrar un empleado existente.} %Actor: Meta
\renewcommand{\caseUsePreconditions}{
	\caseUseRow{Haber realizado con éxito el \CUbuscarAvanzadoEmpleados (Buscar avanzado empleados).} %Precondiciones
}
\renewcommand{\caseUsePostconditions}{
	\caseUseRow{Ninguna.} %Postcondiciones
}
\renewcommand{\caseUseScene}{ %Escenario principal
    \addCaseUseStep{El administrador indica el usuario que desea borrar.}
    \addCaseUseStep{ZMGestión borra el empleado indicado y muestra un mensaje informando que la operación se realizó con éxito.}
}
\renewcommand{\alternativeCaseUse}{ %Flujos alternativos
	\newAlternative{A1: El usuario indicado tiene presupuestos asociados.}{1} %Flujo alternativo A1.
	\caseUseRow{La secuencia A1 comienza luego del punto 1 del escenario principal.} %¡Indicar número paso!
    \alternativeRow{ZMGestion muestra un mensaje de error informando que el usuario no se puede borrar.}
    \caseUseRow{El escenario vuelve al punto 1.}

    \newAlternative{A2: El usuario indicado tiene ventas asociadas.}{1} %Flujo alternativo A2.
	\caseUseRow{La secuencia A2 comienza luego del punto 1 del escenario principal.} %¡Indicar número paso!
    \alternativeRow{ZMGestion muestra un mensaje de error informando que el usuario no se puede borrar.}
    \caseUseRow{El escenario vuelve al punto 1.}

    \newAlternative{A3: El usuario indicado tiene órdenes de producción asociadas.}{1} %Flujo alternativo A3.
	\caseUseRow{La secuencia A3 comienza luego del punto 1 del escenario principal.} %¡Indicar número paso!
    \alternativeRow{ZMGestion muestra un mensaje de error informando que el usuario no se puede borrar.}
    \caseUseRow{El escenario vuelve al punto 1.}

    \newAlternative{A4: El usuario indicado tiene lineas de órdenes de producción asociadas.}{1} %Flujo alternativo A4.
	\caseUseRow{La secuencia A4 comienza luego del punto 1 del escenario principal.} %¡Indicar número paso!
    \alternativeRow{ZMGestion muestra un mensaje de error informando que el usuario no se puede borrar.}
    \caseUseRow{El escenario vuelve al punto 1.}
}

\item Caso de uso \caseUseName
\renewcommand*{\arraystretch}{1.3}
\begin{longtable}[c]{|>{\raggedright}p{0.3\textwidth} | >{\raggedright}p{0.2\textwidth} | p{0.5\textwidth} |}
\caption{\hyperref[sec:listadoCasoUso]{\caseUseName}}
\label{tabla:\caseUseShortName}\\
\hline
\rowcolor{tableCaseUseBackground}

\multicolumn{3}{|l|}{\textcolor{tableCaseUseFontColor}{Descripción textual del caso de uso: \caseUseName}} \\ \hline

Fecha de Creación: & \multicolumn{2}{L{\secondColumnWidth}|}{\caseUseCreated}\\ \hline

Fecha de Modificación: & \multicolumn{2}{L{\secondColumnWidth}|}{\caseUseModified} \\ \hline

Versión: & \multicolumn{2}{L{\secondColumnWidth}|}{1} \\ \hline

Resumen: & \multicolumn{2}{L{\secondColumnWidth}|}{\caseUseSummary} \\ \hline

Personas involucradas y metas: & \multicolumn{2}{L{\secondColumnWidth}|}{\caseUsePeople} \\ \hline

Precondiciones: \caseUsePreconditions \hline

Postcondiciones: \caseUsePostconditions \hline

Escenario principal: \caseUseScene \hline

Flujos alternativos: \alternativeCaseUse \hline

Requisitos de interfaz de usuario: \caseUseRequirementsGUI \hline
\multirow{3}{*}{Requisitos funcionales:}  & Tiempo de respuesta: & \caseUseResponseTime \\ \cline{2-3} 
& Concurrencia: & \caseUseConcurrence \\ \cline{2-3} 
& Disponibilidad: & \caseUseAvailability \\ \hline
\end{longtable}

\setcounter{rownumbers}{0}

\renewcommand{\alternativeCaseUse}{
	\caseUseRow{No existen flujos alternativos.}
}

%DIAGRAMA DE ACTIVIDAD
%\lineabreak[0]
%\activityDiagram{\caseUseShortName}{Diagrama de actividad - \caseUseName}

%GestionRoles

\renewcommand{\caseUseShortName}{crearRol} %cammelCase name

\renewcommand{\caseUseCreated}{31/01/2020} %Fecha creación
\renewcommand{\caseUseModified}{31/01/2020} %Fecha modificación
\renewcommand{\caseUseName}{\CUcrearRol - Crear rol } %{\CUcammelCase - Title}

\renewcommand{\caseUseSummary}{Este caso de uso permite a un administrador de ZMGestion crear un nuevo rol.} %Resumen
\renewcommand{\caseUsePeople}{Administrador: quiere crear un nuevo rol.} %Actor: Meta
\renewcommand{\caseUsePreconditions}{
	\caseUseRow{Haber iniciado sesión en el sistema y tener el permiso necesario para realizar esta función.} %Precondiciones
}
\renewcommand{\caseUsePostconditions}{
	\caseUseRow{Ninguna.} %Postcondiciones
}
\renewcommand{\caseUseScene}{ %Escenario principal
    \addCaseUseStep{El administrador accede a la pantalla para crear roles.}
    \addCaseUseStep{ZMGestion muestra un formulario para que el administrador ingrese el nombre del rol, descripción y seleccione todos los permisos que desea otorgarle. Indicando todos los campos son obligatorios excepto el de descripción.}
    \addCaseUseStep{El administrador completa los campos requeridos del formulario.}
    \addCaseUseStep{ZMGestion crea el rol y muestra un mensaje indicando el éxito de la operación.}
}
\renewcommand{\alternativeCaseUse}{ %Flujos alternativos
	\newAlternative{A1: El nombre del rol ingresado ya esta en uso.}{3} %Flujo alternativo A1.
	\caseUseRow{La secuencia A1 comienza luego del punto 3 del escenario principal.} %¡Indicar número paso!
    \alternativeRow{ZMgestion muestra un mensaje indicando que el nombre ingresado ya se encuentra en uso.}
    \caseUseRow{El escenario vuelve al punto 2.}    
    \caseUseRow{}

	\newAlternative{A2:El administrador ha dejado un campo obligatorio vacio.}{3} %Flujo alternativo A2.
    \caseUseRow{La secuencia A2 comienza luego del punto 3 del escenario principal.}%¡Indicar número paso!
    \alternativeRow{ZMGestion muestra un mensaje de error indicando que dicho campo es requerido.}
    \caseUseRow{EL escenario vuelve al punto 2.}
    \caseUseRow{}
}

\item Caso de uso \caseUseName
\renewcommand*{\arraystretch}{1.3}
\begin{longtable}[c]{|>{\raggedright}p{0.3\textwidth} | >{\raggedright}p{0.2\textwidth} | p{0.5\textwidth} |}
\caption{\hyperref[sec:listadoCasoUso]{\caseUseName}}
\label{tabla:\caseUseShortName}\\
\hline
\rowcolor{tableCaseUseBackground}

\multicolumn{3}{|l|}{\textcolor{tableCaseUseFontColor}{Descripción textual del caso de uso: \caseUseName}} \\ \hline

Fecha de Creación: & \multicolumn{2}{L{\secondColumnWidth}|}{\caseUseCreated}\\ \hline

Fecha de Modificación: & \multicolumn{2}{L{\secondColumnWidth}|}{\caseUseModified} \\ \hline

Versión: & \multicolumn{2}{L{\secondColumnWidth}|}{1} \\ \hline

Resumen: & \multicolumn{2}{L{\secondColumnWidth}|}{\caseUseSummary} \\ \hline

Personas involucradas y metas: & \multicolumn{2}{L{\secondColumnWidth}|}{\caseUsePeople} \\ \hline

Precondiciones: \caseUsePreconditions \hline

Postcondiciones: \caseUsePostconditions \hline

Escenario principal: \caseUseScene \hline

Flujos alternativos: \alternativeCaseUse \hline

Requisitos de interfaz de usuario: \caseUseRequirementsGUI \hline
\multirow{3}{*}{Requisitos funcionales:}  & Tiempo de respuesta: & \caseUseResponseTime \\ \cline{2-3} 
& Concurrencia: & \caseUseConcurrence \\ \cline{2-3} 
& Disponibilidad: & \caseUseAvailability \\ \hline
\end{longtable}

\setcounter{rownumbers}{0}

\renewcommand{\alternativeCaseUse}{
	\caseUseRow{No existen flujos alternativos.}
}

%DIAGRAMA DE ACTIVIDAD
%\lineabreak[0]
\activityDiagram{\caseUseShortName}{Diagrama de actividad - \caseUseName}

\renewcommand{\caseUseShortName}{listarRoles} %cammelCase name

\renewcommand{\caseUseCreated}{31/01/2020} %Fecha creación
\renewcommand{\caseUseModified}{31/01/2020} %Fecha modificación
\renewcommand{\caseUseName}{CU10 - Listar roles } %{\CUcammelCase - Title}

\renewcommand{\caseUseSummary}{Este caso de uso permite a un administrador de ZMGestion listar todos los roles.} %Resumen
\renewcommand{\caseUsePeople}{Administrador: quiere listar los roles existentes en el sitema.} %Actor: Meta
\renewcommand{\caseUsePreconditions}{
	\caseUseRow{Haber iniciado sesión en el sistema y tener el permiso necesario para realizar esta función.} %Precondiciones
}
\renewcommand{\caseUsePostconditions}{
	\caseUseRow{Ninguna.} %Postcondiciones
}
\renewcommand{\caseUseScene}{ %Escenario principal
    \addCaseUseStep{El administrador accede a la pantalla para listar los roles.}
    \addCaseUseStep{ZMGestion muestra una lista con todos los roles existentes en el sistema.}
}

%\item Caso de uso \caseUseName
\renewcommand*{\arraystretch}{1.3}
\begin{longtable}[c]{|>{\raggedright}p{0.3\textwidth} | >{\raggedright}p{0.2\textwidth} | p{0.5\textwidth} |}
\caption{\hyperref[sec:listadoCasoUso]{\caseUseName}}
\label{tabla:\caseUseShortName}\\
\hline
\rowcolor{tableCaseUseBackground}

\multicolumn{3}{|l|}{\textcolor{tableCaseUseFontColor}{Descripción textual del caso de uso: \caseUseName}} \\ \hline

Fecha de Creación: & \multicolumn{2}{L{\secondColumnWidth}|}{\caseUseCreated}\\ \hline

Fecha de Modificación: & \multicolumn{2}{L{\secondColumnWidth}|}{\caseUseModified} \\ \hline

Versión: & \multicolumn{2}{L{\secondColumnWidth}|}{1} \\ \hline

Resumen: & \multicolumn{2}{L{\secondColumnWidth}|}{\caseUseSummary} \\ \hline

Personas involucradas y metas: & \multicolumn{2}{L{\secondColumnWidth}|}{\caseUsePeople} \\ \hline

Precondiciones: \caseUsePreconditions \hline

Postcondiciones: \caseUsePostconditions \hline

Escenario principal: \caseUseScene \hline

Flujos alternativos: \alternativeCaseUse \hline

Requisitos de interfaz de usuario: \caseUseRequirementsGUI \hline
\multirow{3}{*}{Requisitos funcionales:}  & Tiempo de respuesta: & \caseUseResponseTime \\ \cline{2-3} 
& Concurrencia: & \caseUseConcurrence \\ \cline{2-3} 
& Disponibilidad: & \caseUseAvailability \\ \hline
\end{longtable}

\setcounter{rownumbers}{0}

\renewcommand{\alternativeCaseUse}{
	\caseUseRow{No existen flujos alternativos.}
}

%DIAGRAMA DE ACTIVIDAD
%\lineabreak[0]
%\activityDiagram{\caseUseShortName}{Diagrama de actividad - \caseUseName}

\renewcommand{\caseUseShortName}{modificarRol} %cammelCase name

\renewcommand{\caseUseCreated}{31/01/2020} %Fecha creación
\renewcommand{\caseUseModified}{31/01/2020} %Fecha modificación
\renewcommand{\caseUseName}{\CUmodificarRol - Modificar rol} %{\CUcammelCase - Title}

\renewcommand{\caseUseSummary}{Este caso de uso permite a un administrador de ZMGestion modificar un rol existente.} %Resumen
\renewcommand{\caseUsePeople}{Administrador: quiere modificar un rol.} %Actor: Meta
\renewcommand{\caseUsePreconditions}{
	\caseUseRow{Haber realizado con éxito el \CUlistarRoles (Listar roles).} %Precondiciones
}
\renewcommand{\caseUsePostconditions}{
	\caseUseRow{Ninguna.} %Postcondiciones
}
\renewcommand{\caseUseScene}{ %Escenario principal
    \addCaseUseStep{El administrador indica el rol que desea modificar.}
    \addCaseUseStep{ZMGestion muestra un formulario autocmpletado con los datos del rol seleccionado para que el administrador modifique: nombre, descripción y permisos asignados. Indicando todos los campos son obligatorios excepto el de descripción.}
    \addCaseUseStep{ZMGestion modifica el rol y muestra un mensaje indicando el éxito de la operación.}
}
\renewcommand{\alternativeCaseUse}{ %Flujos alternativos
	\newAlternative{A1: El nombre del rol ingresado ya está en uso.}{3} %Flujo alternativo A1.
	\caseUseRow{La secuencia A1 comienza luego del punto 3 del escenario principal.} %¡Indicar número paso!
    \alternativeRow{ZMgestion muestra un mensaje indicando que el nombre ingresado ya se encuentra en uso.}
    \caseUseRow{El escenario vuelve al punto 2.}    
    \caseUseRow{}

	\newAlternative{A2:El administrador ha dejado un campo obligatorio vacío.}{3} %Flujo alternativo A2.
    \caseUseRow{La secuencia A2 comienza luego del punto 3 del escenario principal.}%¡Indicar número paso!
    \alternativeRow{ZMGestion muestra un mensaje de error indicando que dicho campo es requerido.}
    \caseUseRow{EL escenario vuelve al punto 2.}
    \caseUseRow{}
}

\item Caso de uso \caseUseName
\renewcommand*{\arraystretch}{1.3}
\begin{longtable}[c]{|>{\raggedright}p{0.3\textwidth} | >{\raggedright}p{0.2\textwidth} | p{0.5\textwidth} |}
\caption{\hyperref[sec:listadoCasoUso]{\caseUseName}}
\label{tabla:\caseUseShortName}\\
\hline
\rowcolor{tableCaseUseBackground}

\multicolumn{3}{|l|}{\textcolor{tableCaseUseFontColor}{Descripción textual del caso de uso: \caseUseName}} \\ \hline

Fecha de Creación: & \multicolumn{2}{L{\secondColumnWidth}|}{\caseUseCreated}\\ \hline

Fecha de Modificación: & \multicolumn{2}{L{\secondColumnWidth}|}{\caseUseModified} \\ \hline

Versión: & \multicolumn{2}{L{\secondColumnWidth}|}{1} \\ \hline

Resumen: & \multicolumn{2}{L{\secondColumnWidth}|}{\caseUseSummary} \\ \hline

Personas involucradas y metas: & \multicolumn{2}{L{\secondColumnWidth}|}{\caseUsePeople} \\ \hline

Precondiciones: \caseUsePreconditions \hline

Postcondiciones: \caseUsePostconditions \hline

Escenario principal: \caseUseScene \hline

Flujos alternativos: \alternativeCaseUse \hline

Requisitos de interfaz de usuario: \caseUseRequirementsGUI \hline
\multirow{3}{*}{Requisitos funcionales:}  & Tiempo de respuesta: & \caseUseResponseTime \\ \cline{2-3} 
& Concurrencia: & \caseUseConcurrence \\ \cline{2-3} 
& Disponibilidad: & \caseUseAvailability \\ \hline
\end{longtable}

\setcounter{rownumbers}{0}

\renewcommand{\alternativeCaseUse}{
	\caseUseRow{No existen flujos alternativos.}
}

%DIAGRAMA DE ACTIVIDAD
%\lineabreak[0]
%\activityDiagram{\caseUseShortName}{Diagrama de actividad - \caseUseName}

\renewcommand{\caseUseShortName}{borrarRol} %cammelCase name

\renewcommand{\caseUseCreated}{31/01/2020} %Fecha creación
\renewcommand{\caseUseModified}{31/01/2020} %Fecha modificación
\renewcommand{\caseUseName}{\CUborrarRol - Borrar rol } %{\CUcammelCase - Title}

\renewcommand{\caseUseSummary}{Este caso de uso permite a un administrador de ZMGestion borrar un rol existente.} %Resumen
\renewcommand{\caseUsePeople}{Administrador: desea borrar un rol.} %Actor: Meta
\renewcommand{\caseUsePreconditions}{
	\caseUseRow{Haber realizado con éxito el \CUlistarRoles (Listar roles).} %Precondiciones
}
\renewcommand{\caseUsePostconditions}{
	\caseUseRow{Ninguna.} %Postcondiciones
}
\renewcommand{\caseUseScene}{ %Escenario principal
    \addCaseUseStep{El administrador indica el rol que desea borrar.}
    \addCaseUseStep{ZMGestion borra el rol y muestra un mensaje indicando el éxito de la operación.}
}
\renewcommand{\alternativeCaseUse}{ %Flujos alternativos
	\newAlternative{A1: Existe al menos un usuario con el rol seleccionado.}{2} %Flujo alternativo A1.
	\caseUseRow{La secuencia A1 comienza luego del punto 2 del escenario principal.} %¡Indicar número paso!
    \alternativeRow{ZMGestion muestra un mensaje de error indicando que el rol no puede ser eliminado puesto que existe al menos un usuario con dicho rol.}
    \caseUseRow{El escenario vuelve al punto 1.}
    \caseUseRow{}
}

\item Caso de uso \caseUseName
\renewcommand*{\arraystretch}{1.3}
\begin{longtable}[c]{|>{\raggedright}p{0.3\textwidth} | >{\raggedright}p{0.2\textwidth} | p{0.5\textwidth} |}
\caption{\hyperref[sec:listadoCasoUso]{\caseUseName}}
\label{tabla:\caseUseShortName}\\
\hline
\rowcolor{tableCaseUseBackground}

\multicolumn{3}{|l|}{\textcolor{tableCaseUseFontColor}{Descripción textual del caso de uso: \caseUseName}} \\ \hline

Fecha de Creación: & \multicolumn{2}{L{\secondColumnWidth}|}{\caseUseCreated}\\ \hline

Fecha de Modificación: & \multicolumn{2}{L{\secondColumnWidth}|}{\caseUseModified} \\ \hline

Versión: & \multicolumn{2}{L{\secondColumnWidth}|}{1} \\ \hline

Resumen: & \multicolumn{2}{L{\secondColumnWidth}|}{\caseUseSummary} \\ \hline

Personas involucradas y metas: & \multicolumn{2}{L{\secondColumnWidth}|}{\caseUsePeople} \\ \hline

Precondiciones: \caseUsePreconditions \hline

Postcondiciones: \caseUsePostconditions \hline

Escenario principal: \caseUseScene \hline

Flujos alternativos: \alternativeCaseUse \hline

Requisitos de interfaz de usuario: \caseUseRequirementsGUI \hline
\multirow{3}{*}{Requisitos funcionales:}  & Tiempo de respuesta: & \caseUseResponseTime \\ \cline{2-3} 
& Concurrencia: & \caseUseConcurrence \\ \cline{2-3} 
& Disponibilidad: & \caseUseAvailability \\ \hline
\end{longtable}

\setcounter{rownumbers}{0}

\renewcommand{\alternativeCaseUse}{
	\caseUseRow{No existen flujos alternativos.}
}

%DIAGRAMA DE ACTIVIDAD
%\lineabreak[0]
%\activityDiagram{\caseUseShortName}{Diagrama de actividad - \caseUseName}

%GestionProductos

\renewcommand{\caseUseShortName}{crearProducto} %cammelCase name

\renewcommand{\caseUseCreated}{27/01/2020} %Fecha creación
\renewcommand{\caseUseModified}{27/01/2020} %Fecha modificación
\renewcommand{\caseUseName}{\CUcrearProducto - Crear producto } %{\CUcammelCase - Title}

\renewcommand{\caseUseSummary}{Este caso de uso permite a los administradores de ZMGestion crear un producto de manera segura y confiable.} %Resumen
\renewcommand{\caseUsePeople}{Administradores: quiere crear un producto.} %Actor: Meta
\renewcommand{\caseUsePreconditions}{
	\caseUseRow{Haber iniciado sesión en el sistema y tener el permiso necesario para realizar esta función.} %Precondiciones
}
\renewcommand{\caseUsePostconditions}{
	\caseUseRow{Ninguna.} %Postcondiciones
}
\renewcommand{\caseUseScene}{ %Escenario principal
    \addCaseUseStep{El administrador accede a la pantalla para crear productos. }
    \addCaseUseStep{ZMGestion muestra un formulario para que el administrador ingrese el nombre del producto, el precio unitario, la cantidad de metros de tela que deben utilizarse para producirlo, tipo de producto, observaciones y dos listas para seleccionar el grupo y categoría de productos a los cuales pertenece el mismo. Indicando que todos los campos menos la cantidad de metros de tela y observaciones son obligatorios.}
    \addCaseUseStep{El administrador completa los campos del formulario.}
    \addCaseUseStep{ZMGestion crea al producto con los campos ingresados por el usuario y muestra un mensaje indicando el éxito de la operación. }
}
\renewcommand{\alternativeCaseUse}{ %Flujos alternativos
	\newAlternative{A1: El nombre del producto ingresado, la categoria y grupo seleccionado ya existe.}{3} %Flujo alternativo A1.
	\caseUseRow{La secuencia A1 comienza luego del punto 3 del escenario principal.} %¡Indicar número paso!
    \alternativeRow{ZMGestion muestra un mensaje de error indicando que el nombre del producto ya esta en uso para el grupo y categoria seleccionado.}
    \caseUseRow{El escenario vuelve al punto 2.}
    \caseUseRow{}

    \newAlternative{A2: El administrador ha dejado un campo obligatorio vacío.}{3} %Flujo alternativo A1.
	\caseUseRow{La secuencia A2 comienza luego del punto 3 del escenario principal.} %¡Indicar número paso!
    \alternativeRow{ZMGestion muestra un mensaje de error indicando que dicho campo es requerido.}
    \caseUseRow{El escenario vuelve al punto 2.}
    \caseUseRow{}

    \newAlternative{A3: El precio ingresado es menor o igual a cero.}{3} %Flujo alternativo A1.
	\caseUseRow{La secuencia A3 comienza luego del punto 3 del escenario principal.} %¡Indicar número paso!
    \alternativeRow{ZMGestion muestra un mensaje de error informando que el precio no puede ser menor o igual que cero<.}
    \caseUseRow{El escenario vuelve al punto 2.}
    \caseUseRow{}

    \newAlternative{A4: La cantidad de tela requerida ingresada es menor a cero.}{3} %Flujo alternativo A1.
	\caseUseRow{La secuencia A4 comienza luego del punto 3 del escenario principal.} %¡Indicar número paso!
    \alternativeRow{ZMGestion muestra un mensaje de error informando que la cantidad de tela necesaria ingresada es inválida.}
    \caseUseRow{El escenario vuelve al punto 2.}
    \caseUseRow{}

}

\item Caso de uso \caseUseName
\renewcommand*{\arraystretch}{1.3}
\begin{longtable}[c]{|>{\raggedright}p{0.3\textwidth} | >{\raggedright}p{0.2\textwidth} | p{0.5\textwidth} |}
\caption{\hyperref[sec:listadoCasoUso]{\caseUseName}}
\label{tabla:\caseUseShortName}\\
\hline
\rowcolor{tableCaseUseBackground}

\multicolumn{3}{|l|}{\textcolor{tableCaseUseFontColor}{Descripción textual del caso de uso: \caseUseName}} \\ \hline

Fecha de Creación: & \multicolumn{2}{L{\secondColumnWidth}|}{\caseUseCreated}\\ \hline

Fecha de Modificación: & \multicolumn{2}{L{\secondColumnWidth}|}{\caseUseModified} \\ \hline

Versión: & \multicolumn{2}{L{\secondColumnWidth}|}{1} \\ \hline

Resumen: & \multicolumn{2}{L{\secondColumnWidth}|}{\caseUseSummary} \\ \hline

Personas involucradas y metas: & \multicolumn{2}{L{\secondColumnWidth}|}{\caseUsePeople} \\ \hline

Precondiciones: \caseUsePreconditions \hline

Postcondiciones: \caseUsePostconditions \hline

Escenario principal: \caseUseScene \hline

Flujos alternativos: \alternativeCaseUse \hline

Requisitos de interfaz de usuario: \caseUseRequirementsGUI \hline
\multirow{3}{*}{Requisitos funcionales:}  & Tiempo de respuesta: & \caseUseResponseTime \\ \cline{2-3} 
& Concurrencia: & \caseUseConcurrence \\ \cline{2-3} 
& Disponibilidad: & \caseUseAvailability \\ \hline
\end{longtable}

\setcounter{rownumbers}{0}

\renewcommand{\alternativeCaseUse}{
	\caseUseRow{No existen flujos alternativos.}
}

%DIAGRAMA DE ACTIVIDAD
%\lineabreak[0]
\activityDiagram{\caseUseShortName}{Diagrama de actividad - \caseUseName}

\renewcommand{\caseUseShortName}{buscarAvanzadoProductos} %cammelCase name

\renewcommand{\caseUseCreated}{28/01/2020} %Fecha creación
\renewcommand{\caseUseModified}{28/01/2020} %Fecha modificación
\renewcommand{\caseUseName}{\CUbuscarAvanzadoProductos - Buscar avanzado productos } %{\CUcammelCase - Title}

\renewcommand{\caseUseSummary}{Este caso de uso permite a un vendedor de ZMGestion  buscar productos a partir de una cadena de búsqueda.} %Resumen
\renewcommand{\caseUsePeople}{Vendendor: quiere encontrar un producto existente en el sistema.} %Actor: Meta
\renewcommand{\caseUsePreconditions}{
	\caseUseRow{Haber iniciado sesión en el sistema y tener el permiso necesario para realizar esta función.} %Precondiciones
}
\renewcommand{\caseUsePostconditions}{
	\caseUseRow{Ninguna.} %Postcondiciones
}
\renewcommand{\caseUseScene}{ %Escenario principal
    \addCaseUseStep{ El vendedor accede a la pantalla para realizar la búsqueda de productos.}
    \addCaseUseStep{ZMGestion muestra un formulario para que el vendedor ingrese una cadena de búsqueda, elegir el grupo y categoría y tambien seleccionar si desea buscar productos dados de baja.}
    \addCaseUseStep{El vendedor completa los campos solicitados.}
    \addCaseUseStep{ZMGestion realiza la busqueda por nombre del producto, categoría y grupo de productos y estado.}
    \addCaseUseStep{ZMGestion lista las coincidencias encontradas.}
}
\renewcommand{\alternativeCaseUse}{ %Flujos alternativos
	\newAlternative{A1: No se encontró ninguna coincidencia.}{3} %Flujo alternativo A1.
	\caseUseRow{La secuencia A1 comienza luego del punto 3 del escenario principal.} %¡Indicar número paso!
    \alternativeRow{ZMGestion informa al usuario que no se encontron resultados para su búsqueda.}
    \caseUseRow{El escenario vuelve al punto 2.}
    \caseUseRow{}
}
\item Caso de uso \caseUseName
\renewcommand*{\arraystretch}{1.3}
\begin{longtable}[c]{|>{\raggedright}p{0.3\textwidth} | >{\raggedright}p{0.2\textwidth} | p{0.5\textwidth} |}
\caption{\hyperref[sec:listadoCasoUso]{\caseUseName}}
\label{tabla:\caseUseShortName}\\
\hline
\rowcolor{tableCaseUseBackground}

\multicolumn{3}{|l|}{\textcolor{tableCaseUseFontColor}{Descripción textual del caso de uso: \caseUseName}} \\ \hline

Fecha de Creación: & \multicolumn{2}{L{\secondColumnWidth}|}{\caseUseCreated}\\ \hline

Fecha de Modificación: & \multicolumn{2}{L{\secondColumnWidth}|}{\caseUseModified} \\ \hline

Versión: & \multicolumn{2}{L{\secondColumnWidth}|}{1} \\ \hline

Resumen: & \multicolumn{2}{L{\secondColumnWidth}|}{\caseUseSummary} \\ \hline

Personas involucradas y metas: & \multicolumn{2}{L{\secondColumnWidth}|}{\caseUsePeople} \\ \hline

Precondiciones: \caseUsePreconditions \hline

Postcondiciones: \caseUsePostconditions \hline

Escenario principal: \caseUseScene \hline

Flujos alternativos: \alternativeCaseUse \hline

Requisitos de interfaz de usuario: \caseUseRequirementsGUI \hline
\multirow{3}{*}{Requisitos funcionales:}  & Tiempo de respuesta: & \caseUseResponseTime \\ \cline{2-3} 
& Concurrencia: & \caseUseConcurrence \\ \cline{2-3} 
& Disponibilidad: & \caseUseAvailability \\ \hline
\end{longtable}

\setcounter{rownumbers}{0}

\renewcommand{\alternativeCaseUse}{
	\caseUseRow{No existen flujos alternativos.}
}

%DIAGRAMA DE ACTIVIDAD
%\lineabreak[0]
%\activityDiagram{AD_\caseUseShortName}{Diagrama de actividad - \caseUseName}

\renewcommand{\caseUseShortName}{darBajaProducto} %cammelCase name

\renewcommand{\caseUseCreated}{28/01/2020} %Fecha creación
\renewcommand{\caseUseModified}{28/01/2020} %Fecha modificación
\renewcommand{\caseUseName}{\CUdarBajaProducto - Dar de baja producto} %{\CUcammelCase - Title}

\renewcommand{\caseUseSummary}{Este caso de uso permite a un administrador dar de baja un producto que se encuentra en estado activo.} %Resumen
\renewcommand{\caseUsePeople}{Administrador: quiere dar de baja un producto.} %Actor: Meta
\renewcommand{\caseUsePreconditions}{
	\caseUseRow{Haber realizado con éxito el \CUbuscarAvanzadoProductos (Buscar avanzado productos).} %Precondiciones
}
\renewcommand{\caseUsePostconditions}{
	\caseUseRow{Ninguna.} %Postcondiciones
}
\renewcommand{\caseUseScene}{ %Escenario principal
    \addCaseUseStep{El administrador indica el producto que desea dar de baja.}
    \addCaseUseStep{ZMGestion cambia el estado del producto a Baja y muestra un mensaje indicando que la operación se realizó con éxito.}
}
\renewcommand{\alternativeCaseUse}{ %Flujos alternativos
	\newAlternative{A1: El producto ya se encontraba en estado de baja.}{1} %Flujo alternativo A1.
	\caseUseRow{La secuencia A1 comienza luego del punto 1 del escenario principal.} %¡Indicar número paso!
    \alternativeRow{ZMGestion muestra un mensaje de error indicando que el producto ya se encontraba en estado de baja.}
    \caseUseRow{El escenario vuelve al punto 1.}
    \caseUseRow{}
}

\item Caso de uso \caseUseName
\renewcommand*{\arraystretch}{1.3}
\begin{longtable}[c]{|>{\raggedright}p{0.3\textwidth} | >{\raggedright}p{0.2\textwidth} | p{0.5\textwidth} |}
\caption{\hyperref[sec:listadoCasoUso]{\caseUseName}}
\label{tabla:\caseUseShortName}\\
\hline
\rowcolor{tableCaseUseBackground}

\multicolumn{3}{|l|}{\textcolor{tableCaseUseFontColor}{Descripción textual del caso de uso: \caseUseName}} \\ \hline

Fecha de Creación: & \multicolumn{2}{L{\secondColumnWidth}|}{\caseUseCreated}\\ \hline

Fecha de Modificación: & \multicolumn{2}{L{\secondColumnWidth}|}{\caseUseModified} \\ \hline

Versión: & \multicolumn{2}{L{\secondColumnWidth}|}{1} \\ \hline

Resumen: & \multicolumn{2}{L{\secondColumnWidth}|}{\caseUseSummary} \\ \hline

Personas involucradas y metas: & \multicolumn{2}{L{\secondColumnWidth}|}{\caseUsePeople} \\ \hline

Precondiciones: \caseUsePreconditions \hline

Postcondiciones: \caseUsePostconditions \hline

Escenario principal: \caseUseScene \hline

Flujos alternativos: \alternativeCaseUse \hline

Requisitos de interfaz de usuario: \caseUseRequirementsGUI \hline
\multirow{3}{*}{Requisitos funcionales:}  & Tiempo de respuesta: & \caseUseResponseTime \\ \cline{2-3} 
& Concurrencia: & \caseUseConcurrence \\ \cline{2-3} 
& Disponibilidad: & \caseUseAvailability \\ \hline
\end{longtable}

\setcounter{rownumbers}{0}

\renewcommand{\alternativeCaseUse}{
	\caseUseRow{No existen flujos alternativos.}
}

%DIAGRAMA DE ACTIVIDAD
%\lineabreak[0]
%\activityDiagram{AD_\caseUseShortName}{Diagrama de actividad - \caseUseName}

\renewcommand{\caseUseShortName}{darAltaProducto} %cammelCase name

\renewcommand{\caseUseCreated}{29/01/2020} %Fecha creación
\renewcommand{\caseUseModified}{29/01/2020} %Fecha modificación
\renewcommand{\caseUseName}{\CUdarAltaProducto - Dar de alta producto} %{\CUcammelCase - Title}

\renewcommand{\caseUseSummary}{Este caso de uso permite a un administrador dar de alta un producto que se encuentra en estado de Baja.} %Resumen
\renewcommand{\caseUsePeople}{Administrador: quiere dar de alta un producto que esta en estado de Baja.} %Actor: Meta
\renewcommand{\caseUsePreconditions}{
	\caseUseRow{Haber realizado con éxito el \CUbuscarAvanzadoProductos (Buscar avanzado productos).} %Precondiciones
}
\renewcommand{\caseUsePostconditions}{
	\caseUseRow{Ninguna.} %Postcondiciones
}
\renewcommand{\caseUseScene}{ %Escenario principal
    \addCaseUseStep{El administrador indica el producto que desea dar de alta.}
    \addCaseUseStep{ZMGestion da de alta el producto y muestra un mensaje indicando el éxito de la operación.}
}
\renewcommand{\alternativeCaseUse}{ %Flujos alternativos
	\newAlternative{A1: El producto indicado ya se encuentra en estado de Alta.}{1} %Flujo alternativo A1.
	\caseUseRow{La secuencia A1 comienza luego del punto 1 del escenario principal.} %¡Indicar número paso!
    \alternativeRow{ZMGestion muestra un mensaje indicando que el producto ya se encuentra en estado de Alta.}
    \caseUseRow{El escenario vuelve al punto 1.}
    \caseUseRow{}
}
\item Caso de uso \caseUseName
\renewcommand*{\arraystretch}{1.3}
\begin{longtable}[c]{|>{\raggedright}p{0.3\textwidth} | >{\raggedright}p{0.2\textwidth} | p{0.5\textwidth} |}
\caption{\hyperref[sec:listadoCasoUso]{\caseUseName}}
\label{tabla:\caseUseShortName}\\
\hline
\rowcolor{tableCaseUseBackground}

\multicolumn{3}{|l|}{\textcolor{tableCaseUseFontColor}{Descripción textual del caso de uso: \caseUseName}} \\ \hline

Fecha de Creación: & \multicolumn{2}{L{\secondColumnWidth}|}{\caseUseCreated}\\ \hline

Fecha de Modificación: & \multicolumn{2}{L{\secondColumnWidth}|}{\caseUseModified} \\ \hline

Versión: & \multicolumn{2}{L{\secondColumnWidth}|}{1} \\ \hline

Resumen: & \multicolumn{2}{L{\secondColumnWidth}|}{\caseUseSummary} \\ \hline

Personas involucradas y metas: & \multicolumn{2}{L{\secondColumnWidth}|}{\caseUsePeople} \\ \hline

Precondiciones: \caseUsePreconditions \hline

Postcondiciones: \caseUsePostconditions \hline

Escenario principal: \caseUseScene \hline

Flujos alternativos: \alternativeCaseUse \hline

Requisitos de interfaz de usuario: \caseUseRequirementsGUI \hline
\multirow{3}{*}{Requisitos funcionales:}  & Tiempo de respuesta: & \caseUseResponseTime \\ \cline{2-3} 
& Concurrencia: & \caseUseConcurrence \\ \cline{2-3} 
& Disponibilidad: & \caseUseAvailability \\ \hline
\end{longtable}

\setcounter{rownumbers}{0}

\renewcommand{\alternativeCaseUse}{
	\caseUseRow{No existen flujos alternativos.}
}

%DIAGRAMA DE ACTIVIDAD
%\lineabreak[0]
\activityDiagram{\caseUseShortName}{Diagrama de actividad - \caseUseName}

\renewcommand{\caseUseShortName}{modificarProducto} %cammelCase name

\renewcommand{\caseUseCreated}{28/01/2020} %Fecha creación
\renewcommand{\caseUseModified}{28/01/2020} %Fecha modificación
\renewcommand{\caseUseName}{\CUmodificarProducto - Modificar producto } %{\CUcammelCase - Title}

\renewcommand{\caseUseSummary}{Este caso de uso permite a un administrador de ZMGestion modificar un producto existente.} %Resumen
\renewcommand{\caseUsePeople}{Administrador: quiere modificar un producto.} %Actor: Meta
\renewcommand{\caseUsePreconditions}{
	\caseUseRow{Haber realizado con éxito el \CUbuscarAvanzadoProductos (Buscar avanzado productos).} %Precondiciones
}
\renewcommand{\caseUsePostconditions}{
	\caseUseRow{Ninguna.} %Postcondiciones
}
\renewcommand{\caseUseScene}{ %Escenario principal
    \addCaseUseStep{El administrador selecciona el producto que desea modificar.}
    \addCaseUseStep{ZMGestion muestra un formulario autocompletado con los datos del producto seleccionado para que el administrador modifique: nombre, categoría, grupo, tipo de producto, cantidad de metros de tela necesario para producirlo, observaciones y precio. Indicando que todos los campos menos la cantidad de metros de tela y observaciones son obligatorios.}
    \addCaseUseStep{El administrador modifica los campos que desea cambiar.}
    \addCaseUseStep{ZMGestion modifica el producto asignando al producto los valores que hayan sido modificados y muestra un mensaje indicando el éxito de la operación.}
}
\renewcommand{\alternativeCaseUse}{ %Flujos alternativos
	\newAlternative{A1: El nombre, el grupo y categoría ingresados ya estan en uso.}{3} %Flujo alternativo A1.
	\caseUseRow{La secuencia A1 comienza luego del punto 3 del escenario principal.} %¡Indicar número paso!
    \alternativeRow{ZMGestion muestra un mensaje de error indicando que el nombre, grupo y categoría ingresado ya se encuentran en uso.}
    \caseUseRow{El escenario vuelve al punto 2.}
    \caseUseRow{}

	\newAlternative{A2: El precio ingresado es menor o igual a cero.}{3} %Flujo alternativo A2.
    \caseUseRow{La secuencia A2 comienza luego del punto 3 del escenario principal.}%¡Indicar número paso!
    \alternativeRow{ZMGestion muestra un mensaje de error indicando que el precio del producto no puede ser menor o igual que cero.}
    \caseUseRow{El escenario vuelve al punto 2.}
    \caseUseRow{}

    \newAlternative{A3: La cantidad de tela ingresada es menor a cero.}{3} %Flujo alternativo A2.
    \caseUseRow{La secuencia A3 comienza luego del punto 3 del escenario principal.}%¡Indicar número paso!
    \alternativeRow{ZMGestion muestra un mensaje de error indicando que la cantidad de tela no puede ser menor que cero.}
    \caseUseRow{El escenario vuelve al punto 2.}
    \caseUseRow{}

    \newAlternative{A4: El administrador ha dejado un campo requerido vacío.}{3} %Flujo alternativo A3.
	\caseUseRow{La secuencia A4 comienza luego del punto 3 del escenario principal.} %¡Indicar número paso!
    \alternativeRow{ZMGestion muestra un mensaje de error indicando que dicho campo es requerido.}
    \caseUseRow{El escenario vuelve al punto 2.}
    \caseUseRow{}
}

\item Caso de uso \caseUseName
\renewcommand*{\arraystretch}{1.3}
\begin{longtable}[c]{|>{\raggedright}p{0.3\textwidth} | >{\raggedright}p{0.2\textwidth} | p{0.5\textwidth} |}
\caption{\hyperref[sec:listadoCasoUso]{\caseUseName}}
\label{tabla:\caseUseShortName}\\
\hline
\rowcolor{tableCaseUseBackground}

\multicolumn{3}{|l|}{\textcolor{tableCaseUseFontColor}{Descripción textual del caso de uso: \caseUseName}} \\ \hline

Fecha de Creación: & \multicolumn{2}{L{\secondColumnWidth}|}{\caseUseCreated}\\ \hline

Fecha de Modificación: & \multicolumn{2}{L{\secondColumnWidth}|}{\caseUseModified} \\ \hline

Versión: & \multicolumn{2}{L{\secondColumnWidth}|}{1} \\ \hline

Resumen: & \multicolumn{2}{L{\secondColumnWidth}|}{\caseUseSummary} \\ \hline

Personas involucradas y metas: & \multicolumn{2}{L{\secondColumnWidth}|}{\caseUsePeople} \\ \hline

Precondiciones: \caseUsePreconditions \hline

Postcondiciones: \caseUsePostconditions \hline

Escenario principal: \caseUseScene \hline

Flujos alternativos: \alternativeCaseUse \hline

Requisitos de interfaz de usuario: \caseUseRequirementsGUI \hline
\multirow{3}{*}{Requisitos funcionales:}  & Tiempo de respuesta: & \caseUseResponseTime \\ \cline{2-3} 
& Concurrencia: & \caseUseConcurrence \\ \cline{2-3} 
& Disponibilidad: & \caseUseAvailability \\ \hline
\end{longtable}

\setcounter{rownumbers}{0}

\renewcommand{\alternativeCaseUse}{
	\caseUseRow{No existen flujos alternativos.}
}

%DIAGRAMA DE ACTIVIDAD
%\lineabreak[0]
\activityDiagram{\caseUseShortName}{Diagrama de actividad - \caseUseName}

\renewcommand{\caseUseShortName}{borrarProducto} %cammelCase name

\renewcommand{\caseUseCreated}{29/01/2020} %Fecha creación
\renewcommand{\caseUseModified}{29/01/2020} %Fecha modificación
\renewcommand{\caseUseName}{\CUborrarProducto - Borrar producto} %{\CUcammelCase - Title}

\renewcommand{\caseUseSummary}{Este caso de uso permite a un administrador borrar un producto.} %Resumen
\renewcommand{\caseUsePeople}{Administradores: quiere borrar un producto.} %Actor: Meta
\renewcommand{\caseUsePreconditions}{
	\caseUseRow{Haber realizado con éxito el \CUbuscarAvanzadoProductos (Buscar avanzado productos).} %Precondiciones
}
\renewcommand{\caseUsePostconditions}{
	\caseUseRow{Ninguna.} %Postcondiciones
}
\renewcommand{\caseUseScene}{ %Escenario principal
    \addCaseUseStep{El administrador indica el producto que desea borrar.}
    \addCaseUseStep{ZMGestion borra el producto y muestra un mensaje indicando el éxito de la operación.}
}
\renewcommand{\alternativeCaseUse}{ %Flujos alternativos
	\newAlternative{A1: El producto indicado fue utilizado para crear al menos un producto final.}{1} %Flujo alternativo A1.
	\caseUseRow{La secuencia A1 comienza luego del punto 1 del escenario principal.} %¡Indicar número paso!
    \alternativeRow{ZMGestion muestra un mensaje de error informando que el producto no puede borrarse.}
    \caseUseRow{El escenario vuelve al punto 1.}
    \caseUseRow{}
}
\item Caso de uso \caseUseName
\renewcommand*{\arraystretch}{1.3}
\begin{longtable}[c]{|>{\raggedright}p{0.3\textwidth} | >{\raggedright}p{0.2\textwidth} | p{0.5\textwidth} |}
\caption{\hyperref[sec:listadoCasoUso]{\caseUseName}}
\label{tabla:\caseUseShortName}\\
\hline
\rowcolor{tableCaseUseBackground}

\multicolumn{3}{|l|}{\textcolor{tableCaseUseFontColor}{Descripción textual del caso de uso: \caseUseName}} \\ \hline

Fecha de Creación: & \multicolumn{2}{L{\secondColumnWidth}|}{\caseUseCreated}\\ \hline

Fecha de Modificación: & \multicolumn{2}{L{\secondColumnWidth}|}{\caseUseModified} \\ \hline

Versión: & \multicolumn{2}{L{\secondColumnWidth}|}{1} \\ \hline

Resumen: & \multicolumn{2}{L{\secondColumnWidth}|}{\caseUseSummary} \\ \hline

Personas involucradas y metas: & \multicolumn{2}{L{\secondColumnWidth}|}{\caseUsePeople} \\ \hline

Precondiciones: \caseUsePreconditions \hline

Postcondiciones: \caseUsePostconditions \hline

Escenario principal: \caseUseScene \hline

Flujos alternativos: \alternativeCaseUse \hline

Requisitos de interfaz de usuario: \caseUseRequirementsGUI \hline
\multirow{3}{*}{Requisitos funcionales:}  & Tiempo de respuesta: & \caseUseResponseTime \\ \cline{2-3} 
& Concurrencia: & \caseUseConcurrence \\ \cline{2-3} 
& Disponibilidad: & \caseUseAvailability \\ \hline
\end{longtable}

\setcounter{rownumbers}{0}

\renewcommand{\alternativeCaseUse}{
	\caseUseRow{No existen flujos alternativos.}
}

%DIAGRAMA DE ACTIVIDAD
%\lineabreak[0]
%\activityDiagram{AD_\caseUseShortName}{Diagrama de actividad - \caseUseName}

%GestionTelas
\renewcommand{\caseUseShortName}{crearTela} %cammelCase name

\renewcommand{\caseUseCreated}{29/01/2020} %Fecha creación
\renewcommand{\caseUseModified}{29/01/2020} %Fecha modificación
\renewcommand{\caseUseName}{\CUcrearTela - Crear tela} %{\CUcammelCase - Title}

\renewcommand{\caseUseSummary}{Este caso de uso permite a un administrador de ZMGestion crear una tela.} %Resumen
\renewcommand{\caseUsePeople}{Administrador: quiere crear una tela.} %Actor: Meta
\renewcommand{\caseUsePreconditions}{
	\caseUseRow{Haber iniciado sesión en el sistema y tener el permiso necesario para realiar esta función.} %Precondiciones
}
\renewcommand{\caseUsePostconditions}{
	\caseUseRow{Ninguna.} %Postcondiciones
}
\renewcommand{\caseUseScene}{ %Escenario principal
    \addCaseUseStep{El administrador accede a la pantalla para crear telas.}
    \addCaseUseStep{ZMGestion muestra un formulario para que el administrador ingrese el nombre de la tela, el precio por metro y observaciones. Indicando que todos los campos son obligatorios menos el de observaciones.}
    \addCaseUseStep{El administrador completa los campos requeridos del formulario.}
    \addCaseUseStep{ZMGestion crea la tela con los campos ingresador por el usuario y muestra un mensaje indicando el éxito de la operación.}
}
\renewcommand{\alternativeCaseUse}{ %Flujos alternativos
	\newAlternative{A1: El nombre ingresado ya se encuentra en uso.}{3} %Flujo alternativo A1.
	\caseUseRow{La secuencia A1 comienza luego del punto 3 del escenario principal.} %¡Indicar número paso!
    \alternativeRow{ZMGestion muestra un mensaje de error indicando que el nombre ya se encuentra en uso.}
    \caseUseRow{EL escenario vuelve al punto 2.}
    \caseUseRow{}

	\newAlternative{A2:El precio ingresado es menor o igual a cero.}{3} %Flujo alternativo A2.
    \caseUseRow{La secuencia A2 comienza luego del punto 3 del escenario principal.}%¡Indicar número paso!
    \alternativeRow{ZMGestion muestra un mensaje de error indicando que el precio no puede ser menor o igual que cero.}
    \caseUseRow{EL escenario vuelve al punto 2.}
    \caseUseRow{}
    \newAlternative{A3:El administrador ha dejado un campo obligatorio vacio.}{3} %Flujo alternativo A2.
    \caseUseRow{La secuencia A3 comienza luego del punto 3 del escenario principal.}%¡Indicar número paso!
    \alternativeRow{ZMGestion muestra un mensaje de error indicando que dicho campo es requerido.}
    \caseUseRow{EL escenario vuelve al punto 2.}
    \caseUseRow{}
}

\item Caso de uso \caseUseName
\renewcommand*{\arraystretch}{1.3}
\begin{longtable}[c]{|>{\raggedright}p{0.3\textwidth} | >{\raggedright}p{0.2\textwidth} | p{0.5\textwidth} |}
\caption{\hyperref[sec:listadoCasoUso]{\caseUseName}}
\label{tabla:\caseUseShortName}\\
\hline
\rowcolor{tableCaseUseBackground}

\multicolumn{3}{|l|}{\textcolor{tableCaseUseFontColor}{Descripción textual del caso de uso: \caseUseName}} \\ \hline

Fecha de Creación: & \multicolumn{2}{L{\secondColumnWidth}|}{\caseUseCreated}\\ \hline

Fecha de Modificación: & \multicolumn{2}{L{\secondColumnWidth}|}{\caseUseModified} \\ \hline

Versión: & \multicolumn{2}{L{\secondColumnWidth}|}{1} \\ \hline

Resumen: & \multicolumn{2}{L{\secondColumnWidth}|}{\caseUseSummary} \\ \hline

Personas involucradas y metas: & \multicolumn{2}{L{\secondColumnWidth}|}{\caseUsePeople} \\ \hline

Precondiciones: \caseUsePreconditions \hline

Postcondiciones: \caseUsePostconditions \hline

Escenario principal: \caseUseScene \hline

Flujos alternativos: \alternativeCaseUse \hline

Requisitos de interfaz de usuario: \caseUseRequirementsGUI \hline
\multirow{3}{*}{Requisitos funcionales:}  & Tiempo de respuesta: & \caseUseResponseTime \\ \cline{2-3} 
& Concurrencia: & \caseUseConcurrence \\ \cline{2-3} 
& Disponibilidad: & \caseUseAvailability \\ \hline
\end{longtable}

\setcounter{rownumbers}{0}

\renewcommand{\alternativeCaseUse}{
	\caseUseRow{No existen flujos alternativos.}
}

%DIAGRAMA DE ACTIVIDAD
%\lineabreak[0]
\activityDiagram{\caseUseShortName}{Diagrama de actividad - \caseUseName}

\renewcommand{\caseUseShortName}{listarTelas} %cammelCase name

\renewcommand{\caseUseCreated}{29/01/2020} %Fecha creación
\renewcommand{\caseUseModified}{29/01/2020} %Fecha modificación
\renewcommand{\caseUseName}{\CUlistarTelas - Listar telas } %{\CUcammelCase - Title}

\renewcommand{\caseUseSummary}{Este caso de uso permite a un vendedor de ZMGestion listar todas las telas.} %Resumen
\renewcommand{\caseUsePeople}{Vendedor: quiere ver todas las telas existentes.} %Actor: Meta
\renewcommand{\caseUsePreconditions}{
	\caseUseRow{Haber iniciado sesión en el sistema y tener el permiso necesario para realizar esta función.} %Precondiciones
}
\renewcommand{\caseUsePostconditions}{
	\caseUseRow{Ninguna.} %Postcondiciones
}
\renewcommand{\caseUseScene}{ %Escenario principal
    \addCaseUseStep{El vendedor accede a la pantalla para ver todas las telas.}
    \addCaseUseStep{ZMGestion muestra una lista con todas las telas existentes.}

}
\renewcommand{\alternativeCaseUse}{ %Flujos alternativos
	\newAlternative{A1: No existe ninguna tela.}{1} %Flujo alternativo A1.
	\caseUseRow{La secuencia A1 comienza luego del punto 1 del escenario principal.} %¡Indicar número paso!
    \alternativeRow{ZMGestion muestra un mensaje indicando que no existe ninguna tela.}
    \caseUseRow{}
}

\item Caso de uso \caseUseName
\renewcommand*{\arraystretch}{1.3}
\begin{longtable}[c]{|>{\raggedright}p{0.3\textwidth} | >{\raggedright}p{0.2\textwidth} | p{0.5\textwidth} |}
\caption{\hyperref[sec:listadoCasoUso]{\caseUseName}}
\label{tabla:\caseUseShortName}\\
\hline
\rowcolor{tableCaseUseBackground}

\multicolumn{3}{|l|}{\textcolor{tableCaseUseFontColor}{Descripción textual del caso de uso: \caseUseName}} \\ \hline

Fecha de Creación: & \multicolumn{2}{L{\secondColumnWidth}|}{\caseUseCreated}\\ \hline

Fecha de Modificación: & \multicolumn{2}{L{\secondColumnWidth}|}{\caseUseModified} \\ \hline

Versión: & \multicolumn{2}{L{\secondColumnWidth}|}{1} \\ \hline

Resumen: & \multicolumn{2}{L{\secondColumnWidth}|}{\caseUseSummary} \\ \hline

Personas involucradas y metas: & \multicolumn{2}{L{\secondColumnWidth}|}{\caseUsePeople} \\ \hline

Precondiciones: \caseUsePreconditions \hline

Postcondiciones: \caseUsePostconditions \hline

Escenario principal: \caseUseScene \hline

Flujos alternativos: \alternativeCaseUse \hline

Requisitos de interfaz de usuario: \caseUseRequirementsGUI \hline
\multirow{3}{*}{Requisitos funcionales:}  & Tiempo de respuesta: & \caseUseResponseTime \\ \cline{2-3} 
& Concurrencia: & \caseUseConcurrence \\ \cline{2-3} 
& Disponibilidad: & \caseUseAvailability \\ \hline
\end{longtable}

\setcounter{rownumbers}{0}

\renewcommand{\alternativeCaseUse}{
	\caseUseRow{No existen flujos alternativos.}
}

%DIAGRAMA DE ACTIVIDAD
%\lineabreak[0]
\activityDiagram{\caseUseShortName}{Diagrama de actividad - \caseUseName}

\renewcommand{\caseUseShortName}{darBajaTela} %cammelCase name

\renewcommand{\caseUseCreated}{29/01/2020} %Fecha creación
\renewcommand{\caseUseModified}{29/01/2020} %Fecha modificación
\renewcommand{\caseUseName}{\CUdarBajaTela - Dar de baja tela } %{\CUcammelCase - Title}

\renewcommand{\caseUseSummary}{Este caso de uso permito a un administrador de ZMGestion dar de baja una tela.} %Resumen
\renewcommand{\caseUsePeople}{Administrador: quiere dar de baja una tela.} %Actor: Meta
\renewcommand{\caseUsePreconditions}{
	\caseUseRow{Haber realizado con éxito el \CUlistarTelas (Listar telas).} %Precondiciones
}
\renewcommand{\caseUsePostconditions}{
	\caseUseRow{Ninguna.} %Postcondiciones
}
\renewcommand{\caseUseScene}{ %Escenario principal
    \addCaseUseStep{El administrador indica la tela que quiere dar de baja.}
    \addCaseUseStep{ZMGestion cambia el estado de la tela a Baja y muestra un mensaje indicando que la operación se realizo con éxito.}
}
\renewcommand{\alternativeCaseUse}{ %Flujos alternativos
	\newAlternative{A1: La tela ya se encuentra en el estado de Baja.}{1} %Flujo alternativo A1.
	\caseUseRow{La secuencia A1 comienza luego del punto 1 del escenario principal.} %¡Indicar número paso!
    \alternativeRow{ZMGestion muestra un mensaje de error indicando que la tela ya se encontraba en estado de baja.}
    \caseUseRow{El escenario vuelve al punto 1.}
    \caseUseRow{}
}

\item Caso de uso \caseUseName
\renewcommand*{\arraystretch}{1.3}
\begin{longtable}[c]{|>{\raggedright}p{0.3\textwidth} | >{\raggedright}p{0.2\textwidth} | p{0.5\textwidth} |}
\caption{\hyperref[sec:listadoCasoUso]{\caseUseName}}
\label{tabla:\caseUseShortName}\\
\hline
\rowcolor{tableCaseUseBackground}

\multicolumn{3}{|l|}{\textcolor{tableCaseUseFontColor}{Descripción textual del caso de uso: \caseUseName}} \\ \hline

Fecha de Creación: & \multicolumn{2}{L{\secondColumnWidth}|}{\caseUseCreated}\\ \hline

Fecha de Modificación: & \multicolumn{2}{L{\secondColumnWidth}|}{\caseUseModified} \\ \hline

Versión: & \multicolumn{2}{L{\secondColumnWidth}|}{1} \\ \hline

Resumen: & \multicolumn{2}{L{\secondColumnWidth}|}{\caseUseSummary} \\ \hline

Personas involucradas y metas: & \multicolumn{2}{L{\secondColumnWidth}|}{\caseUsePeople} \\ \hline

Precondiciones: \caseUsePreconditions \hline

Postcondiciones: \caseUsePostconditions \hline

Escenario principal: \caseUseScene \hline

Flujos alternativos: \alternativeCaseUse \hline

Requisitos de interfaz de usuario: \caseUseRequirementsGUI \hline
\multirow{3}{*}{Requisitos funcionales:}  & Tiempo de respuesta: & \caseUseResponseTime \\ \cline{2-3} 
& Concurrencia: & \caseUseConcurrence \\ \cline{2-3} 
& Disponibilidad: & \caseUseAvailability \\ \hline
\end{longtable}

\setcounter{rownumbers}{0}

\renewcommand{\alternativeCaseUse}{
	\caseUseRow{No existen flujos alternativos.}
}

%DIAGRAMA DE ACTIVIDAD
%\lineabreak[0]
%\activityDiagram{AD_\caseUseShortName}{Diagrama de actividad - \caseUseName}

\renewcommand{\caseUseShortName}{darAltaTela} %cammelCase name

\renewcommand{\caseUseCreated}{30/01/2020} %Fecha creación
\renewcommand{\caseUseModified}{30/01/2020} %Fecha modificación
\renewcommand{\caseUseName}{CU22 - Dar de alta tela} %{\CUcammelCase - Title}

\renewcommand{\caseUseSummary}{Este caso de uso permite a un administrador de ZMGestion dar de alta una tela que se encuentra en el estado de Baja.} %Resumen
\renewcommand{\caseUsePeople}{Administradores: quiere dar de alta una tela que se encuentra en estado de Baja.} %Actor: Meta
\renewcommand{\caseUsePreconditions}{
	\caseUseRow{Haber realizado con éxito el CU20 (Listar telas).} %Precondiciones
}
\renewcommand{\caseUsePostconditions}{
	\caseUseRow{Ninguna.} %Postcondiciones
}
\renewcommand{\caseUseScene}{ %Escenario principal
    \addCaseUseStep{El administrador indica la tela que quiere dar de alta}
    \addCaseUseStep{ZMGestion da de alta la tela y muestra un mesaje indicando el éxito de la operación.}

}
\renewcommand{\alternativeCaseUse}{ %Flujos alternativos
	\newAlternative{A1: La tela indicada ya se encuenta en el estado de Alta.}{1} %Flujo alternativo A1.
	\caseUseRow{La secuencia A1 comienza luego del punto 1 del escenario principal.} %¡Indicar número paso!
    \alternativeRow{ZMGestion muestra un mensaje indicando que la tela ya se ecuentra en el estado de Alta.}    
    \caseUseRow{El escenario vuelve al punto 1.}
    \caseUseRow{}
}

%\item Caso de uso \caseUseName
\renewcommand*{\arraystretch}{1.3}
\begin{longtable}[c]{|>{\raggedright}p{0.3\textwidth} | >{\raggedright}p{0.2\textwidth} | p{0.5\textwidth} |}
\caption{\hyperref[sec:listadoCasoUso]{\caseUseName}}
\label{tabla:\caseUseShortName}\\
\hline
\rowcolor{tableCaseUseBackground}

\multicolumn{3}{|l|}{\textcolor{tableCaseUseFontColor}{Descripción textual del caso de uso: \caseUseName}} \\ \hline

Fecha de Creación: & \multicolumn{2}{L{\secondColumnWidth}|}{\caseUseCreated}\\ \hline

Fecha de Modificación: & \multicolumn{2}{L{\secondColumnWidth}|}{\caseUseModified} \\ \hline

Versión: & \multicolumn{2}{L{\secondColumnWidth}|}{1} \\ \hline

Resumen: & \multicolumn{2}{L{\secondColumnWidth}|}{\caseUseSummary} \\ \hline

Personas involucradas y metas: & \multicolumn{2}{L{\secondColumnWidth}|}{\caseUsePeople} \\ \hline

Precondiciones: \caseUsePreconditions \hline

Postcondiciones: \caseUsePostconditions \hline

Escenario principal: \caseUseScene \hline

Flujos alternativos: \alternativeCaseUse \hline

Requisitos de interfaz de usuario: \caseUseRequirementsGUI \hline
\multirow{3}{*}{Requisitos funcionales:}  & Tiempo de respuesta: & \caseUseResponseTime \\ \cline{2-3} 
& Concurrencia: & \caseUseConcurrence \\ \cline{2-3} 
& Disponibilidad: & \caseUseAvailability \\ \hline
\end{longtable}

\setcounter{rownumbers}{0}

\renewcommand{\alternativeCaseUse}{
	\caseUseRow{No existen flujos alternativos.}
}

%DIAGRAMA DE ACTIVIDAD
%\lineabreak[0]
\activityDiagram{\caseUseShortName}{Diagrama de actividad - \caseUseName}

\renewcommand{\caseUseShortName}{modificarTela} %cammelCase name

\renewcommand{\caseUseCreated}{30/01/2020} %Fecha creación
\renewcommand{\caseUseModified}{30/01/2020} %Fecha modificación
\renewcommand{\caseUseName}{\CUmodificarTela - Modificar tela } %{\CUcammelCase - Title}

\renewcommand{\caseUseSummary}{Este caso de uso permite a un administrador de ZMGestion modificar una tela.} %Resumen
\renewcommand{\caseUsePeople}{Administradores: quiere modificar una tela.} %Actor: Meta
\renewcommand{\caseUsePreconditions}{
	\caseUseRow{Haber realizado con éxito el \CUlistarTelas (Listar telas).} %Precondiciones
}
\renewcommand{\caseUsePostconditions}{
	\caseUseRow{Ninguna.} %Postcondiciones
}
\renewcommand{\caseUseScene}{ %Escenario principal
    \addCaseUseStep{El administrador indica la tela que quiere modificar.}
    \addCaseUseStep{ZMGestion muestra un formulario autocompletado con los datos de la tela seleccionada para que el administrador modifique: nombre, precio por metro y observaciones. Indicando que todos los campos son obligatorios excepto el de observaciones.}
    \addCaseUseStep{El administrador modifca los datos que desea cambiar.}
    \addCaseUseStep{ZMGestion modifica la tela y muestra un mensaje indicando el éxito de la operación}

}
\renewcommand{\alternativeCaseUse}{ %Flujos alternativos
	\newAlternative{A1: El nombre ingresado ya se encuentra en uso.}{3} %Flujo alternativo A1.
	\caseUseRow{La secuencia A1 comienza luego del punto 3 del escenario principal.} %¡Indicar número paso!
    \alternativeRow{ZMGestion muestra un mensaje de error indicando que el nombre ingresado ya se encuentra en uso.}
    \caseUseRow{El escenario vuelve al punto 2.}
    \caseUseRow{}

	\newAlternative{A2: El precio por metro ingresado es menor o igual a cero.}{3} %Flujo alternativo A2.
    \caseUseRow{La secuencia A2 comienza luego del punto 3 del escenario principal.}%¡Indicar número paso!
    \alternativeRow{ZMGestion muestra un mensaje de error indicando que el precio no puede ser menor o igual que cero.}
    \caseUseRow{El escenario vuelve al punto 2.}
    \caseUseRow{}

    \newAlternative{A4: El administrador ha dejado un campo requerido vacío.}{3} %Flujo alternativo A3.
	\caseUseRow{La secuencia A4 comienza luego del punto 3 del escenario principal.} %¡Indicar número paso!
    \alternativeRow{ZMGestion muestra un mensaje de error indicando que dicho campo es requerido.}
    \caseUseRow{El escenario vuelve al punto 2.}
    \caseUseRow{}
}
\renewcommand{\caseUseAvailability}{} %Requisitos funcionales: Disponibilidad

\item Caso de uso \caseUseName
\renewcommand*{\arraystretch}{1.3}
\begin{longtable}[c]{|>{\raggedright}p{0.3\textwidth} | >{\raggedright}p{0.2\textwidth} | p{0.5\textwidth} |}
\caption{\hyperref[sec:listadoCasoUso]{\caseUseName}}
\label{tabla:\caseUseShortName}\\
\hline
\rowcolor{tableCaseUseBackground}

\multicolumn{3}{|l|}{\textcolor{tableCaseUseFontColor}{Descripción textual del caso de uso: \caseUseName}} \\ \hline

Fecha de Creación: & \multicolumn{2}{L{\secondColumnWidth}|}{\caseUseCreated}\\ \hline

Fecha de Modificación: & \multicolumn{2}{L{\secondColumnWidth}|}{\caseUseModified} \\ \hline

Versión: & \multicolumn{2}{L{\secondColumnWidth}|}{1} \\ \hline

Resumen: & \multicolumn{2}{L{\secondColumnWidth}|}{\caseUseSummary} \\ \hline

Personas involucradas y metas: & \multicolumn{2}{L{\secondColumnWidth}|}{\caseUsePeople} \\ \hline

Precondiciones: \caseUsePreconditions \hline

Postcondiciones: \caseUsePostconditions \hline

Escenario principal: \caseUseScene \hline

Flujos alternativos: \alternativeCaseUse \hline

Requisitos de interfaz de usuario: \caseUseRequirementsGUI \hline
\multirow{3}{*}{Requisitos funcionales:}  & Tiempo de respuesta: & \caseUseResponseTime \\ \cline{2-3} 
& Concurrencia: & \caseUseConcurrence \\ \cline{2-3} 
& Disponibilidad: & \caseUseAvailability \\ \hline
\end{longtable}

\setcounter{rownumbers}{0}

\renewcommand{\alternativeCaseUse}{
	\caseUseRow{No existen flujos alternativos.}
}

%DIAGRAMA DE ACTIVIDAD
%\lineabreak[0]
\activityDiagram{\caseUseShortName}{Diagrama de actividad - \caseUseName}

\renewcommand{\caseUseShortName}{borrarTela} %cammelCase name

\renewcommand{\caseUseCreated}{30/01/2020} %Fecha creación
\renewcommand{\caseUseModified}{30/01/2020} %Fecha modificación
\renewcommand{\caseUseName}{\CUborrarTela - Borrar tela } %{\CUcammelCase - Title}

\renewcommand{\caseUseSummary}{Este caso de uso permite a un administrador de ZMGestion borrar una tela.} %Resumen
\renewcommand{\caseUsePeople}{Administrador: quiere borrar una tela.} %Actor: Meta
\renewcommand{\caseUsePreconditions}{
	\caseUseRow{Haber realizado con éxito el \CUlistarTelas (Listar telas).} %Precondiciones
}
\renewcommand{\caseUsePostconditions}{
	\caseUseRow{Ninguna.} %Postcondiciones
}
\renewcommand{\caseUseScene}{ %Escenario principal
    \addCaseUseStep{El administrador indica la tela que desea borrar.}
    \addCaseUseStep{ZMGestion borra la tela y muestra un mensaje indicando el éxito de la operación.}
}
\renewcommand{\alternativeCaseUse}{ %Flujos alternativos
	\newAlternative{A1: La tela indicada fue utilizada para crear al menos un producto final.}{1} %Flujo alternativo A1.
	\caseUseRow{La secuencia A1 comienza luego del punto 1 del escenario principal.} %¡Indicar número paso!
    \alternativeRow{ZMGestion muestra un mensaje de error indicando que la tela no puede borrarse.}
    \caseUseRow{El escenario vuelve al punto 1.}
    \caseUseRow{}

}

\item Caso de uso \caseUseName
\renewcommand*{\arraystretch}{1.3}
\begin{longtable}[c]{|>{\raggedright}p{0.3\textwidth} | >{\raggedright}p{0.2\textwidth} | p{0.5\textwidth} |}
\caption{\hyperref[sec:listadoCasoUso]{\caseUseName}}
\label{tabla:\caseUseShortName}\\
\hline
\rowcolor{tableCaseUseBackground}

\multicolumn{3}{|l|}{\textcolor{tableCaseUseFontColor}{Descripción textual del caso de uso: \caseUseName}} \\ \hline

Fecha de Creación: & \multicolumn{2}{L{\secondColumnWidth}|}{\caseUseCreated}\\ \hline

Fecha de Modificación: & \multicolumn{2}{L{\secondColumnWidth}|}{\caseUseModified} \\ \hline

Versión: & \multicolumn{2}{L{\secondColumnWidth}|}{1} \\ \hline

Resumen: & \multicolumn{2}{L{\secondColumnWidth}|}{\caseUseSummary} \\ \hline

Personas involucradas y metas: & \multicolumn{2}{L{\secondColumnWidth}|}{\caseUsePeople} \\ \hline

Precondiciones: \caseUsePreconditions \hline

Postcondiciones: \caseUsePostconditions \hline

Escenario principal: \caseUseScene \hline

Flujos alternativos: \alternativeCaseUse \hline

Requisitos de interfaz de usuario: \caseUseRequirementsGUI \hline
\multirow{3}{*}{Requisitos funcionales:}  & Tiempo de respuesta: & \caseUseResponseTime \\ \cline{2-3} 
& Concurrencia: & \caseUseConcurrence \\ \cline{2-3} 
& Disponibilidad: & \caseUseAvailability \\ \hline
\end{longtable}

\setcounter{rownumbers}{0}

\renewcommand{\alternativeCaseUse}{
	\caseUseRow{No existen flujos alternativos.}
}

%DIAGRAMA DE ACTIVIDAD
%\lineabreak[0]
\activityDiagram{\caseUseShortName}{Diagrama de actividad - \caseUseName}

%GestionProductosFinales

\renewcommand{\caseUseShortName}{crearProductoFinal} %cammelCase name

\renewcommand{\caseUseCreated}{03/02/2020} %Fecha creación
\renewcommand{\caseUseModified}{03/02/2020} %Fecha modificación
\renewcommand{\caseUseName}{CU25 - Crear producto final} %{\CUcammelCase - Title}

\renewcommand{\caseUseSummary}{Este caso de uso permite a un vendedor de ZMGestion crear un producto final.} %Resumen
\renewcommand{\caseUsePeople}{Vendedores: quiere crear un producto final.} %Actor: Meta
\renewcommand{\caseUsePreconditions}{
	\caseUseRow{Haber iniciado sesión en el sistema y tener el permiso necesario para realizar esta función.} %Precondiciones
}
\renewcommand{\caseUsePostconditions}{
	\caseUseRow{Ninguna.} %Postcondiciones
}
\renewcommand{\caseUseScene}{ %Escenario principal
    \addCaseUseStep{El vendedor accede a la pantalla de creación de productos finales.}
    \addCaseUseStep{ZMGestion le muestra un formulario para que el usuario seleccione un producto, tipo de producto, tela y lustre existentes.}
    \addCaseUseStep{El vendedor selecciona un producto, tipo de producto, tela y lustre.}
    \addCaseUseStep{ZMGestion crea el producto final a partir del producto, tipo de producto, tela y lustre seleccionado por el usuario.}
}
\renewcommand{\alternativeCaseUse}{ %Flujos alternativos
	\newAlternative{A1: El producto final ya existe.}{3} %Flujo alternativo A1.
	\caseUseRow{La secuencia A1 comienza luego del punto 3 del escenario principal.} %¡Indicar número paso!
    \alternativeRow{ZMGestion informa al vendedor que ya existe un producto final con el producto, tipo de producto, tela y lustre seleccionados.}
    \caseUseRow{El escenario vuelve al punto 2.}
    \caseUseRow{}
}

%\item Caso de uso \caseUseName
\renewcommand*{\arraystretch}{1.3}
\begin{longtable}[c]{|>{\raggedright}p{0.3\textwidth} | >{\raggedright}p{0.2\textwidth} | p{0.5\textwidth} |}
\caption{\hyperref[sec:listadoCasoUso]{\caseUseName}}
\label{tabla:\caseUseShortName}\\
\hline
\rowcolor{tableCaseUseBackground}

\multicolumn{3}{|l|}{\textcolor{tableCaseUseFontColor}{Descripción textual del caso de uso: \caseUseName}} \\ \hline

Fecha de Creación: & \multicolumn{2}{L{\secondColumnWidth}|}{\caseUseCreated}\\ \hline

Fecha de Modificación: & \multicolumn{2}{L{\secondColumnWidth}|}{\caseUseModified} \\ \hline

Versión: & \multicolumn{2}{L{\secondColumnWidth}|}{1} \\ \hline

Resumen: & \multicolumn{2}{L{\secondColumnWidth}|}{\caseUseSummary} \\ \hline

Personas involucradas y metas: & \multicolumn{2}{L{\secondColumnWidth}|}{\caseUsePeople} \\ \hline

Precondiciones: \caseUsePreconditions \hline

Postcondiciones: \caseUsePostconditions \hline

Escenario principal: \caseUseScene \hline

Flujos alternativos: \alternativeCaseUse \hline

Requisitos de interfaz de usuario: \caseUseRequirementsGUI \hline
\multirow{3}{*}{Requisitos funcionales:}  & Tiempo de respuesta: & \caseUseResponseTime \\ \cline{2-3} 
& Concurrencia: & \caseUseConcurrence \\ \cline{2-3} 
& Disponibilidad: & \caseUseAvailability \\ \hline
\end{longtable}

\setcounter{rownumbers}{0}

\renewcommand{\alternativeCaseUse}{
	\caseUseRow{No existen flujos alternativos.}
}

%DIAGRAMA DE ACTIVIDAD
%\lineabreak[0]
\activityDiagram{\caseUseShortName}{Diagrama de actividad - \caseUseName}
\input{Capitulos/Capitulo4/CasosUso/GestionProductosFinales/buscarAvanzadoProductosFinales.tex}

\renewcommand{\caseUseShortName}{darBajaProductoFinal} %cammelCase name

\renewcommand{\caseUseCreated}{04/02/2020} %Fecha creación
\renewcommand{\caseUseModified}{04/02/2020} %Fecha modificación
\renewcommand{\caseUseName}{\CUdarBajaProductoFinal - Dar de baja producto final} %{\CUcammelCase - Title}

\renewcommand{\caseUseSummary}{Este caso de uso permite a un administrador dar de baja un producto final que se encuentra en estado activo.} %Resumen
\renewcommand{\caseUsePeople}{Administrador: quiere dar de baja un producto final.} %Actor: Meta
\renewcommand{\caseUsePreconditions}{
	\caseUseRow{Haber realizado con éxito el \CUbuscarAvanzadoProductosFinales (Buscar avanzado productos finales).} %Precondiciones
}
\renewcommand{\caseUsePostconditions}{
	\caseUseRow{Ninguna.} %Postcondiciones
}
\renewcommand{\caseUseScene}{ %Escenario principal
    \addCaseUseStep{El administrador indica el producto final que desea dar de baja.}
    \addCaseUseStep{ZMGestion cambia el estado del producto final a Baja y muestra un mensaje indicando que la operación se realizó con éxito.}
}
\renewcommand{\alternativeCaseUse}{ %Flujos alternativos
	\newAlternative{A1: El producto ya se encontraba en estado de baja.}{1} %Flujo alternativo A1.
	\caseUseRow{La secuencia A1 comienza luego del punto 1 del escenario principal.} %¡Indicar número paso!
    \alternativeRow{ZMGestion muestra un mensaje de error indicando que el producto ya se encontraba en estado de baja.}
    \caseUseRow{El escenario vuelve al punto 1.}
    \caseUseRow{}
}

\item Caso de uso \caseUseName
\renewcommand*{\arraystretch}{1.3}
\begin{longtable}[c]{|>{\raggedright}p{0.3\textwidth} | >{\raggedright}p{0.2\textwidth} | p{0.5\textwidth} |}
\caption{\hyperref[sec:listadoCasoUso]{\caseUseName}}
\label{tabla:\caseUseShortName}\\
\hline
\rowcolor{tableCaseUseBackground}

\multicolumn{3}{|l|}{\textcolor{tableCaseUseFontColor}{Descripción textual del caso de uso: \caseUseName}} \\ \hline

Fecha de Creación: & \multicolumn{2}{L{\secondColumnWidth}|}{\caseUseCreated}\\ \hline

Fecha de Modificación: & \multicolumn{2}{L{\secondColumnWidth}|}{\caseUseModified} \\ \hline

Versión: & \multicolumn{2}{L{\secondColumnWidth}|}{1} \\ \hline

Resumen: & \multicolumn{2}{L{\secondColumnWidth}|}{\caseUseSummary} \\ \hline

Personas involucradas y metas: & \multicolumn{2}{L{\secondColumnWidth}|}{\caseUsePeople} \\ \hline

Precondiciones: \caseUsePreconditions \hline

Postcondiciones: \caseUsePostconditions \hline

Escenario principal: \caseUseScene \hline

Flujos alternativos: \alternativeCaseUse \hline

Requisitos de interfaz de usuario: \caseUseRequirementsGUI \hline
\multirow{3}{*}{Requisitos funcionales:}  & Tiempo de respuesta: & \caseUseResponseTime \\ \cline{2-3} 
& Concurrencia: & \caseUseConcurrence \\ \cline{2-3} 
& Disponibilidad: & \caseUseAvailability \\ \hline
\end{longtable}

\setcounter{rownumbers}{0}

\renewcommand{\alternativeCaseUse}{
	\caseUseRow{No existen flujos alternativos.}
}

%DIAGRAMA DE ACTIVIDAD
%\lineabreak[0]
%\activityDiagram{AD_\caseUseShortName}{Diagrama de actividad - \caseUseName}

\renewcommand{\caseUseShortName}{darAltaProductoFinal} %cammelCase name

\renewcommand{\caseUseCreated}{04/02/2020} %Fecha creación
\renewcommand{\caseUseModified}{04/02/2020} %Fecha modificación
\renewcommand{\caseUseName}{CU28 - Dar de alta producto final} %{\CUcammelCase - Title}

\renewcommand{\caseUseSummary}{Este caso de uso permite a un administrador dar de alta un producto final que se encuentra en estado de Baja.} %Resumen
\renewcommand{\caseUsePeople}{Administrador: quiere dar de alta un producto final que esta en estado de Baja.} %Actor: Meta
\renewcommand{\caseUsePreconditions}{
	\caseUseRow{Haber realizado con éxito el CU26 (Buscar avanzado productos finales).} %Precondiciones
}
\renewcommand{\caseUsePostconditions}{
	\caseUseRow{Ninguna.} %Postcondiciones
}
\renewcommand{\caseUseScene}{ %Escenario principal
    \addCaseUseStep{El administrador indica el producto final que desea dar de alta.}
    \addCaseUseStep{ZMGestion da de alta el producto final y muestra un mensaje indicando el éxito de la operación.}
}
\renewcommand{\alternativeCaseUse}{ %Flujos alternativos
	\newAlternative{A1: El producto final indicado ya se encuentra en estado de Alta.}{1} %Flujo alternativo A1.
	\caseUseRow{La secuencia A1 comienza luego del punto 1 del escenario principal.} %¡Indicar número paso!
    \alternativeRow{ZMGestion muestra un mensaje indicando que el producto finaal indicado ya se encuentra en estado de Alta.}
    \caseUseRow{El escenario vuelve al punto 1.}
    \caseUseRow{}
}
%\item Caso de uso \caseUseName
\renewcommand*{\arraystretch}{1.3}
\begin{longtable}[c]{|>{\raggedright}p{0.3\textwidth} | >{\raggedright}p{0.2\textwidth} | p{0.5\textwidth} |}
\caption{\hyperref[sec:listadoCasoUso]{\caseUseName}}
\label{tabla:\caseUseShortName}\\
\hline
\rowcolor{tableCaseUseBackground}

\multicolumn{3}{|l|}{\textcolor{tableCaseUseFontColor}{Descripción textual del caso de uso: \caseUseName}} \\ \hline

Fecha de Creación: & \multicolumn{2}{L{\secondColumnWidth}|}{\caseUseCreated}\\ \hline

Fecha de Modificación: & \multicolumn{2}{L{\secondColumnWidth}|}{\caseUseModified} \\ \hline

Versión: & \multicolumn{2}{L{\secondColumnWidth}|}{1} \\ \hline

Resumen: & \multicolumn{2}{L{\secondColumnWidth}|}{\caseUseSummary} \\ \hline

Personas involucradas y metas: & \multicolumn{2}{L{\secondColumnWidth}|}{\caseUsePeople} \\ \hline

Precondiciones: \caseUsePreconditions \hline

Postcondiciones: \caseUsePostconditions \hline

Escenario principal: \caseUseScene \hline

Flujos alternativos: \alternativeCaseUse \hline

Requisitos de interfaz de usuario: \caseUseRequirementsGUI \hline
\multirow{3}{*}{Requisitos funcionales:}  & Tiempo de respuesta: & \caseUseResponseTime \\ \cline{2-3} 
& Concurrencia: & \caseUseConcurrence \\ \cline{2-3} 
& Disponibilidad: & \caseUseAvailability \\ \hline
\end{longtable}

\setcounter{rownumbers}{0}

\renewcommand{\alternativeCaseUse}{
	\caseUseRow{No existen flujos alternativos.}
}

%DIAGRAMA DE ACTIVIDAD
%\lineabreak[0]
%\activityDiagram{\caseUseShortName}{Diagrama de actividad - \caseUseName}

\renewcommand{\caseUseShortName}{modificarProductoFinal} %cammelCase name

\renewcommand{\caseUseCreated}{04/02/2020} %Fecha creación
\renewcommand{\caseUseModified}{04/02/2020} %Fecha modificación
\renewcommand{\caseUseName}{CU29 - Modificar producto final} %{\CUcammelCase - Title}

\renewcommand{\caseUseSummary}{Este caso de uso permite a un administrador de ZMGestion modificar un producto final existente.} %Resumen
\renewcommand{\caseUsePeople}{Administrador: quiere modificar un producto final.} %Actor: Meta
\renewcommand{\caseUsePreconditions}{
	\caseUseRow{Haber realizado con éxito el CU26 (Buscar avanzado productos finales).} %Precondiciones
}
\renewcommand{\caseUsePostconditions}{
	\caseUseRow{Ninguna.} %Postcondiciones
}
\renewcommand{\caseUseScene}{ %Escenario principal
    \addCaseUseStep{El administrador selecciona el producto final que desea modificar.}
    \addCaseUseStep{ZMGestion muestra un formulario autocompletado con los datos del producto seleccionado para que el administrador modifique: producto, tipo de producto, tela y/o lustre. Siendo todos los campos requeridos.}
    \addCaseUseStep{El administrador modifica los campos que desea cambiar.}
    \addCaseUseStep{ZMGestion modifica el producto asignando al producto los valores que hayan sido modificados y muestra un mensaje indicando el éxito de la operación.}
}
\renewcommand{\alternativeCaseUse}{ %Flujos alternativos
	\newAlternative{A1: Ya existe un producto final con el producto, tipo de producto, tela y lustre seleccionados.}{3} %Flujo alternativo A1.
	\caseUseRow{La secuencia A1 comienza luego del punto 3 del escenario principal.} %¡Indicar número paso!
    \alternativeRow{ZMGestion muestra un mensaje de error indicando que ya existe un producto final con el producto, tipo de producto, tela y lustre seleccionados.}
    \caseUseRow{El escenario vuelve al punto 2.}
    \caseUseRow{}

    \newAlternative{A2: El administrador ha dejado un campo requerido vacío.}{3} %Flujo alternativo A2.
	\caseUseRow{La secuencia A2 comienza luego del punto 3 del escenario principal.} %¡Indicar número paso!
    \alternativeRow{ZMGestion muestra un mensaje de error indicando que dicho campo es requerido.}
    \caseUseRow{El escenario vuelve al punto 2.}
    \caseUseRow{}

    \newAlternative{A2: El producto final fue utilizado en una linea de presupuesto, linea de orden de producción o linea de venta.}{3} %Flujo alternativo A2.
	\caseUseRow{La secuencia A2 comienza luego del punto 3 del escenario principal.} %¡Indicar número paso!
    \alternativeRow{ZMGestion muestra un mensaje de error indicando que no se puede modificar el producto final.}
    \caseUseRow{El escenario vuelve al punto 2.}
    \caseUseRow{}
}

%\item Caso de uso \caseUseName
\renewcommand*{\arraystretch}{1.3}
\begin{longtable}[c]{|>{\raggedright}p{0.3\textwidth} | >{\raggedright}p{0.2\textwidth} | p{0.5\textwidth} |}
\caption{\hyperref[sec:listadoCasoUso]{\caseUseName}}
\label{tabla:\caseUseShortName}\\
\hline
\rowcolor{tableCaseUseBackground}

\multicolumn{3}{|l|}{\textcolor{tableCaseUseFontColor}{Descripción textual del caso de uso: \caseUseName}} \\ \hline

Fecha de Creación: & \multicolumn{2}{L{\secondColumnWidth}|}{\caseUseCreated}\\ \hline

Fecha de Modificación: & \multicolumn{2}{L{\secondColumnWidth}|}{\caseUseModified} \\ \hline

Versión: & \multicolumn{2}{L{\secondColumnWidth}|}{1} \\ \hline

Resumen: & \multicolumn{2}{L{\secondColumnWidth}|}{\caseUseSummary} \\ \hline

Personas involucradas y metas: & \multicolumn{2}{L{\secondColumnWidth}|}{\caseUsePeople} \\ \hline

Precondiciones: \caseUsePreconditions \hline

Postcondiciones: \caseUsePostconditions \hline

Escenario principal: \caseUseScene \hline

Flujos alternativos: \alternativeCaseUse \hline

Requisitos de interfaz de usuario: \caseUseRequirementsGUI \hline
\multirow{3}{*}{Requisitos funcionales:}  & Tiempo de respuesta: & \caseUseResponseTime \\ \cline{2-3} 
& Concurrencia: & \caseUseConcurrence \\ \cline{2-3} 
& Disponibilidad: & \caseUseAvailability \\ \hline
\end{longtable}

\setcounter{rownumbers}{0}

\renewcommand{\alternativeCaseUse}{
	\caseUseRow{No existen flujos alternativos.}
}

%DIAGRAMA DE ACTIVIDAD
%\lineabreak[0]
%\activityDiagram{\caseUseShortName}{Diagrama de actividad - \caseUseName}

\renewcommand{\caseUseShortName}{borrarProductoFinal} %cammelCase name

\renewcommand{\caseUseCreated}{04/02/2020} %Fecha creación
\renewcommand{\caseUseModified}{04/02/2020} %Fecha modificación
\renewcommand{\caseUseName}{\CUborrarProductoFinal - Borrar producto final} %{\CUcammelCase - Title}

\renewcommand{\caseUseSummary}{Este caso de uso permite a un administrador borrar un producto final.} %Resumen
\renewcommand{\caseUsePeople}{Administradores: quiere borrar un producto final.} %Actor: Meta
\renewcommand{\caseUsePreconditions}{
	\caseUseRow{Haber realizado con éxito el \CUbuscarAvanzadoProductosFinales (Buscar avanzado productos finales).} %Precondiciones
}
\renewcommand{\caseUsePostconditions}{
	\caseUseRow{Ninguna.} %Postcondiciones
}
\renewcommand{\caseUseScene}{ %Escenario principal
    \addCaseUseStep{El administrador indica el producto final que desea borrar.}
    \addCaseUseStep{ZMGestion borra el producto final y muestra un mensaje indicando el éxito de la operación.}
}
\renewcommand{\alternativeCaseUse}{ %Flujos alternativos
    \newAlternative{A1: El producto final indicado fue utilizado en alguna linea de presupuesto.}{1} %Flujo alternativo A1.
    \caseUseRow{La secuencia A1 comienza luego del punto 1 del escenario principal.} %¡Indicar número paso!
    \alternativeRow{ZMGestion muestra un mensaje de error informando que el producto final no puede borrarse.}
    \caseUseRow{El escenario vuelve al punto 1.}
    \caseUseRow{}
    \newAlternative{A2: El producto final indicado fue utilizado en alguna linea de venta o linea de orden de producción.}{1} %Flujo alternativo A2.
	\caseUseRow{La secuencia A2 comienza luego del punto 1 del escenario principal.} %¡Indicar número paso!
    \alternativeRow{ZMGestion muestra un mensaje de error informando que el producto final no puede borrarse.}
    \caseUseRow{El escenario vuelve al punto 1.}
    \caseUseRow{}
}
\item Caso de uso \caseUseName
\renewcommand*{\arraystretch}{1.3}
\begin{longtable}[c]{|>{\raggedright}p{0.3\textwidth} | >{\raggedright}p{0.2\textwidth} | p{0.5\textwidth} |}
\caption{\hyperref[sec:listadoCasoUso]{\caseUseName}}
\label{tabla:\caseUseShortName}\\
\hline
\rowcolor{tableCaseUseBackground}

\multicolumn{3}{|l|}{\textcolor{tableCaseUseFontColor}{Descripción textual del caso de uso: \caseUseName}} \\ \hline

Fecha de Creación: & \multicolumn{2}{L{\secondColumnWidth}|}{\caseUseCreated}\\ \hline

Fecha de Modificación: & \multicolumn{2}{L{\secondColumnWidth}|}{\caseUseModified} \\ \hline

Versión: & \multicolumn{2}{L{\secondColumnWidth}|}{1} \\ \hline

Resumen: & \multicolumn{2}{L{\secondColumnWidth}|}{\caseUseSummary} \\ \hline

Personas involucradas y metas: & \multicolumn{2}{L{\secondColumnWidth}|}{\caseUsePeople} \\ \hline

Precondiciones: \caseUsePreconditions \hline

Postcondiciones: \caseUsePostconditions \hline

Escenario principal: \caseUseScene \hline

Flujos alternativos: \alternativeCaseUse \hline

Requisitos de interfaz de usuario: \caseUseRequirementsGUI \hline
\multirow{3}{*}{Requisitos funcionales:}  & Tiempo de respuesta: & \caseUseResponseTime \\ \cline{2-3} 
& Concurrencia: & \caseUseConcurrence \\ \cline{2-3} 
& Disponibilidad: & \caseUseAvailability \\ \hline
\end{longtable}

\setcounter{rownumbers}{0}

\renewcommand{\alternativeCaseUse}{
	\caseUseRow{No existen flujos alternativos.}
}

%DIAGRAMA DE ACTIVIDAD
%\lineabreak[0]
%\activityDiagram{\caseUseShortName}{Diagrama de actividad - \caseUseName}

%GestionGruposProducto
\input{Capitulos/Capitulo4/CasosUso/GestionGruposProducto/crearGrupoProducto.tex}

\renewcommand{\caseUseShortName}{listarGruposProducto} %cammelCase name

\renewcommand{\caseUseCreated}{04/02/2020} %Fecha creación
\renewcommand{\caseUseModified}{04/02/2020} %Fecha modificación
\renewcommand{\caseUseName}{\CUlistarGruposProducto - Listar grupos de productos } %{\CUcammelCase - Title}

\renewcommand{\caseUseSummary}{Este caso de uso permite a un administrador de ZMGestion listar los grupos de productos.} %Resumen
\renewcommand{\caseUsePeople}{Administradores: quiere listar todos los grupos de productos existentes.} %Actor: Meta
\renewcommand{\caseUsePreconditions}{
	\caseUseRow{Haber iniciado sesión en el sistema y tener el permiso necesario para realizar esta función.} %Precondiciones
}
\renewcommand{\caseUsePostconditions}{
	\caseUseRow{Ninguna.} %Postcondiciones
}
\renewcommand{\caseUseScene}{ %Escenario principal
    \addCaseUseStep{El vendedor accede a la pantalla para ver todos los grupos de productos.}
    \addCaseUseStep{ZMGestion muestra una lista con todos los grupos de producto.}

}
\renewcommand{\alternativeCaseUse}{ %Flujos alternativos
	\newAlternative{A1: No existe ningún grupo de producto.}{1} %Flujo alternativo A1.
	\caseUseRow{La secuencia A1 comienza luego del punto 1 del escenario principal.} %¡Indicar número paso!
    \alternativeRow{ZMGestion muestra un mensaje indicando que no existe ningún grupo de productos.}
    \caseUseRow{}
}

\item Caso de uso \caseUseName
\renewcommand*{\arraystretch}{1.3}
\begin{longtable}[c]{|>{\raggedright}p{0.3\textwidth} | >{\raggedright}p{0.2\textwidth} | p{0.5\textwidth} |}
\caption{\hyperref[sec:listadoCasoUso]{\caseUseName}}
\label{tabla:\caseUseShortName}\\
\hline
\rowcolor{tableCaseUseBackground}

\multicolumn{3}{|l|}{\textcolor{tableCaseUseFontColor}{Descripción textual del caso de uso: \caseUseName}} \\ \hline

Fecha de Creación: & \multicolumn{2}{L{\secondColumnWidth}|}{\caseUseCreated}\\ \hline

Fecha de Modificación: & \multicolumn{2}{L{\secondColumnWidth}|}{\caseUseModified} \\ \hline

Versión: & \multicolumn{2}{L{\secondColumnWidth}|}{1} \\ \hline

Resumen: & \multicolumn{2}{L{\secondColumnWidth}|}{\caseUseSummary} \\ \hline

Personas involucradas y metas: & \multicolumn{2}{L{\secondColumnWidth}|}{\caseUsePeople} \\ \hline

Precondiciones: \caseUsePreconditions \hline

Postcondiciones: \caseUsePostconditions \hline

Escenario principal: \caseUseScene \hline

Flujos alternativos: \alternativeCaseUse \hline

Requisitos de interfaz de usuario: \caseUseRequirementsGUI \hline
\multirow{3}{*}{Requisitos funcionales:}  & Tiempo de respuesta: & \caseUseResponseTime \\ \cline{2-3} 
& Concurrencia: & \caseUseConcurrence \\ \cline{2-3} 
& Disponibilidad: & \caseUseAvailability \\ \hline
\end{longtable}

\setcounter{rownumbers}{0}

\renewcommand{\alternativeCaseUse}{
	\caseUseRow{No existen flujos alternativos.}
}

%DIAGRAMA DE ACTIVIDAD
%\lineabreak[0]
%\activityDiagram{AD_\caseUseShortName}{Diagrama de actividad - \caseUseName}

\renewcommand{\caseUseShortName}{darBajaGrupoProducto} %cammelCase name

\renewcommand{\caseUseCreated}{04/02/2020} %Fecha creación
\renewcommand{\caseUseModified}{04/02/2020} %Fecha modificación
\renewcommand{\caseUseName}{CU33 - Dar de baja grupo de productos} %{\CUcammelCase - Title}

\renewcommand{\caseUseSummary}{Este caso de uso permite a los administradores dar de baja un grupo de productos que se encuentra en estado activo.} %Resumen
\renewcommand{\caseUsePeople}{Administradores: quiere dar de baja un grupo de productos.} %Actor: Meta
\renewcommand{\caseUsePreconditions}{
	\caseUseRow{Haber realizado con éxito el CU32 (Listar grupos de productos).} %Precondiciones
}
\renewcommand{\caseUsePostconditions}{
	\caseUseRow{Ninguna.} %Postcondiciones
}
\renewcommand{\caseUseScene}{ %Escenario principal
    \addCaseUseStep{El administrador indica el grupo de productos que desea dar de baja.}
    \addCaseUseStep{ZMGestion cambia el estado del grupo de productos a Baja y muestra un mensaje indicando el éxito de la operación.}
}
\renewcommand{\alternativeCaseUse}{ %Flujos alternativos
	\newAlternative{A1: El grupo de producto ya se encontraba en estado de baja.}{1} %Flujo alternativo A1.
	\caseUseRow{La secuencia A1 comienza luego del punto 1 del escenario principal.} %¡Indicar número paso!
    \alternativeRow{ZMGestion muestra un mensaje de error indicando que el grupo de productos ya se encontraba en estado de baja.}
    \caseUseRow{El escenario vuelve al punto 1.}
    \caseUseRow{}
}

%\item Caso de uso \caseUseName
\renewcommand*{\arraystretch}{1.3}
\begin{longtable}[c]{|>{\raggedright}p{0.3\textwidth} | >{\raggedright}p{0.2\textwidth} | p{0.5\textwidth} |}
\caption{\hyperref[sec:listadoCasoUso]{\caseUseName}}
\label{tabla:\caseUseShortName}\\
\hline
\rowcolor{tableCaseUseBackground}

\multicolumn{3}{|l|}{\textcolor{tableCaseUseFontColor}{Descripción textual del caso de uso: \caseUseName}} \\ \hline

Fecha de Creación: & \multicolumn{2}{L{\secondColumnWidth}|}{\caseUseCreated}\\ \hline

Fecha de Modificación: & \multicolumn{2}{L{\secondColumnWidth}|}{\caseUseModified} \\ \hline

Versión: & \multicolumn{2}{L{\secondColumnWidth}|}{1} \\ \hline

Resumen: & \multicolumn{2}{L{\secondColumnWidth}|}{\caseUseSummary} \\ \hline

Personas involucradas y metas: & \multicolumn{2}{L{\secondColumnWidth}|}{\caseUsePeople} \\ \hline

Precondiciones: \caseUsePreconditions \hline

Postcondiciones: \caseUsePostconditions \hline

Escenario principal: \caseUseScene \hline

Flujos alternativos: \alternativeCaseUse \hline

Requisitos de interfaz de usuario: \caseUseRequirementsGUI \hline
\multirow{3}{*}{Requisitos funcionales:}  & Tiempo de respuesta: & \caseUseResponseTime \\ \cline{2-3} 
& Concurrencia: & \caseUseConcurrence \\ \cline{2-3} 
& Disponibilidad: & \caseUseAvailability \\ \hline
\end{longtable}

\setcounter{rownumbers}{0}

\renewcommand{\alternativeCaseUse}{
	\caseUseRow{No existen flujos alternativos.}
}

%DIAGRAMA DE ACTIVIDAD
%\lineabreak[0]
%\activityDiagram{\caseUseShortName}{Diagrama de actividad - \caseUseName}

\renewcommand{\caseUseShortName}{darAltaGrupoProducto} %cammelCase name

\renewcommand{\caseUseCreated}{04/02/2020} %Fecha creación
\renewcommand{\caseUseModified}{04/02/2020} %Fecha modificación
\renewcommand{\caseUseName}{\CUdarAltaGrupoProducto - Dar de alta grupo de productos} %{\CUcammelCase - Title}

\renewcommand{\caseUseSummary}{Este caso de uso permite a un administrador dar de alta un grupo de productos que se encuentra en estado de Baja.} %Resumen
\renewcommand{\caseUsePeople}{Administrador: quiere dar de alta un grupo de productos que esta en estado de Baja.} %Actor: Meta
\renewcommand{\caseUsePreconditions}{
	\caseUseRow{Haber realizado con éxito el \CUlistarGruposProducto (Listar grupos de productos).} %Precondiciones
}
\renewcommand{\caseUsePostconditions}{
	\caseUseRow{Ninguna.} %Postcondiciones
}
\renewcommand{\caseUseScene}{ %Escenario principal
    \addCaseUseStep{El administrador indica el grupo de productos que desea dar de alta.}
    \addCaseUseStep{ZMGestion da de alta el grupo de productos seleccionado y muestra un mensaje indicando el éxito de la operación.}
}
\renewcommand{\alternativeCaseUse}{ %Flujos alternativos
	\newAlternative{A1: El grupo de productos indicado ya se encuentra en estado de Alta.}{1} %Flujo alternativo A1.
	\caseUseRow{La secuencia A1 comienza luego del punto 1 del escenario principal.} %¡Indicar número paso!
    \alternativeRow{ZMGestion muestra un mensaje indicando que el grupo de productos ya se encuentra en estado de Alta.}
    \caseUseRow{El escenario vuelve al punto 1.}
    \caseUseRow{}
}
\item Caso de uso \caseUseName
\renewcommand*{\arraystretch}{1.3}
\begin{longtable}[c]{|>{\raggedright}p{0.3\textwidth} | >{\raggedright}p{0.2\textwidth} | p{0.5\textwidth} |}
\caption{\hyperref[sec:listadoCasoUso]{\caseUseName}}
\label{tabla:\caseUseShortName}\\
\hline
\rowcolor{tableCaseUseBackground}

\multicolumn{3}{|l|}{\textcolor{tableCaseUseFontColor}{Descripción textual del caso de uso: \caseUseName}} \\ \hline

Fecha de Creación: & \multicolumn{2}{L{\secondColumnWidth}|}{\caseUseCreated}\\ \hline

Fecha de Modificación: & \multicolumn{2}{L{\secondColumnWidth}|}{\caseUseModified} \\ \hline

Versión: & \multicolumn{2}{L{\secondColumnWidth}|}{1} \\ \hline

Resumen: & \multicolumn{2}{L{\secondColumnWidth}|}{\caseUseSummary} \\ \hline

Personas involucradas y metas: & \multicolumn{2}{L{\secondColumnWidth}|}{\caseUsePeople} \\ \hline

Precondiciones: \caseUsePreconditions \hline

Postcondiciones: \caseUsePostconditions \hline

Escenario principal: \caseUseScene \hline

Flujos alternativos: \alternativeCaseUse \hline

Requisitos de interfaz de usuario: \caseUseRequirementsGUI \hline
\multirow{3}{*}{Requisitos funcionales:}  & Tiempo de respuesta: & \caseUseResponseTime \\ \cline{2-3} 
& Concurrencia: & \caseUseConcurrence \\ \cline{2-3} 
& Disponibilidad: & \caseUseAvailability \\ \hline
\end{longtable}

\setcounter{rownumbers}{0}

\renewcommand{\alternativeCaseUse}{
	\caseUseRow{No existen flujos alternativos.}
}

%DIAGRAMA DE ACTIVIDAD
%\lineabreak[0]
%\activityDiagram{AD_\caseUseShortName}{Diagrama de actividad - \caseUseName}

\renewcommand{\caseUseShortName}{modificarGrupoProducto} %cammelCase name

\renewcommand{\caseUseCreated}{04/02/2020} %Fecha creación
\renewcommand{\caseUseModified}{04/02/2020} %Fecha modificación
\renewcommand{\caseUseName}{\CUmodificarGrupoProducto - Modificar grupo de productos} %{\CUcammelCase - Title}

\renewcommand{\caseUseSummary}{Este caso de uso permite a un administrador de ZMGestion modificar un grupo de productos existente.} %Resumen
\renewcommand{\caseUsePeople}{Administrador: quiere modificar un grupo de productos.} %Actor: Meta
\renewcommand{\caseUsePreconditions}{
	\caseUseRow{Haber realizado con éxito el \CUlistarGruposProducto (Listar grupos de productos).} %Precondiciones
}
\renewcommand{\caseUsePostconditions}{
	\caseUseRow{Ninguna.} %Postcondiciones
}
\renewcommand{\caseUseScene}{ %Escenario principal
    \addCaseUseStep{El administrador selecciona el grupo de producto que desea modificar.}
    \addCaseUseStep{ZMGestion muestra un formulario autocompletado con el nombre del grupo de productos seleccionado.}
    \addCaseUseStep{El administrador modifica el nombre del grupo de productos.}
    \addCaseUseStep{ZMGestion modifica el nombre del grupo de productos con el nuevo valor ingresado por el administrador y muestra un mensaje indicando el éxito de la operación.}
}
\renewcommand{\alternativeCaseUse}{ %Flujos alternativos
	\newAlternative{A1: El nombre del grupo de productos ya está en uso.}{3} %Flujo alternativo A1.
	\caseUseRow{La secuencia A1 comienza luego del punto 3 del escenario principal.} %¡Indicar número paso!
    \alternativeRow{ZMGestion muestra un mensaje de error indicando que el nombre del grupo de productos ya se encuentra en uso.}
    \caseUseRow{El escenario vuelve al punto 2.}
    \caseUseRow{}

    \newAlternative{A2: El administrador no ha ingresado ningún nombre de grupo de productos.}{3} %Flujo alternativo A2.
	\caseUseRow{La secuencia A2 comienza luego del punto 3 del escenario principal.} %¡Indicar número paso!
    \alternativeRow{ZMGestion muestra un mensaje de error indicando que dicho campo es requerido.}
    \caseUseRow{El escenario vuelve al punto 2.}
    \caseUseRow{}
}

\item Caso de uso \caseUseName
\renewcommand*{\arraystretch}{1.3}
\begin{longtable}[c]{|>{\raggedright}p{0.3\textwidth} | >{\raggedright}p{0.2\textwidth} | p{0.5\textwidth} |}
\caption{\hyperref[sec:listadoCasoUso]{\caseUseName}}
\label{tabla:\caseUseShortName}\\
\hline
\rowcolor{tableCaseUseBackground}

\multicolumn{3}{|l|}{\textcolor{tableCaseUseFontColor}{Descripción textual del caso de uso: \caseUseName}} \\ \hline

Fecha de Creación: & \multicolumn{2}{L{\secondColumnWidth}|}{\caseUseCreated}\\ \hline

Fecha de Modificación: & \multicolumn{2}{L{\secondColumnWidth}|}{\caseUseModified} \\ \hline

Versión: & \multicolumn{2}{L{\secondColumnWidth}|}{1} \\ \hline

Resumen: & \multicolumn{2}{L{\secondColumnWidth}|}{\caseUseSummary} \\ \hline

Personas involucradas y metas: & \multicolumn{2}{L{\secondColumnWidth}|}{\caseUsePeople} \\ \hline

Precondiciones: \caseUsePreconditions \hline

Postcondiciones: \caseUsePostconditions \hline

Escenario principal: \caseUseScene \hline

Flujos alternativos: \alternativeCaseUse \hline

Requisitos de interfaz de usuario: \caseUseRequirementsGUI \hline
\multirow{3}{*}{Requisitos funcionales:}  & Tiempo de respuesta: & \caseUseResponseTime \\ \cline{2-3} 
& Concurrencia: & \caseUseConcurrence \\ \cline{2-3} 
& Disponibilidad: & \caseUseAvailability \\ \hline
\end{longtable}

\setcounter{rownumbers}{0}

\renewcommand{\alternativeCaseUse}{
	\caseUseRow{No existen flujos alternativos.}
}

%DIAGRAMA DE ACTIVIDAD
%\lineabreak[0]
%\activityDiagram{\caseUseShortName}{Diagrama de actividad - \caseUseName}

\renewcommand{\caseUseShortName}{borrarGrupoProducto} %cammelCase name

\renewcommand{\caseUseCreated}{04/02/2020} %Fecha creación
\renewcommand{\caseUseModified}{04/02/2020} %Fecha modificación
\renewcommand{\caseUseName}{\CUborrarGrupoProducto - Borrar grupo de productos} %{\CUcammelCase - Title}

\renewcommand{\caseUseSummary}{Este caso de uso permite a un administrador borrar un grupo de productos.} %Resumen
\renewcommand{\caseUsePeople}{Administradores: quiere borrar un grupo de productos.} %Actor: Meta
\renewcommand{\caseUsePreconditions}{
	\caseUseRow{Haber realizado con éxito el \CUlistarGruposProducto (Listar grupos de productos).} %Precondiciones
}
\renewcommand{\caseUsePostconditions}{
	\caseUseRow{Ninguna.} %Postcondiciones
}
\renewcommand{\caseUseScene}{ %Escenario principal
    \addCaseUseStep{El administrador indica el grupo de productos que desea borrar.}
    \addCaseUseStep{ZMGestion borra el grupo de productos y muestra un mensaje indicando el éxito de la operación.}
}
\renewcommand{\alternativeCaseUse}{ %Flujos alternativos
	\newAlternative{A1: Un producto pertenece al grupo de producto que se desea borrar.}{1} %Flujo alternativo A1.
	\caseUseRow{La secuencia A1 comienza luego del punto 1 del escenario principal.} %¡Indicar número paso!
    \alternativeRow{ZMGestion muestra un mensaje de error informando que el grupo de producto no puede borrarse.}
    \caseUseRow{El escenario vuelve al punto 1.}
    \caseUseRow{}
}
\item Caso de uso \caseUseName
\renewcommand*{\arraystretch}{1.3}
\begin{longtable}[c]{|>{\raggedright}p{0.3\textwidth} | >{\raggedright}p{0.2\textwidth} | p{0.5\textwidth} |}
\caption{\hyperref[sec:listadoCasoUso]{\caseUseName}}
\label{tabla:\caseUseShortName}\\
\hline
\rowcolor{tableCaseUseBackground}

\multicolumn{3}{|l|}{\textcolor{tableCaseUseFontColor}{Descripción textual del caso de uso: \caseUseName}} \\ \hline

Fecha de Creación: & \multicolumn{2}{L{\secondColumnWidth}|}{\caseUseCreated}\\ \hline

Fecha de Modificación: & \multicolumn{2}{L{\secondColumnWidth}|}{\caseUseModified} \\ \hline

Versión: & \multicolumn{2}{L{\secondColumnWidth}|}{1} \\ \hline

Resumen: & \multicolumn{2}{L{\secondColumnWidth}|}{\caseUseSummary} \\ \hline

Personas involucradas y metas: & \multicolumn{2}{L{\secondColumnWidth}|}{\caseUsePeople} \\ \hline

Precondiciones: \caseUsePreconditions \hline

Postcondiciones: \caseUsePostconditions \hline

Escenario principal: \caseUseScene \hline

Flujos alternativos: \alternativeCaseUse \hline

Requisitos de interfaz de usuario: \caseUseRequirementsGUI \hline
\multirow{3}{*}{Requisitos funcionales:}  & Tiempo de respuesta: & \caseUseResponseTime \\ \cline{2-3} 
& Concurrencia: & \caseUseConcurrence \\ \cline{2-3} 
& Disponibilidad: & \caseUseAvailability \\ \hline
\end{longtable}

\setcounter{rownumbers}{0}

\renewcommand{\alternativeCaseUse}{
	\caseUseRow{No existen flujos alternativos.}
}

%DIAGRAMA DE ACTIVIDAD
%\lineabreak[0]
%\activityDiagram{AD_\caseUseShortName}{Diagrama de actividad - \caseUseName}

\renewcommand{\caseUseShortName}{listarProductosGrupo} %cammelCase name

\renewcommand{\caseUseCreated}{04/02/2020} %Fecha creación
\renewcommand{\caseUseModified}{04/02/2020} %Fecha modificación
\renewcommand{\caseUseName}{\CUlistarProductosGrupo - Listar productos por grupo} %{\CUcammelCase - Title}

\renewcommand{\caseUseSummary}{Este caso de uso permite a los administradores listar todos los productos que pertenecen a un grupo.} %Resumen
\renewcommand{\caseUsePeople}{} %Actor: Meta
\renewcommand{\caseUsePreconditions}{
	\caseUseRow{Haber realizado con éxito el \CUlistarGruposProducto (Listar grupos de productos).} %Precondiciones
}
\renewcommand{\caseUsePostconditions}{
	\caseUseRow{Ninguna.} %Postcondiciones
}
\renewcommand{\caseUseScene}{ %Escenario principal
    \addCaseUseStep{El administrador indica un grupo de productos.}
    \addCaseUseStep{ZMGestion muestra una lista con todos los productos que pertenecen a dicho grupo de productos.}
}
\renewcommand{\alternativeCaseUse}{ %Flujos alternativos
    \caseUseRow{Ningun flujo alternativo.}
}

\item Caso de uso \caseUseName
\renewcommand*{\arraystretch}{1.3}
\begin{longtable}[c]{|>{\raggedright}p{0.3\textwidth} | >{\raggedright}p{0.2\textwidth} | p{0.5\textwidth} |}
\caption{\hyperref[sec:listadoCasoUso]{\caseUseName}}
\label{tabla:\caseUseShortName}\\
\hline
\rowcolor{tableCaseUseBackground}

\multicolumn{3}{|l|}{\textcolor{tableCaseUseFontColor}{Descripción textual del caso de uso: \caseUseName}} \\ \hline

Fecha de Creación: & \multicolumn{2}{L{\secondColumnWidth}|}{\caseUseCreated}\\ \hline

Fecha de Modificación: & \multicolumn{2}{L{\secondColumnWidth}|}{\caseUseModified} \\ \hline

Versión: & \multicolumn{2}{L{\secondColumnWidth}|}{1} \\ \hline

Resumen: & \multicolumn{2}{L{\secondColumnWidth}|}{\caseUseSummary} \\ \hline

Personas involucradas y metas: & \multicolumn{2}{L{\secondColumnWidth}|}{\caseUsePeople} \\ \hline

Precondiciones: \caseUsePreconditions \hline

Postcondiciones: \caseUsePostconditions \hline

Escenario principal: \caseUseScene \hline

Flujos alternativos: \alternativeCaseUse \hline

Requisitos de interfaz de usuario: \caseUseRequirementsGUI \hline
\multirow{3}{*}{Requisitos funcionales:}  & Tiempo de respuesta: & \caseUseResponseTime \\ \cline{2-3} 
& Concurrencia: & \caseUseConcurrence \\ \cline{2-3} 
& Disponibilidad: & \caseUseAvailability \\ \hline
\end{longtable}

\setcounter{rownumbers}{0}

\renewcommand{\alternativeCaseUse}{
	\caseUseRow{No existen flujos alternativos.}
}

%DIAGRAMA DE ACTIVIDAD
%\lineabreak[0]
%\activityDiagram{\caseUseShortName}{Diagrama de actividad - \caseUseName}

\renewcommand{\caseUseShortName}{} %cammelCase name

\renewcommand{\caseUseCreated}{15/02/2020} %Fecha creación
\renewcommand{\caseUseModified}{15/02/2020} %Fecha modificación
\renewcommand{\caseUseName}{\CU - } %{\CUcammelCase - Title}

\renewcommand{\caseUseSummary}{} %Resumen
\renewcommand{\caseUsePeople}{} %Actor: Meta
\renewcommand{\caseUsePreconditions}{
	\caseUseRow{Haber iniciado sesión en el sistema.} %Precondiciones
}
\renewcommand{\caseUsePostconditions}{
	\caseUseRow{Ninguna.} %Postcondiciones
}
\renewcommand{\caseUseScene}{ %Escenario principal
    \addCaseUseStep{}
    \addCaseUseStep{}
    \addCaseUseStep{}
    \addCaseUseStep{}
    \addCaseUseStep{}
    \addCaseUseStep{}
    \addCaseUseStep{}
    \addCaseUseStep{}
}
\renewcommand{\alternativeCaseUse}{ %Flujos alternativos
	\newAlternative{A1: Error al .}{NUMERO} %Flujo alternativo A1.
	\caseUseRow{La secuencia A1 comienza luego del punto NUMERO del escenario principal.} %¡Indicar número paso!
    \alternativeRow{}
    \alternativeRow{}
    \alternativeRow{}
    \alternativeRow{}
    \alternativeRow{}
    \alternativeRow{}
    
    \caseUseRow{}

	\newAlternative{A2: Error al .}{NUMERO} %Flujo alternativo A2.
    \caseUseRow{La secuencia A2 comienza luego del punto NUMERO del escenario principal.}%¡Indicar número paso!
    \alternativeRow{}
    \alternativeRow{}
    \alternativeRow{}
    \alternativeRow{}
    \alternativeRow{}
    \alternativeRow{}
}
\renewcommand{\caseUseRequirementsGUI}{
	\caseUseRow{Teclado, Mouse y Pantalla} %Requisitos interfaz de usuario
}
\renewcommand{\caseUseResponseTime}{La interfaz debe responder dentro de un tiempo máximo de 10 segundos.} %Requisitos funcionales: Tiempo de respuesta
\renewcommand{\caseUseConcurrence}{} %Requisitos funcionales: Concurrencia
\renewcommand{\caseUseAvailability}{} %Requisitos funcionales: Disponibilidad

\item Caso de uso \caseUseName
\renewcommand*{\arraystretch}{1.3}
\begin{longtable}[c]{|>{\raggedright}p{0.3\textwidth} | >{\raggedright}p{0.2\textwidth} | p{0.5\textwidth} |}
\caption{\hyperref[sec:listadoCasoUso]{\caseUseName}}
\label{tabla:\caseUseShortName}\\
\hline
\rowcolor{tableCaseUseBackground}

\multicolumn{3}{|l|}{\textcolor{tableCaseUseFontColor}{Descripción textual del caso de uso: \caseUseName}} \\ \hline

Fecha de Creación: & \multicolumn{2}{L{\secondColumnWidth}|}{\caseUseCreated}\\ \hline

Fecha de Modificación: & \multicolumn{2}{L{\secondColumnWidth}|}{\caseUseModified} \\ \hline

Versión: & \multicolumn{2}{L{\secondColumnWidth}|}{1} \\ \hline

Resumen: & \multicolumn{2}{L{\secondColumnWidth}|}{\caseUseSummary} \\ \hline

Personas involucradas y metas: & \multicolumn{2}{L{\secondColumnWidth}|}{\caseUsePeople} \\ \hline

Precondiciones: \caseUsePreconditions \hline

Postcondiciones: \caseUsePostconditions \hline

Escenario principal: \caseUseScene \hline

Flujos alternativos: \alternativeCaseUse \hline

Requisitos de interfaz de usuario: \caseUseRequirementsGUI \hline
\multirow{3}{*}{Requisitos funcionales:}  & Tiempo de respuesta: & \caseUseResponseTime \\ \cline{2-3} 
& Concurrencia: & \caseUseConcurrence \\ \cline{2-3} 
& Disponibilidad: & \caseUseAvailability \\ \hline
\end{longtable}

\setcounter{rownumbers}{0}

\renewcommand{\alternativeCaseUse}{
	\caseUseRow{No existen flujos alternativos.}
}

%DIAGRAMA DE ACTIVIDAD
%\lineabreak[0]
%\activityDiagram{AD_\caseUseShortName}{Diagrama de actividad - \caseUseName}

%GestionUbicaciones

\renewcommand{\caseUseShortName}{crearUbicacion} %cammelCase name

\renewcommand{\caseUseCreated}{09/02/2020} %Fecha creación
\renewcommand{\caseUseModified}{09/02/2020} %Fecha modificación
\renewcommand{\caseUseName}{\CUcrearUbicacion - Crear ubicación } %{\CUcammelCase - Title}

\renewcommand{\caseUseSummary}{Este caso de uso permite a un administrador de ZMGestion crear una ubicación} %Resumen
\renewcommand{\caseUsePeople}{Administradores: quiere crear una ubicación.} %Actor: Meta
\renewcommand{\caseUsePreconditions}{
	\caseUseRow{Haber iniciado sesión en el sistema y tener el permiso necesario para realizar esta función.} %Precondiciones
}
\renewcommand{\caseUsePostconditions}{
	\caseUseRow{Ninguna.} %Postcondiciones
}
\renewcommand{\caseUseScene}{ %Escenario principal
    \addCaseUseStep{El administrador accede a la pantalla para crear ubicaciones.}
    \addCaseUseStep{ZMGestion muestra un formulario para que el administrador ingrese el nombre de la ubicación, la calle y el número, el codigo postal y seleccione ciudad, provincia y país. Inidicando que todos los campos son obligatorios.}
    \addCaseUseStep{El administrador completa el formulario.}
    \addCaseUseStep{ZMGestion crea la ubicación y muestra un mensaje indicando el éxito de la operación.}
}
\renewcommand{\alternativeCaseUse}{ %Flujos alternativos
	\newAlternative{A1: El administrador ha dejado un campo obligatorio vacio.}{3} %Flujo alternativo A1.
	\caseUseRow{La secuencia A1 comienza luego del punto 3 del escenario principal.} %¡Indicar número paso!
    \alternativeRow{ZMGestion muestra un mensaje de error indicando que dicho campo es requerido.}
    \caseUseRow{El escenario vuelve al punto 2.}
    \caseUseRow{}

	\newAlternative{A2: La dirección ingresada ya esta siendo utilizada por otra ubicación.}{3} %Flujo alternativo A2.
    \caseUseRow{La secuencia A2 comienza luego del punto 3 del escenario principal.}%¡Indicar número paso!
    \alternativeRow{ZMGestion muestra un mensaje de error indicando que la dirección esta siendo utilizada por otra ubicación.}
    \caseUseRow{El escenario vuelve al punto 2.}
    \caseUseRow{}
}
\renewcommand{\caseUseRequirementsGUI}{
	\caseUseRow{Teclado, Mouse y Pantalla} %Requisitos interfaz de usuario
}
\renewcommand{\caseUseResponseTime}{La interfaz debe responder dentro de un tiempo máximo de 10 segundos.} %Requisitos funcionales: Tiempo de respuesta
\renewcommand{\caseUseConcurrence}{} %Requisitos funcionales: Concurrencia
\renewcommand{\caseUseAvailability}{} %Requisitos funcionales: Disponibilidad

\item Caso de uso \caseUseName
\renewcommand*{\arraystretch}{1.3}
\begin{longtable}[c]{|>{\raggedright}p{0.3\textwidth} | >{\raggedright}p{0.2\textwidth} | p{0.5\textwidth} |}
\caption{\hyperref[sec:listadoCasoUso]{\caseUseName}}
\label{tabla:\caseUseShortName}\\
\hline
\rowcolor{tableCaseUseBackground}

\multicolumn{3}{|l|}{\textcolor{tableCaseUseFontColor}{Descripción textual del caso de uso: \caseUseName}} \\ \hline

Fecha de Creación: & \multicolumn{2}{L{\secondColumnWidth}|}{\caseUseCreated}\\ \hline

Fecha de Modificación: & \multicolumn{2}{L{\secondColumnWidth}|}{\caseUseModified} \\ \hline

Versión: & \multicolumn{2}{L{\secondColumnWidth}|}{1} \\ \hline

Resumen: & \multicolumn{2}{L{\secondColumnWidth}|}{\caseUseSummary} \\ \hline

Personas involucradas y metas: & \multicolumn{2}{L{\secondColumnWidth}|}{\caseUsePeople} \\ \hline

Precondiciones: \caseUsePreconditions \hline

Postcondiciones: \caseUsePostconditions \hline

Escenario principal: \caseUseScene \hline

Flujos alternativos: \alternativeCaseUse \hline

Requisitos de interfaz de usuario: \caseUseRequirementsGUI \hline
\multirow{3}{*}{Requisitos funcionales:}  & Tiempo de respuesta: & \caseUseResponseTime \\ \cline{2-3} 
& Concurrencia: & \caseUseConcurrence \\ \cline{2-3} 
& Disponibilidad: & \caseUseAvailability \\ \hline
\end{longtable}

\setcounter{rownumbers}{0}

\renewcommand{\alternativeCaseUse}{
	\caseUseRow{No existen flujos alternativos.}
}

%DIAGRAMA DE ACTIVIDAD
%\lineabreak[0]
%\activityDiagram{\caseUseShortName}{Diagrama de actividad - \caseUseName}

\renewcommand{\caseUseShortName}{listarUbicaciones} %cammelCase name

\renewcommand{\caseUseCreated}{11/02/2020} %Fecha creación
\renewcommand{\caseUseModified}{11/02/2020} %Fecha modificación
\renewcommand{\caseUseName}{CU40 - Listar ubicaciones } %{\CUcammelCase - Title}

\renewcommand{\caseUseSummary}{Este caso de uso permite a un vendedor de ZMGestion listar todas las ubicaciones existentes en el sistema.} %Resumen
\renewcommand{\caseUsePeople}{Vendedores: quiere listar todas las ubicaciones.} %Actor: Meta
\renewcommand{\caseUsePreconditions}{
	\caseUseRow{Haber iniciado sesión en el sistema y tener el permiso necesario para realizar esta función.} %Precondiciones
}
\renewcommand{\caseUsePostconditions}{
	\caseUseRow{Ninguna.} %Postcondiciones
}
\renewcommand{\caseUseScene}{ %Escenario principal
    \addCaseUseStep{El vendendor accede a la pantalla para ver todas las ubicaciones.}
    \addCaseUseStep{ZMGestion muestra una lista con todas las ubicaciones existentes.}
}
\renewcommand{\alternativeCaseUse}{ %Flujos alternativos
	\newAlternative{A1: No existe ninguna ubicación.}{2} %Flujo alternativo A1.
	\caseUseRow{La secuencia A1 comienza luego del punto 2 del escenario principal.} %¡Indicar número paso!
    \alternativeRow{ZMGestion muestra un mensaje de error indicando que no existe ninguna ubicación.}
    \caseUseRow{}
}

%\item Caso de uso \caseUseName
\renewcommand*{\arraystretch}{1.3}
\begin{longtable}[c]{|>{\raggedright}p{0.3\textwidth} | >{\raggedright}p{0.2\textwidth} | p{0.5\textwidth} |}
\caption{\hyperref[sec:listadoCasoUso]{\caseUseName}}
\label{tabla:\caseUseShortName}\\
\hline
\rowcolor{tableCaseUseBackground}

\multicolumn{3}{|l|}{\textcolor{tableCaseUseFontColor}{Descripción textual del caso de uso: \caseUseName}} \\ \hline

Fecha de Creación: & \multicolumn{2}{L{\secondColumnWidth}|}{\caseUseCreated}\\ \hline

Fecha de Modificación: & \multicolumn{2}{L{\secondColumnWidth}|}{\caseUseModified} \\ \hline

Versión: & \multicolumn{2}{L{\secondColumnWidth}|}{1} \\ \hline

Resumen: & \multicolumn{2}{L{\secondColumnWidth}|}{\caseUseSummary} \\ \hline

Personas involucradas y metas: & \multicolumn{2}{L{\secondColumnWidth}|}{\caseUsePeople} \\ \hline

Precondiciones: \caseUsePreconditions \hline

Postcondiciones: \caseUsePostconditions \hline

Escenario principal: \caseUseScene \hline

Flujos alternativos: \alternativeCaseUse \hline

Requisitos de interfaz de usuario: \caseUseRequirementsGUI \hline
\multirow{3}{*}{Requisitos funcionales:}  & Tiempo de respuesta: & \caseUseResponseTime \\ \cline{2-3} 
& Concurrencia: & \caseUseConcurrence \\ \cline{2-3} 
& Disponibilidad: & \caseUseAvailability \\ \hline
\end{longtable}

\setcounter{rownumbers}{0}

\renewcommand{\alternativeCaseUse}{
	\caseUseRow{No existen flujos alternativos.}
}

%DIAGRAMA DE ACTIVIDAD
%\lineabreak[0]
%\activityDiagram{\caseUseShortName}{Diagrama de actividad - \caseUseName}
\input{Capitulos/Capitulo4/CasosUso/GestionUbicaciones/darBajaUbicacion.tex}

\renewcommand{\caseUseShortName}{darAltaUbicacion} %cammelCase name

\renewcommand{\caseUseCreated}{11/02/2020} %Fecha creación
\renewcommand{\caseUseModified}{11/02/2020} %Fecha modificación
\renewcommand{\caseUseName}{\CUdarAltaUbicacion - Dar de alta ubicación} %{\CUcammelCase - Title}

\renewcommand{\caseUseSummary}{Este caso de uso permite a un administrador de ZMGestion dar de alta a una ubicación que se encuentra en estado de Baja.} %Resumen
\renewcommand{\caseUsePeople}{Administradores: quiere dar de alta una ubicación.} %Actor: Meta
\renewcommand{\caseUsePreconditions}{
	\caseUseRow{Haber realizado con éxito el \CUlistarUbicaciones (Listar ubicaciones).} %Precondiciones
}
\renewcommand{\caseUsePostconditions}{
	\caseUseRow{Ninguna.} %Postcondiciones
}
\renewcommand{\caseUseScene}{ %Escenario principal
    \addCaseUseStep{El administrador indica la ubicación que quiere dar de alta.}
    \addCaseUseStep{ZMGestion cambia el estado de la ubicación a Alta y muestra un mensaje indicando el éxito de la operación.}
}
\renewcommand{\alternativeCaseUse}{ %Flujos alternativos
	\newAlternative{A1: La ubicación ya se encontraba en estado de Alta.}{2} %Flujo alternativo A1.
	\caseUseRow{La secuencia A1 comienza luego del punto 2 del escenario principal.} %¡Indicar número paso!
    \alternativeRow{ZMGestion muestra un mensaje indicando que la ubicación ya se encontraba en estado de Alta.}    
    \caseUseRow{}
}

\item Caso de uso \caseUseName
\renewcommand*{\arraystretch}{1.3}
\begin{longtable}[c]{|>{\raggedright}p{0.3\textwidth} | >{\raggedright}p{0.2\textwidth} | p{0.5\textwidth} |}
\caption{\hyperref[sec:listadoCasoUso]{\caseUseName}}
\label{tabla:\caseUseShortName}\\
\hline
\rowcolor{tableCaseUseBackground}

\multicolumn{3}{|l|}{\textcolor{tableCaseUseFontColor}{Descripción textual del caso de uso: \caseUseName}} \\ \hline

Fecha de Creación: & \multicolumn{2}{L{\secondColumnWidth}|}{\caseUseCreated}\\ \hline

Fecha de Modificación: & \multicolumn{2}{L{\secondColumnWidth}|}{\caseUseModified} \\ \hline

Versión: & \multicolumn{2}{L{\secondColumnWidth}|}{1} \\ \hline

Resumen: & \multicolumn{2}{L{\secondColumnWidth}|}{\caseUseSummary} \\ \hline

Personas involucradas y metas: & \multicolumn{2}{L{\secondColumnWidth}|}{\caseUsePeople} \\ \hline

Precondiciones: \caseUsePreconditions \hline

Postcondiciones: \caseUsePostconditions \hline

Escenario principal: \caseUseScene \hline

Flujos alternativos: \alternativeCaseUse \hline

Requisitos de interfaz de usuario: \caseUseRequirementsGUI \hline
\multirow{3}{*}{Requisitos funcionales:}  & Tiempo de respuesta: & \caseUseResponseTime \\ \cline{2-3} 
& Concurrencia: & \caseUseConcurrence \\ \cline{2-3} 
& Disponibilidad: & \caseUseAvailability \\ \hline
\end{longtable}

\setcounter{rownumbers}{0}

\renewcommand{\alternativeCaseUse}{
	\caseUseRow{No existen flujos alternativos.}
}

%DIAGRAMA DE ACTIVIDAD
%\lineabreak[0]
%\activityDiagram{AD_\caseUseShortName}{Diagrama de actividad - \caseUseName}

\renewcommand{\caseUseShortName}{modificarUbicacion} %cammelCase name

\renewcommand{\caseUseCreated}{11/02/2020} %Fecha creación
\renewcommand{\caseUseModified}{11/02/2020} %Fecha modificación
\renewcommand{\caseUseName}{CU43 - Modificar ubicación} %{\CUcammelCase - Title}

\renewcommand{\caseUseSummary}{Este caso de uso permite a un administrador de ZMGestion modificar una ubicación.} %Resumen
\renewcommand{\caseUsePeople}{Administradores: quiere modificar una ubicación.} %Actor: Meta
\renewcommand{\caseUsePreconditions}{
	\caseUseRow{Haber realizado con éxito el CU40\ (Listar ubicaciones).}%Precondiciones
}
\renewcommand{\caseUsePostconditions}{
	\caseUseRow{Ninguna.} %Postcondiciones
}
\renewcommand{\caseUseScene}{ %Escenario principal
    \addCaseUseStep{El administrador inidca la ubicación que desea modificar.}
    \addCaseUseStep{ZMGestion muestra un formulario autocompletado con los datos de la ubicación seleccionada para que el administrador modifique: nombre de la ubicación, observaciones y dirección (nombre y número de calle, código postal, ciudad, provincia y pais) de la misma. Indicando que todos los campos son obligatorios excepto el de observaciones.}
    \addCaseUseStep{El administrador modifica los datos que desea cambiar de la ubicación.}
    \addCaseUseStep{ZMGestion modifica la ubicación y muestra un mensaje de éxito.}
}
\renewcommand{\alternativeCaseUse}{ %Flujos alternativos
	\newAlternative{A1: El nombre de la ubicación ingresado ya se encuentra en uso.}{3} %Flujo alternativo A1.
	\caseUseRow{La secuencia A1 comienza luego del punto 3 del escenario principal.} %¡Indicar número paso!
    \alternativeRow{ZMGestion muestra un mensaje de error indicando que el nombre ingresado ya se encuentra en uso.}
    \caseUseRow{El escenario vuelve al punto 2.}
    \caseUseRow{}

	\newAlternative{A2: La dirección ingresada ya se encuentra en uso.}{3} %Flujo alternativo A2.
    \caseUseRow{La secuencia A2 comienza luego del punto 3 del escenario principal.}%¡Indicar número paso!
    \alternativeRow{ZMGestion muestra un mensaje de error indicando que la dirección ingresada ya se encuentra en uso.}
    \caseUseRow{El escenario vuelve al punto 2.}
    \caseUseRow{}

    \newAlternative{A3: El administrador ha dejado un campo obligatorio vacío.}{3} %Flujo alternativo A2.
    \caseUseRow{La secuencia A3 comienza luego del punto 3 del escenario principal.}%¡Indicar número paso!
    \alternativeRow{ZMGestion muestra un mensaje de error indicando que dicho campo es requerido.}
    \caseUseRow{El escenario vuelve al punto 2.}
    \caseUseRow{}
}
%\item Caso de uso \caseUseName
\renewcommand*{\arraystretch}{1.3}
\begin{longtable}[c]{|>{\raggedright}p{0.3\textwidth} | >{\raggedright}p{0.2\textwidth} | p{0.5\textwidth} |}
\caption{\hyperref[sec:listadoCasoUso]{\caseUseName}}
\label{tabla:\caseUseShortName}\\
\hline
\rowcolor{tableCaseUseBackground}

\multicolumn{3}{|l|}{\textcolor{tableCaseUseFontColor}{Descripción textual del caso de uso: \caseUseName}} \\ \hline

Fecha de Creación: & \multicolumn{2}{L{\secondColumnWidth}|}{\caseUseCreated}\\ \hline

Fecha de Modificación: & \multicolumn{2}{L{\secondColumnWidth}|}{\caseUseModified} \\ \hline

Versión: & \multicolumn{2}{L{\secondColumnWidth}|}{1} \\ \hline

Resumen: & \multicolumn{2}{L{\secondColumnWidth}|}{\caseUseSummary} \\ \hline

Personas involucradas y metas: & \multicolumn{2}{L{\secondColumnWidth}|}{\caseUsePeople} \\ \hline

Precondiciones: \caseUsePreconditions \hline

Postcondiciones: \caseUsePostconditions \hline

Escenario principal: \caseUseScene \hline

Flujos alternativos: \alternativeCaseUse \hline

Requisitos de interfaz de usuario: \caseUseRequirementsGUI \hline
\multirow{3}{*}{Requisitos funcionales:}  & Tiempo de respuesta: & \caseUseResponseTime \\ \cline{2-3} 
& Concurrencia: & \caseUseConcurrence \\ \cline{2-3} 
& Disponibilidad: & \caseUseAvailability \\ \hline
\end{longtable}

\setcounter{rownumbers}{0}

\renewcommand{\alternativeCaseUse}{
	\caseUseRow{No existen flujos alternativos.}
}

%DIAGRAMA DE ACTIVIDAD
%\lineabreak[0]
%\activityDiagram{\caseUseShortName}{Diagrama de actividad - \caseUseName}

\renewcommand{\caseUseShortName}{borrarUbicacion} %cammelCase name

\renewcommand{\caseUseCreated}{14/02/2020} %Fecha creación
\renewcommand{\caseUseModified}{14/02/2020} %Fecha modificación
\renewcommand{\caseUseName}{CU44 - Borrar ubicación} %{\CUcammelCase - Title}

\renewcommand{\caseUseSummary}{Este caso de uso permite a un administrador de ZMGestion borrar una ubicación. } %Resumen
\renewcommand{\caseUsePeople}{Administradores: quiere borrar una ubicación.}
%Actor: Meta
\renewcommand{\caseUsePreconditions}{
    \caseUseRow{Haber realizado con éxito el CU40 (Listar ubicaciones).}
}
\renewcommand{\caseUsePostconditions}{
	\caseUseRow{Ninguna.} %Postcondiciones
}
\renewcommand{\caseUseScene}{ %Escenario principal
    \addCaseUseStep{El administrador indica la ubicación que desea borrar.}
    \addCaseUseStep{ZMGestion borra la ubicación y muestra un mensaje indicando el éxito de la operación.}
}
\renewcommand{\alternativeCaseUse}{ %Flujos alternativos
	\newAlternative{A1: Existen productos que se encuentran almacenados en dicha ubicación.}{1} %Flujo alternativo A1.
	\caseUseRow{La secuencia A1 comienza luego del punto 1 del escenario principal.} %¡Indicar número paso!
    \alternativeRow{ZMGestion muestra un mensaje de error indicando que existen productos almacenados en dicha ubicación y por lo tanto no puede ser borrada.}
    \caseUseRow{El escenario vuelve al punto 1.}
    \caseUseRow{}
    \newAlternative{A2: Existen usuarios que se encuentran desempeñando sus tareas en dicha ubicación.}{1} %Flujo alternativo A1.
	\caseUseRow{La secuencia A2 comienza luego del punto 1 del escenario principal.} %¡Indicar número paso!
    \alternativeRow{ZMGestion muestra un mensaje de error indicando que existen usuarios en dicha ubicación y por lo tanto no puede ser borrada.}
    \caseUseRow{El escenario vuelve al punto 1.}
    \caseUseRow{}
}



%\item Caso de uso \caseUseName
\renewcommand*{\arraystretch}{1.3}
\begin{longtable}[c]{|>{\raggedright}p{0.3\textwidth} | >{\raggedright}p{0.2\textwidth} | p{0.5\textwidth} |}
\caption{\hyperref[sec:listadoCasoUso]{\caseUseName}}
\label{tabla:\caseUseShortName}\\
\hline
\rowcolor{tableCaseUseBackground}

\multicolumn{3}{|l|}{\textcolor{tableCaseUseFontColor}{Descripción textual del caso de uso: \caseUseName}} \\ \hline

Fecha de Creación: & \multicolumn{2}{L{\secondColumnWidth}|}{\caseUseCreated}\\ \hline

Fecha de Modificación: & \multicolumn{2}{L{\secondColumnWidth}|}{\caseUseModified} \\ \hline

Versión: & \multicolumn{2}{L{\secondColumnWidth}|}{1} \\ \hline

Resumen: & \multicolumn{2}{L{\secondColumnWidth}|}{\caseUseSummary} \\ \hline

Personas involucradas y metas: & \multicolumn{2}{L{\secondColumnWidth}|}{\caseUsePeople} \\ \hline

Precondiciones: \caseUsePreconditions \hline

Postcondiciones: \caseUsePostconditions \hline

Escenario principal: \caseUseScene \hline

Flujos alternativos: \alternativeCaseUse \hline

Requisitos de interfaz de usuario: \caseUseRequirementsGUI \hline
\multirow{3}{*}{Requisitos funcionales:}  & Tiempo de respuesta: & \caseUseResponseTime \\ \cline{2-3} 
& Concurrencia: & \caseUseConcurrence \\ \cline{2-3} 
& Disponibilidad: & \caseUseAvailability \\ \hline
\end{longtable}

\setcounter{rownumbers}{0}

\renewcommand{\alternativeCaseUse}{
	\caseUseRow{No existen flujos alternativos.}
}

%DIAGRAMA DE ACTIVIDAD
%\lineabreak[0]
%\activityDiagram{\caseUseShortName}{Diagrama de actividad - \caseUseName}

%GestionClientes

\renewcommand{\caseUseShortName}{crearCliente} %cammelCase name

\renewcommand{\caseUseCreated}{03/02/2020} %Fecha creación
\renewcommand{\caseUseModified}{03/02/2020} %Fecha modificación
\renewcommand{\caseUseName}{\CUcrearCliente - Crear cliente } %{\CUcammelCase - Title}

\renewcommand{\caseUseSummary}{Este caso de uso permite a un vendedor de ZMGestion crear un cliente.} %Resumen
\renewcommand{\caseUsePeople}{Vendedor: quiere crear un cliente} %Actor: Meta
\renewcommand{\caseUsePreconditions}{
	\caseUseRow{Haber iniciado sesión en el sistema y tener el permiso necesario para realizar esta función.} %Precondiciones
}
\renewcommand{\caseUsePostconditions}{
	\caseUseRow{Ninguna.} %Postcondiciones
}
\renewcommand{\caseUseScene}{ %Escenario principal
    \addCaseUseStep{El vendedor accede a la pantalla para crear clientes.}
    \addCaseUseStep{ZMGestion muestra un formulario para que el vendedor ingrese el tipo de persona (física o jurídica), nombre y apellido (o razón social), tipo y número de documento, correo electrónico, número de telefono y nacionalidad. Indicando que todos los campos son obligatorios exceptuando al de correo electrónico, número de telefono y dependiendo el tipo de persona el nombre y apellido o razon social.}
    \addCaseUseStep{El vendedor completa los campos del formulario.}
    \addCaseUseStep{ZMGestion crea el cliente y muestra un mensaje indicando el éxito de la operación.}
}
\renewcommand{\alternativeCaseUse}{ %Flujos alternativos
	\newAlternative{A1: El correo electrónico ingresado ya se encuentra en uso.}{3} %Flujo alternativo A1.
	\caseUseRow{La secuencia A1 comienza luego del punto 3 del escenario principal.} %¡Indicar número paso!
    \alternativeRow{ZMGestion muestra un mensaje de error indicando que el correo electrónico ingresado ya esta en uso.}
    \caseUseRow{El escenario vuelve al punto 2.}
    \caseUseRow{}

	\newAlternative{A2: El vendedor ha dejado un campo obligatorio vacio.}{3} %Flujo alternativo A2.
    \caseUseRow{La secuencia A2 comienza luego del punto 3 del escenario principal.}%¡Indicar número paso!
    \alternativeRow{ZMGestion muestra un mensaje de error indicando que dicho campo es requerido.}
    \caseUseRow{El escenario vuelve al punto 2.}
    \caseUseRow{}

    \newAlternative{A3:El documento y tipo de docuemnto ingresado ya existe.}{3} %Flujo alternativo A2.
    \caseUseRow{La secuencia A3 comienza luego del punto 3 del escenario principal.}%¡Indicar número paso!
    \alternativeRow{ZMGestion muestra un mensaje de error indicando que el documento y el tipo de documento ya existe.}
    \caseUseRow{El escenario vuelve al punto 2.}
    \caseUseRow{}
}

\item Caso de uso \caseUseName
\renewcommand*{\arraystretch}{1.3}
\begin{longtable}[c]{|>{\raggedright}p{0.3\textwidth} | >{\raggedright}p{0.2\textwidth} | p{0.5\textwidth} |}
\caption{\hyperref[sec:listadoCasoUso]{\caseUseName}}
\label{tabla:\caseUseShortName}\\
\hline
\rowcolor{tableCaseUseBackground}

\multicolumn{3}{|l|}{\textcolor{tableCaseUseFontColor}{Descripción textual del caso de uso: \caseUseName}} \\ \hline

Fecha de Creación: & \multicolumn{2}{L{\secondColumnWidth}|}{\caseUseCreated}\\ \hline

Fecha de Modificación: & \multicolumn{2}{L{\secondColumnWidth}|}{\caseUseModified} \\ \hline

Versión: & \multicolumn{2}{L{\secondColumnWidth}|}{1} \\ \hline

Resumen: & \multicolumn{2}{L{\secondColumnWidth}|}{\caseUseSummary} \\ \hline

Personas involucradas y metas: & \multicolumn{2}{L{\secondColumnWidth}|}{\caseUsePeople} \\ \hline

Precondiciones: \caseUsePreconditions \hline

Postcondiciones: \caseUsePostconditions \hline

Escenario principal: \caseUseScene \hline

Flujos alternativos: \alternativeCaseUse \hline

Requisitos de interfaz de usuario: \caseUseRequirementsGUI \hline
\multirow{3}{*}{Requisitos funcionales:}  & Tiempo de respuesta: & \caseUseResponseTime \\ \cline{2-3} 
& Concurrencia: & \caseUseConcurrence \\ \cline{2-3} 
& Disponibilidad: & \caseUseAvailability \\ \hline
\end{longtable}

\setcounter{rownumbers}{0}

\renewcommand{\alternativeCaseUse}{
	\caseUseRow{No existen flujos alternativos.}
}

%DIAGRAMA DE ACTIVIDAD
%\lineabreak[0]
%\activityDiagram{AD_\caseUseShortName}{Diagrama de actividad - \caseUseName}

\renewcommand{\caseUseShortName}{buscarAvanzadoClientes} %cammelCase name

\renewcommand{\caseUseCreated}{03/02/2020} %Fecha creación
\renewcommand{\caseUseModified}{03/02/2020} %Fecha modificación
\renewcommand{\caseUseName}{\CUbuscarAvanzadoClientes - Buscar avanzado clientes} %{\CUcammelCase - Title}

\renewcommand{\caseUseSummary}{Este caso de uso permite a un vendendor de ZMGestion buscar cliente a partir de una cadena de busqueda.} %Resumen
\renewcommand{\caseUsePeople}{Vendedor: quiere encontrar un cliente existente en el sistema.} %Actor: Meta
\renewcommand{\caseUsePreconditions}{
	\caseUseRow{Haber iniciado sesión en el sistema y tener el permiso necesario para realizar esta función..} %Precondiciones
}
\renewcommand{\caseUsePostconditions}{
	\caseUseRow{Ninguna.} %Postcondiciones
}
\renewcommand{\caseUseScene}{ %Escenario principal
    \addCaseUseStep{El vendedor accede a la pantalla para realizar la búsqueda de clientes.}
    \addCaseUseStep{ZMGestion muestra un formulario para que el vendedor ingrese la cadena de búsqueda, seleccione el tipo de persona y si desea buscar clientes dados de baja.}
    \addCaseUseStep{El vendedor completa los campos seleccionados.}
    \addCaseUseStep{ZMGestion lista las coincidencias encontradas.}
}
\renewcommand{\alternativeCaseUse}{ %Flujos alternativos
	\newAlternative{A1: No se encontró ninguna coincidencia.}{3} %Flujo alternativo A1.
	\caseUseRow{La secuencia A1 comienza luego del punto 3 del escenario principal.} %¡Indicar número paso!
    \alternativeRow{ZMGestion muestra un mensaje de error indicando que no se encontró ningun resultado para su búsqueda.}
    \caseUseRow{El escenario vuelve al punto 2.}
    \caseUseRow{}
}
\item Caso de uso \caseUseName
\renewcommand*{\arraystretch}{1.3}
\begin{longtable}[c]{|>{\raggedright}p{0.3\textwidth} | >{\raggedright}p{0.2\textwidth} | p{0.5\textwidth} |}
\caption{\hyperref[sec:listadoCasoUso]{\caseUseName}}
\label{tabla:\caseUseShortName}\\
\hline
\rowcolor{tableCaseUseBackground}

\multicolumn{3}{|l|}{\textcolor{tableCaseUseFontColor}{Descripción textual del caso de uso: \caseUseName}} \\ \hline

Fecha de Creación: & \multicolumn{2}{L{\secondColumnWidth}|}{\caseUseCreated}\\ \hline

Fecha de Modificación: & \multicolumn{2}{L{\secondColumnWidth}|}{\caseUseModified} \\ \hline

Versión: & \multicolumn{2}{L{\secondColumnWidth}|}{1} \\ \hline

Resumen: & \multicolumn{2}{L{\secondColumnWidth}|}{\caseUseSummary} \\ \hline

Personas involucradas y metas: & \multicolumn{2}{L{\secondColumnWidth}|}{\caseUsePeople} \\ \hline

Precondiciones: \caseUsePreconditions \hline

Postcondiciones: \caseUsePostconditions \hline

Escenario principal: \caseUseScene \hline

Flujos alternativos: \alternativeCaseUse \hline

Requisitos de interfaz de usuario: \caseUseRequirementsGUI \hline
\multirow{3}{*}{Requisitos funcionales:}  & Tiempo de respuesta: & \caseUseResponseTime \\ \cline{2-3} 
& Concurrencia: & \caseUseConcurrence \\ \cline{2-3} 
& Disponibilidad: & \caseUseAvailability \\ \hline
\end{longtable}

\setcounter{rownumbers}{0}

\renewcommand{\alternativeCaseUse}{
	\caseUseRow{No existen flujos alternativos.}
}

%DIAGRAMA DE ACTIVIDAD
%\lineabreak[0]
\activityDiagram{\caseUseShortName}{Diagrama de actividad - \caseUseName}

\renewcommand{\caseUseShortName}{darBajaCliente} %cammelCase name

\renewcommand{\caseUseCreated}{03/02/2020} %Fecha creación
\renewcommand{\caseUseModified}{03/02/2020} %Fecha modificación
\renewcommand{\caseUseName}{\CUdarBajaCliente - Dar de baja cliente } %{\CUcammelCase - Title}

\renewcommand{\caseUseSummary}{Este caso de uso permite a un administrador de ZMGestion dar de baja un cliente que se encuentra en estado de Alta.} %Resumen
\renewcommand{\caseUsePeople}{Administrador: quiere dar de baja un cliente.} %Actor: Meta
\renewcommand{\caseUsePreconditions}{
	\caseUseRow{Haber realizado con éxito el \CUbuscarAvanzadoClientes (Buscar avanzado clientes).} %Precondiciones
}
\renewcommand{\caseUsePostconditions}{
	\caseUseRow{Ninguna.} %Postcondiciones
}
\renewcommand{\caseUseScene}{ %Escenario principal
    \addCaseUseStep{El administrador indica el cliente que quiere dar de baja.}
    \addCaseUseStep{ZMGestion cambia el estado del cliente a Baja y muestra un mensaje indicando el éxito de la operación.}
}
\renewcommand{\alternativeCaseUse}{ %Flujos alternativos
	\newAlternative{A1: El cliente ya se encontraba en estado de Baja.}{1} %Flujo alternativo A1.
	\caseUseRow{La secuencia A1 comienza luego del punto 1 del escenario principal.} %¡Indicar número paso!
    \alternativeRow{ZMGestion muestra un mensaje de error indicando que el cliente ya se encontraba en estado de Baja.}
    \caseUseRow{El escenario vuelve al punto 1.}
    \caseUseRow{}
}

\item Caso de uso \caseUseName
\renewcommand*{\arraystretch}{1.3}
\begin{longtable}[c]{|>{\raggedright}p{0.3\textwidth} | >{\raggedright}p{0.2\textwidth} | p{0.5\textwidth} |}
\caption{\hyperref[sec:listadoCasoUso]{\caseUseName}}
\label{tabla:\caseUseShortName}\\
\hline
\rowcolor{tableCaseUseBackground}

\multicolumn{3}{|l|}{\textcolor{tableCaseUseFontColor}{Descripción textual del caso de uso: \caseUseName}} \\ \hline

Fecha de Creación: & \multicolumn{2}{L{\secondColumnWidth}|}{\caseUseCreated}\\ \hline

Fecha de Modificación: & \multicolumn{2}{L{\secondColumnWidth}|}{\caseUseModified} \\ \hline

Versión: & \multicolumn{2}{L{\secondColumnWidth}|}{1} \\ \hline

Resumen: & \multicolumn{2}{L{\secondColumnWidth}|}{\caseUseSummary} \\ \hline

Personas involucradas y metas: & \multicolumn{2}{L{\secondColumnWidth}|}{\caseUsePeople} \\ \hline

Precondiciones: \caseUsePreconditions \hline

Postcondiciones: \caseUsePostconditions \hline

Escenario principal: \caseUseScene \hline

Flujos alternativos: \alternativeCaseUse \hline

Requisitos de interfaz de usuario: \caseUseRequirementsGUI \hline
\multirow{3}{*}{Requisitos funcionales:}  & Tiempo de respuesta: & \caseUseResponseTime \\ \cline{2-3} 
& Concurrencia: & \caseUseConcurrence \\ \cline{2-3} 
& Disponibilidad: & \caseUseAvailability \\ \hline
\end{longtable}

\setcounter{rownumbers}{0}

\renewcommand{\alternativeCaseUse}{
	\caseUseRow{No existen flujos alternativos.}
}

%DIAGRAMA DE ACTIVIDAD
%\lineabreak[0]
%\activityDiagram{AD_\caseUseShortName}{Diagrama de actividad - \caseUseName}

\renewcommand{\caseUseShortName}{darAltaCliente} %cammelCase name

\renewcommand{\caseUseCreated}{03/02/2020} %Fecha creación
\renewcommand{\caseUseModified}{03/02/2020} %Fecha modificación
\renewcommand{\caseUseName}{\CUdarAltaCliente - Dar de alta cliente} %{\CUcammelCase - Title}

\renewcommand{\caseUseSummary}{Este caso de uso npermite a un administrador de ZMGestion dar de alta un cliente que se encuentra en estado de Baja.} %Resumen
\renewcommand{\caseUsePeople}{Administrador: quiere dar de alta un cliente.} %Actor: Meta
\renewcommand{\caseUsePreconditions}{
	\caseUseRow{Haber realizado con éxito el \CUbuscarAvanzadoClientes (Buscar avanzado clientes).} %Precondiciones
}
\renewcommand{\caseUsePostconditions}{
	\caseUseRow{Ninguna.} %Postcondiciones
}
\renewcommand{\caseUseScene}{ %Escenario principal
    \addCaseUseStep{EL administrador indica el cliente que quiere dar de alta.}
    \addCaseUseStep{ZMGestion cambia el estado del cliente a Alta y muestra un mensaje indicando el éxito de la operación.}
}
\renewcommand{\alternativeCaseUse}{ %Flujos alternativos
	\newAlternative{A1: El cliente seleccionado ya se encontraba en estado de Alta.}{1} %Flujo alternativo A1.
	\caseUseRow{La secuencia A1 comienza luego del punto 1 del escenario principal.} %¡Indicar número paso!
    \alternativeRow{ZMGestion muestra un mensaje de error indicando que el cliente ya se econtraba en estado de Alta.}
    \caseUseRow{El escenario vuelve al punto 1.}
    \caseUseRow{}
}

\item Caso de uso \caseUseName
\renewcommand*{\arraystretch}{1.3}
\begin{longtable}[c]{|>{\raggedright}p{0.3\textwidth} | >{\raggedright}p{0.2\textwidth} | p{0.5\textwidth} |}
\caption{\hyperref[sec:listadoCasoUso]{\caseUseName}}
\label{tabla:\caseUseShortName}\\
\hline
\rowcolor{tableCaseUseBackground}

\multicolumn{3}{|l|}{\textcolor{tableCaseUseFontColor}{Descripción textual del caso de uso: \caseUseName}} \\ \hline

Fecha de Creación: & \multicolumn{2}{L{\secondColumnWidth}|}{\caseUseCreated}\\ \hline

Fecha de Modificación: & \multicolumn{2}{L{\secondColumnWidth}|}{\caseUseModified} \\ \hline

Versión: & \multicolumn{2}{L{\secondColumnWidth}|}{1} \\ \hline

Resumen: & \multicolumn{2}{L{\secondColumnWidth}|}{\caseUseSummary} \\ \hline

Personas involucradas y metas: & \multicolumn{2}{L{\secondColumnWidth}|}{\caseUsePeople} \\ \hline

Precondiciones: \caseUsePreconditions \hline

Postcondiciones: \caseUsePostconditions \hline

Escenario principal: \caseUseScene \hline

Flujos alternativos: \alternativeCaseUse \hline

Requisitos de interfaz de usuario: \caseUseRequirementsGUI \hline
\multirow{3}{*}{Requisitos funcionales:}  & Tiempo de respuesta: & \caseUseResponseTime \\ \cline{2-3} 
& Concurrencia: & \caseUseConcurrence \\ \cline{2-3} 
& Disponibilidad: & \caseUseAvailability \\ \hline
\end{longtable}

\setcounter{rownumbers}{0}

\renewcommand{\alternativeCaseUse}{
	\caseUseRow{No existen flujos alternativos.}
}

%DIAGRAMA DE ACTIVIDAD
%\lineabreak[0]
%\activityDiagram{\caseUseShortName}{Diagrama de actividad - \caseUseName}

\renewcommand{\caseUseShortName}{modificarCliente} %cammelCase name

\renewcommand{\caseUseCreated}{03/02/2020} %Fecha creación
\renewcommand{\caseUseModified}{03/02/2020} %Fecha modificación
\renewcommand{\caseUseName}{CU59 - Modificar cliente} %{\CUcammelCase - Title}

\renewcommand{\caseUseSummary}{Este caso de uso permite a un vendedor de ZMGestion modificar un cliente existente.} %Resumen
\renewcommand{\caseUsePeople}{Vendedor: quiere modificar un cliente.} %Actor: Meta
\renewcommand{\caseUsePreconditions}{
	\caseUseRow{Haber realizado con éxito el CU56 (Buscar avanzado clientes).} %Precondiciones
}
\renewcommand{\caseUsePostconditions}{
	\caseUseRow{Ninguna.} %Postcondiciones
}
\renewcommand{\caseUseScene}{ %Escenario principal
    \addCaseUseStep{El vendedor selecciona el cliente que desea modificar.}
    \addCaseUseStep{ZMGestion muestra un formulario autocompletado con los datos del cliente seleccioando para que el vendedor modifique: el tipo de persona, nombre y apellido (o razón social), número y tipo de documento, correo electrónico, número de teléfono y nacionalidad. }
    \addCaseUseStep{El vendedor modifica los campos que desea cambiar.}
    \addCaseUseStep{ZMGestion modifica el cliente y muestra un mensaje indicando el éxito de la operación.}
}
\renewcommand{\alternativeCaseUse}{ %Flujos alternativos
	\newAlternative{A1: El correo electrónico ingresado ya está en uso.}{3} %Flujo alternativo A1.
	\caseUseRow{La secuencia A1 comienza luego del punto 3 del escenario principal.} %¡Indicar número paso!
    \alternativeRow{ZMGestion muestra un mensaje de error indicando el que correo electrónico ya se encuentra en uso.}
    \caseUseRow{El escenario vuele al punto 2.}
    \caseUseRow{}

	\newAlternative{A2: El tipo y número de documento ya existe.}{3} %Flujo alternativo A2.
    \caseUseRow{La secuencia A2 comienza luego del punto 3 del escenario principal.}%¡Indicar número paso!
    \alternativeRow{ZMGestion muestra un mensaje de error indicando que el tipo y número de documento ya existe.}
    \caseUseRow{El escenario vuele al punto 2.}
    \caseUseRow{}

    \newAlternative{A3: El vendedor ha dejado un campo obligatorio vacío.}{3} %Flujo alternativo A2.
    \caseUseRow{La secuencia A3 comienza luego del punto 3 del escenario principal.}%¡Indicar número paso!
    \alternativeRow{ZMGestion muestra un mensaje de error indicando que dicho campo es requerido.}
    \caseUseRow{El escenario vuelve al punto 2.}
    \caseUseRow{}
}
\renewcommand{\caseUseRequirementsGUI}{
	\caseUseRow{Teclado, Mouse y Pantalla} %Requisitos interfaz de usuario
}
\renewcommand{\caseUseResponseTime}{La interfaz debe responder dentro de un tiempo máximo de 10 segundos.} %Requisitos funcionales: Tiempo de respuesta
\renewcommand{\caseUseConcurrence}{} %Requisitos funcionales: Concurrencia
\renewcommand{\caseUseAvailability}{} %Requisitos funcionales: Disponibilidad

%\item Caso de uso \caseUseName
\renewcommand*{\arraystretch}{1.3}
\begin{longtable}[c]{|>{\raggedright}p{0.3\textwidth} | >{\raggedright}p{0.2\textwidth} | p{0.5\textwidth} |}
\caption{\hyperref[sec:listadoCasoUso]{\caseUseName}}
\label{tabla:\caseUseShortName}\\
\hline
\rowcolor{tableCaseUseBackground}

\multicolumn{3}{|l|}{\textcolor{tableCaseUseFontColor}{Descripción textual del caso de uso: \caseUseName}} \\ \hline

Fecha de Creación: & \multicolumn{2}{L{\secondColumnWidth}|}{\caseUseCreated}\\ \hline

Fecha de Modificación: & \multicolumn{2}{L{\secondColumnWidth}|}{\caseUseModified} \\ \hline

Versión: & \multicolumn{2}{L{\secondColumnWidth}|}{1} \\ \hline

Resumen: & \multicolumn{2}{L{\secondColumnWidth}|}{\caseUseSummary} \\ \hline

Personas involucradas y metas: & \multicolumn{2}{L{\secondColumnWidth}|}{\caseUsePeople} \\ \hline

Precondiciones: \caseUsePreconditions \hline

Postcondiciones: \caseUsePostconditions \hline

Escenario principal: \caseUseScene \hline

Flujos alternativos: \alternativeCaseUse \hline

Requisitos de interfaz de usuario: \caseUseRequirementsGUI \hline
\multirow{3}{*}{Requisitos funcionales:}  & Tiempo de respuesta: & \caseUseResponseTime \\ \cline{2-3} 
& Concurrencia: & \caseUseConcurrence \\ \cline{2-3} 
& Disponibilidad: & \caseUseAvailability \\ \hline
\end{longtable}

\setcounter{rownumbers}{0}

\renewcommand{\alternativeCaseUse}{
	\caseUseRow{No existen flujos alternativos.}
}

%DIAGRAMA DE ACTIVIDAD
%\lineabreak[0]
%\activityDiagram{\caseUseShortName}{Diagrama de actividad - \caseUseName}
\renewcommand{\caseUseShortName}{borrarCliente} %cammelCase name

\renewcommand{\caseUseCreated}{03/02/2020} %Fecha creación
\renewcommand{\caseUseModified}{03/02/2020} %Fecha modificación
\renewcommand{\caseUseName}{\CUborrarCliente - Borrar cliente} %{\CUcammelCase - Title}

\renewcommand{\caseUseSummary}{Este caso de uso permite a un vendedor de ZMGestio borrar un cliente existente.} %Resumen
\renewcommand{\caseUsePeople}{Vendedor: quiere borrar un cliente del sistema.} %Actor: Meta
\renewcommand{\caseUsePreconditions}{
	\caseUseRow{Haber realizado con éxito el \CUbuscarAvanzadoClientes (Buscar avanzado clientes).} %Precondiciones
}
\renewcommand{\caseUsePostconditions}{
	\caseUseRow{Ninguna.} %Postcondiciones
}
\renewcommand{\caseUseScene}{ %Escenario principal
    \addCaseUseStep{El vendedor indica el cliente que desea borrar.}
    \addCaseUseStep{ZMGestion borra el cliente y muestra un mensaje indicando el éxito de la operación.}
}
\renewcommand{\alternativeCaseUse}{ %Flujos alternativos
	\newAlternative{A1: El cliente indicado tiene un presupuesto asociado.}{1} %Flujo alternativo A1.
	\caseUseRow{La secuencia A1 comienza luego del punto 1 del escenario principal.} %¡Indicar número paso!
    \alternativeRow{ZMGestion muestra un mensaje de error indicando que el cliente tiene un presupuesto asociado y por lo tanto no puede ser borrado.}
    \caseUseRow{El escenario vuelve al punto 1.}
    \caseUseRow{}

	\newAlternative{A1: El cliente indicado tiene una venta asociada.}{1} %Flujo alternativo A1.
	\caseUseRow{La secuencia A1 comienza luego del punto 1 del escenario principal.} %¡Indicar número paso!
    \alternativeRow{ZMGestion muestra un mensaje de error indicando que el cliente tiene una venta asociado y por lo tanto no puede ser borrado.}
    \caseUseRow{El escenario vuelve al punto 1.}
    \caseUseRow{}
}

\item Caso de uso \caseUseName
\renewcommand*{\arraystretch}{1.3}
\begin{longtable}[c]{|>{\raggedright}p{0.3\textwidth} | >{\raggedright}p{0.2\textwidth} | p{0.5\textwidth} |}
\caption{\hyperref[sec:listadoCasoUso]{\caseUseName}}
\label{tabla:\caseUseShortName}\\
\hline
\rowcolor{tableCaseUseBackground}

\multicolumn{3}{|l|}{\textcolor{tableCaseUseFontColor}{Descripción textual del caso de uso: \caseUseName}} \\ \hline

Fecha de Creación: & \multicolumn{2}{L{\secondColumnWidth}|}{\caseUseCreated}\\ \hline

Fecha de Modificación: & \multicolumn{2}{L{\secondColumnWidth}|}{\caseUseModified} \\ \hline

Versión: & \multicolumn{2}{L{\secondColumnWidth}|}{1} \\ \hline

Resumen: & \multicolumn{2}{L{\secondColumnWidth}|}{\caseUseSummary} \\ \hline

Personas involucradas y metas: & \multicolumn{2}{L{\secondColumnWidth}|}{\caseUsePeople} \\ \hline

Precondiciones: \caseUsePreconditions \hline

Postcondiciones: \caseUsePostconditions \hline

Escenario principal: \caseUseScene \hline

Flujos alternativos: \alternativeCaseUse \hline

Requisitos de interfaz de usuario: \caseUseRequirementsGUI \hline
\multirow{3}{*}{Requisitos funcionales:}  & Tiempo de respuesta: & \caseUseResponseTime \\ \cline{2-3} 
& Concurrencia: & \caseUseConcurrence \\ \cline{2-3} 
& Disponibilidad: & \caseUseAvailability \\ \hline
\end{longtable}

\setcounter{rownumbers}{0}

\renewcommand{\alternativeCaseUse}{
	\caseUseRow{No existen flujos alternativos.}
}

%DIAGRAMA DE ACTIVIDAD
%\lineabreak[0]
%\activityDiagram{\caseUseShortName}{Diagrama de actividad - \caseUseName}
\renewcommand{\caseUseShortName}{listarDomicilios} %cammelCase name

\renewcommand{\caseUseCreated}{20/03/2020} %Fecha creación
\renewcommand{\caseUseModified}{20/03/2020} %Fecha modificación
\renewcommand{\caseUseName}{\CUlistarDomicilios - Listar domicilios} %{\CUcammelCase - Title}

\renewcommand{\caseUseSummary}{Este caso de uso permite a un vendedor de ZMGestion listar los domicilios de un cliente.} %Resumen
\renewcommand{\caseUsePeople}{Vendedores: quiere listar los domicilios de un cliente existente.} %Actor: Meta
\renewcommand{\caseUsePreconditions}{
	\caseUseRow{Haber realizado con éxito el \CUbuscarAvanzadoClientes\ (Buscar avanzado clientes).} %Precondiciones
}
\renewcommand{\caseUsePostconditions}{
	\caseUseRow{Ninguna.} %Postcondiciones
}
\renewcommand{\caseUseScene}{ %Escenario principal
    \addCaseUseStep{El vendedor indica el cliente al cual le desea listar sus domicilios.}
    \addCaseUseStep{ZMGestion lista los domicilios existentes del cliente seleccionado.}
}
\renewcommand{\alternativeCaseUse}{ %Flujos alternativos
	\newAlternative{A1: El cliente no posee domicilios.}{1} %Flujo alternativo A1.
	\caseUseRow{La secuencia A1 comienza luego del punto 1 del escenario principal.} %¡Indicar número paso!
    \alternativeRow{ZMGestion muestra un mensaje indicando que el cliente seleccionado no posee ningun domicilio.}
    \caseUseRow{El escenario vuelve al punto 1.}
    \caseUseRow{}
}
\item Caso de uso \caseUseName
\renewcommand*{\arraystretch}{1.3}
\begin{longtable}[c]{|>{\raggedright}p{0.3\textwidth} | >{\raggedright}p{0.2\textwidth} | p{0.5\textwidth} |}
\caption{\hyperref[sec:listadoCasoUso]{\caseUseName}}
\label{tabla:\caseUseShortName}\\
\hline
\rowcolor{tableCaseUseBackground}

\multicolumn{3}{|l|}{\textcolor{tableCaseUseFontColor}{Descripción textual del caso de uso: \caseUseName}} \\ \hline

Fecha de Creación: & \multicolumn{2}{L{\secondColumnWidth}|}{\caseUseCreated}\\ \hline

Fecha de Modificación: & \multicolumn{2}{L{\secondColumnWidth}|}{\caseUseModified} \\ \hline

Versión: & \multicolumn{2}{L{\secondColumnWidth}|}{1} \\ \hline

Resumen: & \multicolumn{2}{L{\secondColumnWidth}|}{\caseUseSummary} \\ \hline

Personas involucradas y metas: & \multicolumn{2}{L{\secondColumnWidth}|}{\caseUsePeople} \\ \hline

Precondiciones: \caseUsePreconditions \hline

Postcondiciones: \caseUsePostconditions \hline

Escenario principal: \caseUseScene \hline

Flujos alternativos: \alternativeCaseUse \hline

Requisitos de interfaz de usuario: \caseUseRequirementsGUI \hline
\multirow{3}{*}{Requisitos funcionales:}  & Tiempo de respuesta: & \caseUseResponseTime \\ \cline{2-3} 
& Concurrencia: & \caseUseConcurrence \\ \cline{2-3} 
& Disponibilidad: & \caseUseAvailability \\ \hline
\end{longtable}

\setcounter{rownumbers}{0}

\renewcommand{\alternativeCaseUse}{
	\caseUseRow{No existen flujos alternativos.}
}

%DIAGRAMA DE ACTIVIDAD
%\lineabreak[0]
%\activityDiagram{\caseUseShortName}{Diagrama de actividad - \caseUseName}

%GestionDomicilios
\renewcommand{\caseUseShortName}{crearDomicilio} %cammelCase name

\renewcommand{\caseUseCreated}{20/03/2020} %Fecha creación
\renewcommand{\caseUseModified}{20/03/2020} %Fecha modificación
\renewcommand{\caseUseName}{\CUcrearDomicilio - Crear domicilio} %{\CUcammelCase - Title}

\renewcommand{\caseUseSummary}{Este caso de uso permite a un vendedor de ZMGestion crear un domicilio para un cliente existente.} %Resumen
\renewcommand{\caseUsePeople}{Vendedores: quiere crear un domicilio para un cliente existente.} %Actor: Meta
\renewcommand{\caseUsePreconditions}{
	\caseUseRow{Haber realizado con éxito el \CUbuscarAvanzadoClientes\ (Buscar avanzado clientes).} %Precondiciones
}
\renewcommand{\caseUsePostconditions}{
	\caseUseRow{Ninguna.} %Postcondiciones
}
\renewcommand{\caseUseScene}{ %Escenario principal
    \addCaseUseStep{El vendedor selecciona un cliente al cual le desea crear un domicilio.}
    \addCaseUseStep{ZMGestion muestra un formulario para que el vendedor ingrese calle, número de calle, el código postal y seleccione ciudad, provincia y país. Ademas se le muestra un campo opcional para ingresar observaciones, siendo el resto de los campos obligatorios.}
    \addCaseUseStep{El vendedor ingresa una calle, numero de calle, código postal, observaciones, ciudad, provincia y país.}
    \addCaseUseStep{ZMGestion crea el domicilio para el cliente seleccionado por el vendedor.}
}
\renewcommand{\alternativeCaseUse}{ %Flujos alternativos
    \newAlternative{A1: El vendedor ha dejado un campo obligatorio vacio.}{3} %Flujo alternativo A1.
    \caseUseRow{La secuencia A1 comienza luego del punto 3 del escenario principal.}%¡Indicar número paso!
    \alternativeRow{ZMGestion muestra un mensaje de error indicando que dicho campo es requerido.}
    \caseUseRow{El escenario vuelve al punto 2.}
    \caseUseRow{}
}

\item Caso de uso \caseUseName
\renewcommand*{\arraystretch}{1.3}
\begin{longtable}[c]{|>{\raggedright}p{0.3\textwidth} | >{\raggedright}p{0.2\textwidth} | p{0.5\textwidth} |}
\caption{\hyperref[sec:listadoCasoUso]{\caseUseName}}
\label{tabla:\caseUseShortName}\\
\hline
\rowcolor{tableCaseUseBackground}

\multicolumn{3}{|l|}{\textcolor{tableCaseUseFontColor}{Descripción textual del caso de uso: \caseUseName}} \\ \hline

Fecha de Creación: & \multicolumn{2}{L{\secondColumnWidth}|}{\caseUseCreated}\\ \hline

Fecha de Modificación: & \multicolumn{2}{L{\secondColumnWidth}|}{\caseUseModified} \\ \hline

Versión: & \multicolumn{2}{L{\secondColumnWidth}|}{1} \\ \hline

Resumen: & \multicolumn{2}{L{\secondColumnWidth}|}{\caseUseSummary} \\ \hline

Personas involucradas y metas: & \multicolumn{2}{L{\secondColumnWidth}|}{\caseUsePeople} \\ \hline

Precondiciones: \caseUsePreconditions \hline

Postcondiciones: \caseUsePostconditions \hline

Escenario principal: \caseUseScene \hline

Flujos alternativos: \alternativeCaseUse \hline

Requisitos de interfaz de usuario: \caseUseRequirementsGUI \hline
\multirow{3}{*}{Requisitos funcionales:}  & Tiempo de respuesta: & \caseUseResponseTime \\ \cline{2-3} 
& Concurrencia: & \caseUseConcurrence \\ \cline{2-3} 
& Disponibilidad: & \caseUseAvailability \\ \hline
\end{longtable}

\setcounter{rownumbers}{0}

\renewcommand{\alternativeCaseUse}{
	\caseUseRow{No existen flujos alternativos.}
}

%DIAGRAMA DE ACTIVIDAD
%\lineabreak[0]
%\activityDiagram{\caseUseShortName}{Diagrama de actividad - \caseUseName}
\renewcommand{\caseUseShortName}{borrarDomicilio} %cammelCase name

\renewcommand{\caseUseCreated}{20/03/2020} %Fecha creación
\renewcommand{\caseUseModified}{20/03/2020} %Fecha modificación
\renewcommand{\caseUseName}{\CUborrarDomicilio - Borrar domicilio} %{\CUcammelCase - Title}

\renewcommand{\caseUseSummary}{Este caso de uso permite a un vendedores de ZMGestion borrar un domicilio de un cliente existente.} %Resumen
\renewcommand{\caseUsePeople}{Vendedores: quiere borrar un domicilio de un cliente existente.} %Actor: Meta
\renewcommand{\caseUsePreconditions}{
	\caseUseRow{Haber realizado con éxito el \CUlistarDomicilios\ (Listar domicilios) para un cliente.} %Precondiciones
}
\renewcommand{\caseUsePostconditions}{
	\caseUseRow{Ninguna.} %Postcondiciones
}
\renewcommand{\caseUseScene}{ %Escenario principal
    \addCaseUseStep{El vendedor selecciona un domicilio que desea borrar.}
    \addCaseUseStep{ZMGestion borra el domicilio del cliente.}
}
\renewcommand{\alternativeCaseUse}{ %Flujos alternativos
    \newAlternative{A1: El domicilio indicado ha sido utilizado en un remito.}{3} %Flujo alternativo A1.
    \caseUseRow{La secuencia A1 comienza luego del punto 1 del escenario principal.}%¡Indicar número paso!
    \alternativeRow{ZMGestion muestra un mensaje de error indicando que no se puede borrar el domicilio.}
    \caseUseRow{El escenario vuelve al punto 1.}
    \caseUseRow{}
    \newAlternative{A2: El domicilio indicado ha sido utilizado en una venta.}{3} %Flujo alternativo A2.
    \caseUseRow{La secuencia A2 comienza luego del punto 1 del escenario principal.}%¡Indicar número paso!
    \alternativeRow{ZMGestion muestra un mensaje de error indicando que no se puede borrar el domicilio.}
    \caseUseRow{El escenario vuelve al punto 1.}
    \caseUseRow{}
}

\item Caso de uso \caseUseName
\renewcommand*{\arraystretch}{1.3}
\begin{longtable}[c]{|>{\raggedright}p{0.3\textwidth} | >{\raggedright}p{0.2\textwidth} | p{0.5\textwidth} |}
\caption{\hyperref[sec:listadoCasoUso]{\caseUseName}}
\label{tabla:\caseUseShortName}\\
\hline
\rowcolor{tableCaseUseBackground}

\multicolumn{3}{|l|}{\textcolor{tableCaseUseFontColor}{Descripción textual del caso de uso: \caseUseName}} \\ \hline

Fecha de Creación: & \multicolumn{2}{L{\secondColumnWidth}|}{\caseUseCreated}\\ \hline

Fecha de Modificación: & \multicolumn{2}{L{\secondColumnWidth}|}{\caseUseModified} \\ \hline

Versión: & \multicolumn{2}{L{\secondColumnWidth}|}{1} \\ \hline

Resumen: & \multicolumn{2}{L{\secondColumnWidth}|}{\caseUseSummary} \\ \hline

Personas involucradas y metas: & \multicolumn{2}{L{\secondColumnWidth}|}{\caseUsePeople} \\ \hline

Precondiciones: \caseUsePreconditions \hline

Postcondiciones: \caseUsePostconditions \hline

Escenario principal: \caseUseScene \hline

Flujos alternativos: \alternativeCaseUse \hline

Requisitos de interfaz de usuario: \caseUseRequirementsGUI \hline
\multirow{3}{*}{Requisitos funcionales:}  & Tiempo de respuesta: & \caseUseResponseTime \\ \cline{2-3} 
& Concurrencia: & \caseUseConcurrence \\ \cline{2-3} 
& Disponibilidad: & \caseUseAvailability \\ \hline
\end{longtable}

\setcounter{rownumbers}{0}

\renewcommand{\alternativeCaseUse}{
	\caseUseRow{No existen flujos alternativos.}
}

%DIAGRAMA DE ACTIVIDAD
%\lineabreak[0]
%\activityDiagram{\caseUseShortName}{Diagrama de actividad - \caseUseName}

%GestionPresupuestos

\renewcommand{\caseUseShortName}{crearPresupuesto} %cammelCase name

\renewcommand{\caseUseCreated}{04/02/2020} %Fecha creación
\renewcommand{\caseUseModified}{04/02/2020} %Fecha modificación
\renewcommand{\caseUseName}{\CUcrearPresupuesto - Crear presupuesto} %{\CUcammelCase - Title}

\renewcommand{\caseUseSummary}{Este caso de uso permite a un vendedor de ZMGestion crear un presupuesto para un determinado cliente.} %Resumen
\renewcommand{\caseUsePeople}{Vendedores: quiere crear un presupuesto.} %Actor: Meta
\renewcommand{\caseUsePreconditions}{
	\caseUseRow{Haber iniciado sesión en el sistemay tener el permiso necesario para realizar esta función.} %Precondiciones
}
\renewcommand{\caseUsePostconditions}{
	\caseUseRow{Ninguna.} %Postcondiciones
}
\renewcommand{\caseUseScene}{ %Escenario principal
    \addCaseUseStep{El vendedor accede a la pantalla para crear presupuestos.}
    \addCaseUseStep{ZMGestion le muestra un formulario para que el vendedor seleccione un cliente. Además se muestra un campo autocompletado con el periodo de validez. Si el vendedor cuenta con los permisos necesarios para modificar el periodo de validez el campo se muestra habilitado, caso contrario el campo se le muestra deshabilitado.}
    \addCaseUseStep{El vendedor selecciona un cliente y modifica el periodo de validez.}%3
    \addCaseUseStep{ZMGestion crea un presupuesto en estado de "En creación" para el cliente con el periodo de validez ingresado por el vendedor.}%4
    \addCaseUseStep{ZMGestion pregunta si desea agregar una nueva linea de presupuesto.}%5
    \addCaseUseStep{En caso afirmativo se ejecuta el \CUcrearLineaPresupuesto (Crear linea de presupuesto) y se vuelve a realizar la pregunta. En caso contrario ZMGestión pasa el estado del presupuesto a "Creado".}
    \addCaseUseStep{ZMGestión muestra un mensaje indicando el éxito de la operacion y vuelve al punto 5 del escenario principal.
    }
}
\renewcommand{\alternativeCaseUse}{ %Flujos alternativos
	\newAlternative{A1: No ha seleccionado ningún cliente.}{3} %Flujo alternativo A1.
	\caseUseRow{La secuencia A1 comienza luego del punto 3 del escenario principal.} %¡Indicar número paso!
    \alternativeRow{ZMGestion muestra un mensaje de error indicando que debe seleccionar un cliente.}
    \caseUseRow{El escenario vuelve al punto 2.}
    \caseUseRow{}
	\newAlternative{A2: No ha agregado ninguna linea de presupuesto.}{6} %Flujo alternativo A2.
    \caseUseRow{La secuencia A2 comienza luego del punto 6 del escenario principal.}%¡Indicar número paso!
    \alternativeRow{ZMGestion muestra un mensaje de error indicando que debe agregar al menos una linea de presupuesto.}
    \caseUseRow{El escenario vuelve al punto 5.}
    \caseUseRow{}
}

\item Caso de uso \caseUseName
\renewcommand*{\arraystretch}{1.3}
\begin{longtable}[c]{|>{\raggedright}p{0.3\textwidth} | >{\raggedright}p{0.2\textwidth} | p{0.5\textwidth} |}
\caption{\hyperref[sec:listadoCasoUso]{\caseUseName}}
\label{tabla:\caseUseShortName}\\
\hline
\rowcolor{tableCaseUseBackground}

\multicolumn{3}{|l|}{\textcolor{tableCaseUseFontColor}{Descripción textual del caso de uso: \caseUseName}} \\ \hline

Fecha de Creación: & \multicolumn{2}{L{\secondColumnWidth}|}{\caseUseCreated}\\ \hline

Fecha de Modificación: & \multicolumn{2}{L{\secondColumnWidth}|}{\caseUseModified} \\ \hline

Versión: & \multicolumn{2}{L{\secondColumnWidth}|}{1} \\ \hline

Resumen: & \multicolumn{2}{L{\secondColumnWidth}|}{\caseUseSummary} \\ \hline

Personas involucradas y metas: & \multicolumn{2}{L{\secondColumnWidth}|}{\caseUsePeople} \\ \hline

Precondiciones: \caseUsePreconditions \hline

Postcondiciones: \caseUsePostconditions \hline

Escenario principal: \caseUseScene \hline

Flujos alternativos: \alternativeCaseUse \hline

Requisitos de interfaz de usuario: \caseUseRequirementsGUI \hline
\multirow{3}{*}{Requisitos funcionales:}  & Tiempo de respuesta: & \caseUseResponseTime \\ \cline{2-3} 
& Concurrencia: & \caseUseConcurrence \\ \cline{2-3} 
& Disponibilidad: & \caseUseAvailability \\ \hline
\end{longtable}

\setcounter{rownumbers}{0}

\renewcommand{\alternativeCaseUse}{
	\caseUseRow{No existen flujos alternativos.}
}

%DIAGRAMA DE ACTIVIDAD
%\lineabreak[0]
%\activityDiagram{AD_\caseUseShortName}{Diagrama de actividad - \caseUseName}
\input{Capitulos/Capitulo4/CasosUso/GestionPresupuestos/buscarAvanzadoPresupuestos.tex}

\renewcommand{\caseUseShortName}{modificarPresupuesto} %cammelCase name

\renewcommand{\caseUseCreated}{04/02/2020} %Fecha creación
\renewcommand{\caseUseModified}{04/02/2020} %Fecha modificación
\renewcommand{\caseUseName}{\CUmodificarPresupuesto - Modificar presupuesto} %{\CUcammelCase - Title}

\renewcommand{\caseUseSummary}{Este caso de uso permite a un vendedor modificar un presupuesto existente.} %Resumen
\renewcommand{\caseUsePeople}{Vendedores: quiere modificar un presupuesto existente.} %Actor: Meta
\renewcommand{\caseUsePreconditions}{
	\caseUseRow{Haber realizado con éxito el \CUbuscarAvanzadoPresupuestos\ (Buscar avanzado presupuestos).} %Precondiciones
}
\renewcommand{\caseUsePostconditions}{
	\caseUseRow{Ninguna.} %Postcondiciones
}
\renewcommand{\caseUseScene}{ %Escenario principal
    \addCaseUseStep{El vendedor indica el presupuesto que desea modificar.}
    \addCaseUseStep{ZMGestion muestra un formulario autocompletado para que el usuario modifique el cliente, observaciones el periodo de validez. Si el vendedor cuenta con los permisos necesarios podrá modificar el periodo de validez.}
    \addCaseUseStep{El vendedor modifica el cliente, observaciones y periodo de validez.}
    \addCaseUseStep{Se ejecuta el \CUlistarLineasPresupuesto\ (Listar lineas de presupuesto) para el presupuesto seleccionado.}
    \addCaseUseStep{Si el vendedor desea modificar una linea, ejecuta el \CUmodificarLineaPresupuesto\ (Modificar linea de presupuesto), si desea borrar una linea ejecuta el \CUborrarLineaPresupuesto\ (Borrar linea de presupuesto) y si desea agregar una nueva linea ejecuta el \CUcrearLineaPresupuesto\ (Crear linea de presupuesto).}
    \addCaseUseStep{ZMGestion modifica el presupuesto seleccionado y muestra un mensaje indicando el éxito de la operación.}
}
\renewcommand{\alternativeCaseUse}{ %Flujos alternativos
	\newAlternative{A1: El presupuesto se encuentra en estado `Vendido'.}{1} %Flujo alternativo A1.
	\caseUseRow{La secuencia A1 comienza luego del punto 1 del escenario principal.} %¡Indicar número paso!
    \alternativeRow{ZMGestion muestra un mensaje indicando que no se puede modificar el presupuesto ya que se encuentra en estado "Vendido".}
    \caseUseRow{El escenario vuelve al punto 1.}
    \caseUseRow{}
    \newAlternative{A2: El vendedor modificó el periodo de validez y no cuenta con los permisos necesarios para hacerlo.}{3} %Flujo alternativo A2.
	\caseUseRow{La secuencia A2 comienza luego del punto 3 del escenario principal.} %¡Indicar número paso!
    \alternativeRow{ZMGestion muestra un mensaje indicando que no cuenta con los permisos necesarios para modificar el periodo de validez.}
    \caseUseRow{El escenario vuelve al punto 4.}
    \caseUseRow{}
    \newAlternative{A3: El vendedor ha dejado el campo de cliente o periodo de validez vacío.}{3} %Flujo alternativo A2.
	\caseUseRow{La secuencia A3 comienza luego del punto 3 del escenario principal.} %¡Indicar número paso!
    \alternativeRow{ZMGestion muestra un mensaje indicando que los campos son requeridos.}
    \caseUseRow{El escenario vuelve al punto 2.}
    \caseUseRow{}
    \newAlternative{A4: El vendedor que creó el presupuesto no es el mismo que el que quiere modificarlo y no es un administrador.}{1} %Flujo alternativo A4.
	\caseUseRow{La secuencia A4 comienza luego del punto 1 del escenario principal.} %¡Indicar número paso!
    \alternativeRow{ZMGestion muestra un mensaje informando que no puede modificar el presupuesto de otro vendedor.}
    \caseUseRow{El escenario vuelve al punto 1.}
    \caseUseRow{}
}
\item Caso de uso \caseUseName
\renewcommand*{\arraystretch}{1.3}
\begin{longtable}[c]{|>{\raggedright}p{0.3\textwidth} | >{\raggedright}p{0.2\textwidth} | p{0.5\textwidth} |}
\caption{\hyperref[sec:listadoCasoUso]{\caseUseName}}
\label{tabla:\caseUseShortName}\\
\hline
\rowcolor{tableCaseUseBackground}

\multicolumn{3}{|l|}{\textcolor{tableCaseUseFontColor}{Descripción textual del caso de uso: \caseUseName}} \\ \hline

Fecha de Creación: & \multicolumn{2}{L{\secondColumnWidth}|}{\caseUseCreated}\\ \hline

Fecha de Modificación: & \multicolumn{2}{L{\secondColumnWidth}|}{\caseUseModified} \\ \hline

Versión: & \multicolumn{2}{L{\secondColumnWidth}|}{1} \\ \hline

Resumen: & \multicolumn{2}{L{\secondColumnWidth}|}{\caseUseSummary} \\ \hline

Personas involucradas y metas: & \multicolumn{2}{L{\secondColumnWidth}|}{\caseUsePeople} \\ \hline

Precondiciones: \caseUsePreconditions \hline

Postcondiciones: \caseUsePostconditions \hline

Escenario principal: \caseUseScene \hline

Flujos alternativos: \alternativeCaseUse \hline

Requisitos de interfaz de usuario: \caseUseRequirementsGUI \hline
\multirow{3}{*}{Requisitos funcionales:}  & Tiempo de respuesta: & \caseUseResponseTime \\ \cline{2-3} 
& Concurrencia: & \caseUseConcurrence \\ \cline{2-3} 
& Disponibilidad: & \caseUseAvailability \\ \hline
\end{longtable}

\setcounter{rownumbers}{0}

\renewcommand{\alternativeCaseUse}{
	\caseUseRow{No existen flujos alternativos.}
}

%DIAGRAMA DE ACTIVIDAD
%\lineabreak[0]
\activityDiagram{\caseUseShortName}{Diagrama de actividad - \caseUseName}

\renewcommand{\caseUseShortName}{borrarPresupuesto} %cammelCase name

\renewcommand{\caseUseCreated}{04/02/2020} %Fecha creación
\renewcommand{\caseUseModified}{04/02/2020} %Fecha modificación
\renewcommand{\caseUseName}{\CUborrarPresupuesto - Borrar presupuesto} %{\CUcammelCase - Title}

\renewcommand{\caseUseSummary}{Este caso de uso permite a un administrador borrar un presupuesto.} %Resumen
\renewcommand{\caseUsePeople}{Administradores: quiere borrar un presupuesto.} %Actor: Meta
\renewcommand{\caseUsePreconditions}{
	\caseUseRow{Haber realizado con éxito el \CUbuscarAvanzadoPresupuestos\ (Buscar avanzado presupuestos).} %Precondiciones
}
\renewcommand{\caseUsePostconditions}{
	\caseUseRow{Ninguna.} %Postcondiciones
}
\renewcommand{\caseUseScene}{ %Escenario principal
    \addCaseUseStep{El administrador indica el presupuesto que desea borrar.}
    \addCaseUseStep{ZMGestion borra el presupuesto y muestra un mensaje indicando el éxito de la operación.}
}
\renewcommand{\alternativeCaseUse}{ %Flujos alternativos
    \newAlternative{A1: El presupuesto indicado se encuentra en estado "Vendido".}{1} %Flujo alternativo A1.
    \caseUseRow{La secuencia A1 comienza luego del punto 1 del escenario principal.} %¡Indicar número paso!
    \alternativeRow{ZMGestion muestra un mensaje de error informando que el presupuesto no puede borrarse.}
    \caseUseRow{El escenario vuelve al punto 1.}
    \caseUseRow{}
}
\item Caso de uso \caseUseName
\renewcommand*{\arraystretch}{1.3}
\begin{longtable}[c]{|>{\raggedright}p{0.3\textwidth} | >{\raggedright}p{0.2\textwidth} | p{0.5\textwidth} |}
\caption{\hyperref[sec:listadoCasoUso]{\caseUseName}}
\label{tabla:\caseUseShortName}\\
\hline
\rowcolor{tableCaseUseBackground}

\multicolumn{3}{|l|}{\textcolor{tableCaseUseFontColor}{Descripción textual del caso de uso: \caseUseName}} \\ \hline

Fecha de Creación: & \multicolumn{2}{L{\secondColumnWidth}|}{\caseUseCreated}\\ \hline

Fecha de Modificación: & \multicolumn{2}{L{\secondColumnWidth}|}{\caseUseModified} \\ \hline

Versión: & \multicolumn{2}{L{\secondColumnWidth}|}{1} \\ \hline

Resumen: & \multicolumn{2}{L{\secondColumnWidth}|}{\caseUseSummary} \\ \hline

Personas involucradas y metas: & \multicolumn{2}{L{\secondColumnWidth}|}{\caseUsePeople} \\ \hline

Precondiciones: \caseUsePreconditions \hline

Postcondiciones: \caseUsePostconditions \hline

Escenario principal: \caseUseScene \hline

Flujos alternativos: \alternativeCaseUse \hline

Requisitos de interfaz de usuario: \caseUseRequirementsGUI \hline
\multirow{3}{*}{Requisitos funcionales:}  & Tiempo de respuesta: & \caseUseResponseTime \\ \cline{2-3} 
& Concurrencia: & \caseUseConcurrence \\ \cline{2-3} 
& Disponibilidad: & \caseUseAvailability \\ \hline
\end{longtable}

\setcounter{rownumbers}{0}

\renewcommand{\alternativeCaseUse}{
	\caseUseRow{No existen flujos alternativos.}
}

%DIAGRAMA DE ACTIVIDAD
\activityDiagram{\caseUseShortName}{Diagrama de actividad - \caseUseName}

\renewcommand{\caseUseShortName}{generarPresupuestoPDF} %cammelCase name

\renewcommand{\caseUseCreated}{04/02/2020} %Fecha creación
\renewcommand{\caseUseModified}{04/02/2020} %Fecha modificación
\renewcommand{\caseUseName}{\CUgenerarPresupuestoPDF\ - Generar presupuesto en formato PDF.} %{\CUcammelCase - Title}

\renewcommand{\caseUseSummary}{Este caso de uso permite a un vendedor generar un presupuesto en formato PDF.} %Resumen
\renewcommand{\caseUsePeople}{Vendedores: quiere generar el presupuesto en formato PDF.} %Actor: Meta
\renewcommand{\caseUsePreconditions}{
	\caseUseRow{Haber realizado con éxito el \CUbuscarAvanzadoPresupuestos\ (Buscar avanzado presupuestos).} %Precondiciones
}
\renewcommand{\caseUsePostconditions}{
	\caseUseRow{Ninguna.} %Postcondiciones
}
\renewcommand{\caseUseScene}{ %Escenario principal
    \addCaseUseStep{El administrador indica el presupuesto que desea generar en formato PDF.}
    \addCaseUseStep{ZMGestion genera un archivo PDF con los datos del presupuesto.}
}
\renewcommand{\alternativeCaseUse}{ %Flujos alternativos
    \caseUseRow{Ningún flujo alternativo.}
}
\item Caso de uso \caseUseName
\renewcommand*{\arraystretch}{1.3}
\begin{longtable}[c]{|>{\raggedright}p{0.3\textwidth} | >{\raggedright}p{0.2\textwidth} | p{0.5\textwidth} |}
\caption{\hyperref[sec:listadoCasoUso]{\caseUseName}}
\label{tabla:\caseUseShortName}\\
\hline
\rowcolor{tableCaseUseBackground}

\multicolumn{3}{|l|}{\textcolor{tableCaseUseFontColor}{Descripción textual del caso de uso: \caseUseName}} \\ \hline

Fecha de Creación: & \multicolumn{2}{L{\secondColumnWidth}|}{\caseUseCreated}\\ \hline

Fecha de Modificación: & \multicolumn{2}{L{\secondColumnWidth}|}{\caseUseModified} \\ \hline

Versión: & \multicolumn{2}{L{\secondColumnWidth}|}{1} \\ \hline

Resumen: & \multicolumn{2}{L{\secondColumnWidth}|}{\caseUseSummary} \\ \hline

Personas involucradas y metas: & \multicolumn{2}{L{\secondColumnWidth}|}{\caseUsePeople} \\ \hline

Precondiciones: \caseUsePreconditions \hline

Postcondiciones: \caseUsePostconditions \hline

Escenario principal: \caseUseScene \hline

Flujos alternativos: \alternativeCaseUse \hline

Requisitos de interfaz de usuario: \caseUseRequirementsGUI \hline
\multirow{3}{*}{Requisitos funcionales:}  & Tiempo de respuesta: & \caseUseResponseTime \\ \cline{2-3} 
& Concurrencia: & \caseUseConcurrence \\ \cline{2-3} 
& Disponibilidad: & \caseUseAvailability \\ \hline
\end{longtable}

\setcounter{rownumbers}{0}

\renewcommand{\alternativeCaseUse}{
	\caseUseRow{No existen flujos alternativos.}
}

%DIAGRAMA DE ACTIVIDAD
%\lineabreak[0]
\activityDiagram{\caseUseShortName}{Diagrama de actividad - \caseUseName}

\renewcommand{\caseUseShortName}{enviarPresupuestoEmail} %cammelCase name

\renewcommand{\caseUseCreated}{04/02/2020} %Fecha creación
\renewcommand{\caseUseModified}{04/02/2020} %Fecha modificación
\renewcommand{\caseUseName}{\CUenviarPresupuestoEmail - Enviar presupuesto por correo electrónico} %{\CUcammelCase - Title}

\renewcommand{\caseUseSummary}{Este caso de uso permite a un vendedor enviar un presupuesto por correo electrónico.} %Resumen
\renewcommand{\caseUsePeople}{Vendedores: quiere enviar un presupuesto al correo electrónico del cliente al que se le realizó el presupuesto.} %Actor: Meta
\renewcommand{\caseUsePreconditions}{
	\caseUseRow{Haber realizado con éxito el \CUbuscarAvanzadoPresupuestos\ (Buscar avanzado presupuestos).} %Precondiciones
}
\renewcommand{\caseUsePostconditions}{
	\caseUseRow{Ninguna.} %Postcondiciones
}
\renewcommand{\caseUseScene}{ %Escenario principal
    \addCaseUseStep{El administrador indica el presupuesto que desea enviar.}
    \addCaseUseStep{ZMGestion ejecuta el \CUgenerarPresupuestoPDF\ (Generar presupuesto en formato PDF).}
    \addCaseUseStep{ZMGestion envia el presupuesto en formato al correo electrónico del cliente al que se le realizó el presupuesto.}
}
\renewcommand{\alternativeCaseUse}{ %Flujos alternativos
	\newAlternative{A1: El cliente al que se le realizó el presupuesto no tiene un correo electrónico.}{1} %Flujo alternativo A1.
	\caseUseRow{La secuencia A1 comienza luego del punto 1 del escenario principal.} %¡Indicar número paso!
    \alternativeRow{ZMGestion muestra un mensaje de error informando que el presupuesto no puede enviarse ya que el cliente no tiene un correo electrónico.}
    \caseUseRow{El escenario vuelve al punto 1.}
    \caseUseRow{}
}
\item Caso de uso \caseUseName
\renewcommand*{\arraystretch}{1.3}
\begin{longtable}[c]{|>{\raggedright}p{0.3\textwidth} | >{\raggedright}p{0.2\textwidth} | p{0.5\textwidth} |}
\caption{\hyperref[sec:listadoCasoUso]{\caseUseName}}
\label{tabla:\caseUseShortName}\\
\hline
\rowcolor{tableCaseUseBackground}

\multicolumn{3}{|l|}{\textcolor{tableCaseUseFontColor}{Descripción textual del caso de uso: \caseUseName}} \\ \hline

Fecha de Creación: & \multicolumn{2}{L{\secondColumnWidth}|}{\caseUseCreated}\\ \hline

Fecha de Modificación: & \multicolumn{2}{L{\secondColumnWidth}|}{\caseUseModified} \\ \hline

Versión: & \multicolumn{2}{L{\secondColumnWidth}|}{1} \\ \hline

Resumen: & \multicolumn{2}{L{\secondColumnWidth}|}{\caseUseSummary} \\ \hline

Personas involucradas y metas: & \multicolumn{2}{L{\secondColumnWidth}|}{\caseUsePeople} \\ \hline

Precondiciones: \caseUsePreconditions \hline

Postcondiciones: \caseUsePostconditions \hline

Escenario principal: \caseUseScene \hline

Flujos alternativos: \alternativeCaseUse \hline

Requisitos de interfaz de usuario: \caseUseRequirementsGUI \hline
\multirow{3}{*}{Requisitos funcionales:}  & Tiempo de respuesta: & \caseUseResponseTime \\ \cline{2-3} 
& Concurrencia: & \caseUseConcurrence \\ \cline{2-3} 
& Disponibilidad: & \caseUseAvailability \\ \hline
\end{longtable}

\setcounter{rownumbers}{0}

\renewcommand{\alternativeCaseUse}{
	\caseUseRow{No existen flujos alternativos.}
}

%DIAGRAMA DE ACTIVIDAD
%\lineabreak[0]
%\activityDiagram{AD_\caseUseShortName}{Diagrama de actividad - \caseUseName}

\renewcommand{\caseUseShortName}{listarLineasPresupuesto} %cammelCase name

\renewcommand{\caseUseCreated}{04/02/2020} %Fecha creación
\renewcommand{\caseUseModified}{04/02/2020} %Fecha modificación
\renewcommand{\caseUseName}{CU71 - Listar lineas de presupuesto} %{\CUcammelCase - Title}

\renewcommand{\caseUseSummary}{Este caso de uso permite a un vendedor de ZMGestion listar las lineas de presupuesto de un presupuesto determinado.} %Resumen
\renewcommand{\caseUsePeople}{Vendedores: quiere listar las lineas de presupuesto de un presupuesto existente.} %Actor: Meta
\renewcommand{\caseUsePreconditions}{
	\caseUseRow{Haber realizado con éxito el CU65 (Buscar avanzado presupuestos).} %Precondiciones
}
\renewcommand{\caseUsePostconditions}{
	\caseUseRow{Ninguna.} %Postcondiciones
}
\renewcommand{\caseUseScene}{ %Escenario principal
    \addCaseUseStep{El vendedor indica el presupuesto del cual desea listar sus lineas de presupuesto.}
    \addCaseUseStep{ZMGestion lista las lineas de presupuesto existentes del presupuesto seleccionado.}
}
\renewcommand{\alternativeCaseUse}{ %Flujos alternativos
	\newAlternative{A1: El presupuesto no posee lineas de presupuesto.}{1} %Flujo alternativo A1.
	\caseUseRow{La secuencia A1 comienza luego del punto 1 del escenario principal.} %¡Indicar número paso!
    \alternativeRow{ZMGestion muestra un mensaje indicando que el presupuesto seleccionado no posee ninguna linea de presupuesto.}
    \caseUseRow{El escenario vuelve al punto 1.}
    \caseUseRow{}
}
%\item Caso de uso \caseUseName
\renewcommand*{\arraystretch}{1.3}
\begin{longtable}[c]{|>{\raggedright}p{0.3\textwidth} | >{\raggedright}p{0.2\textwidth} | p{0.5\textwidth} |}
\caption{\hyperref[sec:listadoCasoUso]{\caseUseName}}
\label{tabla:\caseUseShortName}\\
\hline
\rowcolor{tableCaseUseBackground}

\multicolumn{3}{|l|}{\textcolor{tableCaseUseFontColor}{Descripción textual del caso de uso: \caseUseName}} \\ \hline

Fecha de Creación: & \multicolumn{2}{L{\secondColumnWidth}|}{\caseUseCreated}\\ \hline

Fecha de Modificación: & \multicolumn{2}{L{\secondColumnWidth}|}{\caseUseModified} \\ \hline

Versión: & \multicolumn{2}{L{\secondColumnWidth}|}{1} \\ \hline

Resumen: & \multicolumn{2}{L{\secondColumnWidth}|}{\caseUseSummary} \\ \hline

Personas involucradas y metas: & \multicolumn{2}{L{\secondColumnWidth}|}{\caseUsePeople} \\ \hline

Precondiciones: \caseUsePreconditions \hline

Postcondiciones: \caseUsePostconditions \hline

Escenario principal: \caseUseScene \hline

Flujos alternativos: \alternativeCaseUse \hline

Requisitos de interfaz de usuario: \caseUseRequirementsGUI \hline
\multirow{3}{*}{Requisitos funcionales:}  & Tiempo de respuesta: & \caseUseResponseTime \\ \cline{2-3} 
& Concurrencia: & \caseUseConcurrence \\ \cline{2-3} 
& Disponibilidad: & \caseUseAvailability \\ \hline
\end{longtable}

\setcounter{rownumbers}{0}

\renewcommand{\alternativeCaseUse}{
	\caseUseRow{No existen flujos alternativos.}
}

%DIAGRAMA DE ACTIVIDAD
%\lineabreak[0]
%\activityDiagram{\caseUseShortName}{Diagrama de actividad - \caseUseName}

%GestionLineasPresupuesto

\renewcommand{\caseUseShortName}{crearLineaPresupuesto} %cammelCase name

\renewcommand{\caseUseCreated}{04/02/2020} %Fecha creación
\renewcommand{\caseUseModified}{04/02/2020} %Fecha modificación
\renewcommand{\caseUseName}{\CUcrearLineaPresupuesto\ - Crear linea de presupuesto} %{\CUcammelCase - Title}

\renewcommand{\caseUseSummary}{Este caso de uso permite a un vendedor crear una linea de presupuesto para un presupuesto determinado.} %Resumen
\renewcommand{\caseUsePeople}{Vendedores: quiere crear una linea de presupuesto.} %Actor: Meta
\renewcommand{\caseUsePreconditions}{
	\caseUseRow{Haber ejecutado con éxito el \CUbuscarAvanzadoPresupuestos (Buscar avanzado presupuestos)} %Precondiciones
}
\renewcommand{\caseUsePostconditions}{
	\caseUseRow{Si el producto, tela y lustre no pertenece a ningún producto final existente se ejecuta el \CUcrearProductoFinal\ (Crear producto final) con el producto, tela y lustre seleccionados.} %Postcondiciones
}
\renewcommand{\caseUseScene}{ %Escenario principal
    \addCaseUseStep{El vendedor desea agregar una linea de presupuesto a un presupuesto.}
    \addCaseUseStep{ZMGestion le muestra un formulario para que el vendedor seleccione un producto, tela, lustre, precio unitario e indique la cantidad solicitada de dicho producto. En caso de no contar con los permisos necesarios para asignar precios el campo de precio unitario se muestra deshabilitado.}
    \addCaseUseStep{El vendedor selecciona producto, tela, lustre y la cantidad solicitada.}
    \addCaseUseStep{ZMGestion autocompleta el campo precio unitario con el precio actual del producto seleccionado con la tela solicitada. Si el vendedor cuenta con los permisos necesario para modificar este campo se le permite la edición, caso contrario se le muestra el campo deshabilitado con el precio.}
    \addCaseUseStep{El vendedor cuenta con los permisos necesarios y modifica el precio.}
    \addCaseUseStep{ZMGestion crea la linea de presupuesto y la asocia al presupuesto seleccionado.}
}
\renewcommand{\alternativeCaseUse}{ %Flujos alternativos
	\newAlternative{A1: La cantidad solicitada es menor o igual a cero.}{3} %Flujo alternativo A1.
	\caseUseRow{La secuencia A1 comienza luego del punto 3 del escenario principal.} %¡Indicar número paso!
    \alternativeRow{ZMGestion muestra un mensaje de error indicando que debe ingresar una cantidad mayor a cero.}
    \caseUseRow{El escenario vuelve al punto 2.}
    \caseUseRow{}

    \newAlternative{A2: El vendedor no cuenta con los permisos necesarios para modificar el precio.}{5} %Flujo alternativo A3.
    \caseUseRow{La secuencia A2 comienza luego del punto 5 del escenario principal.}%¡Indicar número paso!
    \alternativeRow{ZMGestion muestra un mensaje de error indicando que no cuenta con los permisos necesarios para modificar el precio unitario.}
    \caseUseRow{El escenario vuelve al punto 4.}
}

\item Caso de uso \caseUseName
\renewcommand*{\arraystretch}{1.3}
\begin{longtable}[c]{|>{\raggedright}p{0.3\textwidth} | >{\raggedright}p{0.2\textwidth} | p{0.5\textwidth} |}
\caption{\hyperref[sec:listadoCasoUso]{\caseUseName}}
\label{tabla:\caseUseShortName}\\
\hline
\rowcolor{tableCaseUseBackground}

\multicolumn{3}{|l|}{\textcolor{tableCaseUseFontColor}{Descripción textual del caso de uso: \caseUseName}} \\ \hline

Fecha de Creación: & \multicolumn{2}{L{\secondColumnWidth}|}{\caseUseCreated}\\ \hline

Fecha de Modificación: & \multicolumn{2}{L{\secondColumnWidth}|}{\caseUseModified} \\ \hline

Versión: & \multicolumn{2}{L{\secondColumnWidth}|}{1} \\ \hline

Resumen: & \multicolumn{2}{L{\secondColumnWidth}|}{\caseUseSummary} \\ \hline

Personas involucradas y metas: & \multicolumn{2}{L{\secondColumnWidth}|}{\caseUsePeople} \\ \hline

Precondiciones: \caseUsePreconditions \hline

Postcondiciones: \caseUsePostconditions \hline

Escenario principal: \caseUseScene \hline

Flujos alternativos: \alternativeCaseUse \hline

Requisitos de interfaz de usuario: \caseUseRequirementsGUI \hline
\multirow{3}{*}{Requisitos funcionales:}  & Tiempo de respuesta: & \caseUseResponseTime \\ \cline{2-3} 
& Concurrencia: & \caseUseConcurrence \\ \cline{2-3} 
& Disponibilidad: & \caseUseAvailability \\ \hline
\end{longtable}

\setcounter{rownumbers}{0}

\renewcommand{\alternativeCaseUse}{
	\caseUseRow{No existen flujos alternativos.}
}

%DIAGRAMA DE ACTIVIDAD
%\lineabreak[0]
%\activityDiagram{AD_\caseUseShortName}{Diagrama de actividad - \caseUseName}
\input{Capitulos/Capitulo4/CasosUso/GestionLineasPresupuesto/modificarLineaPresupuesto.tex}

\renewcommand{\caseUseShortName}{borrarLineaPresupuesto} %cammelCase name

\renewcommand{\caseUseCreated}{04/02/2020} %Fecha creación
\renewcommand{\caseUseModified}{04/02/2020} %Fecha modificación
\renewcommand{\caseUseName}{\CUborrarLineaPresupuesto\ - Borrar linea de presupuesto} %{\CUcammelCase - Title}

\renewcommand{\caseUseSummary}{Este caso de uso permite a un vendedor borrar una linea de presupuesto.} %Resumen
\renewcommand{\caseUsePeople}{Vendedores: quiere borrar una linea de presupuesto.} %Actor: Meta
\renewcommand{\caseUsePreconditions}{
	\caseUseRow{Haber ejecutado con éxito el \CUlistarLineasPresupuesto (Listar lineas de presupuesto)} %Precondiciones
}
\renewcommand{\caseUsePostconditions}{
	\caseUseRow{Ninguna.} %Postcondiciones
}
\renewcommand{\caseUseScene}{ %Escenario principal
    \addCaseUseStep{El vendedor indica la linea de presupuesto que desea borrar.}
    \addCaseUseStep{ZMGestion borra la linea de presupuesto.}
}
\renewcommand{\alternativeCaseUse}{ %Flujos alternativos

    %VER QUE A1 TAMBIEN SE PUEDE HACER REVISANDO SI LA LINEA ESTA UTILIZADA O NO

	\newAlternative{A1: El presupuesto al cual pertenece la linea se encuentra en estado `Vendido'.}{1} %Flujo alternativo A1.
	\caseUseRow{La secuencia A1 comienza luego del punto 1 del escenario principal.} %¡Indicar número paso!
    \alternativeRow{ZMGestion muestra un mensaje de error indicando que no se puede borrar la linea de presupuesto.}
    \caseUseRow{El escenario vuelve al punto 1.}
    \newAlternative{A2: El vendedor que creó el presupuesto no es el mismo que el que quiere borrar la linea de presupuesto y no es un administrador.}{1} %Flujo alternativo A2.
	\caseUseRow{La secuencia A2 comienza luego del punto 1 del escenario principal.} 
    \alternativeRow{ZMGestion muestra un mensaje de error indicando que puede borrar el presupuesto de otro vendedor.}
    \caseUseRow{El escenario vuelve al punto 1.}
}

\item Caso de uso \caseUseName
\renewcommand*{\arraystretch}{1.3}
\begin{longtable}[c]{|>{\raggedright}p{0.3\textwidth} | >{\raggedright}p{0.2\textwidth} | p{0.5\textwidth} |}
\caption{\hyperref[sec:listadoCasoUso]{\caseUseName}}
\label{tabla:\caseUseShortName}\\
\hline
\rowcolor{tableCaseUseBackground}

\multicolumn{3}{|l|}{\textcolor{tableCaseUseFontColor}{Descripción textual del caso de uso: \caseUseName}} \\ \hline

Fecha de Creación: & \multicolumn{2}{L{\secondColumnWidth}|}{\caseUseCreated}\\ \hline

Fecha de Modificación: & \multicolumn{2}{L{\secondColumnWidth}|}{\caseUseModified} \\ \hline

Versión: & \multicolumn{2}{L{\secondColumnWidth}|}{1} \\ \hline

Resumen: & \multicolumn{2}{L{\secondColumnWidth}|}{\caseUseSummary} \\ \hline

Personas involucradas y metas: & \multicolumn{2}{L{\secondColumnWidth}|}{\caseUsePeople} \\ \hline

Precondiciones: \caseUsePreconditions \hline

Postcondiciones: \caseUsePostconditions \hline

Escenario principal: \caseUseScene \hline

Flujos alternativos: \alternativeCaseUse \hline

Requisitos de interfaz de usuario: \caseUseRequirementsGUI \hline
\multirow{3}{*}{Requisitos funcionales:}  & Tiempo de respuesta: & \caseUseResponseTime \\ \cline{2-3} 
& Concurrencia: & \caseUseConcurrence \\ \cline{2-3} 
& Disponibilidad: & \caseUseAvailability \\ \hline
\end{longtable}

\setcounter{rownumbers}{0}

\renewcommand{\alternativeCaseUse}{
	\caseUseRow{No existen flujos alternativos.}
}

%DIAGRAMA DE ACTIVIDAD
%\lineabreak[0]
%\activityDiagram{\caseUseShortName}{Diagrama de actividad - \caseUseName}

%GestionObservaciones
\renewcommand{\caseUseShortName}{crearObservacion} %cammelCase name

\renewcommand{\caseUseCreated}{05/03/2020} %Fecha creación
\renewcommand{\caseUseModified}{05/03/2020} %Fecha modificación
\renewcommand{\caseUseName}{\CUcrearObservacion - Crear observación} %{\CUcammelCase - Title}

\renewcommand{\caseUseSummary}{Este caso de uso permite a un administrador de ZMGestion crear una observación en una línea de orden de producción existente.} %Resumen
\renewcommand{\caseUsePeople}{Administradores: quiere crear una observación en una línea de orden de producción.} %Actor: Meta
\renewcommand{\caseUsePreconditions}{
	\caseUseRow{Haber realizado con éxito el \CUlistarLineasOrdenProduccion\ (Listar líneas de orden de producción).} %Precondiciones
}
\renewcommand{\caseUsePostconditions}{
	\caseUseRow{Ninguna.} %Postcondiciones
}
\renewcommand{\caseUseScene}{ %Escenario principal
    \addCaseUseStep{El administrador selecciona una línea de orden de producción a la cual le desea crear una observación.}
    \addCaseUseStep{ZMGestion muestra un formulario para que el administrador ingrese una observación.}
    \addCaseUseStep{El administrador ingresa una observación.}
    \addCaseUseStep{ZMGestion crea la observación para la línea de orden de producción seleccionada por el administrador.}
}
\renewcommand{\alternativeCaseUse}{ %Flujos alternativos
	\newAlternative{A1: No ha ingresado ninguna observación.}{3} %Flujo alternativo A1.
	\caseUseRow{La secuencia A1 comienza luego del punto 3 del escenario principal.} %¡Indicar número paso!
    \alternativeRow{ZMGestion muestra un mensaje de error indicando que debe ingresar una observación.}
    \caseUseRow{El escenario vuelve al punto 2.}
    \caseUseRow{}
}

\item Caso de uso \caseUseName
\renewcommand*{\arraystretch}{1.3}
\begin{longtable}[c]{|>{\raggedright}p{0.3\textwidth} | >{\raggedright}p{0.2\textwidth} | p{0.5\textwidth} |}
\caption{\hyperref[sec:listadoCasoUso]{\caseUseName}}
\label{tabla:\caseUseShortName}\\
\hline
\rowcolor{tableCaseUseBackground}

\multicolumn{3}{|l|}{\textcolor{tableCaseUseFontColor}{Descripción textual del caso de uso: \caseUseName}} \\ \hline

Fecha de Creación: & \multicolumn{2}{L{\secondColumnWidth}|}{\caseUseCreated}\\ \hline

Fecha de Modificación: & \multicolumn{2}{L{\secondColumnWidth}|}{\caseUseModified} \\ \hline

Versión: & \multicolumn{2}{L{\secondColumnWidth}|}{1} \\ \hline

Resumen: & \multicolumn{2}{L{\secondColumnWidth}|}{\caseUseSummary} \\ \hline

Personas involucradas y metas: & \multicolumn{2}{L{\secondColumnWidth}|}{\caseUsePeople} \\ \hline

Precondiciones: \caseUsePreconditions \hline

Postcondiciones: \caseUsePostconditions \hline

Escenario principal: \caseUseScene \hline

Flujos alternativos: \alternativeCaseUse \hline

Requisitos de interfaz de usuario: \caseUseRequirementsGUI \hline
\multirow{3}{*}{Requisitos funcionales:}  & Tiempo de respuesta: & \caseUseResponseTime \\ \cline{2-3} 
& Concurrencia: & \caseUseConcurrence \\ \cline{2-3} 
& Disponibilidad: & \caseUseAvailability \\ \hline
\end{longtable}

\setcounter{rownumbers}{0}

\renewcommand{\alternativeCaseUse}{
	\caseUseRow{No existen flujos alternativos.}
}

%DIAGRAMA DE ACTIVIDAD
%\lineabreak[0]
%\activityDiagram{\caseUseShortName}{Diagrama de actividad - \caseUseName}
\renewcommand{\caseUseShortName}{listarObservaciones} %cammelCase name
\renewcommand{\caseUseCreated}{05/03/2020} %Fecha creación
\renewcommand{\caseUseModified}{05/03/2020} %Fecha modificación
\renewcommand{\caseUseName}{\CUlistarObservaciones - Listar observaciones.} %{\CUcammelCase - Title}

\renewcommand{\caseUseSummary}{Este caso de uso permite a los usaurios de ZMGestion listar las observaciones de una línea de presupuesto, venta, orden de producción o remito.} %Resumen
\renewcommand{\caseUsePeople}{Usuarios: quiere listar las observaciones de una linea de presupuesto, venta, orden de produccióno o remito.} %Actor: Meta
\renewcommand{\caseUsePreconditions}{
    \caseUseRow{Haber realizado con éxito alguno de los siguientes casos de uso: 
        \begin{itemize}
            \item \CUlistarLineasPresupuesto\ (Listar líneas de presupuesto).
            \item \CUlistarLineasVenta\ (Listar líneas de venta).
            \item \CUlistarLineasOrdenProduccion\ (Listar líneas de orden de producción).
            \item \CUlistarLineasRemito\ (Listar líneas de remito.).
        \end{itemize} 
    }
} %Precondiciones

\renewcommand{\caseUsePostconditions}{
	\caseUseRow{Ninguna.} %Postcondiciones
}
\renewcommand{\caseUseScene}{ %Escenario principal
    \addCaseUseStep{El fabricante selecciona una línea de presupuesto, venta orden de producción o remito a la cual le desea listar sus observaciones.}
    \addCaseUseStep{ZMGestion lista las observaciones de la línea seleccionada.}
}
\renewcommand{\alternativeCaseUse}{ %Flujos alternativos
	\newAlternative{A1: La línea seleccionada no posee observaciones.}{1} %Flujo alternativo A1.
	\caseUseRow{La secuencia A1 comienza luego del punto 1 del escenario principal.} %¡Indicar número paso!
    \alternativeRow{ZMGestion muestra un mensaje indicando que la línea seleccionada no posee observaciones.}
    \caseUseRow{El escenario vuelve al punto 1.}
    \caseUseRow{}
}

\item Caso de uso \caseUseName
\renewcommand*{\arraystretch}{1.3}
\begin{longtable}[c]{|>{\raggedright}p{0.3\textwidth} | >{\raggedright}p{0.2\textwidth} | p{0.5\textwidth} |}
\caption{\hyperref[sec:listadoCasoUso]{\caseUseName}}
\label{tabla:\caseUseShortName}\\
\hline
\rowcolor{tableCaseUseBackground}

\multicolumn{3}{|l|}{\textcolor{tableCaseUseFontColor}{Descripción textual del caso de uso: \caseUseName}} \\ \hline

Fecha de Creación: & \multicolumn{2}{L{\secondColumnWidth}|}{\caseUseCreated}\\ \hline

Fecha de Modificación: & \multicolumn{2}{L{\secondColumnWidth}|}{\caseUseModified} \\ \hline

Versión: & \multicolumn{2}{L{\secondColumnWidth}|}{1} \\ \hline

Resumen: & \multicolumn{2}{L{\secondColumnWidth}|}{\caseUseSummary} \\ \hline

Personas involucradas y metas: & \multicolumn{2}{L{\secondColumnWidth}|}{\caseUsePeople} \\ \hline

Precondiciones: \caseUsePreconditions \hline

Postcondiciones: \caseUsePostconditions \hline

Escenario principal: \caseUseScene \hline

Flujos alternativos: \alternativeCaseUse \hline

Requisitos de interfaz de usuario: \caseUseRequirementsGUI \hline
\multirow{3}{*}{Requisitos funcionales:}  & Tiempo de respuesta: & \caseUseResponseTime \\ \cline{2-3} 
& Concurrencia: & \caseUseConcurrence \\ \cline{2-3} 
& Disponibilidad: & \caseUseAvailability \\ \hline
\end{longtable}

\setcounter{rownumbers}{0}

\renewcommand{\alternativeCaseUse}{
	\caseUseRow{No existen flujos alternativos.}
}

%DIAGRAMA DE ACTIVIDAD
%\lineabreak[0]
%\activityDiagram{\caseUseShortName}{Diagrama de actividad - \caseUseName}
\renewcommand{\caseUseShortName}{borrarObservacion} %cammelCase name

\renewcommand{\caseUseCreated}{05/03/2020} %Fecha creación
\renewcommand{\caseUseModified}{05/03/2020} %Fecha modificación
\renewcommand{\caseUseName}{\CUborrarObservacion - Borrar observación} %{\CUcammelCase - Title}

\renewcommand{\caseUseSummary}{Este caso de uso permite a un administrador de ZMGestion borrar una observación de una línea de orden de producción existente.} %Resumen
\renewcommand{\caseUsePeople}{Administradores: quiere borrar una observación de una línea de orden de producción.} %Actor: Meta
\renewcommand{\caseUsePreconditions}{
	\caseUseRow{Haber realizado con éxito el \CUlistarObservacionesLineaOrdenProduccion\ (Listar observaciones de línea de orden de producción) y contar con los permisos necesarios para realizar esta función.} %Precondiciones
}
\renewcommand{\caseUsePostconditions}{
	\caseUseRow{Ninguna.} %Postcondiciones
}
\renewcommand{\caseUseScene}{ %Escenario principal
    \addCaseUseStep{El administrador selecciona la observación que desea borrar de una línea de orden de producción.}
    \addCaseUseStep{ZMGestion borra la observación seleccionada por el administrador.}
}
\renewcommand{\alternativeCaseUse}{ %Flujos alternativos
    \caseUseRow{Ninguna.}
}

\item Caso de uso \caseUseName
\renewcommand*{\arraystretch}{1.3}
\begin{longtable}[c]{|>{\raggedright}p{0.3\textwidth} | >{\raggedright}p{0.2\textwidth} | p{0.5\textwidth} |}
\caption{\hyperref[sec:listadoCasoUso]{\caseUseName}}
\label{tabla:\caseUseShortName}\\
\hline
\rowcolor{tableCaseUseBackground}

\multicolumn{3}{|l|}{\textcolor{tableCaseUseFontColor}{Descripción textual del caso de uso: \caseUseName}} \\ \hline

Fecha de Creación: & \multicolumn{2}{L{\secondColumnWidth}|}{\caseUseCreated}\\ \hline

Fecha de Modificación: & \multicolumn{2}{L{\secondColumnWidth}|}{\caseUseModified} \\ \hline

Versión: & \multicolumn{2}{L{\secondColumnWidth}|}{1} \\ \hline

Resumen: & \multicolumn{2}{L{\secondColumnWidth}|}{\caseUseSummary} \\ \hline

Personas involucradas y metas: & \multicolumn{2}{L{\secondColumnWidth}|}{\caseUsePeople} \\ \hline

Precondiciones: \caseUsePreconditions \hline

Postcondiciones: \caseUsePostconditions \hline

Escenario principal: \caseUseScene \hline

Flujos alternativos: \alternativeCaseUse \hline

Requisitos de interfaz de usuario: \caseUseRequirementsGUI \hline
\multirow{3}{*}{Requisitos funcionales:}  & Tiempo de respuesta: & \caseUseResponseTime \\ \cline{2-3} 
& Concurrencia: & \caseUseConcurrence \\ \cline{2-3} 
& Disponibilidad: & \caseUseAvailability \\ \hline
\end{longtable}

\setcounter{rownumbers}{0}

\renewcommand{\alternativeCaseUse}{
	\caseUseRow{No existen flujos alternativos.}
}

%DIAGRAMA DE ACTIVIDAD
%\lineabreak[0]
%\activityDiagram{\caseUseShortName}{Diagrama de actividad - \caseUseName}

%GestionLineasOrdenesProduccion
\renewcommand{\caseUseShortName}{crearLineaOrdenProduccion} %cammelCase name

\renewcommand{\caseUseCreated}{02/03/2020} %Fecha creación
\renewcommand{\caseUseModified}{02/03/2020} %Fecha modificación
\renewcommand{\caseUseName}{\CUcrearLineaOrdenProduccion\ - Crear linea de orden de producción.} %{\CUcammelCase - Title}

\renewcommand{\caseUseSummary}{Este caso de uso permite a un administrador crear una linea de orden de producción para una determinada orden de producción.} %Resumen
\renewcommand{\caseUsePeople}{Administradores: quiere crear una linea de orden de producción.} %Actor: Meta
\renewcommand{\caseUsePreconditions}{
	\caseUseRow{Estar ejecutando el \CUcrearOrdenProduccion (Crear orden de producción) o el \CUmodificarOrdenProduccion (Modificar orden de producción).} %Precondiciones
}
\renewcommand{\caseUsePostconditions}{
	\caseUseRow{Si el producto, tela y lustre no pertenece a ningun producto final existente se ejecuta el \CUcrearProductoFinal\ (Crear producto final) con el producto, tela y lustre seleccionados.} %Postcondiciones
}
\renewcommand{\caseUseScene}{ %Escenario principal
    \addCaseUseStep{El administrador desea agregar una linea de orden de producción a una determinada orden de producción.}
    \addCaseUseStep{ZMGestion le muestra un formulario para que el administrador seleccione un producto, tela, lustre e indique la cantidad a producir de dicho producto. Indicando que el producto y la cantidad son obligatorios.}
    \addCaseUseStep{El administrador completa el formulario.}
    \addCaseUseStep{ZMGestion crea la linea de orden de producción en estado `Pendiente de producción' y la asocia a la orden de producción correspondiente.}
}
\renewcommand{\alternativeCaseUse}{ %Flujos alternativos
    \newAlternative{A1: La cantidad indicada es menor o igual a cero.}{3} %Flujo alternativo A1.
    \caseUseRow{La secuencia A1 comienza luego del punto 3 del escenario principal.} %¡Indicar número paso!
    \alternativeRow{ZMGestion muestra un mensaje de error indicando que debe ingresar una cantidad mayor a cero.}
    \caseUseRow{El escenario vuelve al punto 2.}
    \caseUseRow{}
    \newAlternative{A2: El producto, tela y lustre indicado ya se encuentra en la orden de producción.}{3} %Flujo alternativo A2.
    \caseUseRow{La secuencia A2 comienza luego del punto 3 del escenario principal.} %¡Indicar número paso!
    \alternativeRow{ZMGestion muestra un mensaje de error indicando que la combinación de producto, tela y lustre ingresado ya se encuentra en la orden de producción.}
    \caseUseRow{El escenario vuelve al punto 2.}
    \caseUseRow{}
    \newAlternative{A3: El tipo de producto seleccionado no es del tipo `Producible'.}{3} %Flujo alternativo A1.
    \caseUseRow{La secuencia A3 comienza luego del punto 3 del escenario principal.} %¡Indicar número paso!
    \alternativeRow{ZMGestion muestra un mensaje de error indicando que el producto seleccionado no puede ser asignado a una orden de producción.}
    \caseUseRow{El escenario vuelve al punto 2.}
    \caseUseRow{}

    \newAlternative{A4: El administrador ha dejado un campo obligatorio vacio.}{3} %Flujo alternativo A1.
    \caseUseRow{La secuencia A4 comienza luego del punto 3 del escenario principal.} %¡Indicar número paso!
    \alternativeRow{ZMGestion muestra un mensaje de error indicando que ha dejado un campo obligatorio vacio.}
    \caseUseRow{El escenario vuelve al punto 2.}
    \caseUseRow{}
}

\item Caso de uso \caseUseName
\renewcommand*{\arraystretch}{1.3}
\begin{longtable}[c]{|>{\raggedright}p{0.3\textwidth} | >{\raggedright}p{0.2\textwidth} | p{0.5\textwidth} |}
\caption{\hyperref[sec:listadoCasoUso]{\caseUseName}}
\label{tabla:\caseUseShortName}\\
\hline
\rowcolor{tableCaseUseBackground}

\multicolumn{3}{|l|}{\textcolor{tableCaseUseFontColor}{Descripción textual del caso de uso: \caseUseName}} \\ \hline

Fecha de Creación: & \multicolumn{2}{L{\secondColumnWidth}|}{\caseUseCreated}\\ \hline

Fecha de Modificación: & \multicolumn{2}{L{\secondColumnWidth}|}{\caseUseModified} \\ \hline

Versión: & \multicolumn{2}{L{\secondColumnWidth}|}{1} \\ \hline

Resumen: & \multicolumn{2}{L{\secondColumnWidth}|}{\caseUseSummary} \\ \hline

Personas involucradas y metas: & \multicolumn{2}{L{\secondColumnWidth}|}{\caseUsePeople} \\ \hline

Precondiciones: \caseUsePreconditions \hline

Postcondiciones: \caseUsePostconditions \hline

Escenario principal: \caseUseScene \hline

Flujos alternativos: \alternativeCaseUse \hline

Requisitos de interfaz de usuario: \caseUseRequirementsGUI \hline
\multirow{3}{*}{Requisitos funcionales:}  & Tiempo de respuesta: & \caseUseResponseTime \\ \cline{2-3} 
& Concurrencia: & \caseUseConcurrence \\ \cline{2-3} 
& Disponibilidad: & \caseUseAvailability \\ \hline
\end{longtable}

\setcounter{rownumbers}{0}

\renewcommand{\alternativeCaseUse}{
	\caseUseRow{No existen flujos alternativos.}
}

%DIAGRAMA DE ACTIVIDAD
%\lineabreak[0]
%\activityDiagram{\caseUseShortName}{Diagrama de actividad - \caseUseName}

\renewcommand{\caseUseShortName}{modificarLineaOrdenProduccion} %cammelCase name

\renewcommand{\caseUseCreated}{04/03/2020} %Fecha creación
\renewcommand{\caseUseModified}{04/03/2020} %Fecha modificación
\renewcommand{\caseUseName}{\CUmodificarLineaOrdenProduccion - Modificar línea de orden de producción} %{\CUcammelCase - Title}

\renewcommand{\caseUseSummary}{Este caso de uso permite a un administrador de ZMGestion modificar una línea de orden de producción.} %Resumen
\renewcommand{\caseUsePeople}{Administradores: desea modificar una línea de orden de producción.} %Actor: Meta
\renewcommand{\caseUsePreconditions}{
	\caseUseRow{Haber ejecutado con éxito el \CUlistarLineasOrdenProduccion (Listar lineas de orden de producción).} %Precondiciones
}
\renewcommand{\caseUsePostconditions}{
	\caseUseRow{Ninguna.} %Postcondiciones
}
\renewcommand{\caseUseScene}{ %Escenario principal
    \addCaseUseStep{El administrador indica la línea de orden de producción que desea modificar.}
    \addCaseUseStep{ZMGestion muestra un formulario autocompletado con los datos de la linea de orden de producción seleccionada, producto, tela, lustre y cantidad a producirse. Indicando que el producto y la cantidad son obligatorios.}
    \addCaseUseStep{El administrador modifica el formulario.}
    \addCaseUseStep{ZMGestion modifica la línea de orden de producción y muestra un mensaje indicando el éxito de la operación.}

}
\renewcommand{\alternativeCaseUse}{ %Flujos alternativos
	\newAlternative{A1: La línea de orden de producción seleccionada no se encuentra en estado de `Pendiente de producción'.}{1} %Flujo alternativo A1.
	\caseUseRow{La secuencia A1 comienza luego del punto 1 del escenario principal.} %¡Indicar número paso!
    \alternativeRow{ZMgestion muestra un mensaje de error indicando que la línea de orden de producción no puede modificarse.}
    \caseUseRow{El escenario vuelve al punto 1.}
    \caseUseRow{}

	\newAlternative{A2: La cantidad a producirse es menor o igual que cero.}{3} %Flujo alternativo A2.
    \caseUseRow{La secuencia A2 comienza luego del punto 3 del escenario principal.}%¡Indicar número paso!
    \alternativeRow{ZMGestion muestra un mensaje de error indicando que la cantidad a producirse debe ser mayor que cero.}
    \caseUseRow{El escenario vuelve al punto 2.}
    \caseUseRow{}

    \newAlternative{A3: Existe una línea de orden de producción, producto, tela y lustre, igual en la orden de producción.}{3} %Flujo alternativo A2.
    \caseUseRow{La secuencia A3 comienza luego del punto 3 del escenario principal.}%¡Indicar número paso!
    \alternativeRow{ZMGestion muestra un mensaje de error indicando que ya existe una línea de orden de producción identica en la orden de producción.}
    \caseUseRow{El escenario vuelve al punto 2.}
    \caseUseRow{}

    \newAlternative{A4: El tipo de producto seleccionado no es del tipo `Producible'.}{3} %Flujo alternativo A4.
    \caseUseRow{La secuencia A4 comienza luego del punto 3 del escenario principal.} %¡Indicar número paso!
    \alternativeRow{ZMGestion muestra un mensaje de error indicando que el producto seleccionado no puede ser asignado a una orden de producción.}
    \caseUseRow{El escenario vuelve al punto 2.}
    \caseUseRow{}

    \newAlternative{A5: El administrador ha dejado un campo obligatorio vacio.}{3} %Flujo alternativo A1.
    \caseUseRow{La secuencia A5 comienza luego del punto 3 del escenario principal.} %¡Indicar número paso!
    \alternativeRow{ZMGestion muestra un mensaje de error indicando que ha dejado un campo obligatorio vacio.}
    \caseUseRow{El escenario vuelve al punto 2.}
    \caseUseRow{}
}

\item Caso de uso \caseUseName
\renewcommand*{\arraystretch}{1.3}
\begin{longtable}[c]{|>{\raggedright}p{0.3\textwidth} | >{\raggedright}p{0.2\textwidth} | p{0.5\textwidth} |}
\caption{\hyperref[sec:listadoCasoUso]{\caseUseName}}
\label{tabla:\caseUseShortName}\\
\hline
\rowcolor{tableCaseUseBackground}

\multicolumn{3}{|l|}{\textcolor{tableCaseUseFontColor}{Descripción textual del caso de uso: \caseUseName}} \\ \hline

Fecha de Creación: & \multicolumn{2}{L{\secondColumnWidth}|}{\caseUseCreated}\\ \hline

Fecha de Modificación: & \multicolumn{2}{L{\secondColumnWidth}|}{\caseUseModified} \\ \hline

Versión: & \multicolumn{2}{L{\secondColumnWidth}|}{1} \\ \hline

Resumen: & \multicolumn{2}{L{\secondColumnWidth}|}{\caseUseSummary} \\ \hline

Personas involucradas y metas: & \multicolumn{2}{L{\secondColumnWidth}|}{\caseUsePeople} \\ \hline

Precondiciones: \caseUsePreconditions \hline

Postcondiciones: \caseUsePostconditions \hline

Escenario principal: \caseUseScene \hline

Flujos alternativos: \alternativeCaseUse \hline

Requisitos de interfaz de usuario: \caseUseRequirementsGUI \hline
\multirow{3}{*}{Requisitos funcionales:}  & Tiempo de respuesta: & \caseUseResponseTime \\ \cline{2-3} 
& Concurrencia: & \caseUseConcurrence \\ \cline{2-3} 
& Disponibilidad: & \caseUseAvailability \\ \hline
\end{longtable}

\setcounter{rownumbers}{0}

\renewcommand{\alternativeCaseUse}{
	\caseUseRow{No existen flujos alternativos.}
}

%DIAGRAMA DE ACTIVIDAD
%\lineabreak[0]
%\activityDiagram{\caseUseShortName}{Diagrama de actividad - \caseUseName}

\renewcommand{\caseUseShortName}{borrarLineaOrdenProduccion} %cammelCase name

\renewcommand{\caseUseCreated}{02/03/2020} %Fecha creación
\renewcommand{\caseUseModified}{02/03/2020} %Fecha modificación
\renewcommand{\caseUseName}{\CUborrarLineaOrdenProduccion\ - Borrar linea de orden de producción.} %{\CUcammelCase - Title}

\renewcommand{\caseUseSummary}{Este caso de uso permite a un administrador borrar una linea de orden de producción de una orden de producción determinada.} %Resumen
\renewcommand{\caseUsePeople}{Administradores: quiere borrar una linea de orden de producción.} %Actor: Meta
\renewcommand{\caseUsePreconditions}{
	\caseUseRow{Haber ejecutado con éxito el \CUlistarLineasOrdenProduccion (Listar lineas de orden de producción).} %Precondiciones
}
\renewcommand{\caseUsePostconditions}{
	\caseUseRow{Ninguna.} %Postcondiciones
}
\renewcommand{\caseUseScene}{ %Escenario principal
    \addCaseUseStep{El vendedor indica la linea de orden de producción que desea borrar.}
    \addCaseUseStep{ZMGestion borra la linea de orden de producción.}
}
\renewcommand{\alternativeCaseUse}{ %Flujos alternativos

    %VER QUE A1 TAMBIEN SE PUEDE HACER REVISANDO SI LA LINEA ESTA UTILIZADA O NO

	\newAlternative{A1: La linea seleccionada tiene un estado distinto de `Pendiente de producción'.}{1} %Flujo alternativo A1.
	\caseUseRow{La secuencia A1 comienza luego del punto 1 del escenario principal.} %¡Indicar número paso!
    \alternativeRow{ZMGestion muestra un mensaje de error indicando que no se puede borrar la linea de orden de producción.}
    \caseUseRow{El escenario vuelve al punto 1.}
}

\item Caso de uso \caseUseName
\renewcommand*{\arraystretch}{1.3}
\begin{longtable}[c]{|>{\raggedright}p{0.3\textwidth} | >{\raggedright}p{0.2\textwidth} | p{0.5\textwidth} |}
\caption{\hyperref[sec:listadoCasoUso]{\caseUseName}}
\label{tabla:\caseUseShortName}\\
\hline
\rowcolor{tableCaseUseBackground}

\multicolumn{3}{|l|}{\textcolor{tableCaseUseFontColor}{Descripción textual del caso de uso: \caseUseName}} \\ \hline

Fecha de Creación: & \multicolumn{2}{L{\secondColumnWidth}|}{\caseUseCreated}\\ \hline

Fecha de Modificación: & \multicolumn{2}{L{\secondColumnWidth}|}{\caseUseModified} \\ \hline

Versión: & \multicolumn{2}{L{\secondColumnWidth}|}{1} \\ \hline

Resumen: & \multicolumn{2}{L{\secondColumnWidth}|}{\caseUseSummary} \\ \hline

Personas involucradas y metas: & \multicolumn{2}{L{\secondColumnWidth}|}{\caseUsePeople} \\ \hline

Precondiciones: \caseUsePreconditions \hline

Postcondiciones: \caseUsePostconditions \hline

Escenario principal: \caseUseScene \hline

Flujos alternativos: \alternativeCaseUse \hline

Requisitos de interfaz de usuario: \caseUseRequirementsGUI \hline
\multirow{3}{*}{Requisitos funcionales:}  & Tiempo de respuesta: & \caseUseResponseTime \\ \cline{2-3} 
& Concurrencia: & \caseUseConcurrence \\ \cline{2-3} 
& Disponibilidad: & \caseUseAvailability \\ \hline
\end{longtable}

\setcounter{rownumbers}{0}

\renewcommand{\alternativeCaseUse}{
	\caseUseRow{No existen flujos alternativos.}
}

%DIAGRAMA DE ACTIVIDAD
%\lineabreak[0]
%\activityDiagram{\caseUseShortName}{Diagrama de actividad - \caseUseName}

\section*{Descripción de escenarios}
\begin{figure}[H]
\centering
\includegraphics[width=\textwidth,height=\textheight,keepaspectratio]{Escenarios/AD-01-00}
\caption{Escenario - AD-01-00}
\label{fig:AD-01-00}
\end{figure}

Descripcion escenario...
\clearpage

\begin{figure}[H]
\centering
\includegraphics[width=\textwidth,height=\textheight,keepaspectratio]{Escenarios/AD-09-00}
\caption{Escenario - AD-09-00}
\label{fig:AD-09-00}
\end{figure}
Este es el escenario que se muestra luego de crear una venta a partir de uno o más presupuestos. Un click en el boton \textbf{AD-09-01} navegará al escenario \textbf{AD-12-00} para ver la venta recientemente creada. Un click en el botón \textbf{AD-09-02} cerrará la ventana y navegará al escenario \textbf{AD-03-00}.
\\
\begin{figure}[H]
\centering
\includegraphics[width=\textwidth,height=\textheight,keepaspectratio]{Escenarios/AD-11-00}
\caption{Escenario - AD-11-00}
\label{fig:AD-11-00}
\end{figure}
Este es el escenario que permite a los usuarios crear y modificar ventas, el campo \textbf{AD-11-01} indicará la acción que se va a realizar, pudiendo ser 'Nueva venta' o 'Editar venta'. Con el botón \textbf{AD-11-02} se podrá cerrar la ventana y volver al escenario \textbf{AD-10-00}.
El campo \textbf{AD-10-03} permite al usuario indicar el cliente para el cúal se esta creando la venta o indicar el nuevo cliente en caso que se esté editando el presupuesto. Un click en botón \textbf{AD-10-04} eliminará la seleccion que se encuentre en el campo \textbf{AD-10-03}. El botón \textbf{AD-10-05} permite crear un nuevo cliente navegando al escenario \textbf{AD-29-00}. La lista desplegable \textbf{AD-10-06} permite al usuario indicar la ubicación en la cúal se esta creando la venta o bien modificar la misma. En la lista desplegable \textbf{AD-11-07} se mostrarán los domicilios del cliente seleccionado en el campo \textbf{AD-11-03}, de los cuales el usuario deberá elegir uno para la venta. El botón \textbf{AD-11-08} navega al escenario \textbf{AD-31-00} para crear un domicilio para el cliente.
El campo \textbf{AD-10-09} se muestra cuando la venta no tiene asociado ninguna linea de venta. El boton \textbf{AD-10-10} permite al usuario crear una linea de venta y asociarla a la venta, navega al escenario \textbf{AD-05-00}. Si el usuario hace click en el botón \textbf{AD-10-11} creará la venta y navegará al escenario \textbf{AD-13-00}, si hace click en el botón \textbf{AD-04-09} cerrará la ventana navegando al escenario \textbf{AD-10-00}.
\clearpage

\begin{figure}[H]
\centering
\includegraphics[width=\textwidth,height=\textheight,keepaspectratio]{Escenarios/AD-13-00}
\caption{Escenario - AD-13-00}
\label{fig:AD-13-00}
\end{figure}
Este es el escenario que se muestrá luego de crear con exito una venta. El campo \textbf{AD-13-01} muestra un mensaje indicando el éxito de la operación. Un click en el botón \textbf{AD-13-02} navegará al escenario \textbf{AD-15-00} para agregar un comprobante a la venta. Un click en el botón \textbf{AD-13-03} cerrará la venta navegando al escenario \textbf{AD-10-00}.
\clearpage

\begin{figure}[H]
\centering
\includegraphics[width=\textwidth,height=\textheight,keepaspectratio]{Escenarios/AD-14-00}
\caption{Escenario - AD-14-00}
\label{fig:AD-14-00}
\end{figure}
Este escenario muestra toda la información referida a los comprobantes, junto con las acciones disponibles.
El botón \textbf{AD-14-01} permite navegar al escenario \textbf{AD-02-00}. El campo \textbf{AD-10-02} permite filtrar los comprobantes por su número de identificación. La lista desplegable \textbf{AD-14-03} permite filtrar a los comprobantes de acuerdo al tipo. El campo \textbf{AD-14-04} permite al usuario buscar comprobantes de acuerdo al empleado que la realizó. El campo \textbf{AD-15-04} cuenta con el boton \textbf{AD-14-05} que permite eliminar el empleado seleccionado en el campo \textbf{AD-14-05}.
El campo \textbf{AD-14-06} muestra la información relacionada a los comprobantes especificando el código de identificación, el tipo , el monto del comprobante y observaciones El botón \textbf{AD-14-07} permite al usuario dar de baja un comprobante, el botón \textbf{AD-14-08} permite al usuario editar el comprobante navegando al escenario \textbf{AD-15-00} y el botón \textbf{AD-14-09} permite al usuario borrar el comprobante. 
En  \textbf{AD-14-10} se mostrarán las páginas de resultado, pudiendo cambiar de página. En \textbf{AD-14-11} se mostrará cuantos resultados se están visualizando y el total.
\\
\begin{figure}[H]
\centering
\includegraphics[width=\textwidth,height=\textheight,keepaspectratio]{Escenarios/AD-15-00}
\caption{Escenario - AD-15-00}
\label{fig:AD-15-00}
\end{figure}

Descripcion escenario...
\clearpage

\begin{figure}[H]
\centering
\includegraphics[width=\textwidth,height=\textheight,keepaspectratio]{Escenarios/AD-16-00}
\caption{Escenario - AD-16-00}
\label{fig:AD-16-00}
\end{figure}
Este escenario muestra toda la información referida a los remitos, junto con las acciones disponibles.
El botón \textbf{AD-16-01} permite navegar al escenario. El campo \textbf{AD-16-02} permite al usuario buscar remitos de acuerdo al empleado que la realizó el campo cuenta con el botón \textbf{AD-16-03} que permite borrar el texto ingresado en el campo. La lista desplegable \textbf{AD-16-04} permite filtrar de acuerdo a la ubicación de entrada de los remitos. La lista desplegable \textbf{AD-16-5} permite filtrar de acuerdo al tipo de remito. La lista desplegable \textbf{AD-16-06} permite al usuario filtrar por los estados en los cuales puede encontrarse el remito.  Los campos \textbf{AD-16-07} y \textbf{AD-16-08} permiten al usuario filtrar los ventas de acuerdo a un rango de fechas en el cual fueron creados. El botón \textbf{AD-16-09} permite visualizar más filtros de búsqueda disponibles, como ser el producto, la tela y el lustre.
El botón \textbf{AD-16-10} permite al usuario crear un nuevo remito y navega al escenario \textbf{AD-17-00}.
El campo \textbf{AD-16-11} muestra la información relacionada a los remitos especificando el código, el detalle de remito, la fecha de creación, la fecha de entrega, el tipo de remito, el estado en el cúal se encuentra y la ubicación de entrada en caso de existir. El botón \textbf{AD-16-12} permite navegar al escenario \textbf{AD-19-00} para ver el remito, el botón \textbf{AD-16-13} permite al usuario borrar el remito y el botón \textbf{AD-16-14} permite al usuario dar de baja un remito en caso de poder realizar la acción.. 
En \textbf{AD-16-15} se mostrarán las páginas de resultado, pudiendo cambiar de página. En \textbf{AD-16-16} se mostrará cuantos resultados se están visualizando y el total.
\\

\begin{figure}[H]
\centering
\includegraphics[width=\textwidth,height=\textheight,keepaspectratio]{Escenarios/AD-19-00}
\caption{Escenario - AD-19-00}
\label{fig:AD-19-00}
\end{figure}

Descripcion escenario...
\clearpage

\begin{figure}[H]
\centering
\includegraphics[width=\textwidth,height=\textheight,keepaspectratio]{Escenarios/AD-20-00}
\caption{Escenario - AD-20-00}
\label{fig:AD-20-00}
\end{figure}

Descripcion escenario...
\clearpage

\begin{figure}[H]
\centering
\includegraphics[width=\textwidth,height=\textheight,keepaspectratio]{Escenarios/AD-21-00}
\caption{Escenario - AD-21-00}
\label{fig:AD-21-00}
\end{figure}

Descripcion escenario...
\clearpage

\begin{figure}[H]
\centering
\includegraphics[width=\textwidth,height=\textheight,keepaspectratio]{Escenarios/AD-22-00}
\caption{Escenario - AD-22-00}
\label{fig:AD-22-00}
\end{figure}

Este escenario muestra toda la información referida a las órdenes de producción, junto con las acciones disponibles.
El botón \textbf{AD-32-01} permite navegar al escenario \textbf{AD-02-00}. El campo \textbf{AD-32-02} permite ingresar un fabricante para filtrar las órdenes de producción que tengan al menos una tarea designada a un determinado fabricante, el campo cuenta con el botón \textbf{AD-32-03} que permite borrar el texto ingresado en el campo. En el campo \textbf{AD-32-04} se puede indicar un empleado revisor para filtrar las órdenes de producción que hayan sido revisadas por un determinado empleado. Indicando un empleado en el campo \textbf{AD-32-05} se pueden filtrar todas las órdenes de producción que fueron creadas por un determinado empleado. La lista desplegable \textbf{AD-32-06} permite al usuario filtrar por los estados en los cuáles puede encontrarse la orden de producción. Los campos  \textbf{AD-32-07} y \textbf{AD-32-08} permiten al usuario filtrar las órdenes de producción de acuerdo a un rango de fechas en la cual fueron creadas. El botón \textbf{AD-32-09} permite visualizar más filtros de búsqueda disponibles, como ser el producto, la tela y el lustre, filtrando las órdenes de producción que contengan un detalle con alguno de éstos.
El botón \textbf{AD-32-10} permite al usuario crear una nueva orden de producción y navega al escenario \textbf{AD-23-00}.
El botón \textbf{AD-32-11} permite al usuario seleccionar una o más órdenes de producción del resultado de búsqueda. Si existen órdenes de producción seleccionadas se mostrarán botones con las opción de borrar las órdenes de producción seleccionadas. El campo \textbf{AD-32-12} muestra la información relacionada a las órdenes de producción especificando el código, el detalle de la orden de producción, la fecha de creación y el estado en el cual se encuentra. El botón \textbf{AD-32-13} permite navegar al escenario \textbf{AD-25-00} para ver la orden de producción, el botón \textbf{AD-32-14} permite al usuario editar la orden de producción navegando al escenario \textbf{AD-23-00} y el botón \textbf{AD-32-15} permite al usuario borrar la orden de producción. 
En \textbf{AD-32-16} se mostrarán las páginas de resultado, pudiendo cambiar de página. En \textbf{AD-32-17} se mostrará cuantos resultados se están visualizando y el total.
\clearpage

\begin{figure}[H]
    \centering
    \begin{minipage}[b]{0.4\textwidth}
        \includegraphics[width=\textwidth,height=\textheight,keepaspectratio]{Escenarios/AD-23-00-DesdeCero}
        \caption{AD-23-00 Desde cero}
        \label{fig:AD-23-00}
    \end{minipage}
    \hfill
    \begin{minipage}[b]{0.4\textwidth}
        \includegraphics[width=\textwidth,height=\textheight,keepaspectratio]{Escenarios/AD-23-00-Venta}
        \caption{AD-23-00 Para venta}
        \label{fig:AD-23-00}
    \end{minipage}
    \end{figure}

Descripcion escenario...
\clearpage

\begin{figure}[H]
\centering
\includegraphics[width=\textwidth,height=\textheight,keepaspectratio]{Escenarios/AD-24-00}
\caption{Escenario - AD-24-00}
\label{fig:AD-24-00}
\end{figure}

Descripcion escenario...
\clearpage

\begin{figure}[H]
\centering
\includegraphics[width=\textwidth,height=\textheight,keepaspectratio]{Escenarios/AD-25-00}
\caption{Escenario - AD-25-00}
\label{fig:AD-25-00}
\end{figure}

Este escenario permite a un usuario visualizar una orden de producción, en \textbf{AD-25-01} se puede ver el estado de la orden de producción. Con el botón \textbf{AD-25-02} se puede cerrar la ventana, volviendo al escenario \textbf{AD-22-00}. En \textbf{AD-25-03} se muestra información acerca del empleado que creó la orden de producción y la fecha de creación. En \textbf{AD-25-04} se muestra el detalle de la orden de producción, donde se puede observar la cantidad a producir de un determinado mueble. También se muestra el estado en el cual se encuentra cada una de estas líneas. El botón \textbf{AD-25-05} permite ver las tareas asociadas a la orden de producción, navegando al escenario \textbf{AD-27-00} y el botón \textbf{AD-25-06} permite cancelar la linea de orden de producción indicada. Con el botón \textbf{AD-25-07} se podrá visualizar el PDF de la orden de producción, navegando al escenario \textbf{AD-26-00} teniendo la opción de imprimir.
\clearpage

\begin{figure}[H]
\centering
\includegraphics[width=\textwidth,height=\textheight,keepaspectratio]{Escenarios/AD-26-00}
\caption{Escenario - AD-26-00}
\label{fig:AD-26-00}
\end{figure}

Este escenario permite al usuario visualizar un documento PDF de la orden de producción en la sección \textbf{AD-26-01}. La sección \textbf{AD-26-02} muestras las opciones de impresión. Con el botón \textbf{AD-26-03} se puede navegar al escenario \textbf{AD-25-00}. Un click en el botón \textbf{AD-26-04} permitirá al usuario guardar o imprimir el documento PDF.
\clearpage

%\begin{figure}[H]
\centering
\includegraphics[width=\textwidth,height=\textheight,keepaspectratio]{Escenarios/AD-27-00}
\caption{Escenario - AD-27-00}
\label{fig:AD-27-00}
\end{figure}

Descripcion escenario...
\clearpage

\begin{figure}[H]
\centering
\includegraphics[width=\textwidth,height=\textheight,keepaspectratio]{Escenarios/AD-28-00}
\caption{Escenario - AD-28-00}
\label{fig:AD-28-00}
\end{figure}
Este escenario muestra toda la información referida a los clientes, junto con las acciones disponibles.
El botón \textbf{AD-28-01} permite navegar al escenario \textbf{AD-02-00}. El campo \textbf{AD-28-02} permite ingresar el nombre, apellido o razón social para filtrar los clientes. La lista desplegable \textbf{AD-28-04} permite al usuario filtrar por el tipo de cliente. El botón \textbf{AD-28-03} permite visualizar más filtros de búsqueda disponibles.
El botón \textbf{AD-28-05} permite al usuario crear un nuevo cliente y navega al escenario \textbf{AD-29-00}.
El botón \textbf{AD-28-06} permite al usuario seleccionar uno o más clientes del resultado de la búsqueda. El campo \textbf{AD-28-07} muestra la información relacionada a los clientes  especificando los nombres, apellidos, razón social, tipo de cliente, número de documento y número de teléfono. El botón \textbf{AD-28-08} permite navegar al escenario \textbf{AD-30-00} para ver los domicilios del cliente, el botón \textbf{AD-28-09} permite al usuario dar de alta o de baja el cliente, según el estado en el cual se encuentra. Un click en el botón \textbf{AD-28-10} navega al escenario \textbf{AD-29-00} para editar los datos del cliente y el botón \textbf{AD-28-11} permite al usuario borrar al cliente.
En  \textbf{AD-28-12} se mostrarán las páginas de resultado, pudiendo cambiar de página. En \textbf{AD-28-13} se mostrará cuantos resultados se están visualizando y el total.
\\

\begin{figure}[H]
\centering
\includegraphics[width=\textwidth,height=\textheight,keepaspectratio]{Escenarios/AD-29-00}
\caption{Escenario - AD-29-00}
\label{fig:AD-29-00}
\end{figure}
Este escenario permite a los usuarios crear o modificar clientes. El campo \textbf{AD-29-01} indica la operación que se esta realizando, pudiendo ser 'Crear cliente' o 'Editar cliente'.
Con el botón \textbf{AD-29-02} se podrá cerrar la ventana y volver al escenario \textbf{AD-28-00}.
En el caso que la operación sea 'Editar', los campos: \textbf{AD-29-04}, \textbf{AD-29-05}, \textbf{AD-29-06}, \textbf{AD-29-07}, \textbf{AD-29-08}, \textbf{AD-29-09}, \textbf{AD-29-10}, y \textbf{AD-29-11} estarán autocompletados con los datos del cliente a editar.
El botón \textbf{AD-29-03} permite indicar que se está creando o modificando un cliente que es una persona jurídica, en dicho caso los campos \textbf{AD-29-04} y \textbf{AD-29-05} se reemplazan por un único para indicar la razón social. El cliente debe completar el formulario para crear o editar un cliente, indicando en el campo \textbf{AD-29-04} los nombres, en el campo \textbf{AD-29-05} los apellidos, seleccionado de la lista desplegable \textbf{AD-29-06} el tipo de documento, especifando el número de documento en el campo \textbf{AD-29-07}, el número de teléfono en el campo \textbf{AD-29-08}, el correo electrónico en el campo \textbf{AD-29-09}, fecha de nacimiento en el campo \textbf{AD-29-10} y seleccionado la nacionalidad en de la lista desplegabale \textbf{AD-29-11}. 
Un click en el botón \textbf{AD-29-12} desplegara un formulario para que el usuario complete con los datos del domicilio del cliente.
Un click en el botón \textbf{AD-29-13} creará o modificará al cliente según corresponda y navegará al escenario \textbf{AD-28-00}, mientras que click en el botón \textbf{AD-29-14} cerrará la ventana y navegará al escenario \textbf{AD-28-00}.
\clearpage

\begin{figure}[H]
\centering
\includegraphics[width=\textwidth,height=\textheight,keepaspectratio]{Escenarios/AD-30-00}
\caption{Escenario - AD-30-00}
\label{fig:AD-30-00}
\end{figure}
Este escenario muestra la información y acciones referidas a los domicilios de un cliente. 
El botón \textbf{AD-30-01} permite navegar al escenario \textbf{AD-28-00} o \textbf{AD-02-00} según el click del usuario.
El botón \textbf{AD-30-02} permite a los usuarios crear un nuevo domicilio para el cliente navegando al escenario \textbf{AD-31-00}.
El campo \textbf{AD-30-03} muestra la información de los domicilios del cliente, como ser el domicilio, código postal, ciudad, provincia y país. El botón \textbf{AD-30-04} permite al usuario borrar un domicilio del cliente.
\\
\begin{figure}[H]
\centering
\includegraphics[width=\textwidth,height=\textheight,keepaspectratio]{Escenarios/AD-31-00}
\caption{Escenario - AD-31-00}
\label{fig:AD-31-00}
\end{figure}
Este escenario permite al usuario agregar un nuevo domicilio para un cliente.
Con el botón \textbf{AD-31-01} se podrá cerrar la ventana y volver al escenario \textbf{AD-30-00}.
El usuario deberá indicar en el campo \textbf{AD-31-02} la dirección del cliente, en el campo \textbf{AD-31-03} el código posta, seleccionar provincia de la lista desplegable \textbf{AD-31-04} y ciudad de la lista desplegable \textbf{AD-31-05}.
Un click en el botón \textbf{AD-31-06} creará el domicilio asignandolo al cliente y navegará al escenario \textbf{AD-30-00}.
\clearpage

\begin{figure}[H]
\centering
\includegraphics[width=\textwidth,height=\textheight,keepaspectratio]{Escenarios/AD-32-00}
\caption{Escenario - AD-32-00}
\label{fig:AD-32-00}
\end{figure}

Descripcion escenario...
\clearpage

\begin{figure}[H]
\centering
\includegraphics[width=\textwidth,height=\textheight,keepaspectratio]{Escenarios/AD-33-00}
\caption{Escenario - AD-33-00}
\label{fig:AD-33-00}
\end{figure}

Descripcion escenario...
\clearpage

\begin{figure}[H]
\centering
\includegraphics[width=\textwidth,height=\textheight,keepaspectratio]{Escenarios/AD-34-00}
\caption{Escenario - AD-34-00}
\label{fig:AD-34-00}
\end{figure}

Descripcion escenario...
\clearpage

\begin{figure}[H]
\centering
\includegraphics[width=\textwidth,height=\textheight,keepaspectratio]{Escenarios/AD-37-00}
\caption{Escenario - AD-37-00}
\label{fig:AD-37-00}
\end{figure}

Descripcion escenario...
\clearpage

\begin{figure}[H]
\centering
\includegraphics[width=\textwidth,height=\textheight,keepaspectratio]{Escenarios/AD-38-00}
\caption{Escenario - AD-38-00}
\label{fig:AD-38-00}
\end{figure}

Este escenario permite crear una ubicación. Con el botón \textbf{AD-38-01} se podrá volver al escenario \textbf{AD-37-00}. Se debe ingresar el nombre de la ubicación en \textbf{AD-38-02}, las observaciones que se deseen realizar en \textbf{AD-38-03}, el domicilio en \textbf{AD-38-04}, código postal en \textbf{AD-38-05} y las listas desplegables \textbf{AD-38-06}, \textbf{AD-38-07} y \textbf{AD-38-08}  permiten indicar el pais, provincia y ciudad, respectivamente, del domicilio que se indicó. El botón \textbf{AD-38-10} permite crear la ubicación con los valores indicados en los campos y listas desplegables mencionadas anteriormente.
\clearpage

\begin{figure}[H]
\centering
\includegraphics[width=\textwidth,height=\textheight,keepaspectratio]{Escenarios/AD-39-00}
\caption{Escenario - AD-39-00}
\label{fig:AD-39-00}
\end{figure}

Descripcion escenario...
\clearpage

\begin{figure}[H]
\centering
\includegraphics[width=\textwidth,height=\textheight,keepaspectratio]{Escenarios/AD-40-00}
\caption{Escenario - AD-40-00}
\label{fig:AD-40-00}
\end{figure}

Este escenario permite crear o modificar una tela, en \textbf{AD-40-01} se indicará la operación que se esté realizando mostrando 'Crear Telas' o 'Modificar Telas'. Con el botón \textbf{AD-40-02} se podrá volver al escenario \textbf{AD-39-00}. Se debe ingresar el nombre de la tela en \textbf{AD-40-03} y el precio en \textbf{AD-40-04}. En caso de estar modificando un producto, los campos \textbf{AD-40-03} y \textbf{AD-40-04} estarán autocompletados con los valores actuales de la tela, pudiendo modificar los mismos. Al hacer click en el botón \textbf{AD-40-05} se modificará o creará la tela según corresponda, mostrando en el botón el texto 'Crear' o 'Modificar' respectivamente. Con el botón \textbf{AD-40-06} se podrá volver al escenario \textbf{AD-39-00} cancelando la operación que se esté realizando.
\clearpage

\begin{figure}[H]
\centering
\includegraphics[width=\textwidth,height=\textheight,keepaspectratio]{Escenarios/AD-43-00}
\caption{Escenario - AD-43-00}
\label{fig:AD-43-00}
\end{figure}

Descripcion escenario...
\clearpage

\begin{figure}[H]
\centering
\includegraphics[width=\textwidth,height=\textheight,keepaspectratio]{Escenarios/AD-44-00}
\caption{Escenario - AD-44-00}
\label{fig:AD-44-00}
\end{figure}

Este escenario permite a un usuario ver el historial de precios de un producto en una gráfica. En \textbf{AD-44-01} se puede visualizar el nombre del producto al cual se le está viendo los precios. El botón \textbf{AD-44-02} permite cerrar la ventana, volviendo al escenario \textbf{AD-42-00}. En \textbf{AD-44-03} se puede visualizar una gráfica con el precio que tuvo el producto a lo largo del tiempo.
\clearpage

\begin{figure}[H]
\centering
\includegraphics[width=\textwidth,height=\textheight,keepaspectratio]{Escenarios/AD-46-00}
\caption{Escenario - AD-46-00}
\label{fig:AD-46-00}
\end{figure}

Descripcion escenario...
\clearpage

\begin{figure}[H]
\centering
\includegraphics[width=\textwidth,height=\textheight,keepaspectratio]{Escenarios/AD-48-00}
\caption{Escenario - AD-48-00}
\label{fig:AD-48-00}
\end{figure}

Descripcion escenario...
\clearpage

\begin{figure}[H]
\centering
\includegraphics[width=\textwidth,height=\textheight,keepaspectratio]{Escenarios/AD-49-00}
\caption{Escenario - AD-49-00}
\label{fig:AD-49-00}
\end{figure}

Descripcion escenario...
\clearpage

\begin{figure}[H]
\centering
\includegraphics[width=\textwidth,height=\textheight,keepaspectratio]{Escenarios/AD-51-00}
\caption{Escenario - AD-51-00}
\label{fig:AD-51-00}
\end{figure}

Descripcion escenario...
\clearpage

%\input{Capitulo10.tex}
\end{document}