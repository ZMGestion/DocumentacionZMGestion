\chapter{Introducción}
\label{ch:capitulo1} 
\markboth{CAPÍTULO \ref*{ch:capitulo1}. Introducción}{CAPÍTULO \ref*{ch:capitulo1}. Introducción}
\section{Introducción}
\paragraph\indent
El presente documento corresponde al trabajo de graduación de la carrera de Ingeniería en Computación de la Facultad de Ciencias Exactas y Tecnología - Universidad Nacional de Tucumán, de los alumnos Nicolás Bachs y Loïk Choua.

\paragraph\indent
Esta documentación surgió a partir de sucesivas versiones del mismo que fueron revisadas por los usuarios finales del sistema, los tutores y jurados del proyecto.

\paragraph\indent
Zimmerman Muebles es una empresa dedicada a la fabricación y comercialización de muebles y artículos de decoración. En los últimos años la empresa ha automatizado algunas cuestiones administrativas de forma heterogénea y aislada. Sin embargo, la automatización de los procesos de presupuestación, venta, producción, entrega de productos, como la previsión de un proyecto integral que contemple todas las áreas concentrando la información e una única fuente de datos no han sido realizados. 
Con esa inquietud la gerencia de Zimmerman Muebles contactó a uno de los integrantes del grupo de este trabajo de graduación consultando sobre el desarrollo de un sistema que contemple y resuelva todas las necesidades existentes.

\paragraph\indent
El desarrollo de este sistema fue utilizado como caso práctico para la resolución de trabajos prácticos de las asignaturas Ingeniería de Software I y Laboratorio de Bases de Datos por los integrantes de estre grupo.

\section{Objetivos}
\subsection{Generales}
\paragraph\indent
Con respecto al trabajo de graduación, el objetivo principal es desarrollar el sistema de gestión para una mueblería, poniendo en práctica todos los conocimientos y habilidades adquiridas durante el transcurso de la carrera.

\subsection{Específicos}
\paragraph\indent
Desarrollar un sistema (mediante el uso de tecnologias web) para una mueblería, con el fin de optimizar el proceso de manejo de stock, presupuestos, ventas y entrega de productos, y que cumpla con las siguientes características:
\begin{itemize}
	\item permita automatizar los procesos del negocio,
	\item sea intuitivo y fácil de utilizar para los empleados,
	\item se pueda implementar rápidamente,
	\item permita almacenar y procesar datos que contribuyan a la toma de decisiones,
	\item sea capaz de brindar información resumida y procesada acerca de las operaciones que se llevan a cabo,
	\item verifique los requisitos de software ANSI/IEEE 830 que se describen en este documento.
\end{itemize}
