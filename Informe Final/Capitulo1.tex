\chapter{Introducción}
\label{ch:capitulo1} 
\markboth{CAPÍTULO \ref*{ch:capitulo1}. Introducción}{CAPÍTULO \ref*{ch:capitulo1}. Introducción}
\section{Introducción}
\paragraph\indent
Esta especificación tiene como objetivo analizar y documentar las necesidades funcionales que deberán ser soportadas por el sistema a desarrollar en forma clara, precisa, completa y verificable. Para ello, se identificarán los requisitos que ha de satisfacer el nuevo sistema mediante entrevistas, recolección de datos, el estudio de los problemas de las unidades afectadas y sus necesidades actuales. Además de identificar los requisitos se deberán establecer prioridades, lo cual proporciona un punto de referencia para validar el sistema final que compruebe que se ajusta a las necesidades del usuario. Esta documentación surgió a partir de sucesivas versiones del mismo que fueron revisadas por los usuarios finales del sistema, los tutores y jurados del proyecto.
\section{Objetivos}
\subsection{Generales}
\paragraph\indent
Con respecto al trabajo de graduación, el objetivo principal poner en práctica todos los conocimientos y habilidades adquiridas durante el transcurso de la carrera, mediante la construcción de un producto de software que satisfaga las necesidades y genere valor agregado a un cliente determinado.

\subsection{Específicos}
\begin{itemize}
	\item Diseñar un sistema intuitivo, fácil de utilizar y que pueda ser implementado rápidamente.
	\item Diseñar el sistema de manera que se puedan añadir nuevas funcionalidades al mismo, teniendo en cuenta el constante avance de la tecnología.
\end{itemize}
